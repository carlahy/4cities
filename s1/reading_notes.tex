\documentclass{article}

\usepackage[utf8]{inputenc}
\usepackage[left=1.5in,right=1.5in,bottom=1in]{geometry}
\setlength\parindent{0pt}
\setlength{\parskip}{1em}
\setcounter{secnumdepth}{0}
\usepackage{outlines}
\usepackage{graphicx}
\graphicspath{ {imgs} }
\usepackage{hyperref}
\usepackage{chronology}


\usepackage[
backend=biber,
style=alphabetic,
sorting=ynt
]{biblatex}
\addbibresource{reading.bib}


\title{Reading Notes}
\author{Carla Hyenne }

\begin{document}

\maketitle

\tableofcontents

\pagebreak

\section{Neoliberalism}

\subsection{Roger Keith, \textit{Urban Neoliberalism: Rolling with the changes in a globalising world}}

Keywords: neoliberalism, roll back/roll out/roll with it urbanism, Brenner/Schmidt, Schipper, governmentality, ecological dominance

\begin{outline}
	\1  Urbanisation and neoliberalisation are processes that are material and discursive. They have real ramifications, through which modern capitalist societies are reproduced
	\1\textbf{Neoliberalism} is the role that the state plays in facilitating market rule, and the expansion of market mechanisms and thinking to extra-economic sectors that it entails
		\2 \textbf{Roll-back neoliberalism}: the state does not play a prominent role in early neoliberalism
		\2 \textbf{Roll-out neoliberalism}: the state facilitates neoliberalism, with new forms of institution-building and governmental intervention, concerned specifically with the regulation, disciplining, and containment of those marginalised and dispossessed by the 1980 neoliberalism\marginpar{Revanchist state policies in favour of gentrification}
	\1 Foucault-inspired critique
		\2 \textbf{Governmentality}: under neoliberalism, individuals govern themselves and are responsible for their own well-being, rather than relying on the government/the state to intervene. Political subjects "govern at a distance"
		\2 In the neoliberal city, we can expect people to govern themselves according to the model of the enterprise and the norms of competition 
	\1 Neo-Marxist critique
		\2 Neoliberalism is a capitalist project orchestrated by a hegemonic ruling-class, in order to create an \textbf{`ecological dominance'} where one system imposes itself on others
		\2 Ecological dominance is what has happened. There is, in the 21st century, a near complete commodification of urban life and space; exchange and value-oriented activities are dominant; general liberalisation; strengthening of competitive power and of the value of shareholders; $\rightarrow$ everything is now competitive\marginpar{Financialisation of everything}
	\1 When did the neoliberalism, which affects the urban, start?
		\2 Harvey: late 1970s with the end of the Fordism/Keynesianism crisis, Chile's experimental neoliberal government after the putsch, the election of Reagan and Thatcher
		\2 UK and USA policy makers created "urban enterprise zones"
		\2 Reagan and Thatcher's campaigns worked as disciplining strategies against the urban working class, which occupied (factually or supposedly) uncontrollable spaces of violent contestations or just general discontent
		\2 Post-fordism (accumulation based on flexible processes) and world city formation (creating urban decision-making centres for global capitalism) were two concrete processes through which neoliberalisation proceeded
	\1 Urbanisation and neoliberalisation have changed urban infrastructure: there is a \textbf{`splintering urbanism'}, of sharply segregated, class-divided, privatised and access-controlled infrastructure in cities and suburbs
		\2 Neoliberal urbanism makes it easy for residents with resources and power to get access to water/mobility/health/other critique infrastructures, and harder to poor and marginalised communities
		\2 New forms of segregation appear, where the poor are driven out of gentrified centres, into the `in-between' spaces or inner and outer suburbs
	\1 \textbf{Smart cities} have had incredible success in neoliberal urbanism: it created a new techno-economic strategy for companies and workers in `creative economies', created new techno-social and techno-spatial `constellations', associated with `millenials' and de-regulated inner-city urbanism, with displacement and gentrification in former inner-city working class neigbhourhoods
\end{outline}

\begin{chronology}[5]{1960}{2000}{100ex}[\textwidth]
\event[1960]{1972}{Riotous period}
\event{1963}{First signs of gentrification}
\event{1970}{Fallout from the Fordism/Keynesian crises}
\event{1973}{Chile's experimental neoliberal government}
\event{1979}{Thatcher and Regan}
\event{1978}{Emergence of neoliberalism that affects urban processes \cite{keil2016urban}}
\event{1980}{Frankfurt, Germany: lab of neoliberalism  \cite{keil2016urban}}
\event{1981}{Inner-city becomes attractive to investors}
\event[1980]{1990}{Rampant suburbanisation, `yuppie' culture}
\event{1986}{Post-fordism, world city formation}
\end{chronology}

\section{Chronology}

%\begin{chronology}[5]{1700}{1900}{100ex}[\textwidth]
%\event{1984}{one}
%\event[1985]{1986}{two}
%\event{\decimaldate{25}{12}{2001}}{three}
%\end{chronology}


\begin{chronology}[5]{1900}{1950}{100ex}[\textwidth]
%\event{1984}{one}
%\event[1985]{1986}{two}
%\event{\decimaldate{25}{12}{2001}}{three}
\end{chronology}


\begin{chronology}[5]{1950}{2000}{100ex}[\textwidth]
\event{1978}{Emergence of neoliberalism that affects urban processes \cite{keil2016urban}}
\event{1980}{Frankfurt, Germany: lab of neoliberalism  \cite{keil2016urban}}
\end{chronology}


\begin{chronology}[5]{2000}{2030}{100ex}[\textwidth]
\event{2001}{one}
\event[2007]{2008}{two}
%\event{\decimaldate{25}{12}{2001}}{three}
\end{chronology}

\pagebreak

\printbibliography

%%%%% ENVIRONMENTS %%%%%

\if{false}
\begin{outline}
	\1 
\end{outline}
\fi


\end{document}