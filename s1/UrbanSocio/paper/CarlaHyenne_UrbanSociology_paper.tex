\documentclass{article}

\linespread{1.5}
\usepackage[utf8]{inputenc}
\usepackage[left=1.5in,right=1.5in,bottom=1in]{geometry}
\setlength\parindent{0pt}
\setlength{\parskip}{1em}
\setcounter{secnumdepth}{0}
\usepackage{outlines}
\usepackage{graphicx}
\graphicspath{ {imgs} }
\usepackage{hyperref}
\usepackage{tabularx}
\usepackage{longtable}
\usepackage{color,soul}
\usepackage{comment}
\usepackage{enumitem}
\setlist[itemize]{noitemsep, topsep=0pt}
\usepackage{changepage}
\usepackage[normalem]{ulem}

% defines \RaggedRight
\usepackage{ragged2e}
% for nicer table rules
\usepackage{booktabs}
% define a new column type for convenience
%\newcolumntype{L}[1]{>{\hsize=#1\hsize\RaggedRight} X}
\newcolumntype{L}{>{\raggedright\arraybackslash}}

\usepackage[
backend=biber,
style=apa,
citestyle=authoryear,
sorting=nyt,
]{biblatex}
\addbibresource{references.bib}

\title{How can gig economy workers mobilise for their rights? Investigating strategies from Latino migrant food-delivery workers in New York City \\[5ex]Research Proposal \\[3ex]}
\author{Carla Hyenne}
\date{January 16th, 2021}

\begin{document}

\maketitle

\section{Abstract}

The goal of this research proposal is to investigate how Latino migrants working in NYC’s food delivery industry mobilised to demand their workers’ rights, a mobilisation that culminated in the formation of Los Deliveristas Unidos (LDU).
I situate the movement within social movement theory, specifically the framework developed by Nicholls from which I take three dimensions: the community and network ties between actors, the specialised resources that power the movement, and the institutional and organisational structures that exist in the city that enable LDU to demand workers' rights.
The approach I propose is to conduct semi-formal interviews with LDU stakeholders, and to the extent possible, external actors like the local government and police departments.
The aspiration of this research is that it can enable other gig economy workers, in other cities and continents, to gain fair employment conditions, social justice, and a right to the city.

\pagebreak

\section{Introduction}

The gig economy is a recent phenomenon that has emerged in cities world-wide. The appeal of flexibility, the low barrier-to-entry, and weekly remuneration have made it favourable for certain people to take up such work. In New York City, immigrants from Latin America constitute a significant majority of the food delivery workforce \parencite{ldu_report2021}. Extensive reporting and research has been done on the poor working conditions of delivery work, which have worsened during the Covid 19 pandemic (\parencite{newyorker2020uncertain}, \parencite{nytimes2021risk}).

One silver lining for NYC delivery workers has been the formation of the Los Deliveristas Unidos (LDU) movement. Created in March 2020, LDU are a group of thousands of mostly Latino migrants working for delivery apps, and making demands on their workers’ rights. The demands go from the right to access restaurant bathrooms’, to a minimum wage. 

Delivery workers in NYC have been campaigning for their rights since the rise in popularity of food delivery platforms in the early 2010s \parencite{TODO}, but to little avail. Their conditions have seldom improved. However, there has been more progress since 2020. What has changed? On one hand, the pandemic was key in publicly displaying the delivery workers’ struggles, especially because they were essential workers. 
They risked their health and their families’ to deliver food to those who could afford to stay home, did not have access to PPE or basic utilities like bathrooms, and worked extremely long shifts on less than minimum wage, and were left without work when restaurants were mandated to close. 
Moreover, workers are independent contractors. This deprives them from labour protections like a minimum wage and worker’s compensation, and bans them from unionising \parencite{dunn2019hustle}.
The supply of delivery workers increased because restaurants had to let employees go, as did the demand with people sheltering in place. NYC delivery workers went from 50,000 to 80,000 workers in 2020 \parencite{TODO}. At the same time, crime like theft and personal attacks also increased. 53\% of NYC delivery workers reportedly had their bikes stolen \parencite{brictv}.
Although these derogatory conditions existed prior to the pandemic, we can assume that the gravity of the health crisis accelerated the mobilisation of delivery workers.

LDU emerged because of the lack of protection from the system \parencite{TODO} - the digital platforms employing the workers, the government, the court, and the city. It has gained incredible momentum, and a rally in April 2021 attracted thousands of delivery workers \parencite{aponte_2021}. The movement was spawned by the Worker’s Justice Project (WJP), an NGO that organises immigrant and low-wage workers fighting for better work conditions and social justice. 
LDU has united with 32BJ, a powerful Service Employees International Union, and, in September 2021, following pressure from LDU, the NYC council voted to pass a bill requiring restaurants to provide bathroom access \parencite{vice2021bathroom}.

A report on food delivery workers was conducted by WJP, LDU and Cornell University. It looks at the demographics and everyday experiences of delivery workers: their age, gender, ethnicity, migration background, pay, equipment, and more \parencite{ldu_report2021}, and provides insights on the food-delivery ecosystem of NYC, but doesn’t analyse the forces that made LDU the success it is.

What is particularly interesting in the case of LDU, is that an important Latino community existed in NYC prior to the delivery worker groups. In NYC, Latinos constitute the largest share of migrants and of undocumented migrants, and reside predominantly in Queens and the Bronx \parencite{nycimmigrantpopulation2021}. Since Latinos represent a significant majority of food delivery workers in NYC, I want to investigate how the existence of a Latino community has contributed to the movement’s popularity.

In this research, I situate the LDU movement within social movement theory as articulated by Nicholls and use this framework to guide the research proposal. It provides a framework to explore what factors have made LDU a success. By success, I mean the extent to which LDU has been able to achieve tangible results regarding working conditions.

I will also draw on the right to the city as a framework to illustrate how the gig economy movements in general are an urban phenomenon and an urban struggle. I will also use rhetoric of undocumented migrant struggles to illustrate the multifaceted nature of food delivery workers in NYC.

This research is worthwhile because of the nature of the gig economy - a growing but poorly regulated economy, employing disadvantaged populations - and because it may shed light on power structures and relations within the urban landscape that could enable other gig economy movements in other cities, on other continents, to have similar success.

<Maybe expand on how this research differs from other research>

\section{Research Background and Question}

TODO:Find two papers related to urban social movements: eg. Doc Adams, Nichols, Sans Papier, social network papers/theories
Research background

\subsection{Delivery workers’ right to the city}

The right to the city as articulated by Lefebvre (\citeyear{lefebvre1995writings}) is interpreted by Don Mitchel (\citeyear{mitchell2003right}) as the right to inhabit public space, to participate in spatial production and to renegotiate the city \parencite{lee2018delivering}. Delivery workers are significant in the city. As one LDU and WJP member says, ``the city keeps saying we’re essential workers, and we want them to act like it and protect us’’ \parencite{ldutestimonials}. The city here is a metaphor for the system responsible for the delivery economy. It highlights the multi-dimensionality of the movement, involving private companies (digital platforms employing workers and providing the apps), politics (the national and local governments responsible for labour rights), society (those who deliver, and those who are delivered to), spatiality (the use of urban space by delivery workers, in their daily work and in their campaign), policing (the safety of the streets, or lack thereof).

The delivery workers’ right to the city is undermined by the system. On one hand, streets are unsafe to use, there is a constant threat of physical assault and electric bike theft, but on the other they are essential for the city. 
The assaults were particularly problematic on Willis Avenue bridge, which connects Manhattan to the Bronx and is used by many delivery workers going home after work. After failed efforts at engaging local police to provide better safety on this bridge, a group of Latino workers took the matter into their own hands. They self-organised a nightwatch, spread the word through Facebook and WhatsApp groups, and put up signs recommending to cross in groups of five or more (\parencite{vox2021}, \parencite{curbed2021}). 
Generally, lockdowns meant less street activity which inevitably reduced the number of ``eyes on the streets’’ \parencite{jacobs2007uses}. This made delivery workers easy targets.
Another concern with the police, is that a significant number of Latinos are undocumented. As a result, bike thefts go unreported, and injuries related to work or assault are not treated \parencite{brictv}.

The Deliveristas are fighting to make the streets safer because the system as described, isn't. They understand their right to a safe workplace that the city streets are. Through rallies, signs, recognisable work and LDU clothing, and their growing numbers in the city, they are making their presence evermore visible and their voices heard.

\subsection{Delivery workers social movements}

Nicholls (\citeyear{nicholls2008urban}) describes urban social movements as needing: one, a multitude of network relations between many actors and groups, and two, an institutional structure to work with.
This provides the basis for the framework I will use to research LDU. 

Network relations will be analysed by looking at community and network ties, and the specialised resources that emerge from these connections.
Diverse networks must exist and exhibit both strong and weak ties. Strong ties are useful to create and orchestrate specialised resources, and to build ``strong norms, trust, emotions and interpretive frameworks’’\parencite{nicholls2008urban}. People forge strong ties when they meet often over a shared interest. This creates an environment where individuals are willing to engage and take risks, and where resources are organised for the group to be efficient, effective and run smoothly. 
Weak ties are equally as important to connect distinct groups together, and encourage the sharing of specialised resources with broader movements. They create an interdependency between groups and generally strengthen the movement.

Social groups exist in the Latino population in NYC. This can be seen through the number of hispanic services, shops, and activities like community gardens \parencite{saldivar2004culturing}. Social organisations also exist. Perhaps the most relevant are the numerous immigrant organisations like Hispanic Federation, who support those arriving from Latin America. 

The movement’s primary goals are to gain rights for food delivery workers, but it also has an ethnic dimension. The movement’s name is in Spanish, all of their resources are available in Spanish, virtually all members are Latino, and some are not proficient in English. There is also a spatial dimension to the Latino communities, which are located predominantly in Queens and the Bronx \parencite{nycimmigrantpopulation2021}. 

The LDU movement in turn has created a community. Their Facebook page posts about ``sharing culture and faith’’, bike and motorcycle repair workshops, and drives to help people in the Latino community obtain an Individual Taxpayer Identification Number \parencite{facebookldu}.

The elements of community and neighbourhood regarding the Latino population will be central in my research - the main assumption being that spatial proximity has strengthened Latino communities, which in turn contributed to LDU’s strength.

TODO: I don't talk at all about the institutions in the city
\sout{Uitermark notes that cities are ``the relational conduits where movements connect and develop''} % \parencite{uitermark2012cities}

\sout{As Uitermark notes, a relational approach should be taken to understand how cities affect the socio-spatial development of urban social movements.} %\parencite{uitermark2012cities}

\subsection{Research question}

Given the above, my hypothesis is that there exists rich social capital in the Latino community, reinforced by the spatial concentration of Latinos in select neighbourhoods, and that the existence of these social ties directly contributed to the launch of the delivery worker movement, materialised by LDU. 
I also assume that the existence of institutions like WJP and 32BJ, have facilitated the growth of LDU, and that the presence of these institutions is directly linked to the ethnic diversity of NYC.

Henceforth, my research question is:

\begin{center}
How have Latino migrants working for NYC's food delivery industry mobilised their communities and networks to successfully unionise, and made use of the institutional structures in the city to demand fair working conditions?
\end{center}

To address the question, I will research three dimensions of social movements previously elaborated: the presence of strong and weak ties, the specialised resources emerging from these ties, and the institutional structures of the city. This will address the following three aspects, in no particular order. It is expected that they overlap and provide input to one another.

\subsubsection{Communities, neighbourhoods and networks}

First is to understand what communities exist in the Latino population. Latinos live predominantly in Queens and the Bronx, so I will investigate whether spatiality has an impact on the communities. I also need to understand the social bonds between Latinos inside and outside their close circles, what networks exist and the interactions between them, notably with external organisations. For example, with those who can represent them in court.

\subsubsection{Specialised Resources}

Second is to find out what specialised resources emerged from (strong) community and network ties. What are the high-quality resources that the city made available to LDU? How did they proliferate, notably through weak ties?
It is also useful to consider what resources are made available to delivery workers by their employer. This will help explain what it is that workers are demanding, to whom, and how.

\subsubsection{Institutional structures in the city}

Third is to understand what institutional structures exist in NYC that enabled LDU. The obvious organisation is the WJP which launched LDU. Others include 32BJ, a union with political influence; other gig economy movements, like Black and Asian worker movements; and political institutions like the police force and local government. It will be important to pay attention to the power relations between the civic and political organisms, but also between delivery workers and their employer.
The lack of opportunities for workers to exert agency in the gig economy is directly caused by the power that platforms exert over them \parencite{anwar2020hidden}.

\section{Research methodology}

The research will rely extensively on interactions with actors related to the movement like LDU members and organisers, Latino and non-Latino delivery workers, organisations like WJP and 32BJ, admins of Whatsapp and Facebook groups for delivery workers, etc. If possible, speaking with the local police departments and the local government will give an outside perspective to the movement.

The research will be qualitative. I will conduct semi-structured interviews with the different actors. The first point of contact will be LDU, because they have contact details online and will be able to provide further contacts. My connections will be made by recommendation of someone, by someone. My strategy will also include uncovering unknown unknown, that is, what I do not yet know but that may be important to the research. I expect this to happen naturally by talking to a variety of actors. Below, I outline the general interview interests by actor, and they may contribute to answering the research question.

\setlength\LTleft{-3cm}
\begin{longtable}{ Lp{.35\textwidth} Lp{.60\textwidth} Lp{.4\textwidth}}
\toprule
\textbf{Actors} & \textbf{What knowledge they can bring} & \textbf{What question they help answer}  \\
\midrule
Los Deliveristas Unidos & 
	How was LDU formed?
	What is the internal organisation?
	Who are the stakeholders? What are their responsibilities?
	What relations does LDU have with external organisations/institutions (governments, digital platforms, gig-economy movements, delivery workers, etc.)? &
	Specialised Resources; Institutional structures in the city \\ 
	\hline
	Members and organisers of Facebook/Whatsapp groups for delivery workers & 
		Why, how and by who were the groups started?
		How has LDU changed the groups? Do they exist today? What are the groups’ relations to LDU? How are they similar or different in their goals, communications and actions?
delivery workers, etc.)? &
	Communities, neighbourhoods and networks; Specialised Resources  \\ 
	\hline
	Self-organised interventions such as the night-watch on Willis Avenue Bridge &
	Why, how and by who were the groups started?
	How has LDU changed the interventions? Do they still take place? How, if at all, are the interventions coordinated with LDU? How are they similar or different in their goals, communications and actions?
	&
	Communities, neighbourhoods and networks; Specialised Resources  \\ 
	\hline
	Latino food delivery workers &
	Where they live, how they experience their neighbourhood, what is their community(ies), their background and experience as delivery workers
	Are they involved with LDU? If yes, why and how did they get involved, what is their role in LDU. If not, why not, what is preventing them from joining
	Are they part of other movements?
	Does LDU represent their aspirations?
	&
	Communities, neighbourhoods and networks; Specialised Resources  \\ 
	\hline
	Non-Latino food delivery workers (eg. Asian, Black, White, etc.) &
	Where they live, how they experience their neighbourhood, what is their community(ies), their background and experience as delivery workers
	Are they involved with LDU? If yes, why and how did they get involved, what is their role in LDU. If not, why not, what is preventing them from joining
	Are they part of other movements?
	How does the fact that LDU is mostly Latino influence their experience, and the outcome of the movement in their opinion?
	Does LDU represent their aspirations?
	 &
	Communities, neighbourhoods and networks; Specialised Resources  \\ 
	\hline
	NGOs, unions, and other institutions involved with LDU &
	Why and how did they get involved with LDU? What is their role? What are their resources, what do they bring to LDU that there would not be otherwise? Who do they work with (incl. other movements, public/private institutions)
	&
	Institutional structures in the city  \\ 
	\hline
	Local governments &
	What interactions do they have with LDU? Who communicates with, how, and with whom from LDU (or representative bodies)? What is their role and their power with regards to demands for fair working conditions? What are the (dis)advantages to working with movements like LDU? 
	&
	Institutional structures in the city  \\ 
\bottomrule
\caption{An non-exhaustive list of actors who will be interviewed, what knowledge they will bring, and how that will help answer the research question.}
\label{table:interviews}
\end{longtable}

\subsection{Limits of the Research}

The obstacles for the research are the language barrier and the geographical distance from NYC. The Latino community is Spanish speaking and some have limited proficiency in English. Having a limited Spanish proficiency reduces the options of interviewees, which means not being able to access the right people without a translator, and reduces the opportunity to genuinely connect with people.

If the research is conducted from Europe there may be difficulties in meeting and exchanging with the interviewees, especially in a group setting. Video conferencing is a great tool but does not replace real-life interactions. The global health crisis may further complicate travel and planned interviews. Additionally, the long working hours of delivery workers may make it difficult to find the time, since everyone also has life obligations.

NYC has been chosen as the focus of this research because of the activity already taking place there. However it would be very interesting to carry the research in other social, political and economic contexts like Europe which is itself very diverse.

Lastly, the research will only focus only on LDU, which represents almost only the Latino population of delivery workers. Although other groups like Black and Asian communities work with LDU, and the results of the campaigns affect all delivery workers regardless of their background, these other groups are not the primary focus of the research. This may exclude interesting dimensions of the gig economy movement. 

TODO: excluding undocumented migrant struggles \sout{It would be particularly interesting to research the intersectionality of the Latino community and the delivery workers, and to a lesser extent, the undocumented migrants. Perhaps strong community ties between Latino migrants has powered the mobilisation in a way that could not have been possible otherwise.}

\subsection{Plan of work and Time schedule}

TODO


\section{Conclusion}

Given the prevalence of food delivery workers in cities today, and their lack of welfare protection, it is worth investigating the opportunities and assets they have to come together and to be heard by those who have regulatory power.

The gig economy is an industry that has emerged in cities on all continents, and whose presence will keep growing for years to come. Unfortunately, gig workers lack protection that `conventional’ employees have, such as a minimum wage, worker’s compensation like health care, sick pay and vacation days, and the right to unionise. Their situation is exacerbated from the virtual absence of their employer, tech platforms like Uber Eats or Doordash, who shirk responsibilities by employing workers as independent contractors and who’s algorithms discriminate workers already in precarious situations \parencite{kellogg2020algorithms}.

Through this proposed research, I aim to understand how delivery workers in NYC have formed the Los Deliveristas Unidos organisation, which has gained incredible momentum since 2020. I focus on the Latino community, because they make up the greatest share of delivery workers in NYC and are the most represented by the movement. By using three dimensions of the social movement framework of Nichols and TODO, I aim to understand the community and network ties within the Latino community, the specialised resources they mobilised, and the institutional structures within the city that empower LDU to make influential demands.
Ultimately, understanding how movements such as LDU can form is important because it can enable other gig economy workers in other cities to claim their right to safety, to social justice and to the streets. 

\pagebreak

\printbibliography

\end{document}
