\documentclass{article}


\usepackage[utf8]{inputenc}
\usepackage[left=1in,right=1in,bottom=1in]{geometry}
\setlength\parindent{0pt}
\setlength{\parskip}{1em}
\setcounter{secnumdepth}{0}

\title{Urban Sociology}
\author{Carla Hyenne }
\date{\today}

\begin{document}

\maketitle

\tableofcontents

\pagebreak

\section{Introduction}
\date{September 28th, 2021}

\subsection{Urbanized World}

According to the UN, "for the first time in human history a majority of the world's population lives in urban areas". However, there are varying definitions of the "urban". For example
\begin{itemize}
  \item \textbf{The number of people} living in a space should be of a certain density. This density changes per country, and ranges from scale of 1 to 10's of thousands
  \item \textbf{The type of employment} in a space should be mostly not agricultural, ie. what you expect in a non-urban, rural area
\end{itemize}

Furthermore, maps representing the scale of urbanization of countries can be misleading, because some "highly urbanized" countries like Australia or Brazil may be mostly undeveloped or non-habited land. 
  
\subsection{What is a city?}

Is Brussels a city? A city can be understood by the density of its people, commuting patterns, administrative boundaries, the concentration of activities, the type of activities (non-rural activities), and more.

But, are cities always urban? Do rural cities exist? These are questions we can ask ourselves, especially as the rural is always given in opposition to the urban.

\subsection{Founding Fathers of Sociology}

\subsubsection{Emergence of Urban Sociology}

Urban sociology was founded in the 19th century, and was influenced by the industrialization era. 
In the mid-19th century, cities of 1 million habitants or more appeared in Western Europe. This was a time of rapid urbanization and industrialization. 

\subsubsection{Max Weber}

Weber, a german sociologist, researched how capitalism emerge, and how \textit{ideas} move capitalism forward. This opposes Marx's view that \textit{things} move capitalism forward.

In \textit{The Nature of the City}, he concentrates on the medieval city. He says that in cities, people lack personal connections due to the size, but that the size is not enough to define a city.

The city is both a fortress and a market: the fortress represents the political and administrative activities: the regulation of land ownership, taxes, authority, military. The market represents economic activity: the specialisation, the marketplace, consumer and producer cities.

All his arguments are based on the medieval city, but is the medieval city still relevant?  On one hand, the political, economic, social activities are still concentrated in the cities, but the scale of the city is different, the city doesn't have a strong military presence, and production is often pushed out of cities.

Weber defines an urbanite as "a man who does not supply his own food need on his own land". This could help scope the urban vs. the rural: urban activity does not require land (for eg. shoe making), but rural activity does (agriculture). 

He defines the urban community as:

\begin{itemize}
  \item A fortification
  \item A market
  \item A court of its own, and at least partial autonomous law
  \item A related form of association
  \item At least partial autonomy and autocephaly (elections)
\end{itemize}

\subsubsection{Karl Marx}

Marx, a german philosopher, researched the negative impacts of capitalism. He defined the city (or town) as a place of concentration of the population, of instruments of production, of capital, and of pleasures and needs, ie. of consumerism.
Like many others, he saw the country side in opposition to cities. 

In \textit{Das Kapital}, Marx explains that Industrial capitalism changed the nature of social relations, and set in motion economic forces that would transform human society. In cities, workers clash against capitalists. Urbanisation reinforces capitalism because it is at the same time:

\begin{itemize}
  \item A natural outcome of the development of capitalism, and
  \item A launched for sustaining capitalism
\end{itemize}

The above means that cities are growing because of capitalism. Capitalism requires more factories, which means more people (workers or investors) and more infrastructure (food, housing) is required, which cycles back to requiring more factories to increase production.

Marx also reflected on the negative effects of the link between capital accumulation and urbanisation, which generates miserable living conditions. 

\subsubsection{Friedrich Engels}

Engels was a german empirical sociologist, meaning he actually left his house and visited the cities he wrote about (contrary to Weber and Marx). He visited Manchester, and wrote about the conditions of working class in England. 

He concentrates on the negative effects of living in industrial cities, where industrialization is the cause of the bad living circumstances. He describes the dwellings as slums, with bad infrastructure with no ventilation, over crowding, dirty streets without sewers. There is clearly a segregation of the classes, where workers live in separate, miserable quarters from the middle classes. 

Molenbeek has been described as "little Manchester" due to its bad living conditions, which are still there today. Some houses are very degraded, with some apparently do not have bathrooms. 

\subsubsection{Emile Durkheim}

Durkheim is considered conservative. He wrote about suicide in cities, and explained that the rise of suicide was a symptom of the conditions in cities where there is a lack or change in social trends. 

In \textit{The Division of Labor in Society}, he describes modern society as a move from \textbf{mechanical to organic solidarity}. Mechanical refers to the small scale, in-differentiated societies who share common values and norms, and have no individuality. "Everyone is the same" and there is a \textit{collection consciousness}. On the other hand, organic solidarity is a product of the complex division of labour, specialised occupations, where individuals are \textit{interdependent} and function as organs of a living body ("a shoe maker can't each their shoes").

\subsection{What is urban sociology?}

Urban sociology is the study of patterns of everyday life in cities: its social structures, social processes, and social interactions. It is an \textbf{empirical discipline}. There are three main questions:

\begin{enumerate}
  \item \textit{Social order or social cohesion}: what keeps an urban society organised? What keeps urban society together?
  \item \textit{Social inequality}: how are power and privilege divided in urban society? What consequences does this have on social groups?
  \item \textit{Social identity}: how to urban societal changes influence one's self?
\end{enumerate}

How does space fit in with sociology? How does it compare to geography?

\section{Social, psychological, and physical consequences of urbanization}
\date{October 5th, 2021}

\subsection{Readings}

\subsubsection{Urbanism as a Way of Life, Louis Wirth}

\begin{itemize}
  \item The rapidity of urbanisation makes it hard to follow and understand social changes, and we do not have a good sociological definition of the city/urbanism
  \item The urban mode of life is not confined to cities
  \item Urbanism is a complex set of traits that make up the characteristics of city life, and urbanisation develops and extends these traits. Urbanisation is not capitalism or industrialisation!
  \item Speaks of size, density and heterogeneity as characteristics of what makes a space a city
  \item Density: diversification and specialisation, close physical contact but distant social relations, complex pattern of segregation, predominance of social control, accentuated friction
  \item Heterogeneity: break down rigid social structures, produces increased mobility, instability, insecurity, intersecting and tangential social groups with high membership turnover. The city encourages diversity by definition, by bringing people from distant places specifically because these people are different
  \item Urbanites are highly dependent on each other, in so far as they need each other's activities to survive (compared to the rural, who can be more self-sufficient). Conversely, urbanites are less dependent on specific people, rather to a group of people who perform certain activities (it doesn't matter \textit{who} will drive your bus to work, it only matters that some driver does). Cities are characterised by \textbf{secondary} rather than \textbf{primary} contacts
  \item Acquaintances in cities are utilitarian, the role that people play in our lives is regarded as a means to achieve our own ends. The specialization of tasks in cities segments people and creates a utilitarian nature.
  \item A city needs immigration (domestic or international) to grow, because growth cannot be sustained purely by reproduction of its residents. 
  \item In a city, the individual acts within a group rather than on their own.
\end{itemize}

\subsubsection{Community and Society, Tonnies}

\begin{itemize}
  \item Gemeinschaft (community) vs. Gesellschaft (society)
  \item Gemeinschaft
  \item Gesellschaft
\end{itemize}


\subsubsection{Simmel}


\begin{itemize}
  \item 
  \item
\end{itemize}

\section{The Chicago School}
\date{September 28th, 2021}

\section{The Legacy of the Chicago School}
\date{September 28th, 2021}

\section{Undocumented migrant struggles in/over the city}
\date{September 28th, 2021}

\section{Urban Politics}
\date{September 28th, 2021}

\section{Social networks, Neighborhoods, and Communities}
\date{September 28th, 2021}

\section{Global Cities}
\date{September 28th, 2021}

\section{Urban struggle and urban social movements}
\date{September 28th, 2021}

\section{Crime, policing and public space}
\date{September 28th, 2021}

\section{Urban Culture and lifestyle. The postindustrial city}
\date{September 28th, 2021}



%A very helpfull reference is~\cite{dietz1982maintaining}.
%\bibliographystyle{plain}
%\bibliography{References.bib}
\end{document}
