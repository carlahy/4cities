\documentclass{article}


\usepackage[utf8]{inputenc}
\usepackage[left=1in,right=1in,bottom=1in]{geometry}
\setlength\parindent{0pt}
\setlength{\parskip}{1em}
\usepackage[normalem]{ulem}
\setcounter{secnumdepth}{0}
\usepackage{outlines}
\usepackage{hyperref}

\title{Urban Sociology}
\author{Carla Hyenne }
\date{\today}

\begin{document}

\maketitle

\tableofcontents

\subsection{Syllabus}

\begin{tabular}{|c|c|c|}
	\hline
	Date & Lecture & Post? \\ 
	 \hline
	28/09/21 & Introduction &  \\  
	 \hline
	4/10/21 & Social, psychological, and physical consequences of urbanization & No \\   
	 \hline
	 11/10/21 & The Chicago School & Yes \\
	 \hline
	 18/10/21 & The Legacy of the Chicago School & Yes \\   
	 \hline
	 25/10/21 & Undocumented migrant struggles in the city &  \\   
	  \hline
	 /11/21 & Urban Politics &  \\   
	 \hline
	 /11/21 & Social networks, neighbourhoods, communities &  \\   
	 \hline
	 /11/21 & Global cities &  \\   
	  \hline
	 /11/21 & Urban struggle and urban social movements &  \\   
	 \hline
	 /12/21 & Crime, policing, public space &  \\   
	 \hline
	 /12/21 & Urban culture and lifestyle, the post industrial city &  \\   
	
	
	 \hline
\end{tabular}

\pagebreak

%%%%% LECTURE 1 %%%%%

\section{Introduction}
\date{September 28th, 2021}

\subsection{Urbanized World}

According to the UN, "for the first time in human history a majority of the world's population lives in urban areas". However, there are varying definitions of the "urban". For example
\begin{itemize}
  \item \textbf{The number of people} living in a space should be of a certain density. This density changes per country, and ranges from scale of 1 to 10's of thousands
  \item \textbf{The type of employment} in a space should be mostly not agricultural, ie. what you expect in a non-urban, rural area
\end{itemize}

Furthermore, maps representing the scale of urbanization of countries can be misleading, because some "highly urbanized" countries like Australia or Brazil may be mostly undeveloped or non-habited land. 
  
\subsection{What is a city?}

Is Brussels a city? A city can be understood by the density of its people, commuting patterns, administrative boundaries, the concentration of activities, the type of activities (non-rural activities), and more.

But, are cities always urban? Do rural cities exist? These are questions we can ask ourselves, especially as the rural is always given in opposition to the urban.

\subsection{Founding Fathers of Sociology}

\subsubsection{Emergence of Urban Sociology}

Urban sociology was founded in the 19th century, and was influenced by the industrialization era. 
In the mid-19th century, cities of 1 million habitants or more appeared in Western Europe. This was a time of rapid urbanization and industrialization. 

\subsubsection{Max Weber}

Weber, a german sociologist, researched how capitalism emerge, and how \textit{ideas} move capitalism forward. This opposes Marx's view that \textit{things} move capitalism forward.

In \textit{The Nature of the City}, he concentrates on the medieval city. He says that in cities, people lack personal connections due to the size, but that the size is not enough to define a city.

The city is both a fortress and a market: the fortress represents the political and administrative activities: the regulation of land ownership, taxes, authority, military. The market represents economic activity: the specialisation, the marketplace, consumer and producer cities.

All his arguments are based on the medieval city, but is the medieval city still relevant?  On one hand, the political, economic, social activities are still concentrated in the cities, but the scale of the city is different, the city doesn't have a strong military presence, and production is often pushed out of cities.

Weber defines an urbanite as "a man who does not supply his own food need on his own land". This could help scope the urban vs. the rural: urban activity does not require land (for eg. shoe making), but rural activity does (agriculture). 

He defines the urban community as:

\begin{itemize}
  \item A fortification
  \item A market
  \item A court of its own, and at least partial autonomous law
  \item A related form of association
  \item At least partial autonomy and autocephaly (elections)
\end{itemize}

\subsubsection{Karl Marx}

Marx, a german philosopher, researched the negative impacts of capitalism. He defined the city (or town) as a place of concentration of the population, of instruments of production, of capital, and of pleasures and needs, ie. of consumerism.
Like many others, he saw the country side in opposition to cities. 

In \textit{Das Kapital}, Marx explains that Industrial capitalism changed the nature of social relations, and set in motion economic forces that would transform human society. In cities, workers clash against capitalists. Urbanisation reinforces capitalism because it is at the same time:

\begin{itemize}
  \item A natural outcome of the development of capitalism, and
  \item A launched for sustaining capitalism
\end{itemize}

The above means that cities are growing because of capitalism. Capitalism requires more factories, which means more people (workers or investors) and more infrastructure (food, housing) is required, which cycles back to requiring more factories to increase production.

Marx also reflected on the negative effects of the link between capital accumulation and urbanisation, which generates miserable living conditions. 

\subsubsection{Friedrich Engels}

Engels was a german empirical sociologist, meaning he actually left his house and visited the cities which he wrote about (contrary to Weber and Marx). He visited Manchester, and wrote about the conditions of working class in England. 

He concentrates on the \textbf{negative effects of living in industrial cities, where industrialization is the cause of the bad living circumstances}. He describes the dwellings as slums, with bad infrastructure with no ventilation, over crowding, dirty streets without sewers. There is clearly a segregation of the classes, where workers live in separate, miserable quarters from the middle classes. 

Molenbeek has been described as "little Manchester" due to its bad living conditions, which are still there today. Some houses are very degraded, with some apparently do not have bathrooms. 

\subsubsection{Emile Durkheim}

Durkheim is considered conservative. He wrote about suicide in cities, and explained that the rise of suicide was a symptom of the conditions in cities where there is a lack or change in social trends. 

In \textit{The Division of Labor in Society}, he describes modern society as a move from \textbf{mechanical to organic solidarity}. Mechanical refers to the small scale, in-differentiated societies who share common values and norms, and have no individuality. "Everyone is the same" and there is a \textit{collection consciousness}. On the other hand, organic solidarity is a product of the complex division of labour, specialised occupations, where individuals are \textit{interdependent} and function as organs of a living body ("a shoe maker can't each their shoes").

\subsection{What is urban sociology?}

Urban sociology is the study of patterns of everyday life in cities: its social structures, social processes, and social interactions. It is an \textbf{empirical discipline}. There are three main questions:

\begin{enumerate}
  \item \textit{Social order or social cohesion}: what keeps an urban society organised? What keeps urban society together?
  \item \textit{Social inequality}: how are power and privilege divided in urban society? What consequences does this have on social groups?
  \item \textit{Social identity}: how to urban societal changes influence one's self?
\end{enumerate}

How does space fit in with sociology? How does it compare to geography?

%%%%% LECTURE 2 %%%%%

\section{Social, psychological, and physical consequences of urbanization}
\date{October 5th, 2021}

\subsection{Readings}

\subsubsection{Urbanism as a Way of Life, Louis Wirth}

\begin{itemize}
  \item The rapidity of urbanisation makes it hard to follow and understand social changes, and we do not have a good sociological definition of the city/urbanism
  \item The urban mode of life is not confined to cities
  \item Urbanism is a complex set of traits that make up the characteristics of city life, and urbanisation develops and extends these traits. Urbanisation is not capitalism or industrialisation!
  \item Speaks of size, density and heterogeneity as characteristics of what makes a space a city
  \item Density: diversification and specialisation, close physical contact but distant social relations, complex pattern of segregation, predominance of social control, accentuated friction
  \item Heterogeneity: break down rigid social structures, produces increased mobility, instability, insecurity, intersecting and tangential social groups with high membership turnover. The city encourages diversity by definition, by bringing people from distant places specifically because these people are different
  \item Urbanites are highly dependent on each other, in so far as they need each other's activities to survive (compared to the rural, who can be more self-sufficient). Conversely, urbanites are less dependent on specific people, rather to a group of people who perform certain activities (it doesn't matter \textit{who} will drive your bus to work, it only matters that some driver does). Cities are characterised by \textbf{secondary} rather than \textbf{primary} contacts
  \item Acquaintances in cities are utilitarian, the role that people play in our lives is regarded as a means to achieve our own ends. The specialization of tasks in cities segments people and creates a utilitarian nature.
  \item A city needs immigration (domestic or international) to grow, because growth cannot be sustained purely by reproduction of its residents. 
  \item In a city, the individual acts within a group rather than on their own.
\end{itemize}

\subsubsection{Community and Society, Tonnies}

\begin{itemize}
  \item Gemeinschaft (community) vs. Gesellschaft (society)
  \item Gemeinschaft
  \item Gesellschaft
\end{itemize}

\subsubsection{Simmel}

\iffalse
\begin{itemize}
  \item 
  \item
\end{itemize}
\fi


\subsection{Last week}

\begin{outline}
	\1 The sociological definition of the city is more than size alone. It has social, economic, political dimensions (Max Weber)
	\1 Urbanisation changes social relationships between individuals: impersonal relationships (Weber), organic solidarity (Durkheim), class struggle between bourgeoisie and working class (Marx)
	\1 Industrialisation and urbanisation brought about the emergence of the social sciences: the city as a problem
\end{outline}

In this lesson, we get a deeper understanding of the sociological view on "the city" and urbanism, and on the impact of urbanization on European society (Tönnies, Simmel) and American society (Wirth). These authors describe the social and psychological consequences of urbanisation.

\subsection{Urbanisation vs. Urbanism}

\textbf{Urbanisation} is a process, the development and expansion of urbanism, refers to the origin of cities and the process of city building; has a historical perspective on the societal development and change; is the rise and fall, growth and decline of cities. $\rightarrow$ \textit{how cities change}

\textbf{Urbanism} is a way of living that makes urban communities, a set of characteristics and activities in the city;  cultural, spiritual, meanings, symbols of the city; patterns of everyday life; individual experiences and processes of adjustment to the city environment; social conflicts and political organisation; $\rightarrow$ \textit{what people do in cities}

Weber, Engels, Durkheim, Tönnies focus on the \textbf{relation between the historical development of the city (urbanisation) and its way of life (urbanism)}. They are interested in the question of modernity, and not the city as such.

Simmel, Wirth focus on \textbf{patterns of activity and ways of thinking found in the city (urbanism)}. They are interested in the city as such.

\subsection{Tönnies, \textit{Community and Society}}

A German philosopher, he studied the impact of urbanisation on European societies. He questioned the social order - how is the new society organised? - and came up with two types of social formation, Gemeinschaft and Gesellschaft, or village life pre-industrialisation vs. urban life in the industrial period.
Tönnies had an evolutionary and teleological view of the development of society: it was inevitable to move from 'community' to 'society', and we cannot go back. This leads to a weakening of social ties, and the loss of a shared set of belonging. 

\begin{outline}
	\1 Gemeinschaft: community
		\2 Folk culture, religious 
		\2 Organised around family, village, town (small social units) 
 		\2 Relationships are sentimental\footnote{This is a romanticised view of community, pre-modern life} 
		\2 Law and morality are not written, but know through folkways, mores, religions 
		\2 Clear social norms, "social will"
		\2 More egalitarian, not centred around money
		
	\1 Gesellschaft: society
		\2 Based on complex trade, complex division of labour, industry 
		\2 Larger social units of metropolis and nation-state (civilisation of the state)
		\2 Secondary associational, instrumental relationships
		\2 People are isolated individuals with rational will (free agents), are hostile, even competitive toward one another
		\2 Law and morality are written, there are conventions and agreements (contracts and contractual relationships), political legislation and public opinion, breeding mutual fear
		\2 Public opinion is important and influences us
		\2 Prevalence of ratio, or science, of thinking rationally
		\2 Money and wealth are important, influence the degree of freedom one has in society, and creates inequalities
\end{outline}


Resemblance with Max Weber's ideal types: community and society are ideal types that don't exist in society. Ideal types are a paradigm or model that does not fully conform to social reality, but that are useful for analytical purposes.

Resemblance with Durkheim's theory of solidarity: mechanical solidarity is Gemeinschaft, and organic solidarity is Gesellschaft. There is a division of labour in society, and the body is a metaphor for society.

Resemblance with Marx: money has value in society, and increases segregation (capitalism is the driving force). There is hostility towards each other, especially due to money, class consciousness and class struggle. There is a need for a revolution.

Resemblance with Simmel: the evolution from community to society has an impact on people's psyche. People change their temperament and character, "restless striving". There is individualism and a focus on money.

Tönnies has a negative image of human beings, depicting them as greedy and hostile, and also of the city as a divided city between the rich and the poor.

\subsection{Simmel, \textit{The Metropolis and Mental Life}}

Simmel was a German sociologist, and cofounder of the German Society for Sociology with Weber and Tönnies. He was heavily influenced by the early Chicago School (Wirth) and was concerned with the \textit{social psychology of Modernity}, within the city because that is where the subtle aspects of modernity were displayed more clearly. He did no empirical work and was an "armchair sociologist". 

Simmel views the city in cultural and socio-psychological terms, so on a micro-level. His explanation of \textbf{how urban life transforms individual consciousness}:


\begin{outline}
	\1 \textbf{Blasé attitude}: we are emotionally reserved and indifferent, because there are too many stimuli to process and we cannot pay attention to everyone
	\1 \textbf{Rational calculation}: punctuality, calculability and exactness are important to be on time, and to make money. This makes people predictable and efficient
	\1 \textbf{Individuality and individual freedom}: you can be who you want (personal expression, personal peculiarities, independent), but this can be lonely because it becomes harder to find people like you. There are many people yet little social interactions with them
	\1 \textbf{Mass culture} and objective culture is de-personalised
	\1 \textbf{Manifold and complex relationships}: there are many types of relationships, between neighbours, retail, colleagues, friends, family, strangers...
\end{outline}

For Simmel, cities are the seat of the money economy and commerce (=Weber), of an advanced economic division of labour (=Durkheim), of cosmopolitanism. 
The city has a functional magnitude beyond its immediate boundaries. 
There are two faces of modernity, on one hand, people are losing their individuality to society, but on the other, there is potential for emancipation.


\subsection{Walter Benjamin, \textit{The Arcades Project}}

Benjamin is a Marxist influenced by Simmel. He has a positive, romantic view of the city and street life. He describes a new type of urbanite, the flâneur: you can lose yourself in a crowd, strolling aimlessly and enjoying the "capitalist showcase", even if you acknowledge that you don't know who made the goods and you are completely disconnected from the social relations. You are letting the city impact and change your psyche.

\subsection{Louis Wirth, \textit{Urbanism as a Way of Life}}

German-born sociologist who grew up in the US. He was an important figure in the development of the Chicago School, and the first to call himself an "urban sociologist". He was interested in the impact of ghetto life on the jews' psyche. 

His work is inspired by Simmel: the way the city influences individual behaviour and produces an \textbf{urban way of life}, how life in the city produced a \textbf{distinctive urban culture}, and how the \textbf{city is a social entity} with social life.

For Wirth, the growth of cities and urbanisation in the world is one of the most impressive facts of modern times. The shift  to an urban society has brought about profound changes in virtually every phase of social life. 
His \textbf{Theory of Urbanism describes the social effects of size, density and heterogeneity of the city}. This theory has potential to predict impacts of urbanisation, as it can be empirically tested and revised.

Wirth's sociological definition of a city is a "relatively large, dense and permanent settlement of heterogeneous individuals". The city is a \textbf{social entity}! The role of an urban sociologist is to create theories of urbanism, study the differences between urban and rural modes of living, and discover forms of social action and organisation inside the city $\rightarrow$ \textit{urban sociologists must define theories of the city}.

\subsubsection{Social Effects of the City}

Social effects of size:
\begin{outline}
	\1 There is individual variability
	\1 Social relations are distant
	\1 Human relations are segmentalised, anonymous, utilitarian \textit{$\rightarrow$ interdependence and division of labour}
	\1 Anomie, the lack of usual social or ethical standards within a group, because people are highly individual
\end{outline}

Social effects of density:
\begin{outline}
	\1 Diversification and specialisation
	\1 Close physical contacts, but distant social relations, leading to loneliness
	\1 Glaring contrasts between people, and communities
	\1 A complex pattern of social segregation, based on class, income, jobs, social status, customs, habits, tastes, preferences... \textit{"the city as a mosaic of social worlds"}
	\1 Relativist perspectives and tolerance of differences, we become desensitised to things like poverty in the streets
	\1 Predominance of formal social control, 
	\1 Accentuated friction, irritation, tension in and between social groups/individuals
	\1 Rapid tempo and complex technology and infrastructure
\end{outline}

Social effects of heterogeneity:
\begin{outline}
	\1 Breakdown of rigid social structures
	\1 Increase in social mobility, instability, insecurity: there is less home ownership (physical footloose-ness), and this insecurity is accepted
	\1 Affiliation of individuals with a variety of intersecting and tangential social groups, with high membership turnover (as people moving within/outside of a city)
	\1 Cosmopolitanism: everyone is entitled to equal respect and consideration
	\1 Impersonal market, where we don't talk to each other during exchanges of goods and money
\end{outline}

Urban personality:

\begin{outline}
	\1 Increase in negative behaviours due to less social control and connections: personal disorganisation, mental breakdowns, suicide, delinquency, crime, corruption
	\1 Urbanites develop themselves through voluntary membership of groups, "fictional kinship groups", you find social solidarity through interest units
	\1 There is more social control
	\1 Masses are subject to manipulation by the media
\end{outline}

\subsection{Herbert Gans, Suburbanism as Way of Life}

Ran out of time in lecture...

%%%%% LECTURE 3 %%%%%

\section{The Chicago School - Understanding urban growth and `the unknown' urbanite}
\date{September 28th, 2021}

\subsubsection{Park, Human Ecology}

Web of life: beings are anchored in their environment, and are interrelated and mutually interdependent\footnote{Darwin's example of the relation of cats and red clovers}. This interdependence is symbiotic rather than societal. 

The balance of nature: there is a kind of equilibrium between beings and with their environment, and when that equilibrium is broken, the conditions of life change. The equilibrium can broken by a famine, epidemic, or invasion by a species. In reality, this "balance of nature" doesn't exist because something always comes along to disturb it before it is achieved. Our world today is highly mobile (people, things, money, microbes), so the equilibrium is always changing. In a society (human or not), after a crisis there is an increase in competition, and only when this competition decreases does cooperation exist that allow the society to exist again.

Competition, dominance, succession: dominance and succession are two principles that establish and maintain societal order. Every urban landscape can be explained by dominance, ie. competition: industries compete for the most valuable and strategic land, and the central shopping and banking districts have the highest value; the peripheries lose in value gradually, but can also be re-written to be valuable as the centre expands; high value/low value areas exist in competition yet in interdependence, because one cannot exist without the other. Thus, power, competition, dominance, shapes the landscape;
Succession describes a cyclic process of passing from an unstable, to a stable state of society. When society becomes unstable for some reason, competition increases, society changes, and then is stable again (state of equilibrium) until the next 'crisis'.

Biological economics: the economics of living beings? human ecology is neither geography, nor economics

Symbiosis and society: 

Questions/Comments: 

- if we took an individual A and put them in a city, would we expect the city to shape that individual in the same way as another individual B in the same city? -> no, of course not; Thus is there some resistance, or some choice, that an individual makes on how/if the city will shape them? And if so, wouldn't we say that the city doesn't influence the individual, rather than the individual chooses how to be influenced by it?

- "it is when, and to the extent that, competition declines that the kind of order which we call society may be said to exist" $\rightarrow$ how does this correlate with capitalism? aren't our capitalist societies, by definition, competing within and against each other? $\rightarrow$ but competition has to decline to a point that cooperation can exist. was there ever a point of time that we had so much competition that society could not exist or function? Would we consider the world wards such a time? but wasn't society still surviving, or was it another kind of society? 

- Park argues society grows (succession) when competition occurs, ie. a crisis has happened, and at some point, competition declines and equilibrium is found. Is there ever a growth without crisis and competition?

- The parallel of society to the flora/fauna kingdoms implies a natural order, and a way of existing (with domination, competition, fragile equilibrium) that we cannot circumvent as society. 

\begin{outline}
	\1 In "Human Ecology", Park talks about competition and dominance, explaining that competition shapes the urban landscape ("The area of dominance in any community is usually the area of highest land values" The City p. 8), and that a crisis must happen for society to grow. These concepts are referring to capitalism.
However, the word "capitalism" is never mentioned or discussed as such. Why is this? Capitalism was a known and implemented system at the time of writing. Is the omission by accident or purpose? What were the political stances of the Chicago School? Is Park referring to the same capitalism that we have today?
\end{outline}

\sout{Furthermore, competition and dominance are explained as ecological principles: "It is when, and to the extent that, competition declines that the kind of order which we call society may be said to exist" (The City, p. 7). 
Given that under capitalism, competition will never cease, this implies that society cannot live in the "co-operation" that supersedes competition. 
In Park's words, capitalism becomes a natural fact, inherent of human/animal/plant systems alike.}


\subsubsection{Burgess, The City}

Process of urban metabolism and mobility

\begin{outline}
	\1 Expansion as physical growth
		\2 Expansion is measured by the physical growth of the city. The city plans parks, boulevards, civic centres, etc, and considers land for development far beyond its city limits, in order to anticipate and control the city growth.
	\1 Expansion as a process
		\2 The city is organised in concentric circles, going from CBD $\rightarrow$ zone in transition (businesses, light manufacturing) $\rightarrow$ zone of working class homes $\rightarrow$ zone of middle class and family residences $\rightarrow$ commuter zones (suburbs, satellite cities). Thus expansion is the process by which a zone grows into an outer zone, pushing it further from the centre, extension and succession.
		\2 	Expansion is also concentration and decentralisation. Concentration: the convergence of transport in the CBD, where the political/economic/cultural life is centred. Decentralisation: ???
		\2 	Expansion can be measured not only by physical growth, but also changes in social organisation and personality types. 
	\1 Social organisation and disorganisation as processes of metabolism
		\2 The city is a place where people can be organically integrated. But, cities are growing faster than the reproduction rate, meaning that a large part of the population are immigrants (Burgess categorises this as a 'disturbance' of the metabolism); the assimilation of culture by immigrants is 'abnormal', since culture is typically learned by birth.
		\2 There is disorganisation in the city, and it is a normal process of reorganisation; it is a feeling of disorientation when a person arrives to the city and is confronted to new norms; the city must shape the individual.
		\2 There is segregation in the city: a type of organisation into economic and cultural groups. It allows groups to emancipate, but limits their development in some ways.
		\2 	The division of labour: a disorganisation, reorganisation and increasing differentiation. There is a huge variety of jobs, and ethnicities tend to perform a set jobs over others.
	 	\2 Expansion and metabolism" indicates an excessive increase in crime, disease, disorder, vice, insanity, suicide
	\1 Mobility as the pulse of the community
		\2 
\end{outline}

Questions:

- What is a centralised decentralised system? I understand centralisation as the process that agglomerated towns into a city, but how does decentralisation fit in to the Chicago school's concentric model? If an industry, or political function, is decentralised, then the organisation as zones would not apply anymore.
\sout{Surely, in the city, there is a more centralised system of governance than there was previously, when each town would have their own politics (I assume). Or is decentralisation referring to the city governing itself more independently than towns were, vis-à-vis of the state governance?}\

- Burgess asks how individuals are incorporated into the city and become an organic part of their society. As we learnt, thinking about the city as an living, natural organism is not accurate (there is nothing natural about cities), but still, cities absorb people in a way. 

- Why is "the city as an organism" a bad metaphor? 

\subsection{The Chicago School}

The Chicago School is a group of sociologists established in Chicago in 1892. Chicago is a city that surged from 2 million people in 1907, to three million in 1923 (a significant growth for the time). It was a transportation hub, with railways and the Chicago river. It is an industrial city, with grain, lumber and meat (``the meat city''). It is a place with a lot of activity.

It borrows from biology, envisions the city as a living organism with natural ways of growing\footnote{The city as a natural organism is the main critique of the Chicago School, because it is too simplistic as a model}. It conducts empirical studies, ethnographies, uses progressive methods, and studies marginalised groups. 

\subsubsection{Chicago School classics}

\begin{outline}
	\1 First work by WI Thomas, \textit{The Polish Peasant in Europe and America},  1908
		\2 An ethnography of Polish migrants, and how they lived in Chicago versus Warsaw. At the time, Chicago had the 3rd largest population of Polish people in the world
	\1 Nels Anderson, \textit{The Hobo: the sociology of the homeless man}, 1923
		\2 An ethnography, Anderson lived within the hobo community for some time
	\1 Louis Wirth, \textit{The Ghetto}, 1928
		\2 Author was himself Jewish, had an insider view
	\1 Harvey Zorbaugh, \textit{The Gold Coast and the Slum}, 1929
	\1 Paul Cressey, \textit{The Taxi-Dance Hall}, 1932
\end{outline}

\subsubsection{Similarities of the Chicago School studies}

\begin{outline}
	\1 The Chicago School sees the city as a mystery, \textbf{a strange collection of social worlds}
	\1 Things happen in a \textbf{natural way}, the city is a social organism with `natural areas' (the slum, the ghetto, hobohemia, using Burgess map as a basis). The city is divided in to social maps with different social communities
	\1 The methodology is centered on \textbf{participant observation}
	\1 They describe the \textbf{`unknown' urbanite}, ie., those who are not the white collared academics like themselves
	\1 \textbf{Social phenomena} like poverty, crime, vice,... were explained as \textbf{products of social disorganisation}, particularly the breaking up of primary social relations\marginpar{Similar to Durkheim, Tönnies}
\end{outline}

\subsubsection{Importance of the Chicago School}

\begin{outline}
	\1 \textbf{An interactionist perspective}, studies how people and social groups interact (micro level)
	\1 They do \textbf{empirical research}, like participant observation and ethnography
	\1 They connect social phenomena to \textbf{spatial patterns and temporal processes}. The locatedness of social facts in time and space, ie. they are located in their context; the Chicago mapping project which located social characteristics in space (and time)
\end{outline}

\subsubsection{Research Committee}

\subsection{Jane Addams Hull House}

\subsection{Robert Park, \textit{Human Ecology}}

\subsection{Ernest Burgess, \textit{The Growth of the City}}

\subsubsection{Social organisation and disorganisation as a process of metabolism}

\subsubsection{Mobility as the pulse of the community}

\subsection{Nels Anderson, \textit{The Hobo}}

\subsubsection{Anderson's ethnography}

\subsection{Criticism on the Chicago School}


%%%%% LECTURE 4 %%%%%

\section{The Legacy of the Chicago School: Urban ethnographies, urban poverty, and symbolic interaction}
\date{September 28th, 2021}

\subsection{Readings}

\subsubsection{Wacquant, Wilson, \textit{The Cost of Racial and Class Exclusion in the Inner City}}

\begin{outline}
	\1 Describes the relationship between inner-city dislocations, and the struggles and structural changes in society, economy and polity (political organisation). This connection is usually not made when explaining the conditions of people living in ghettos, ie. disconnected and segregated urban spaces. It asks: what are the features of the social structure in which ghetto residents try to survive? 
		\2 Compares class composition, welfare trajectories, economic and financial assets, social capital of blacks in ghettos, to those in low-poverty areas
	\1 The conditions of the people living in ghettos are not put into socio, political, economic context, but rather interpreted as individual cases or worst, a self-imposed phenomenon (people live in ghettos due to their own faults, culpability, if some people can make it out so can others)
	\1 The geographical concentration of black people in dilapidated territorial enclaves epitomises the social and economic marginalisation of these people
		\2 The higher the socio-spatial concentration of poverty, the more obstacles the residents face (neighbourhood effects?)
		\2 There is an exodus in the inner-city of jobs, working families, which leads to the deterioration of housing, schools, business, leisure places, community organisations, exacerbated by government policies
		\2 The ghetto is a "closed opportunity structure", endless cycle of poverty that few escape, and is self-reinforcing
\end{outline}

\subsubsection{Wacquant, \textit{Revisiting territories of relegation}}

\subsubsection{Elijah Anderson, \textit{Code of the Streets}}

\begin{outline}
	\1 \textbf{This week's comment}:
	What I like about this week's texts is that they emphasise society's collective responsibility towards ghettos, slums, and any neighbourhood where survival is the residents' main concern. They focus on the socio, political and economic conditions that created this situation, rather than blaming individuals.
 
This got me thinking about the neighbourhood effect we discussed in the last lecture. For example, if the violent youths were taken out of their neighbourhoods where the code of the street applies, would the dangerous street behaviour prevail? I think so: the conditions which led to the violent environment in the first place would still exist. The neighbourhoods are not dangerous because the people living there are innately violent, but because the lack of support from socio, political, economic systems end up fostering a violent environment.

Wacquant and Wilson seem to support the idea of neighbourhood effects when talking about the `exodus' of jobs and decent families from the inner-city. This phenomenon deteriorated the housing, schools, businesses, places of leisure and community, and created a self-reinforcing system: the people who can afford to will move out of the dilapidated, possibly dangerous inner-city district, which strengthens the concentration of poverty.

Do you think otherwise? Do you think that without the most violent minority, the code of the street would not be maintained as such and the neighbourhoods would become less aggressive, and move towards a `decent' community?

	\1 Neighbourhood effect: \textit{``Simply living in such an environment places young people at special risk of falling victim to aggressive behaviour''}
	
	\1 Without the violent minority, the neighbourhood could become more liveable and less aggressive: \textit{"Most people in inner-city communities are not totally invested in the code, but the significant minority of hard-cord street youths who are have to maintain the code" and that "many less alienated young black ... want a nonviolent setting in which to live"}. 

	\1 Vice versa of neighbourhood effect, would the street-oriented continue their behaviour in a new neighbourhood that previously wasn't governed by the code? This is a scenario that I assume would never, or rarely, take place. But I think not, because their living situation will have changed. Living in a better neighbourhood means better access to institutions and jobs.
	
		\1 If the street community were to get significantly weaker in numbers compared to the decent, the socio, economic, political situation that perpetuates the low-income, black communities will still exist, and the phenomenon of the code will win over. The street exists because of the lack of support from the system (what is this system?). It is a self-reinforcing system. It isn't worth for them to try to gain respect in the mainstream system, because they could gain respect faster and with more impact in the streets.
	
	\1 What socio, political, economic factors would have to change for there to be a tipping point from a violent to a 'less violent' neighbourhood?		

\end{outline}
		
\subsubsection{Questions \& comments}

\begin{outline}
	\1  What I liked about these texts is that they research the conditions of ghetto residents and the forces that come together to create such living conditions, but contrary to previous texts, they emphasise that the person who resides in delapidated areas is not to blame for their situation. Rather, the texts focus on the collective responsibility that the society has in regards to such spaces existing. They focus on the socio, political and economic forces that led to such conditions.

This shift of focus from the individual to a higher level, pushes society to take responsibility for the living conditions of ghetto residents. When the blame is on the individual, and the thinking is that if they only wanted to, they could escape the ghetto, society can ignore the everyday struggle and survival of the ghetto residents.

Overall, this week's texts paint marginalised communities in a better light than that of the Chicago School's.
	\1 Gemeinschaft in the streets? 
		\2 In "Code of the Streets", the code of the streets is described as "a set of informal rules governing interpersonal public behaviour".
	\1
\end{outline}


%%%%% LECTURE 5 %%%%%

\section{Undocumented migrant struggles in/over the city}
\date{September 28th, 2021}

%%%%% LECTURE 6 %%%%%

\section{Urban Politics}
\date{September 28th, 2021}


%%%%% LECTURE 7 %%%%%

\section{Social networks, Neighborhoods, and Communities}
\date{September 28th, 2021}


%%%%% LECTURE 8 %%%%%

\section{Global Cities}
\date{September 28th, 2021}


%%%%% LECTURE 9 %%%%%

\section{Urban struggle and urban social movements}
\date{September 28th, 2021}


%%%%% LECTURE 10 %%%%%

\section{Crime, policing and public space}
\date{September 28th, 2021}


%%%%% LECTURE 11 %%%%%

\section{Urban Culture and lifestyle. The postindustrial city}
\date{September 28th, 2021}


% ENVIRONMENTS

\if{false}
\begin{outline}
	\1
\end{outline}
\fi

\end{document}
