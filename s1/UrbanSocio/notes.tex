\documentclass{article}


\usepackage[utf8]{inputenc}
\usepackage[left=1in,right=1in,bottom=1in]{geometry}
\setlength\parindent{0pt}
\setlength{\parskip}{1em}
\usepackage[normalem]{ulem}
\setcounter{secnumdepth}{0}
\usepackage{outlines}
\usepackage{hyperref}

\title{Urban Sociology}
\author{Carla Hyenne }
\date{\today}

\begin{document}

\maketitle

\tableofcontents

\subsection{Syllabus}

\begin{tabular}{|c|c|c|}
	\hline
	Date & Lecture & Post? \\ 
	 \hline
	28/09/21 & Introduction & n/a  \\  
	 \hline
	4/10/21 & Social, psychological, and physical consequences of urbanization & No \\   
	 \hline
	 11/10/21 & The Chicago School & Yes \\
	 \hline
	 18/10/21 & The Legacy of the Chicago School & Yes \\   
	 \hline
	 26/10/21 & Undocumented migrant struggles in the city & Yes  \\   
	  \hline
	 16/11/21 & Urban Politics &  No \\   
	 \hline
	 23/11/21 & Social networks, neighbourhoods, communities & Yes  \\   
	 \hline
	 30/11/21 & Global cities and its consequences &  \\   
	  \hline
	 7/12/21 & Urban struggle and urban social movements &  \\   
	 \hline
	 14/12/21 & Crime, policing, public space &  \\   
	 \hline
	 21/12/21 & Urban culture and lifestyle, the post industrial city &  \\ 
	 \hline
\end{tabular}

\pagebreak

%%%%%%%%%%%%%%%%%%%%%%%%%%%%%%%%%%%%%%%%
%												LECTURE 1
%%%%%%%%%%%%%%%%%%%%%%%%%%%%%%%%%%%%%%%%

\section{Introduction}
\date{September 28th, 2021}

\subsection{Urbanized World}

According to the UN, "for the first time in human history a majority of the world's population lives in urban areas". However, there are varying definitions of the "urban". For example
\begin{itemize}
  \item \textbf{The number of people} living in a space should be of a certain density. This density changes per country, and ranges from scale of 1 to 10's of thousands
  \item \textbf{The type of employment} in a space should be mostly not agricultural, ie. what you expect in a non-urban, rural area
\end{itemize}

Furthermore, maps representing the scale of urbanization of countries can be misleading, because some "highly urbanized" countries like Australia or Brazil may be mostly undeveloped or non-habited land. 
  
\subsection{What is a city?}

Is Brussels a city? A city can be understood by the density of its people, commuting patterns, administrative boundaries, the concentration of activities, the type of activities (non-rural activities), and more.

But, are cities always urban? Do rural cities exist? These are questions we can ask ourselves, especially as the rural is always given in opposition to the urban.

\subsection{Founding Fathers of Sociology}

\subsubsection{Emergence of Urban Sociology}

Urban sociology was founded in the 19th century, and was influenced by the industrialization era. 
In the mid-19th century, cities of 1 million habitants or more appeared in Western Europe. This was a time of rapid urbanization and industrialization. 

\subsubsection{Max Weber}

Weber, a german sociologist, researched how capitalism emerge, and how \textit{ideas} move capitalism forward. This opposes Marx's view that \textit{things} move capitalism forward.

In \textit{The Nature of the City}, he concentrates on the medieval city. He says that in cities, people lack personal connections due to the size, but that the size is not enough to define a city.

The city is both a fortress and a market: the fortress represents the political and administrative activities: the regulation of land ownership, taxes, authority, military. The market represents economic activity: the specialisation, the marketplace, consumer and producer cities.

All his arguments are based on the medieval city, but is the medieval city still relevant?  On one hand, the political, economic, social activities are still concentrated in the cities, but the scale of the city is different, the city doesn't have a strong military presence, and production is often pushed out of cities.

Weber defines an urbanite as "a man who does not supply his own food need on his own land". This could help scope the urban vs. the rural: urban activity does not require land (for eg. shoe making), but rural activity does (agriculture). 

He defines the urban community as:

\begin{itemize}
  \item A fortification
  \item A market
  \item A court of its own, and at least partial autonomous law
  \item A related form of association
  \item At least partial autonomy and autocephaly (elections)
\end{itemize}

\subsubsection{Karl Marx}

Marx, a german philosopher, researched the negative impacts of capitalism. He defined the city (or town) as a place of concentration of the population, of instruments of production, of capital, and of pleasures and needs, ie. of consumerism.
Like many others, he saw the country side in opposition to cities. 

In \textit{Das Kapital}, Marx explains that Industrial capitalism changed the nature of social relations, and set in motion economic forces that would transform human society. In cities, workers clash against capitalists. Urbanisation reinforces capitalism because it is at the same time:

\begin{itemize}
  \item A natural outcome of the development of capitalism, and
  \item A launched for sustaining capitalism
\end{itemize}

The above means that cities are growing because of capitalism. Capitalism requires more factories, which means more people (workers or investors) and more infrastructure (food, housing) is required, which cycles back to requiring more factories to increase production.

Marx also reflected on the negative effects of the link between capital accumulation and urbanisation, which generates miserable living conditions. 

\subsubsection{Friedrich Engels}

Engels was a german empirical sociologist, meaning he actually left his house and visited the cities which he wrote about (contrary to Weber and Marx). He visited Manchester, and wrote about the conditions of working class in England. 

He concentrates on the \textbf{negative effects of living in industrial cities, where industrialization is the cause of the bad living circumstances}. He describes the dwellings as slums, with bad infrastructure with no ventilation, over crowding, dirty streets without sewers. There is clearly a segregation of the classes, where workers live in separate, miserable quarters from the middle classes. 

Molenbeek has been described as "little Manchester" due to its bad living conditions, which are still there today. Some houses are very degraded, with some apparently do not have bathrooms. 

\subsubsection{Emile Durkheim}

Durkheim is considered conservative. He wrote about suicide in cities, and explained that the rise of suicide was a symptom of the conditions in cities where there is a lack or change in social trends. 

In \textit{The Division of Labor in Society}, he describes modern society as a move from \textbf{mechanical to organic solidarity}. Mechanical refers to the small scale, in-differentiated societies who share common values and norms, and have no individuality. "Everyone is the same" and there is a \textit{collection consciousness}. On the other hand, organic solidarity is a product of the complex division of labour, specialised occupations, where individuals are \textit{interdependent} and function as organs of a living body ("a shoe maker can't each their shoes").

\subsection{What is urban sociology?}

Urban sociology is the study of patterns of everyday life in cities: its social structures, social processes, and social interactions. It is an \textbf{empirical discipline}. There are three main questions:

\begin{enumerate}
  \item \textit{Social order or social cohesion}: what keeps an urban society organised? What keeps urban society together?
  \item \textit{Social inequality}: how are power and privilege divided in urban society? What consequences does this have on social groups?
  \item \textit{Social identity}: how to urban societal changes influence one's self?
\end{enumerate}

How does space fit in with sociology? How does it compare to geography?

%%%%%%%%%%%%%%%%%%%%%%%%%%%%%%%%%%%%%%%%
%												LECTURE 2
%%%%%%%%%%%%%%%%%%%%%%%%%%%%%%%%%%%%%%%%

\section{Social, psychological, and physical consequences of urbanization}
\date{October 5th, 2021}

\subsection{Last week}

\begin{outline}
	\1 The sociological definition of the city is more than size alone. It has social, economic, political dimensions (Max Weber)
	\1 Urbanisation changes social relationships between individuals: impersonal relationships (Weber), organic solidarity (Durkheim), class struggle between bourgeoisie and working class (Marx)
	\1 Industrialisation and urbanisation brought about the emergence of the social sciences: the city as a problem
\end{outline}

In this lesson, we get a deeper understanding of the sociological view on "the city" and urbanism, and on the impact of urbanization on European society (Tönnies, Simmel) and American society (Wirth). These authors describe the social and psychological consequences of urbanisation.

\subsection{Urbanisation vs. Urbanism}

\textbf{Urbanisation} is a process, the development and expansion of urbanism, refers to the origin of cities and the process of city building; has a historical perspective on the societal development and change; is the rise and fall, growth and decline of cities. $\rightarrow$ \textit{how cities change}

\textbf{Urbanism} is a way of living that makes urban communities, a set of characteristics and activities in the city;  cultural, spiritual, meanings, symbols of the city; patterns of everyday life; individual experiences and processes of adjustment to the city environment; social conflicts and political organisation; $\rightarrow$ \textit{what people do in cities}

Weber, Engels, Durkheim, Tönnies focus on the \textbf{relation between the historical development of the city (urbanisation) and its way of life (urbanism)}. They are interested in the question of modernity, and not the city as such.

Simmel, Wirth focus on \textbf{patterns of activity and ways of thinking found in the city (urbanism)}. They are interested in the city as such.

\subsection{Tönnies, \textit{Community and Society}}

A German philosopher, he studied the impact of urbanisation on European societies. He questioned the social order - how is the new society organised? - and came up with two types of social formation, Gemeinschaft and Gesellschaft, or village life pre-industrialisation vs. urban life in the industrial period.
Tönnies had an evolutionary and teleological view of the development of society: it was inevitable to move from 'community' to 'society', and we cannot go back. This leads to a weakening of social ties, and the loss of a shared set of belonging. 

\begin{outline}
	\1 Gemeinschaft: community
		\2 Folk culture, religious 
		\2 Organised around family, village, town (small social units) 
 		\2 Relationships are sentimental\footnote{This is a romanticised view of community, pre-modern life} 
		\2 Law and morality are not written, but know through folkways, mores, religions 
		\2 Clear social norms, "social will"
		\2 More egalitarian, not centred around money
		
	\1 Gesellschaft: society
		\2 Based on complex trade, complex division of labour, industry 
		\2 Larger social units of metropolis and nation-state (civilisation of the state)
		\2 Secondary associational, instrumental relationships
		\2 People are isolated individuals with rational will (free agents), are hostile, even competitive toward one another
		\2 Law and morality are written, there are conventions and agreements (contracts and contractual relationships), political legislation and public opinion, breeding mutual fear
		\2 Public opinion is important and influences us
		\2 Prevalence of ratio, or science, of thinking rationally
		\2 Money and wealth are important, influence the degree of freedom one has in society, and creates inequalities
\end{outline}


Resemblance with Max Weber's ideal types: community and society are ideal types that don't exist in society. Ideal types are a paradigm or model that does not fully conform to social reality, but that are useful for analytical purposes.

Resemblance with Durkheim's theory of solidarity: mechanical solidarity is Gemeinschaft, and organic solidarity is Gesellschaft. There is a division of labour in society, and the body is a metaphor for society.

Resemblance with Marx: money has value in society, and increases segregation (capitalism is the driving force). There is hostility towards each other, especially due to money, class consciousness and class struggle. There is a need for a revolution.

Resemblance with Simmel: the evolution from community to society has an impact on people's psyche. People change their temperament and character, "restless striving". There is individualism and a focus on money.

Tönnies has a negative image of human beings, depicting them as greedy and hostile, and also of the city as a divided city between the rich and the poor.

\subsection{Simmel, \textit{The Metropolis and Mental Life}}

Simmel was a German sociologist, and cofounder of the German Society for Sociology with Weber and Tönnies. He was heavily influenced by the early Chicago School (Wirth) and was concerned with the \textit{social psychology of Modernity}, within the city because that is where the subtle aspects of modernity were displayed more clearly. He did no empirical work and was an "armchair sociologist". 

Simmel views the city in cultural and socio-psychological terms, so on a micro-level. His explanation of \textbf{how urban life transforms individual consciousness}:


\begin{outline}
	\1 \textbf{Blasé attitude}: we are emotionally reserved and indifferent, because there are too many stimuli to process and we cannot pay attention to everyone
	\1 \textbf{Rational calculation}: punctuality, calculability and exactness are important to be on time, and to make money. This makes people predictable and efficient
	\1 \textbf{Individuality and individual freedom}: you can be who you want (personal expression, personal peculiarities, independent), but this can be lonely because it becomes harder to find people like you. There are many people yet little social interactions with them
	\1 \textbf{Mass culture} and objective culture is de-personalised
	\1 \textbf{Manifold and complex relationships}: there are many types of relationships, between neighbours, retail, colleagues, friends, family, strangers...
\end{outline}

For Simmel, cities are the seat of the money economy and commerce (=Weber), of an advanced economic division of labour (=Durkheim), of cosmopolitanism. 
The city has a functional magnitude beyond its immediate boundaries. 
There are two faces of modernity, on one hand, people are losing their individuality to society, but on the other, there is potential for emancipation.


\subsection{Walter Benjamin, \textit{The Arcades Project}}

Benjamin is a Marxist influenced by Simmel. He has a positive, romantic view of the city and street life. He describes a new type of urbanite, the flâneur: you can lose yourself in a crowd, strolling aimlessly and enjoying the "capitalist showcase", even if you acknowledge that you don't know who made the goods and you are completely disconnected from the social relations. You are letting the city impact and change your psyche.

\subsection{Louis Wirth, \textit{Urbanism as a Way of Life}}

German-born sociologist who grew up in the US. He was an important figure in the development of the Chicago School, and the first to call himself an "urban sociologist". He was interested in the impact of ghetto life on the jews' psyche. 

His work is inspired by Simmel: the way the city influences individual behaviour and produces an \textbf{urban way of life}, how life in the city produced a \textbf{distinctive urban culture}, and how the \textbf{city is a social entity} with social life.

For Wirth, the growth of cities and urbanisation in the world is one of the most impressive facts of modern times. The shift  to an urban society has brought about profound changes in virtually every phase of social life. 
His \textbf{Theory of Urbanism describes the social effects of size, density and heterogeneity of the city}. This theory has potential to predict impacts of urbanisation, as it can be empirically tested and revised.

Wirth's sociological definition of a city is a "relatively large, dense and permanent settlement of heterogeneous individuals". The city is a \textbf{social entity}! The role of an urban sociologist is to create theories of urbanism, study the differences between urban and rural modes of living, and discover forms of social action and organisation inside the city $\rightarrow$ \textit{urban sociologists must define theories of the city}.

\subsubsection{Social Effects of the City}

Social effects of size:
\begin{outline}
	\1 There is individual variability
	\1 Social relations are distant
	\1 Human relations are segmentalised, anonymous, utilitarian \textit{$\rightarrow$ interdependence and division of labour}
	\1 Anomie, the lack of usual social or ethical standards within a group, because people are highly individual
\end{outline}

Social effects of density:
\begin{outline}
	\1 Diversification and specialisation
	\1 Close physical contacts, but distant social relations, leading to loneliness
	\1 Glaring contrasts between people, and communities
	\1 A complex pattern of social segregation, based on class, income, jobs, social status, customs, habits, tastes, preferences... \textit{"the city as a mosaic of social worlds"}
	\1 Relativist perspectives and tolerance of differences, we become desensitised to things like poverty in the streets
	\1 Predominance of formal social control, 
	\1 Accentuated friction, irritation, tension in and between social groups/individuals
	\1 Rapid tempo and complex technology and infrastructure
\end{outline}

Social effects of heterogeneity:
\begin{outline}
	\1 Breakdown of rigid social structures
	\1 Increase in social mobility, instability, insecurity: there is less home ownership (physical footloose-ness), and this insecurity is accepted
	\1 Affiliation of individuals with a variety of intersecting and tangential social groups, with high membership turnover (as people moving within/outside of a city)
	\1 Cosmopolitanism: everyone is entitled to equal respect and consideration
	\1 Impersonal market, where we don't talk to each other during exchanges of goods and money
\end{outline}

Urban personality:

\begin{outline}
	\1 Increase in negative behaviours due to less social control and connections: personal disorganisation, mental breakdowns, suicide, delinquency, crime, corruption
	\1 Urbanites develop themselves through voluntary membership of groups, "fictional kinship groups", you find social solidarity through interest units
	\1 There is more social control
	\1 Masses are subject to manipulation by the media
\end{outline}

\subsection{Herbert Gans, Suburbanism as Way of Life}

Ran out of time in lecture...

%%%%%%%%%%%%%%%%%%%%%%%%%%%%%%%%%%%%%%%%
%												LECTURE 3
%%%%%%%%%%%%%%%%%%%%%%%%%%%%%%%%%%%%%%%%

\section{The Chicago School - Understanding urban growth and `the unknown' urbanite}
\date{September 28th, 2021}

\subsection{The Chicago School}

The Chicago School is a group of sociologists established in Chicago in 1892. Chicago is a city that surged from 2 million people in 1907, to three million in 1923 (a significant growth for the time). It was a transportation hub, with railways and the Chicago river. It is an industrial city, with grain, lumber and meat (``the meat city''). It is a place with a lot of activity.

It borrows from biology, envisions the city as a living organism with natural ways of growing\footnote{The city as a natural organism is the main critique of the Chicago School, because it is too simplistic as a model}. It conducts empirical studies, ethnographies, uses progressive methods, and studies marginalised groups. 

\subsubsection{Chicago School classics}

\begin{outline}
	\1 First work by WI Thomas, \textit{The Polish Peasant in Europe and America},  1908
		\2 An ethnography of Polish migrants, and how they lived in Chicago versus Warsaw. At the time, Chicago had the 3rd largest population of Polish people in the world
	\1 Nels Anderson, \textit{The Hobo: the sociology of the homeless man}, 1923
		\2 An ethnography, Anderson lived within the hobo community for some time
	\1 Louis Wirth, \textit{The Ghetto}, 1928
		\2 Author was himself Jewish, had an insider view
	\1 Harvey Zorbaugh, \textit{The Gold Coast and the Slum}, 1929
	\1 Paul Cressey, \textit{The Taxi-Dance Hall}, 1932
\end{outline}

\subsubsection{Similarities of the Chicago School studies}

\begin{outline}
	\1 The Chicago School sees the city as a mystery, \textbf{a strange collection of social worlds}
	\1 Things happen in a \textbf{natural way}, the city is a social organism with `natural areas' (the slum, the ghetto, hobohemia, using Burgess map as a basis). The city is divided in to social maps with different social communities
	\1 The methodology is centered on \textbf{participant observation}
	\1 They describe the \textbf{`unknown' urbanite}, ie., those who are not the white collared academics like themselves
	\1 \textbf{Social phenomena} like poverty, crime, vice,... were explained as \textbf{products of social disorganisation}, particularly the breaking up of primary social relations\marginpar{Similar to Durkheim, Tönnies}
\end{outline}

\subsubsection{Importance of the Chicago School}

\begin{outline}
	\1 \textbf{An interactionist perspective}, studies how people and social groups interact (micro level)
	\1 They do \textbf{empirical research}, like participant observation and ethnography
	\1 They connect social phenomena to \textbf{spatial patterns and temporal processes}. The locatedness of social facts in time and space, ie. they are located in their context; the Chicago mapping project which located social characteristics in space (and time)
\end{outline}

\subsubsection{Research Committee}

\subsection{Jane Addams Hull House}

\subsection{Robert Park, \textit{Human Ecology}}

\subsection{Ernest Burgess, \textit{The Growth of the City}}

\subsubsection{Social organisation and disorganisation as a process of metabolism}

\subsubsection{Mobility as the pulse of the community}

\subsection{Nels Anderson, \textit{The Hobo}}

\subsubsection{Anderson's ethnography}

\subsection{Criticism on the Chicago School}

%%%%%%%%%%%%%%%%%%%%%%%%%%%%%%%%%%%%%%%%
%												LECTURE 4
%%%%%%%%%%%%%%%%%%%%%%%%%%%%%%%%%%%%%%%%

\section{The Legacy of the Chicago School: Urban ethnographies, urban poverty, and symbolic interaction}
\textit{October 19th, 2021}


%%%%%%%%%%%%%%%%%%%%%%%%%%%%%%%%%%%%%%%%
%												LECTURE 5
%%%%%%%%%%%%%%%%%%%%%%%%%%%%%%%%%%%%%%%%

\section{Undocumented migrant struggles in/over the city}
\textit{October 26th, 2021, Thomas Swerts $\rightarrow see slides for comprehensive content$}

\subsection{Guiding questions}

\begin{outline}
	\1 Why are undocumented migrant struggles (UMS) necessarily urban struggles?
	\1 How can undocumented migrants gain recognition in hostile environments?
	\1 Why are ethnographic observations and participatory action research (PAR) suitable methods for studying UMS in the city?
\end{outline}

\subsection{Undocumented migration and the city}

Who is an undocumented migrant? Those without citizenship or legal residency.

Why are they undocumented? Many reasons: through birth and legal loopholes; fleeing a dangerous situation and could not bring documents; visa or papers expired; 

The `undocumented' status can fluctuate, depending on whether one is employed or not. 

Language matters and can be criminalising, like using ``undocumented'' vs. ``illegal aliens''.

The definition of undocumented is hard to grasp: there is no official data, and estimates are based on criminal records/arrests.

\marginpar{Does how the undocumented organise change based on public space and socio-demographics?}

Being undocumented is a new phenomenon, based on state policies starting in the 1970s. They targeted certain populations for crime, vice, violence, ie. the `impossible subjects'.

\subsubsection{Urban concentration}

Why are undocumented concentrated in cities?

\begin{outline}
	\1 Informal labour market access
	\1 Can be inconspicuous and not stand out
	\1 Shelters exist for migrants, eg. arrival infrastructure
	\1 Family and friend ties, existing community, organised solidarity
	\1 Administrative institutions, useful for asylum seekers
	\1 Policy reasons, eg. the city can be an open and friendly place
\end{outline}

Sanctuary cities are cities that experiment with different policies, such as don't ask/don't tell (regarding immigration status), local id cards (so everyone has id), providing access to public services...

\marginpar{How does visibility change policy?}

\subsection{A challenge for urban sociologists}

The \textbf{Chicago School} paid attention to migration as a driver of the urbanisation process. They conducted ethnography studies that reinforced `victimhood' which doesn't leave much room for agency. 

But

\begin{outline}
	\1 Urban informality/illegality associated with social deviance, unwanted populations and disorganisation
	\1 Perspectives that focus on social `organisation' of marginalised communities hardly pay attention to more structural/institutional factors at play
\end{outline}

The \textbf{new Urban Sociology}'s focus is how flows of capital have ignored flows of people as a major force that shapes urbanisation. 

``The massive forced and unforced migrations of people... will have as much if not greater significance in shaping urbanisation in the twenty-first century as the powerful dynamic of unrestrained capital mobility and accumulation'' - David Harvey, 1996

There is a need to move away from ethnographies that portray marginalised communities as victims or vulnerably subjects, and portray the city as background rather than object of action.

We need an epistemological shift towards the perspective of the outsider, the marginal, the subaltern, the unauthorised, the illegal, the deviant


\textbf{Migrant city}: a complete different city map, focus on mobility

\textbf{Migrant city-making}: undocumented as actors who can shape the city

\subsection{Everyday illegality in the city}

\subsubsection{Lived experiences of Illegality}

\begin{outline}
	\1 \textbf{Embodied lived experiences}
		\2 Mobility: very limited, risky to move because of policy raids on the metro, although people taken in by the police are rarely directed to the deportation centres because they are already too full, and instead given the order to leave the territory
		\2 Borders and bordering practices: everywhere, internal
	\1 \textbf{Discursive lived experiences}
		\2 Illegalising and criminalising discourses and bureaucratic procedures
		\2 Learning to be illegal: as children, they don't know that they are undocumented; the transition to adulthood is a turning point, as it becomes hard to get education, funding, employment, etc.
	\1 \textbf{Emotional lived experiences}
		\2 Constantly living in fear, not being able to participate, feeling of exclusion
\end{outline}

These lived experiences of illegality stop people from being politically active.

\subsection{Understanding undocumented migrant struggles (UMS)}

\subsubsection{What is undocumented migrant activism?}

\textbf{Undocumented migrant activism} can be defined as a sustained and collective effort of people who are in a precarious legal position to organise themselves and mobilise supports as to claim their rights, make their grievances heard and strive to change their predicament and that of others like them.

Translated to migrants in the gig economy, \textbf{migrant gig-worker activism} can be defined as a sustained and collective effort of people who are in a precarious \textbf{socio-economic position} to organise themselves and mobilise supports as to claim their rights, make their grievances heard and strive to change their predicament and that of others like them.\marginpar{Gig economy migrants}

Barriers to political participation:

\begin{outline}
	\1 Lack of voting or other political rights to representation
	\1 Lack of resources
	\1 Stigmatisation in public discourse
	\1 Fear of getting deported
\end{outline}

\subsubsection{Hard to research}

Research ethics of power and empowerment

\subsubsection{Findings from fieldwork in Chicago and Brussels}

The importance of story telling in social movements, UMS is similar to social movement theory

\subsubsection{Spatiality of UMS}


%%%%%%%%%%%%%%%%%%%%%%%%%%%%%%%%%%%%%%%%
%												LECTURE 6
%%%%%%%%%%%%%%%%%%%%%%%%%%%%%%%%%%%%%%%%

\section{Urban Politics}
\date{September 28th, 2021}

%%%%%%%%%%%%%%%%%%%%%%%%%%%%%%%%%%%%%%%%
%												LECTURE 7
%%%%%%%%%%%%%%%%%%%%%%%%%%%%%%%%%%%%%%%%

\section{Social networks, Neighbourhoods, and Communities}
\date{November 23rd, 2021}

\subsection{Studying community and the neighbourhood from a social network perspective}

\begin{outline}
	\1 
\end{outline}

\subsection{Social implications of a lack of community and social networks (social capital) within society}

\begin{outline}
	\1
\end{outline}

\subsection{}

\begin{outline}
	\1
\end{outline}

\subsection{}

\begin{outline}
	\1
\end{outline}

%%%%%%%%%%%%%%%%%%%%%%%%%%%%%%%%%%%%%%%%
%												LECTURE 8
%%%%%%%%%%%%%%%%%%%%%%%%%%%%%%%%%%%%%%%%

\section{Global Cities}
\date{November 30th, 2021}

\subsection{Cities in history}

Before industrial age in the feudal times:

\begin{outline}
	\1 Cities were relatively autonomous centres of political and economic power (ref. Weber). 
	\1 City corporations and craft guilds
\end{outline}

Industrialisation in the 19th and 20th century:

\begin{outline}
	\1 
	\1 
	\1 City became a production site
\end{outline}

De-industrialisation in 1960s and 1970s:

\begin{outline}
	\1 Urban decline
	\1 White middle-class flight to the suburbs
	\1 Minorities had a larger presence in cities
	\1 Outsource of production, emergence of multinational corporations
\end{outline}

Post-industrial society, 1974, Daniel Bel

\begin{outline}
	\1 Cities' economy moved from manufacturing to one specialised in services
	\1 A new factor of production, \textbf{knowledge}, is becoming more important than land, labour and capital as the key to economic organisation
	\1 Emergence of a \textbf{new professional and technical class} (ref. Florida)
\end{outline}

\subsection{Global city? Back to the future}

\textbf{Is the city becoming an autonomous force again, or do these trends lead to the death of the city?}

Late 1980s: social scientists found that the subnational and supranational levels gained importance. The EU grew, and cities and regions became important political and economic levels to take into account.

\textbf{Glocalisation}: there is at the same time universalising and particularising tendencies, everything became more universal but at the same time we had to take into account local trends. 

Cities are assumed to cope better with globalisation than the nation-states. Cities can deal better with trans and multinational corporations.
Traditional social sciences is guilty of methodological nationalism, and ignored the city/region/continental scale. To this day, national states do not have any impact on national trade and corporations (they do not pay taxes or regulate them, eg. Google).

Globalisation was a hype in social sciences in the 1990s-2000s.

\subsection{Understanding globalisation}

Globalisation is \textbf{the process of becoming global}
\begin{outline}
	\1 Not static but always changing, `time space compression' (Harvey, 1989)
	\1 Intensification of worldwide social relationships; events that occur on one side of the world, can influence events in other areas of the world (Giddens, 1990)
	\1 Increased circulation and movement in various scapes: ethnoscapes (increasing flow of people), technoscapes (increasing flow of technology), financescapes (increasing flow of money), mediscapes ((increasing flow of media channels), ideoscapes (increasing flow of ideas) (Appadurai, 1996)
\end{outline}

Globalisation is an umbrella term for economic, political, social, cultural interlinked processes.
Globalisation is not new, already existed in colonialism, imperialism, etc. Economic flows between Congo and Belgium 

\subsection{Saskia Sassen, \textit{The Global City}, 1991 and 2005}

Argues that a combination of spatial dispersal and global disperssion has created a new strategic role for cities like London, NY, Tokyo. Global cities are:

\begin{outline}
	\1 Highly concentrated command points: concentration of economic and political power
	\1 Key locations for finance and specialised service firms (APS)
	\1 Sites of production
	\1 Markets for the products
\end{outline}

This has implications for how the city is organised and the social hierarchies. 
Focuses on the place where the work of globalisation `gets done', and anchors her research in a place.

The global city is a new type of conceptual architecture

\begin{outline}
	\1 Simultaneous dispersal and integration of economic activities; firms are located near each other because they need each other
	\1 Outsourcing of central functions such as accounting, legal, pr, telecommunications, etc; two key sites are the headquarter and specialised service firm
	\1 Specialised firms are subject to agglomeration economic; global cities are production sites of higher order information, ie. dealing with concepts, numbers, money, and not production of object
	\1 Outsourcing allows companies to opt for any location, and not be subject to agglomeration economies
	\1 Strengthening of cross border and city-to-city transactions and networks; \textbf{transnational urban systems}
\end{outline}

There is social polarisation in global cities, creating divided cities:

\begin{outline}
	\1 A divided city
		\2 High end professional jobs need low-wage service sector jobs
		\2 Middle class manufacturing jobs disappear, and middle class cannot afford to live in the city anymore
		\2 Growing informalisation
	\1 Growing informalisation
		\2 Uber eats, deliveroo, sorting centres
		\2 Undocumented (sans-papiers) and minors are employed in informal markets
\end{outline}

Global city research needs to focus on place:

\begin{outline}
	\1 Focus on how the global materialises itself in a place.
	\1 You need activity (work) and infrastructure to run the economy
\end{outline}

Impact of new communication technologies:

\begin{outline}
	\1 Impacts how we look at the CBD; it is still central, but because there are new communication technologies it may not be as concentrated as before
\end{outline}

Nexus for new politico-cultural alignments:

\begin{outline}
	\1 New identities, new struggles, new possibilities for urban citizenships; new strategies to make themselves heard\marginpar{Los Deliveristas Unidos?}
\end{outline}

\subsection{European cities, the informational society and the global economy}

Manuel Castells, \textit{The Information Age}, 1996

Contemporary society is undergoing economic transformation on a scale at least equal or faster to that experienced in the transition from agrarian to industrial society.
It is because of technologies that society changes occur at such a fast pace.

Knowledge and information, and capacity of technologies to diffuse them lie at the core of this recent transformation and have become more important to the production of goods and services than more traditional factors like labour, capital, energy, physical communication.

There is an information technology revolution.

What is happening in European cities? How does the technological revolution affect (European) cities?

Major social trends affecting European cities:

\begin{outline}
	\1 \textbf{Technological informational revolution}: creates an infrastructure of current world system
	\1 \textbf{The informational society}: if you don't have the capacity to deal with information; if you are unable to understand or access information through the internet, you are at a disadvantage
	\1 \textbf{The global economy}:
	\1 \textbf{European integration}: its regions like Europe that have become powerful political systems, regional cities have become more important that nations
	\1 \textbf{European identity}: new identities emerged, European and regional/local/urban identities (eg. people associating as Bruxellois, or as Flemish, Catalan or Basque movements)
	\1 \textbf{Social movements}: new types of social movements like ecological, women, child care movements, etc.
	\1 \textbf{Marginality}:
	\1 \textbf{Dual city}: presence of the low and upper class, but absence of middle class
\end{outline}

The informational city is the urban expression of the informational society:
\begin{outline}
	\1 Global city
	\1 Dual city: new managerial-technocratic elite, versus the ghettos where people work in informal, low paid and skilled jobs; there is a cosmopolitan elite who celebrates diversity, verus local communities that are very secluded living in `tribal' worlds (eg. orthodox jews)
\end{outline}

In the informational cities, there is a new spatial logic. We no longer operate in the \textit{space of places}, but organisations operate in the \textit{space of flows}. Everyone and everything is connected to each other, through information technologies. Space does not become important anymore, it doesn't matter where we are but how connected we are to the global network. If you don't have a connection to the global network (are isolated in local communities), then you don't matter.

Cities are nodes in this global network. There is a new international and interregional division of labour. A global economy is characterised by variable geometry and uneven spatial development, creating winners and losers among cities. 

\subsection{Global care chains}

Hochschild, 2003 

How does globalisation impact care? Globalisation doesn't only impact cities, but also the people living in cities.

Global care chains are a series of personal links between people across the globe, based on the paid or unpaid work of caring. Made up usually of women, and can be global, national or local. A migrating nanny can take care of someone else's child within the same city, or in a different continent from their own homes and children.

Chains vary in number of links and the connective strength. Many people can be involved in the chain, the nanny can also have another job in another household. 

Typical example:
\begin{outline}
	\1 Older daughter from a poor family cares for her siblings
	\1 Her mother works as a nanny caring for the children of a migrating nanny
	\1 Migrating nanny works for the child of a family in a rich country
\end{outline}

There is a lot of emotional labour that goes into caring for another person's children. You transfer your feelings to another child. There's a surplus love or emotional surplus value, and the rich children may receive more love than others: from the nanny and from their mother.

The paid care is racially patterned, takes a toll of women and children.

\subsection{Cultural globalisation}

ref. slides 

\begin{outline}
	\1
\end{outline}

%%%%%%%%%%%%%%%%%%%%%%%%%%%%%%%%%%%%%%%%
%												LECTURE 9
%%%%%%%%%%%%%%%%%%%%%%%%%%%%%%%%%%%%%%%%

\section{Urban struggle and urban social movements}
\date{September 28th, 2021}


%%%%%%%%%%%%%%%%%%%%%%%%%%%%%%%%%%%%%%%%
%												LECTURE 10
%%%%%%%%%%%%%%%%%%%%%%%%%%%%%%%%%%%%%%%%

\section{Crime, policing and public space}
\date{September 28th, 2021}

%%%%%%%%%%%%%%%%%%%%%%%%%%%%%%%%%%%%%%%%
%												LECTURE 11
%%%%%%%%%%%%%%%%%%%%%%%%%%%%%%%%%%%%%%%%

\section{Urban Culture and lifestyle. The postindustrial city}
\date{September 28th, 2021}

%%%%%%%%%%%%%%%%%%%%%%%%%%%%%%%%%%%%%%%%
%												READINGS
%%%%%%%%%%%%%%%%%%%%%%%%%%%%%%%%%%%%%%%%

\section{Readings}

\subsection{Lecture 2}

\subsubsection{Urbanism as a Way of Life, Louis Wirth}

\begin{itemize}
  \item The rapidity of urbanisation makes it hard to follow and understand social changes, and we do not have a good sociological definition of the city/urbanism
  \item The urban mode of life is not confined to cities
  \item Urbanism is a complex set of traits that make up the characteristics of city life, and urbanisation develops and extends these traits. Urbanisation is not capitalism or industrialisation!
  \item Speaks of size, density and heterogeneity as characteristics of what makes a space a city
  \item Density: diversification and specialisation, close physical contact but distant social relations, complex pattern of segregation, predominance of social control, accentuated friction
  \item Heterogeneity: break down rigid social structures, produces increased mobility, instability, insecurity, intersecting and tangential social groups with high membership turnover. The city encourages diversity by definition, by bringing people from distant places specifically because these people are different
  \item Urbanites are highly dependent on each other, in so far as they need each other's activities to survive (compared to the rural, who can be more self-sufficient). Conversely, urbanites are less dependent on specific people, rather to a group of people who perform certain activities (it doesn't matter \textit{who} will drive your bus to work, it only matters that some driver does). Cities are characterised by \textbf{secondary} rather than \textbf{primary} contacts
  \item Acquaintances in cities are utilitarian, the role that people play in our lives is regarded as a means to achieve our own ends. The specialization of tasks in cities segments people and creates a utilitarian nature.
  \item A city needs immigration (domestic or international) to grow, because growth cannot be sustained purely by reproduction of its residents. 
  \item In a city, the individual acts within a group rather than on their own.
\end{itemize}

\subsubsection{Community and Society, Tonnies}

\begin{itemize}
  \item Gemeinschaft (community) vs. Gesellschaft (society)
  \item Gemeinschaft
  \item Gesellschaft
\end{itemize}

\subsubsection{Simmel}


\subsection{Lecture 3: Chicago School}


\subsubsection{Park, Human Ecology}

Web of life: beings are anchored in their environment, and are interrelated and mutually interdependent\footnote{Darwin's example of the relation of cats and red clovers}. This interdependence is symbiotic rather than societal. 

The balance of nature: there is a kind of equilibrium between beings and with their environment, and when that equilibrium is broken, the conditions of life change. The equilibrium can broken by a famine, epidemic, or invasion by a species. In reality, this "balance of nature" doesn't exist because something always comes along to disturb it before it is achieved. Our world today is highly mobile (people, things, money, microbes), so the equilibrium is always changing. In a society (human or not), after a crisis there is an increase in competition, and only when this competition decreases does cooperation exist that allow the society to exist again.

Competition, dominance, succession: dominance and succession are two principles that establish and maintain societal order. Every urban landscape can be explained by dominance, ie. competition: industries compete for the most valuable and strategic land, and the central shopping and banking districts have the highest value; the peripheries lose in value gradually, but can also be re-written to be valuable as the centre expands; high value/low value areas exist in competition yet in interdependence, because one cannot exist without the other. Thus, power, competition, dominance, shapes the landscape;
Succession describes a cyclic process of passing from an unstable, to a stable state of society. When society becomes unstable for some reason, competition increases, society changes, and then is stable again (state of equilibrium) until the next 'crisis'.

Biological economics: the economics of living beings? human ecology is neither geography, nor economics

Symbiosis and society: 

Questions/Comments: 

- if we took an individual A and put them in a city, would we expect the city to shape that individual in the same way as another individual B in the same city? -> no, of course not; Thus is there some resistance, or some choice, that an individual makes on how/if the city will shape them? And if so, wouldn't we say that the city doesn't influence the individual, rather than the individual chooses how to be influenced by it?

- "it is when, and to the extent that, competition declines that the kind of order which we call society may be said to exist" $\rightarrow$ how does this correlate with capitalism? aren't our capitalist societies, by definition, competing within and against each other? $\rightarrow$ but competition has to decline to a point that cooperation can exist. was there ever a point of time that we had so much competition that society could not exist or function? Would we consider the world wards such a time? but wasn't society still surviving, or was it another kind of society? 

- Park argues society grows (succession) when competition occurs, ie. a crisis has happened, and at some point, competition declines and equilibrium is found. Is there ever a growth without crisis and competition?

- The parallel of society to the flora/fauna kingdoms implies a natural order, and a way of existing (with domination, competition, fragile equilibrium) that we cannot circumvent as society. 

\begin{outline}
	\1 In "Human Ecology", Park talks about competition and dominance, explaining that competition shapes the urban landscape ("The area of dominance in any community is usually the area of highest land values" The City p. 8), and that a crisis must happen for society to grow. These concepts are referring to capitalism.
However, the word "capitalism" is never mentioned or discussed as such. Why is this? Capitalism was a known and implemented system at the time of writing. Is the omission by accident or purpose? What were the political stances of the Chicago School? Is Park referring to the same capitalism that we have today?
\end{outline}

\sout{Furthermore, competition and dominance are explained as ecological principles: "It is when, and to the extent that, competition declines that the kind of order which we call society may be said to exist" (The City, p. 7). 
Given that under capitalism, competition will never cease, this implies that society cannot live in the "co-operation" that supersedes competition. 
In Park's words, capitalism becomes a natural fact, inherent of human/animal/plant systems alike.}


\subsubsection{Burgess, The City}

Process of urban metabolism and mobility

\begin{outline}
	\1 Expansion as physical growth
		\2 Expansion is measured by the physical growth of the city. The city plans parks, boulevards, civic centres, etc, and considers land for development far beyond its city limits, in order to anticipate and control the city growth.
	\1 Expansion as a process
		\2 The city is organised in concentric circles, going from CBD $\rightarrow$ zone in transition (businesses, light manufacturing) $\rightarrow$ zone of working class homes $\rightarrow$ zone of middle class and family residences $\rightarrow$ commuter zones (suburbs, satellite cities). Thus expansion is the process by which a zone grows into an outer zone, pushing it further from the centre, extension and succession.
		\2 	Expansion is also concentration and decentralisation. Concentration: the convergence of transport in the CBD, where the political/economic/cultural life is centred. Decentralisation: ???
		\2 	Expansion can be measured not only by physical growth, but also changes in social organisation and personality types. 
	\1 Social organisation and disorganisation as processes of metabolism
		\2 The city is a place where people can be organically integrated. But, cities are growing faster than the reproduction rate, meaning that a large part of the population are immigrants (Burgess categorises this as a 'disturbance' of the metabolism); the assimilation of culture by immigrants is 'abnormal', since culture is typically learned by birth.
		\2 There is disorganisation in the city, and it is a normal process of reorganisation; it is a feeling of disorientation when a person arrives to the city and is confronted to new norms; the city must shape the individual.
		\2 There is segregation in the city: a type of organisation into economic and cultural groups. It allows groups to emancipate, but limits their development in some ways.
		\2 	The division of labour: a disorganisation, reorganisation and increasing differentiation. There is a huge variety of jobs, and ethnicities tend to perform a set jobs over others.
	 	\2 Expansion and metabolism" indicates an excessive increase in crime, disease, disorder, vice, insanity, suicide
	\1 Mobility as the pulse of the community
		\2 
\end{outline}

Questions:

- What is a centralised decentralised system? I understand centralisation as the process that agglomerated towns into a city, but how does decentralisation fit in to the Chicago school's concentric model? If an industry, or political function, is decentralised, then the organisation as zones would not apply anymore.
\sout{Surely, in the city, there is a more centralised system of governance than there was previously, when each town would have their own politics (I assume). Or is decentralisation referring to the city governing itself more independently than towns were, vis-à-vis of the state governance?}\

- Burgess asks how individuals are incorporated into the city and become an organic part of their society. As we learnt, thinking about the city as an living, natural organism is not accurate (there is nothing natural about cities), but still, cities absorb people in a way. 

- Why is "the city as an organism" a bad metaphor? 

\subsection{Lecture 4: Legacy of the Chicago School}

\subsubsection{Wacquant, Wilson, \textit{The Cost of Racial and Class Exclusion in the Inner City}}

\begin{outline}
	\1 Describes the relationship between inner-city dislocations, and the struggles and structural changes in society, economy and polity (political organisation). This connection is usually not made when explaining the conditions of people living in ghettos, ie. disconnected and segregated urban spaces. It asks: what are the features of the social structure in which ghetto residents try to survive? 
		\2 Compares class composition, welfare trajectories, economic and financial assets, social capital of blacks in ghettos, to those in low-poverty areas
	\1 The conditions of the people living in ghettos are not put into socio, political, economic context, but rather interpreted as individual cases or worst, a self-imposed phenomenon (people live in ghettos due to their own faults, culpability, if some people can make it out so can others)
	\1 The geographical concentration of black people in dilapidated territorial enclaves epitomises the social and economic marginalisation of these people
		\2 The higher the socio-spatial concentration of poverty, the more obstacles the residents face (neighbourhood effects?)
		\2 There is an exodus in the inner-city of jobs, working families, which leads to the deterioration of housing, schools, business, leisure places, community organisations, exacerbated by government policies
		\2 The ghetto is a "closed opportunity structure", endless cycle of poverty that few escape, and is self-reinforcing
\end{outline}

\subsubsection{Wacquant, \textit{Revisiting territories of relegation}}

\subsubsection{Elijah Anderson, \textit{Code of the Streets}}

\begin{outline}
	\1 \textbf{This week's comment}:
	What I like about this week's texts is that they emphasise society's collective responsibility towards ghettos, slums, and any neighbourhood where survival is the residents' main concern. They focus on the socio, political and economic conditions that created this situation, rather than blaming individuals.
 
This got me thinking about the neighbourhood effect we discussed in the last lecture. For example, if the violent youths were taken out of their neighbourhoods where the code of the street applies, would the dangerous street behaviour prevail? I think so: the conditions which led to the violent environment in the first place would still exist. The neighbourhoods are not dangerous because the people living there are innately violent, but because the lack of support from socio, political, economic systems end up fostering a violent environment.

Wacquant and Wilson seem to support the idea of neighbourhood effects when talking about the `exodus' of jobs and decent families from the inner-city. This phenomenon deteriorated the housing, schools, businesses, places of leisure and community, and created a self-reinforcing system: the people who can afford to will move out of the dilapidated, possibly dangerous inner-city district, which strengthens the concentration of poverty.

Do you think otherwise? Do you think that without the most violent minority, the code of the street would not be maintained as such and the neighbourhoods would become less aggressive, and move towards a `decent' community?

	\1 Neighbourhood effect: \textit{``Simply living in such an environment places young people at special risk of falling victim to aggressive behaviour''}
	
	\1 Without the violent minority, the neighbourhood could become more liveable and less aggressive: \textit{"Most people in inner-city communities are not totally invested in the code, but the significant minority of hard-cord street youths who are have to maintain the code" and that "many less alienated young black ... want a nonviolent setting in which to live"}. 

	\1 Vice versa of neighbourhood effect, would the street-oriented continue their behaviour in a new neighbourhood that previously wasn't governed by the code? This is a scenario that I assume would never, or rarely, take place. But I think not, because their living situation will have changed. Living in a better neighbourhood means better access to institutions and jobs.
	
		\1 If the street community were to get significantly weaker in numbers compared to the decent, the socio, economic, political situation that perpetuates the low-income, black communities will still exist, and the phenomenon of the code will win over. The street exists because of the lack of support from the system (what is this system?). It is a self-reinforcing system. It isn't worth for them to try to gain respect in the mainstream system, because they could gain respect faster and with more impact in the streets.
	
	\1 What socio, political, economic factors would have to change for there to be a tipping point from a violent to a 'less violent' neighbourhood?		

\end{outline}
		
\subsubsection{Questions \& comments}

\begin{outline}
	\1  What I liked about these texts is that they research the conditions of ghetto residents and the forces that come together to create such living conditions, but contrary to previous texts, they emphasise that the person who resides in delapidated areas is not to blame for their situation. Rather, the texts focus on the collective responsibility that the society has in regards to such spaces existing. They focus on the socio, political and economic forces that led to such conditions.

This shift of focus from the individual to a higher level, pushes society to take responsibility for the living conditions of ghetto residents. When the blame is on the individual, and the thinking is that if they only wanted to, they could escape the ghetto, society can ignore the everyday struggle and survival of the ghetto residents.

Overall, this week's texts paint marginalised communities in a better light than that of the Chicago School's.
	\1 Gemeinschaft in the streets? 
		\2 In "Code of the Streets", the code of the streets is described as "a set of informal rules governing interpersonal public behaviour".
	\1
\end{outline}

\subsection{Lecture 5: Undocumented migrant struggles in/over the city}

\subsubsection{Swerts, \textit{Creating Space for Citizenship: The Liminal Politics of Undocumented Activism}}

\begin{outline}
	\1 How precarious communities, like the undocumented, appropriate public space to try to gain visibility
		\2 \textbf{Liminal politics} is ``the process by which precarious populations, like undocumented migrants, constitute themselves as political subjects by creating, using and appropriating in-between spaces''
	\1 Theatrical and performative stagings of citizenship: undocumented migrants, by being active in public space, show that they are citizens
	\1 Developing political scripts in \textbf{safe spaces}
		\2 \textbf{Sanctuary cities}: cities that have prohibited asking people for immigration papers, and generally made it harder/defunded deportation policies and practices
		\2 Compares Chicago and Brussels:
			\3 Reyes case: an undocumented student who was arrested for drunk driving, and faced deportation until a group got together and successfully campaigned against it. This led to the formation of a youth organisation, which provided a safe space for undocumented youths to come and tell their stories
			\3 Brussels is a very different landscape: the political layers, between regions, makes it very complicated. The government is hostile towards undocumented, conducting raids on immigrant neighbourhoods and ad-hoc ID checks in public spaces like the metro. Undocumented have formed a solidarity movement, meeting in places like church, and helping each other out with basic needs. It creates a safe place for sans-papiers to come together and feel part of a community
			\3 20th municipality of Brussels as the municipality of the 100.000 sans-papiers who reside in Brussels
		\2 Safe spaces are places for political training, innovation, experimentation.
	\1 Allied institutions are required for the safe spaces to exist. Are allies required for the undocumented to strengthen their movement?
	\1 Backstage: the safe spaces created by different communities; Frontstage: the public space, the streets, where what is organised in the back stage comes to life
\end{outline}

\subsubsection{Swerts, \textit{Undocumented Immigrant Activism and the Political: Disrupting the Order or Reproducing the Status Qo?}}

\begin{outline}
	\1 Between the pressure to conform to societal norms and discourses (\textbf{reproduction}), and disrupting the status qo (\textbf{disruption})
	\1 ``Illegalisation of migrants by national governments'' shifts the focus of the wrong-doer from the migrants, to the governments 
\end{outline}

\subsubsection{Questions \& comments}

\begin{outline}
	\1 How exactly is `urban citizenship' defined? For me, citizenship is something material that you receive by birth, or by adoption from a country. It manifests with a passport. I guess you can have symbolic citizenship, where you declare yourself an important enough part of a group/society. When Swerts says ``... interstitial spaces like refugee camps, small-scale community-led spaces and places of `sanctuary', can produce new forms of urban citizenship'' (Creating Spaces for Citizenship, p 381), what does he mean?
	
	\1 How do we understand the differences and overlaps between citizenship and residency? I have not lived in the country in which I am a citizen, for fifteen years. Instead, I've been a resident of the cities I lived in, which gives me some rights, even if not all of the rights that a citizen has. I understand that residency is only available to those who hold a citizenship (passport) from a compatible country, or who obtain a visa. 
	The problem with undocumented migrants, is that they have no such paper (passport or visa) to legitimise their living in a city/country, and thus cannot obtain either residency or citizenship
	
	\1 Undocumented migrants are at the margins of society, in an `in-between' place, in the ``frontiers, zones and camps through which distressed, displaced and dispossessed people are rendered `invisible' and `inaudible' and thereby condemned to the status of strangers, outsiders, aliens'' (Swerts, Creating Space for Citizenship, p. 381). 
	This is probably linked to the politics of the city pushing such people to the margins. If so, will undocumented activism ever reach or influence the people who sit in government and have the power to fulfil these migrants' requests? Or, is the activism aimed at another group - perhaps ordinary citizens, or other undocumented migrant groups in another space/city, who could strengthen the voice and reach of the activism in numbers?
	Ie., what are the undocumented migrants targeting, and who is the most effective target of their activism?
	Even if the undocumented can gather in safe spaces, develop political strategies and build strength in numbers, would they be able to achieve their goals without the help of documented citizens in government or in similarly powerful positions? The fact that the Belgian government refuses to address the undocumented, and considers their actions `blackmailing', it sounds like there is little hope for the undocumented seeking to be heard by the Belgian government.

	\1 I had never heard of sanctuary cities, and I found it fascinating that a city like Chicago can declare itself as such, and make it illegal for police to ask about immigration status, and reduce funding for deportation enforcement. I wonder how difficult it was for Chicago's mayor to enact this, and what effect it has had on the city. Sanctuary cities are not the mainstream by far, but it seems that (based on my very limited research) there are no cons. On the contrary, it even strengthened some economies. This makes me think that the only opposition to sanctuary cities, is a political one, especially since the analysis of sanctuary cities is focused on criminality, insinuating that undocumented migrants have a relation with the violence and crime of a place.
	 Moreover, what (political?) conditions tip a city to be a sanctuary city, compared to another? Brussels is such a hostile city that towards undocumented immigrants, that raids are organised on immigrant neighbourhoods and ID checks conducted in public space (p. 387). What in Brussels' politics make it an anti-sanctuary?
	 
	 On an unrelated note, I thought the phrasing of the ``illegalisation of migrants by national governments'' (Swerts, Undocumented Immigrant Activism) was powerful and interesting. It shifts the wrong-doing from the migrants onto the governments, who are in a position to change the condition of the undocumented. It shows that a person cannot be illegal simply by entering a city, rather, it is the political system that makes them so.
	

	Sanctuary cities are definitely not the mainstream, but could they be? 
	\sout{What would a sanctuary city look like in Europe, ie. outside of the US?}
	
	\1 The articles address undocumented migrants, and their struggle for representation and citizenship. However, it still seems like those who are privileged are the ones who can assimilate the best. For example, Reyes, who was an undocumented migrant, but his education and good grades helped rally a crowd to contest his deportation. I wonder if this would have been the same, if Reyes did not have a college education, but rather worked odd jobs with little income. I'm trying to reconcile Reyes' case and undocumented migrant activism: the case shows that those who have power, in this case through education, can create movement. But those who live invisibly, informally, could not. School provided a way for undocumented students to organise and feel safe, but what about those who cannot access this education? 
	
	\1 On an unrelated note, I thought the phrasing of the ``illegalisation of migrants by national governments'' (Swerts, Undocumented Immigrant Activism) was powerful and interesting, because it shifts the wrong-doing from the migrants, to the governments, who are in a position to change the  
	
\end{outline}

\subsection{Lecture 9: Urban Politics}

\subsection{Lecture 10: Social networks, Neighbourhoods, and Communities}

\subsubsection{Wellman and Leighton, \textit{Networks, neighbourhoods and communities: approaches to the study of the community question}}

\begin{outline}
	\1 Definition of community: ``networks of impersonal ties (outside of the household) which provide sociability and support to members, residence in a common locality, and solidarity sentiments and activities'' (p. 365)
	\1 Neighbourhood vs. community:
		\2 Neighbourhood implies a spatiality, and entangling community and neighbourhood encourages relationships to be local; ie. spatial determinism
		\2 Presence of local ties does not necessarily create discrete neighbourhood, there is a range and overlap explained by the mobility of the people
		\2 Disregards spaces of interaction outside of the neighbourhood, eg. work place
		\2 Too much emphasis on spatiality as a causal variable
		\2 Assumes that if community cannot be seen in the neighbourhood through ``organised behaviour and sentiments'', it does not exist
	\1 Network analytical approach 
		\2 Does not take spatiality as a starting point, and instead looks for ``social linkages and flows of resources'', and associates spatiality after, according to these flows $\rightarrow$ makes it possible to discover \textbf{network based communities} that aren't linked to neighbourhoods or ``solidary sentiments''
	\1 The community arguments:
		\2 The community lost: absence of local solidarities
		\2 The community saved: persistence of local solidarities
		\2 The community liberated: denies neighbourhood basis to community
	\1 Community lost
		\2 Because it only looks at community existing in neighbourhoods, mobility was seen as a loss of community
	\1 Community saved
		\2 Humans are gregarious and will create communities even under challenging circumstances like poverty, oppression, catastrophe
		\2 In policy, this means: importance of active neighbourhood communities and organisations, neighbourhood unit as the planning ideal, preservation of neighbourhood against institutions
	\1 Community liberated
		\2 Community ties have weakened due to `industrial bureaucratic nature of social systems' (community lost) but community ties are still important and present (community saved)
		\2 The community does not need to be tied to a neighbourhood, and can reach large distances
		\2 Structural and technological developments liberated the community from the confines of a neighbourhood
	\1 Liberated networks
		\2 One of the characteristics: ``some network ties can be mobilized for general-purpose or specific assistance in dealing with routine or emergency matters. The likelihood of mobilisation depends more on the quality of the two-person tie than on the nature of the larger network structure'' (p.378)
		\2 Reinforcement of other social networks, outside of those of the neighbourhood
	\1 ``Dense saved network patterns are better suited for internal control of resources while ramified liberated patterns are better suited for obtaining access to external resources'' (p. 384)
	\1 Most communities have members to communicate and exchange resources with outside world, and some for allocating those resources internally 
\end{outline}

\subsubsection{Putnam, \textit{Bowling Alone: America's declining social capital}}

\begin{outline}
	\1 Americans are spending time individually, and losing the sense of community 
\end{outline}

\subsubsection{Weekly question}

\begin{outline}
	\1 I was also thinking about socioeconomic status when reading Wellman and Leighton's text, especially after our class on neighbourhood effects and social mix in Urban Social Geography. I would be very surprised if communities of different socioeconomic backgrounds exhibited the same community and network patterns. I also wonder to what extent we can disregard neighbourhoods when analysing communities and networks. In all cities I know, there exists segregation based on socioeconomic status.
For one, neighbourhoods with a strong migrant population can serve as familiar spaces for new immigrants to settle and understand the city with the help of people with similar experiences and backgrounds. Neighbourhood connections may not be so important when you know how to and/or are able to connect to people outside, but can be crucial otherwise.
The homeless population also comes to mind. They may or may not be confined to a neighbourhood, but either way, would you say their community is liberated because it can span across large distances? I don't think this is what the authors had in mind when describing liberation as pushing the boundaries of space..
\end{outline}

\subsection{Urban Struggle and Urban Social Movements}


\subsubsection{Lefebvre, \textit{The Right to the City} p. 147-160, 1995}

\begin{outline}
	\1 There should be a place for the expression of `anthropological needs', ie. the human experience, where ``exchange would not go through exchange value, commerce and profit''
	\1 The city today is not lived and understood ``practically'', but is a place where consumption dominates (eg. commodification of culture, art, life). How do we reconcile the past and the future? We will not go back to what the city was before, ie. a traditional place, but we also do not want to go towards an alienated urban agglomeration (p. 148)
	\1 Old humanism does not exist anymore, but that doesn't mean that man doesn't exist or that nothing matters. Lefebvre calls for a ``new humanism'' focused on the urban society (p. 150) 
	\1 It seems like Lefebvre is writing about the struggle between leaving an old way of life, ie. the cultural and traditional places. But what are these places? He writes in 1995, and it seems to me like the time and place he describes has been gone for a long time. The industrial era (1800s to early 1900s) also changed `traditional humanism', society reorganised itself, and the industrial/urban way of life that emerged was not the `traditional place' that Lefebvre talks about. So what is this 'traditional city' that he refers to?
	His tone is nostalgic but at the same time brutal (examples around p. 148-150), but also hopeful - he says we need a new way to characterise humanity, focused on urban society, because urban life is the future. I'm kind of confused at what his sentiment is - nostalgic, frustrated, hopeful? All of them? 
	\1 The best of rural life has to be revitalised in the city: that is, the festivities 
	\1 Experimental utopia: every one is a utopian, and utopia is experimental. People are trying to understand 1, how we can find out what  makes a place successful, and 2, what that thing that makes a place successful is. But what is ``successful'' here? What makes people happy in these places?
	\1 ``Only groups, social classes and class fractions capable of revolutionary initiative can take over and realize to fruition solutions to urban problems'' (p. 154) this implies that the people in power, who are anti-revolutionary otherwise they wouldn't be doing the job they are and that doesn't appear to solve the problems that people need solving, want to silence the revolutionaries because they are the ones with the real power. It makes me think about democracy, and what is just? We have people who govern us and the city, who try to implement solutions to create successful places, but actually the ones who have the most power to change things are the `revolutionaries', ie. everyday citizens who can organise themselves, be resourceful, and fix their own problems despite the higher powers
	\1 ``In itself reformist, the strategy of urban renewal becomes `inevitably' reformist'' (p. 154)
	\1 Working class
		\2 The working class is the only one able to put an end to segregation (and other ills of the urban), because they are the ones who are inherently against it since it affects them most in their daily life. This probably explains why revolutionaries are historically often working class
		\2 Without the working class, features of the urban like integration/disintegration, segregation, doesn't mean anything
		\2 The working class pressures governments (and society, maybe indirectly) for the recognition of rights 
	\1 ``Although necessary, policy is not enough. It changes during the course of its implementation. Only social force, capable of investing itself in the urban through a long political experience, can take charge of the realization of a programme concerning urban society'' (p. 156) $\rightarrow$ change comes not from government, but from the people who enact it and take ownership of their situation
	\1 Commodification of nature: ``the `naturality' which is counterfeited and traded in, is destroyed by commercialized, industrialized and institutionally organized leisure pursuits'' (p. 158)
	\1 Lefebvre adopts a romanticised view of the rural (``colonized by them [urban dwellers], the countryside has lost the qualities, features and charms of peasant life'', ``the urban ravages the countryside'', ``return to the hear to the traditional city'' (p. 158)). 
	\1 ``The need and the `right' to nature contradict the right to the city without being able to evade it'' (p. 158) 
\end{outline}

\subsubsection{Harvey, \textit{The right to the city}}

\begin{outline}
	\1 Whose right to the city? Groups who fight for \textit{their} right, are fighting against people also protecting \textit{their} rights. So, whose rights are more right? Marx noted that in this case, force always wins. Which seems ironic when talking about `rights', which have a universal and undeniable connotation, but in reality it isn't so
	\1 We change ourselves by changing our world: the world is the physical expression of who/what we want to be $\rightarrow$ a dialectical relationship
	\1 What is social justice? Is it always in the advantage of the stronger? Other possibilities include being egalitarian, utilitarian (Bentham), contractual (Rousseau), cosmopolitan (Kant), state imposed justice (Hobbesian); but in the end, the ruling class always wins
	\1 Capitalism
\end{outline}

\subsubsection{Mayer, \textit{The `Right to the City' in the context of shifting mottos of urban social movements}, 2009}

\begin{outline}
	\1 Macro trends transformed the environment (cities, political circles) and the movements that operate in these environments 
	\1 Do cities still have the environments that ``harbour the prerequisites for revolutionary forces for social change''? 
	\1 ``Urban development patterns have become increasingly similar across the advanced capitalist countries, and forms of urban governance have converged so much that it is not surprising that the movements challenging and resisting them, across the global North anyway, have gone through similar cycles'' (p. 363)
	\1 1960s: first wave of broad urban mobilisation was a reaction to Fordism. It was driven by youth, students, migrants, anti-war and leftists mobilisations; driven by ``the inhospitability of our cities'', focused on housing, rent strikes, campaigns against urban renewal, youth and community centres
		\2 Activism relocated from the factory to the neighbourhood
	\1 1980s: second wave of urban mobilisation was a reaction to austerity and global shift to neoliberalism.
		\2 ``increasing unemployment and poverty, a `new' housing  need, riots in housing estates and new waves of squattings changed the make-up of the urban movements'' (p. 364)
		\2 Local governments became cooperative, using community-based organisations and shifting from ``protest to program''; previously confrontational groups shifted towards developing and delivering more or less alternative services
		\2 This created a division between recognised/official groups, and other groups whose needs were not addressed, who radicalised
		\2 New middle class movements emerged in the 1980s: NIMBY, environmental
	\1 1990s: new regime of radically prioritising market mechanisms and roll-out neoliberalism
		\2 New urban development policies were polarising: on one hand, movements defended their privileges; on the other hand they politicised `whose' city it was supposed to be (p. 365)
	\1 2000s: ``urbanisation has gone global through the integration of financial markets that have used their flexibility to debt-finance urban development around the world''
		\2 Social movement organisations are reproduced through local social and employment programs, or community developments, and are often seen as doing a better job than private or state actors could
		\2 Corporate urban development, neoliberalisation of the welfare state
		\2 Movements are localising, and campaigning on issues like privatisation and infringement on social rights. But since these phenomenons and movements happen worldwide, they can connect to one another. They attack global neoliberalism (as global corporations, investors, developers)
	\1 Contemporary urban situation: loss of civil, political, social, economic rights; gated communities and privatisation of public space has created (invisible) barriers between the poor and wealthy, and reinforced social divides
	\1 The responsibility of enforcing the ``right to the city'' falls on the people who inhabit urban space, and not on those responsible for producing it - a fundamental contradiction 
		\2 Using one category ``urban inhabitants'' is wrong and hurtful, because it considers the city as one homogeneous group of people, including ``economic and political actors who participate in and profit from the production of poverty and discrimination''; but actually the city is stratified by class and power
\end{outline}

\subsubsection{Weekly questions}

\begin{outline}
	\1 Where does the commodification of culture and art fit into the discussion of requiring a place where ``exchange would not go through exchange value, commerce and profit'', and the human experience can be experienced without money?
	\1 Is utopia possible? I think no, or we (or I) have to redefine it. It feels more obvious today that utopia is impossible, Ref. covid 19, anti vax protests. I guess that's why Lefebvre describes utopia as `experimental'. 
	\1 Lefebvre says that the working class is responsible for the fights within the city, against segregation and disintegration for example, because they are the ones who have to suffer the consequences (abandonment by the state, poverty, exclusion, etc.). The working class is seldom represented in government, and so it seems that revolutionary behaviour as Lefebvre describes it is necessary for working class representation to exist. So without their voice and their actions, is it right to say that the government would not be confronted and would not engage with issues specific to working class?
	\1 I found Lefebvre's text hard to read. I don't think I could understand many of his points, they were abstract and the language was complex in a way that completely lost me at some points. Chapeau to those who could dissect eg. ``example''
\end{outline}


%%%%%%%%%%%%%%%%%%%%%%%%%%%%%%%%%%%%%%
% 							ENVIRONMENTS
%%%%%%%%%%%%%%%%%%%%%%%%%%%%%%%%%%%%%%

\if{false}

\begin{outline}
	\1
\end{outline}

\subsubsection{, \textit{}}
\fi


\end{document}
