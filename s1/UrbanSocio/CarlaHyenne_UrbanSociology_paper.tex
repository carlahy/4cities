\documentclass{article}

\usepackage[utf8]{inputenc}
\usepackage[left=1.5in,right=1.5in,bottom=1in]{geometry}
\usepackage{setspace} \doublespacing
\setlength\parindent{0pt}
\setlength{\parskip}{1em}
\setcounter{secnumdepth}{0}
\usepackage{outlines}
\usepackage{graphicx}
\graphicspath{ {imgs} }
\usepackage{hyperref}

\title{Urban Sociology}
\author{Carla Hyenne }

\begin{document}

\maketitle

Requirements:
\begin{itemize}
	\item 3000 words double spaced, excluding refs/figures
	\item include at least 3 references to texts of syllabus
	\item include at least 2 references to new urban sociology texts
\end{itemize}

%%%%%%%%%%%%%%%%%%%%%%%%%%%%%%%%%%%%%
%												NOTES
%%%%%%%%%%%%%%%%%%%%%%%%%%%%%%%%%%%%%

\section{Notes}

\subsection{Gig economy}

\begin{outline}
	\1 Adam Badger, Geographer, PhD in food delivery gig economy  \url{https://scholar.google.com/citations?user=trzrwWAAAAAJ}
	\1 Lecture 5, Undocumented Activism: 
		\2 Are people working in the gig economy organising themselves in such a way as to be recognised and heard?
		\2 Can their activism be related to urban space? They have become a common sight within urban space, their bikes/cares/motorcycles are ubiquitous. However, is there any activism and if so, can it be seen on the streets? Its not visible in Brussels, or Munich... are there other European cities where it is visible? How do the conditions change from country to country? What about in the US? Are there less worker welfare protection policies?
		\2 Gig economy companies like Uber, make it virtually impossible for its workers to unionise. They have no formal way of contacting each other, of reaching out to their peers, if not by word of mouth or by ad-hoc measures\marginpar{Find out how they organise}
		\2 Acts of civil disobedience: these sound interesting, they are planned events that go against the status qo. Are there such things staged by gig economists?
		\2 Are gig economy workers uniting in safe spaces?
	\1 A parallel to the gig economy: the black economy. 
\end{outline}

\subsection{Idea bucket}

\begin{outline}
	\1 Empirical study of a city, on a sociological perspective (can maybe reuse astana). Astana was majority non-kazakh a century ago, but efforts to make it the capital drove kazakhs to it (especially given saturation of almaty), they are now the majority
	\1 Civil disobedience 
	\1 The black economy: it makes up a part of the economy of the city, eg Brussels where 100.000 people are undocumented yet working and contributing to the society in some way. Who are these people who work in the black economy? how are they shaping the city, and influencing the politics?
	\1 sanctuary cities and safe spaces: do they work, how do they work, why don't they work, do they change urban space?
		\2 What conditions need to exist in order for space spaces to exist and be constructive towards progressing the cause?
		\2 Rally by undocumented group, by the ICE agency in Chicago (Immigration Customs and Enforcement)
\end{outline}

%%%%%%%%%%%%%%%%%%%%%%%%%%%%%%%%%%%%%
%										ENVIRONMENTS
%%%%%%%%%%%%%%%%%%%%%%%%%%%%%%%%%%%%%
\if{false}

\begin{outline}
	\1
\end{outline}

\fi

\end{document}