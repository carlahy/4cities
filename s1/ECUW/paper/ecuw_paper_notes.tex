\documentclass{article}

\usepackage[utf8]{inputenc}
\usepackage[left=1.5in,right=1.5in,bottom=1in]{geometry}
\setlength\parindent{0pt}
\setlength{\parskip}{1em}
\setcounter{secnumdepth}{0}
\usepackage{outlines}
\usepackage{graphicx}
\graphicspath{ {imgs} }
\usepackage{hyperref}

\usepackage[
backend=biber,
style=alphabetic,
sorting=ynt
]{biblatex}
\addbibresource{ecuw_paper.bib}

\title{ECUW - Astana}
\author{Carla Hyenne}

\begin{document}

\maketitle

\tableofcontents

\pagebreak

\section{Readings}

\subsection{Natalia Koch, \textit{The monumental and the miniature: imagining `modernity' in Astana}}\cite{koch2010monumental}

\begin{outline}
	\1 The Astana project hints at the idea of creating a new `social order'
	\1 Compared to other ex-soviet countries, Kazakhstan has had a rather smooth political transition
\end{outline}


See this link for papers on Kazakhstan, published in Eurasian Geography and Economy \url{https://scholar.google.com/scholar?hl=en&as_sdt=0%2C5&q=eurasian+geography+and+economics+kazakhstan&btnG=&oq=eurasian+geography+and+economics+kazk} 

\subsection{Natalia Koch, \textit{Bordering on the modern: power, practice and exclusion in Astana}}\cite{koch2014bordering}

\textit{Keywords: modernism, emotional landscapes of control}

\begin{outline}
	\1 What were ordinary residents' and citizens' role in the capital-city making project of Astana?
	\1 In creating modernity, what or who were the city planners and residents trying to `exclude'
	\1 Astana is built on `tabula rasa'
	\1 Excluding the soviet past
		\2 Astana is built on top of an old soviet city, Tselinograd, which was renamed in 1998
		\2 To build Astana, a lot of run-down soviet era buildings and neighbourhoods (eg. wooden) were torn down, and the soviet idea of standardisation was no more. Argument was sanitisation, and prettification (buildings all looked the same, 5 story brown)
\end{outline}

\subsection{Shonin Anacker, \textit{Geographies of Power in Nazarbayev's Astana}}\cite{anacker2004geographies}

\begin{outline}
	\1 Links between the move and the development of official nationalism in Nur Sultan's post-communist governance 
	\1 Features of the new capital's layout, changes in ethnic composition due to migration patterns after capital relocation
	\1 Astana is a nation building project
		\2 Proximate power relations: coercion, authority
		\2 Far reaching symbolism: manipulation, seduction
	\1 Post-colonial city: Almaty had signs of the Russian colonial era in its urban space and population. Almaty was a `Russian' city in the middle of KZ territory. KZ wanted to protect its north steppes from Russian invasion, but there was no major city near by. 
	\1 Building a capital city
		\2 A conservative estimate of the cost of Astana is 10 billion USD. 
\end{outline}


%%%%%%%%%%%%%%%%%%%%%%%%%%%%%%%%%%%%%%%%%%%%%%%%%%%%
% 																	NOTES																
%%%%%%%%%%%%%%%%%%%%%%%%%%%%%%%%%%%%%%%%%%%%%%%%%%%%

\section{Notes}

\subsection{Links to syllabus and research questions}

\begin{outline}
	\1 Lecture 5: ``Planning is not a neutral technical activity: it is influenced by the changing agenda of governments and other interest groups, and their views on the role of the state, the market and civil society in shaping the built and natural environment''
		\2 Planning changes based on historical, cultural, geographical, constitutional/legal, administrative, economic, political factors
	\1 Astana is a nation building project
	\1 Capital city
		\2 What makes a capital city? In 1995, when Nur Sultan signed the treaty ``On the Moving of the Capital'', Aqmola was an industrial city of roughyl 200,000 people. `Overnight', Aqmola's residents became the Kazakh capital's residents. 
		\2 It is clear today, to any one looking at Astana's social, economic, political features that the city is a capital. If only for the fact that Astana is the Kazakh word for capital. But what happened in the twenty or so years since the declaration of the new capital, that made Astana what it is today - a capital city?
		\2 What socio, political, economic forces transformed Astana into a capital. Is Astana a global city?
\end{outline}

%%%%%%%%%%%%%%%%%%%%%%%%%%%%%%%%%%%%%%%%%%%%%%%%%%%%
% 																	SEMINARS																
%%%%%%%%%%%%%%%%%%%%%%%%%%%%%%%%%%%%%%%%%%%%%%%%%%%%

\section{Seminar 2 - Data and research strategies}

\begin{outline}
	\1 Present how you found empirical data on city, and how you analysed this data
\end{outline}

\pagebreak

\printbibliography

\end{document}

%%%%% ENVIRONMENTS %%%%%

\if{false}

\subsection{\textit{}}

\begin{outline}
	\1
\end{outline}

\fi