\documentclass{article}

\usepackage[utf8]{inputenc}
\usepackage[left=1.5in,right=1.5in,bottom=1in]{geometry}
\setlength\parindent{0pt}
\setlength{\parskip}{1em}
\setcounter{secnumdepth}{0}
\usepackage{outlines}
\usepackage{graphicx}
\graphicspath{ {imgs} }
\usepackage{hyperref}

\usepackage[
backend=biber,
style=alphabetic,
sorting=ynt
]{biblatex}
\addbibresource{ecuw_paper.bib}

\title{ECUW - Astana}
\author{Carla Hyenne}

\begin{document}

\maketitle

\tableofcontents

\pagebreak

\section{Research}

\subsection{Geography}

\subsubsection{General info}

\begin{outline}
  \1 Akmoly settlement started in 1830, it's advantageous geography was clear. Roads connected to the East/West/South areas
  \1 Large urban areas, favourable geographical position, proximity to the major economic centers of the region, considerably demographic capacity, good transportation facilities, relatively favourable climate
  \1 1971, Tselinograd becomes center of the oblast
  \1 Former capital Almaty is still largest city in the country (what are the differences in social, political, economic powers between Almaty and Astana?)
  \1 Located on the Ishim River, very flat steppe region, very spacious landscape, in between the north of KZ and the very sparsely settled center, because of the river. 
  \1 Astana is divided by the river: the north has the old boroughs, and the south has the new boroughs
\end{outline}

\subsection{Economy}

\subsubsection{General info}

\begin{outline}
  \1 Served as a route to transport equipments during war, industries supplied WWII material (to Soviets). After WWII Akmoly was beacon of economic revival in Soviet Union
  \1 Became part of the Virgin Lands Campaign by Nikita Khrushchev, in order to boost agricultural (grain) production to help with the food shortages in the Soviet Union
  \1 Based on trade, industrial production (mainly of building material, foodstuff, mechanical engineering), transport, communication, construction
  \1 Astana International Financial center opened in 2018 to become a hub for financial services in Central Asia
  \1 Headquarters for state-owned corporations
  \1 The shift of the capital was an economic boost for Astana, and the economic development attracted foreign investors. The New City "special economic zone" was created to develop industry and make the city attractive to investors
  	\1 Was given "the market economy country" status by the EU and the US, in 2000 and 2002 - what does that mean? \url{https://en.wikipedia.org/wiki/Economy_of_Kazakhstan}
\end{outline}

\subsubsection{Statistics}

From the world bank:

\begin{outline}
	\1 Country overview: \url{https://data.worldbank.org/country/kazakhstan?view=chart}
		\2 There is a significant drop in GDP between 2013 and 2016. Russia has a similar drop in GDP over the same time (\url{https://data.worldbank.org/country/russian-federation?view=chart}), this could indicate that the factors that led to the decline in wealth was the same, and that perhaps the way these two countries operate means that they were hit the same way by these factors, or that these two countries are intertwined in some way
		\2 The EU does not have this decline in GDP between 2013-2016. On the contrary, EU's GDP grew between 2015-2018, even though it declined 2014-2015. Maybe this drop can be linked to what affected the economy of KZ.
	\1 Economic crisis 2015
		\2 \url{https://www.worldbank.org/en/country/kazakhstan/publication/economic-update-summer-2016} explains KZ's economic crisis as: a drop in oil prices, slowing doing of China's growth, and 
		\2 KZ switched to a floating exchange-rate regime, and inflation targeting regime 
		\2 KZ economy is highly dependent on oil, and supplies 5.5\% of the EU's extra-EU petroleum\footnote{https://ec.europa.eu/eurostat/documents/3217494/10934584/KS-EX-20-001-EN-N.pdf/8ac3b640-0c7e-65e2-9f79-d03f00169e17?t=1590936683000, p. 120)}
		\2 It has plans to diversify its economy, and reduce its reliance on petrol export. It wants to increase governance and reduce role of the state in the economy. There is a privatisation plan 2016-2020, which is a large-scale privatisation program, to reduce state intervention in the economy to 15\%, which is a target set by the OECD, Organisation for Economic Cooperation and Development (sounds like a move further away from socialism)
		\2 The 2014 slowdown in growth is linked to the Ukrainian crisis \url{https://en.wikipedia.org/wiki/Economy_of_Kazakhstan#2014_and_2015_developments}
\end{outline}

From the government \url{https://stat.gov.kz/}: 

\begin{outline}

\end{outline}

\subsection{Politics}

\begin{outline}
  \1 Is, or was, an oblast: a type of federal subject of Russian Federation. What did this mean for Akmoly? 
  \1 Renamed from Akmoly to Tselinograd (1961) to represent city's role in the Virgin Lands Campaign, it means "city of virgin lands"
  \1 1991, Kazakhstan gains its independence after dissolution of Soviet union, name is restored to Akmola.
  \1 1994, decree "on the transfer of the capital of Kazakhstan" which is officially moved from Almaty to Akmoly in 1997, and renamed to Astana in 1998 (Why did the name change?). A major driver for the movement of the capital was the growing Kazakh population in Astana, which was done on purpose to alleviate burden on Almaty which was running out of space for expansion. Almaty is also located on earthquake prone land, and near Chinese border
  \1 1999, Astana is awarded medal of City of Peace by UNESCO
  \1 2019, Nur Sultan resigns and city is renamed to him, even if there is resistance by residents (out of habit? opposition?) who continue to call it Astana
  \1 Platform for high profile diplomatic talks, and summits on global issues
  \1 KZ is rich with oil money, yet a lot of it has just gone into enriching the capital and the buildings to project power and influence
  \1 KZ needs to navigate Russia, China and US 
\end{outline}


\subsection{City Scape/Development}

\begin{outline}
  \1 1960s completely transformed Tselinograd: three high-rise housing districts began, new monumental public buildings (including Virgin Land Palace, Palace of Youth, House of Soviets, new airport, sports venues).
  \1 Divided into four districts
  \1 In 1998, the KZ government launched a competition for renowned architects and urban planners to design the new capital. The winner, Kurokawa, decided to preserve and redevelop the existing city, and create a new city at the south and east of the river. 
  \1 Given the above, the North of the rail way is the industrial part and the poor residential areas. Between the railway and the river is the city center, where intense building is happening. South is the new area with government administrations, diplomatic quarters, government buildings.
  \1 Centrally planned city
  \1 For reference, the presidential palace that is modelled after the white house, is eight times bigger than it. What kind of message is this sending? 
  \1 Astana is estimated to have cost 15 billion usd (2007) \cite{laszczkowski2011building}
\end{outline}

\subsection{Population}

\begin{outline}
  \1 Astana has 1.1M residents, mostly Kazakhs (80\%), Russians (17\%), and other central Asians representing <1\% each. The majority Kazakh population is a recent phenomenon, and shift in the last decades
  \1 There was a drive to attract Kazakhs north-ward from Almaty, due to density limits of Almaty. this was key in shifting the capital
  \1 Attracts migrant workers, legal and illegal
  \1 Attracts young professionals
  \1 To what extent has the planning of Astana influenced the socio-demographics of the city?
  \1 The population tripled in 10 years, with mass migration from within the country, at the prospect of employment and better life in the new capital \cite{laszczkowski2011building}
  \1 When the city was Tselinograd, it was a majority German population, who were descendants of Volga Germans, that Stalin (USSR dictator) had banned to Central Asia 
\end{outline}

KZ government stats:

\begin{outline}
  \1 Great source of population statistics \url{https://stat.gov.kz/official/industry/61/statistic/8} 
\end{outline}

Its hard to see the city as anything but Nazarbayev's monument to himself. The city is named after him, and so many building are named after him (Nazarbayev university, Nazarbayev airport, Nazarbayev centre, Nazarbayev central concert hall

- spectacular urbanism, what is behind this image? is it using modernity as a legitimisation of authoritarianism?

\subsection{Post-Soviet Nation-Building}

\begin{outline}
	\1 Soviet collapse
		\2 The collapse of the Soviet union led to socio-economic crises, and preceded charismatic leaderships in the 20th century\cite{isaacs2010papa}
		
	\1 Nation-building: population, culture, history, identity, language
		\2 First decade after the collapse of the Soviet union, ie. in the 1990s, post-Soviet countries dealt mostly with nationalisation
		\2 Goals were to: maintain Kazakhstan's integrity, avoiding ethnic conflicts between Russians/Russified Kazakhs/provinces; create a multi-ethnic and multi-religion nation, with not discrimination between groups
		\2 Biggest challenge was: creating national integrity, based on Kazakh ethnicity; at time of collapse, KZ's ethnic population was most diverse within post-soviet countries and Kazakh population was not the majority ethnicity (Kazakhs were 44\%, Russians 37\%, to be checked on national statistics); thus, Kazakh identity had to be strengthened without marginalising Russians 
		\2 The GINI coefficient, ie. the measure of inequality, now resembles that of European countries rather than that of soviet times, and a large and growing number of the population (circa 35\% in 2017) live in urban areas

	\1 State-building: politics, government, democratisation
		\2	At the collapse of the Soviet Union, leaders in the communist party jumped at the opportunity to become leaders of their independent nation. They changed the narrative from building a communist project, to a nation project. Their regimes, at the exception of Kyrgyzstan, were  characterised by authoritarianism\cite{batsaikhan2017central}
		\2 Nazarbayev's legitimacy:
			\3 Nazarbayev succeeded at (re) building the nation and state, keeping the peace and bringing prosperity to the country; thus, his legitimacy 
			\3 Ensuring international legitimacy was important for the state building project
			\3 Economic prosperity soon after the soviet collapse gave NSN legitimacy, especially amongst the elites/new middle classes
		\2 BUT, the government has voted into law things like the right to cut off internet and mobile phone access; non of the elections since Nazarbayev's rise to power are considered free and fair; political opposition is either shut down or chased out of the country; this all bares resemblance to Russia and the USSR
			
	\1 Economic reforms, marketisation
		\2 When the USSR collapsed, Kazakhstan (and other Central Asian countries) lost their markets. This led to partial de-industrialisation and an economic crises, because the new nations could not compete or take part in the global market (yet), under new conditions \cite{batsaikhan2017central}
		\2 The economy was on the brink of collapse when the soviet union dissolved, much like other soviet states' economies
		\2 The lack of transport networks, which were built mostly in Russia and soviet republics the West of Central Asia, are an obstacle for trade with the outside world (ie. outside Central Asia and Russia) and economic development\cite{batsaikhan2017central}
		\2 There has been economic growth since the 2000s, mostly because of the export of commodities such as oil, natural gas, and metals
		\2 Kazakhstan has made the transition to a market economy. Import and export activities make up a great share of the national GDP, however this is predominantly from their specialisation in the commodity market (oil, gas, metals) rather than from an open-trade regime\cite{batsaikhan2017central}
		\2 Kazakhstan's economy was based predominantly on mining and commodity; it needed/needs a strategy to move away from commodity-based growth, towards new diverse economic and political reforms that reshape the economy
			\3 Its still strongly based on mining and quarrying, an industry which expanded 
		\2 Types of economic reform\cite{batsaikhan2017central}: large scale privatisation, small scale privatisation, goernance and enterprise restructuring, price liberalisation, trade and forex system, competition policy
		\2 Providing economic prosperity
		\2 Economy bounced back thanks to better resource use and oil industry (KZ has high supply, and prices were high), which meant the standards of living rose, parts of the population grew wealthier, new middle class with disposable income
		\2 The EU is the largest export market for Kazakhstan, and Russia has a increasingly smaller share
		\2 Since 2015, Kazakhstan belongs to the World Trade Organisation
		\2 ``Kazakhstan and Kyrgyzstan have achieved some progress in building market oriented financial sectors. Kazakhstan has attracted meaningful foreign investment into this sector. It also has the largest banking sector as measured by the ratio of credit to the private sector to GDP, which was 58.9 percent of GDP in 2007 but then declined as a result of the 2007–2008 banking crisis''\cite{batsaikhan2017central}
			\3 Check how much Astana receives in FDI, compared to other major cities and regions; and how many financial institutions exist in Astana; if the foreign investments were hit by the 2008-2009 crisis, Astana should have been hit economically, more than other regions/cities, and if other regions are not hit by the 2008 crisis but the country as a whole is, then it means that Astana's economy greatly influences KZ's economy and prosperity;
		\2 ``The decline in the prices of oil, natural gas, metals and agricultural raw materials in the second half of 2014 meant that Central Asia suffered a huge adverse shock. The vulnerability of Central Asian economies to changes in the world commodity markets was exposed and the need for their structural diversification towards more manufacturing and services became even more urgent (see Linn, 2016).'' \cite{batsaikhan2017central}
			\3 There was a commodity crisis in 2014, in which the prices dropped and the macroeconomy deteriorated; KZ tenge depreciated, there was increase in inflation, growth slowed, national finances got worse;
\end{outline}

\subsection{Post-soviet urban theory}

Introduction to Post-Socialist Cities and Urban Theory\cite{ferenvcuhova2016introduction}

\begin{outline}
	\1 There is a lack of post-soviet, or post-communism, urban theory and research; all the ideas that are applied to the post-soviet city come from the West, the flow of urban knowledge is unidirectional; urban research is predominantly western, but ``is circulated as universal knowledge'' (Robinson, p. 3)
	\1 Reasons include: the post-soviet cities' economies must be corrected towards a modern, capitalist system, and rid of socialist urban ``impurities'', liberated; because post-sovietism is a transition process, it's hard to export because it's too much situated in space and time; 
	\1 Post-soviet cities are portrayed as non-modern, behind the Western cities and society ideals, and seen as needing to catch up to the modern western norms, if observed through a western urban lens; 
	\1 Post-socialism is still used today (2010s onward) to describe urban phenomena in once-soviet cities, like cycling, or urban regeneration projects; thus it appears to be still relevant, even though the event is 30 years old
	\1 Two main transitions associated with post-socialism are democratisation and marketisation
	\1 ``the material urban environment has been used strategically by political and economic actors to create new path-shaping outcomes'' $\rightarrow$ the built environment is used by government to create path dependencies? 
	\1 The urbanisation of transition
	\1 The socialist past is exploited by governments in post-soviet (European) countries, to promote reforms that withdraw the state from the provision of social security (ie state welfare, eg. social housing); claims for social security are portrayed as non-democratic, non-progressive
		\2 This argument is used to force labour force to accept the economic reforms as progressive, and claiming social benefits as backwards 
	\1 Path dependency, and path-shaping in post-socialist transformations, rolling path-dependencies
\end{outline}

Argument ideas: to what extent does the (western/European) discourse on global cities, apply to Astana?
path dependence

\subsection{Media sources}

\begin{outline}
	\1 \url{https://eurasianet.org/}
	\1 \url{https://tol.org/client/article/category/regions/central-asia/kazakhstan}
	\1 Meet our partners section in \url{https://www.theguardian.com/world/2014/jun/09/post-soviet-states-new-east-network-guardian-welcome}
\end{outline}

\subsection{Politicians}

Tokayev: ``You know what happens in winter? The city becomes filled with smoke... It disgraces us before the people, diplomatic staff and foreigners who live here. And we are positioning ourselves as an international hub, a global city in the heart of Eurasia?!'' as quoted in \cite{darkhanumirbekov2019}, speaking about the state of the city during the winter.

\subsection{Global cities}

We will see that given Astana’s role in modernising Kazakhstan, Astana looks like an aspiring global city: has an international airport, geographically central within Central Asia, has concentration of APS functions, attracts foreign investments, is the innovation hub of the country,...

However, we will discover that the global city, a measurement of the success of (predominantly globally northern) cities, is not a measure which is useful or fair to evaluate Astana, and that post-socialism should not always be interpreted as the transitioning into a capitalist, neo-liberal system characteristic of the Global North

Finally, we will briefly discuss what is next for the Kazakh capital - if Astana is not an aspiring global city, what discourse can we use to interpret its position as a national capital, in a globalising Central Asia? (eg. comparative urbanism, “third way”...)
We will conclude that a different approach than the global cities rhetoric is required to understand globalising Central Asian cities, because the North-centric global city discourse, that explains cities’ success in advanced countries and economies, is not a discourse that is applicable to Astana, one of the most powerful and globally influential cities in Central Asia

Given its current position, should Astana aspire to be a global city? $\rightarrow$ no, because it does not have the resources or the institutional structure that many global cities, ie. cities in the global north, do (geopolitics, infrastructure, state-controlled capitalism,...) $\rightarrow$ does this mean that Astana hasn’t been successful, or that it is not relevant in the global context? $\rightarrow$ no, but it means that the measure against which to evaluate a city’s success does not apply to Astana $\rightarrow$ so, what is a better way to conceptualise Astana as a city in a globalised world? (the third way discourse, taking example of China,...)


%%%%%%%%%%%%%%%%%%%%%%%%%%%%%%%%%%%%%%%%%%%%%%%%%%%%
% 																	NOTES																
%%%%%%%%%%%%%%%%%%%%%%%%%%%%%%%%%%%%%%%%%%%%%%%%%%%%

\section{Notes}

\subsection{Links to syllabus and research questions}

\begin{outline}
	\1 Lecture 5: ``Planning is not a neutral technical activity: it is influenced by the changing agenda of governments and other interest groups, and their views on the role of the state, the market and civil society in shaping the built and natural environment''
		\2 Planning changes based on historical, cultural, geographical, constitutional/legal, administrative, economic, political factors
	\1 Astana is a nation building project
	\1 Capital city
		\2 What makes a capital city? In 1995, when Nur Sultan signed the treaty ``On the Moving of the Capital'', Aqmola was an industrial city of roughyl 200,000 people. `Overnight', Aqmola's residents became the Kazakh capital's residents. 
		\2 It is clear today, to any one looking at Astana's social, economic, political features that the city is a capital. If only for the fact that Astana is the Kazakh word for capital. But what happened in the twenty or so years since the declaration of the new capital, that made Astana what it is today - a capital city?
		\2 What socio, political, economic forces transformed Astana into a capital. Is Astana a global city?
	\1 History \url{https://globaledge.msu.edu/countries/kazakhstan/history}
		\2 2009: France and KZ sign energy and business deals costing 6bn usd
		\2 2010: Kazakhstan becomes the first former Soviet state to chair the Organization of Security and Cooperation in Europe (OSCE) security and rights group.
		\2 2014: Kazakhstan signs an agreement with Russia and Belarus to create the Eurasian Economic Union to create a shared market and integrate their economic policies. 
\end{outline}

%%%%%%%%%%%%%%%%%%%%%%%%%%%%%%%%%%%%%%%%%%%%%%%%%%%%
% 																	STATISTICS																
%%%%%%%%%%%%%%%%%%%%%%%%%%%%%%%%%%%%%%%%%%%%%%%%%%%%

\begin{outline}
	\1 GDP per head in Astana, vs. Almaty, vs. other cities in KZ
		\2 What does this tell us about Astana? Does it show the power it has?
	\1 Where did the investments come from, for Astana? Internal or external to the country?
		\2 Investments in one place, mean disinvestment in another. Who or what was left behind, or deprioritised, as a result of the building of the new capital?
	\1 Are the buildings in Astana public or private?
		\2 Is there a governmental strategy regarding real estate?
\end{outline}

\subsection{World bank data}

\subsection{KZ gov stats}

\url{https://stat.gov.kz/}
%%%%%%%%%%%%%%%%%%%%%%%%%%%%%%%%%%%%%%%%%%%%%%%%%%%%
% 																	READINGS																
%%%%%%%%%%%%%%%%%%%%%%%%%%%%%%%%%%%%%%%%%%%%%%%%%%%%

\section{Readings}

\subsection{\textit{The monumental and the miniature: Imagining modernity in Astana}, Koch, 2010}

\begin{outline}
	\1 How Astana is a modernist project, that is used for legitimising authoritarianism
	\1 Astana's urban landscape and the modernist architecture is a break with soviet architecture
	\1 Asks: what is the role of the capital in national identity projects? \footnote{It could be interesting to research specifically about the capital. What role does the capital play within the nation? It must have great importance if it was moved and built from scratch over the last 30 years, just after KZ's independence from the Soviets. What about other post-soviet capitals? Does Astana share any similar features? What are the socio, political, economic dynamics between Astana, post-soviet states and cities, and Russia/the rest of the world?}
	\1 KZ, and its government, are promoting a `Eurasian' narrative for the state
		\2 One of the reasons that the capital moved to Astana: it is further away from the Chinese border, more central within the country
		\2 Symbols in Astana's architecture and image are not nationalist, modernity and progress are used to promote the diverse population and the Eurasian image (even if there is a strong inclination/bias towards Kazakhstani heritage)
	\1 Presents the monumental (big buildings) and the miniature (photographs) in shaping the image of Astana
	\1 Modernism in Central Asia has its specifics compared to other parts of the world
		\2 Imperial Russia/Soviet socialism justified its colonies by the promise of progress, modernisation and economic prosperity
	\1 "High modernism is an ideology that relies heavily on "visual images of heroic progress toward a totally transformed future" (Scott, 1998:95)"
	\1 The Kazakhstan-30 program is built of 5-year plans, reminiscent of the Soviet era; this sort of short scale justifies to the population that sacrifices are worth it for the better tomorrow that is just around the corner.
	\1 Astana as a proxy for Nazerbayev's cult of personality:
		\2 Astana Day celebration is on his birthday, July 6th. It marks the day of the capital change
		\2 The city is named after him, and so many building are named after him (Nazarbayev university, Nazarbayev airport, Nazarbayev centre, Nazarbayev central concert hall
\end{outline}

\subsection{Natalia Koch, \textit{The monumental and the miniature: imagining `modernity' in Astana}}\cite{koch2010monumental}

\begin{outline}
	\1 The Astana project hints at the idea of creating a new `social order'
	\1 Compared to other ex-soviet countries, Kazakhstan has had a rather smooth political transition
\end{outline}


See this link for papers on Kazakhstan, published in Eurasian Geography and Economy \url{https://scholar.google.com/scholar?hl=en&as_sdt=0%2C5&q=eurasian+geography+and+economics+kazakhstan&btnG=&oq=eurasian+geography+and+economics+kazk} 

\subsection{Natalia Koch, \textit{Bordering on the modern: power, practice and exclusion in Astana}}\cite{koch2014bordering}

\textit{Keywords: modernism, emotional landscapes of control}

\begin{outline}
	\1 What were ordinary residents' and citizens' role in the capital-city making project of Astana?
	\1 In creating modernity, what or who were the city planners and residents trying to `exclude'
	\1 Astana is built on `tabula rasa'
	\1 Excluding the soviet past
		\2 Astana is built on top of an old soviet city, Tselinograd, which was renamed in 1998
		\2 To build Astana, a lot of run-down soviet era buildings and neighbourhoods (eg. wooden) were torn down, and the soviet idea of standardisation was no more. Argument was sanitisation ie. public health, and prettification (buildings all looked the same, 5 story brown, wanting to break from the soviet era styles)
		\2 ``the entire narrative echoes so many colonial discourses, in which the local urban developments are characterised as chaotic, primitive, and unhygienic - and definitely non-modern'' (Koch, p. 435)
		\2 Kazakhs typically lived communally, as nomads, this continued during soviet era, but this changed during the modernisation of KZ in the 1990s; people were forced to live in individual housing units, which was not something they were used to
		\2 Kazakhs who wish to remain living this `communal' lifestyle, live in spaces that are targeted as spaces for demolition to make way for the `new social and political economic order of the city and the state'
	\1 Task of Chykanayev, master planner for Astana, has the role of addressing healthy vs. diseased spaces, and to `impose a harmonious social and spatial order'
		\2 Imposing social order through space becomes a `natural' solution to unhealthy urban space
		\2 Destruction of poor soviet neighbourhoods = political act, ordering the disorder 
	\1 Much of Astana does not reflect the hyper modern Left Bank image, but rather looks like a village (p. 437)
		\2 People see Astana as a big village, and even if people move from rural areas to Astana for some time, they still keep their `rural mentality', and thus isn't changed by an urban way of life. This is because Astana still has rural mentality, migrants, and structures $\rightarrow$ rural mentality is explained as a lack of culture, consideration, cosmopolitan way of living, urban/modern aspirations 
	\1 Division of north and south KZ: north = "urban, modern, civilised, wealthy, Russified", and south = "rural, traditional, uncivilised, poor, Kazakh, dark skinned" (p. 438) \marginpar{Understand how image has shaped KZ culture promotion}
		\2 This division is mapped onto the country, is this why the capital was moved to Astana? so that Astana could be seen as the `northern', not traditional, modern image?
		\2 Soviet era elites characterised the south as nationalists 
\end{outline}

\subsection{Shonin Anacker, \textit{Geographies of Power in Nazarbayev's Astana}}\cite{anacker2004geographies}

\begin{outline}
	\1 Links between the move and the development of official nationalism in Nur Sultan's post-communist governance 
	\1 Features of the new capital's layout, changes in ethnic composition due to migration patterns after capital relocation
	\1 Astana is a nation building project
		\2 Proximate power relations: coercion, authority
		\2 Far reaching symbolism: manipulation, seduction
	\1 Post-colonial city: Almaty had signs of the Russian colonial era in its urban space and population. Almaty was a `Russian' city in the middle of KZ territory. KZ wanted to protect its north steppes from Russian invasion, but there was no major city near by. 
	\1 Building a capital city
		\2 A conservative estimate of the cost of Astana is 10 billion USD. 
\end{outline}

\subsection{Rico Isaacs, \textit{Nursult Nazarbayec and the Discourse of Charismatic Leadership and Nation-Building in Post-Soviet Kazakhstan}}\cite{isaacs2010papa}

\begin{outline}
	\1 Nazarbayev as only politician able to meet challenges of post-soviet nation building, he has created an authoritarian regime\marginpar{how did he make his regime acceptable in KZ?}
		\2 He portrayed a kind of post-soviet charisma, distinct from soviet charisma
		\2 Weber's charismatic ideal-type: NSN doesn't meet all the criteria of this ideal type, but some
	\1 NSN's authority is accepted as believable domestically and internationally
	\1 Nazarbayev's legitimacy was reinforced by the post-soviet economic prosperity, where a new middle class with disposable income emerged 
	\1 ``Territorial integrity, inter-ethnic stability, economic development, and international legitimacy have been central to nation-building in Kazakhstan'' (p. 441)
\end{outline}

\subsection{Natalia Koch, Why Not a World City}\cite{koch2013not}

\begin{outline}
	\1 Nazarbayev's goal for Astana is to ``raise global awareness and prestige of this young Soviet successor state'' (p. 109)
	\1 The third way discourse: ``the script suggests that state control of information, capital and foreign investment is a desirable avenue for entering the global capitalist market, without reliving the perceived social disorder accompanying extreme deregulation in the post-Soviet space in the 1990s'' (p. 115)
	\2 There is a wide rejection of western ``wild capitalism'' and political liberalisation ideologies
\end{outline}

\subsection{The Post-Socialist City Urban Form and Space Transformations in Central and Eastern Europe after Socialism, Kiril Stanilov}

\begin{outline}
	\1 It takes a long time for the built environment to change (decades), compared to political and economic structures (could be days or months); Astana's urban form changed very quickly, compared to central and eastern European post-soviet cities
\end{outline}

%%%%%%%%%%%%%%%%%%%%%%%%%%%%%%%%%%%%%%%%%%%%%%%%%%%%
% 																	SEMINARS																
%%%%%%%%%%%%%%%%%%%%%%%%%%%%%%%%%%%%%%%%%%%%%%%%%%%%

\section{Seminar 2 - Data and research strategies}

\begin{outline}
	\1 Present how you found empirical data on city, and how you analysed this data
\end{outline}

- Need to adapt definition to city context if it isn't specific enough - eg. what is a migrant in a given city/country?
- Find city council data
- Has an impact on identity
- Find information that is not directly related, but could be argued to influence your data
- Land use maps, to compare policies
- Make your own maps to validate your argument
- Finding opposing literature to what we want to study and argue 
- Using music (rap?) and culture to understand the 
- Being critical with sources: gender, ethnicities.... 

- operationalising question: find 'what is wrong' and from that, operationalise the question around it
- convince them that the research question still fits within the course - argument it, make sure they can follow
- eg. taking a trend within the city, and see if that's a general trend within europe
	- are there migration trends, economic growth trends, other trends, that are comparable to the trends of post-soviet, or capital cities?
- question 6
	- there is a lot on question 6! can talk about the transition of capital from Almaty to Astana
- question 15
	- population change isn't only ethnicity, also class, age, education.... everything goes
	- life expectancy, how are these ethnically shaped? it'll be hard to get stats on different populations within KZ or Astana
	- largest migrant groups are usually men between 25-45, how does that shape the migration demographic?
- can be focused within one single area of a city, eg. an airport, a square, one neighbourhood. starting super specific and zooming out 
- have to discuss background and context of city, but very lightly; the argument should be present throughout the paper, don't talk about other aspects of the city if they don't add to the argument

- business partnerships with Europe (TA mentions students writing papers)
- has access to russia for the 3 months? 
- astana has changed names, geographies 
- argue and be critical of eurocentrism, why should i be trying to fit Europe within a non-europe/Eurasian contexts

%%%%%%%%%%%%%%%%%%%%%%%%%%%%%%%%%%%%%%%%%%%%%%%%%%%%
% 																	UEG Lecture 3																
%%%%%%%%%%%%%%%%%%%%%%%%%%%%%%%%%%%%%%%%%%%%%%%%%%%%

\section{Urban Economic Geography link}

Is Astana a city of rent?\marginpar{city of rent} Looking at Astana's spectacular, hyper modern architecture, it is clear that it stands out from the rest of the country's landscape. The second largest city, Almaty, was the country's capital until 1995 (98?), when it was decided that the capital would move to Aqmola. At the time, Aqmola was an industrial settlement of 200,000 inhabitants. Clearly, the landscape had to be changed to reflect the vision that post-soviet, independent Kazakhstan and its president, Nur Sultan Nazarbayev, wanted to showcase. 

What kind of city is Astana today? It is the wealthiest city in KZ (compare GDP of inhabitants of astana to those of almaty, and third biggest city). 

Do the developments address the needs of the people, or of the investors and of the government first?
Does the government rely on the valorisation of the land and real estate, to finance public services? Who owns the fancy buildings, are they public or private?

What is the development of the housing market in Astana?
is there a growing conflictuality between Astana and the other KZ cities?

As investments were made in Astana, what cities or people were left behind?\marginpar{Investment/disinvestment, inclusion/exclusion}

Astana does not suffer from any space constraints, because it is surrounded by hundreds of kilometers (?) of steppe. 


%%%%%%%%%%%%%%%%%%%%%%%%%%%%%%%%%%%%%%%%%%%%%%%%%%%%
% 																PHOTO ESSAY																
%%%%%%%%%%%%%%%%%%%%%%%%%%%%%%%%%%%%%%%%%%%%%%%%%%%%

\section{Old notes from Photo Essay}

An ex-soviet state, who's president (Nur Sultan Nazerbayev) was in power for over 20 years, since the country's independence
The capital has an interesting history, moving from Almaty (the biggest city) to Nur Sultan
As a country, KZ is geopolitically complicated: it needs to navigate three of the world's most dominant countries, Russia, China, and the USA. It borders China and Russia, and thus needs to keep close and friendly ties with them. At the same time, not become dependent.
The renaming of Astana to Nur Sultan in 2018 (to be verified) is almost too good to help explain the ideology of the government. President Nazerbayev has an airport, a university, and a city named after him, and the museum of kazakh history (verify) focuses heavily on him as a leader, with a x meter tall statue of him at the centre of a museum.
 The fact that he renamed Astana himself, and that he created his own position as "advisor to the president", also illustrates his authority and, without wanting to draw too many parallels, is reminiscent of Stalin as the "father of the nation" or even Kim Jeong Il as "supreme leader". 
 As a Western woman visiting Kazakhstan and post-soviet, Central Asia for the first time, the propaganda felt so obvious.
 Nonetheless, the people of KZ and especially Astana (I assume, because the rest of the country is not as well off, you can even see it just by crossing the river to the south side of Astana) acknowledge the positive changes that Nazerbayev brought to the city: (verify everything here) better education, infrastructure, modernity...
 
Walking through Astana feels like you have landed in a post-modern, perhaps utopian, city, or even in a science fiction. One massive boulevard (include meters here) cuts through and connects the city. Its lined with impressive building after impressive building. I do not know of another place in the world which has such a display of eclectic, futurist architecture.

Even if we ignore the look of the architecture, the sheer scale of buildings is staggering. On the map, nothing looks particularly far apart. But the length and width of the boulevards are astonishing. The size of the mosque is disorienting, dwarfing. As you can see, it is hard to even get a picture. 
This enormous scale makes it hard to reach any destination without a car or a taxi (insert uber equivalent name here), even if taking a car ride at peak hours might seem slower than walking pace. 
For reference, the presidential palace that is modelled after the white house, is eight times bigger than it. What kind of message is this sending? 

One can't help to wonder, what is behind this architecture? What is the purpose of such a display? Especially as you cross the Isim (?) river towards the south side of Astana, where the modernism stops and low rise, dirty white and brown buildings are the majority. The overwhelming architecture is no where to be seen, and you start to get a sense of neighbourhood and community life. Is this disinvestment? Will creative destruction take over?

Astana is surrounded by steppe, and will not run into space constraints any time soon. 

Did the capital move from Almaty to Astana so that the government could build a city from scratch, and impress the rest of the world? Seem like a big player against Russia and China? Is it meant to intimidate and remind them that KZ is a powerful country who will not let other (communist) countries govern them? 


The people of KZ continue to call the city Astana. Even I, as a tourist and someone who has little ties to the country besides a handful of friends, find it odd to call it Nur Sultan. Is this a form of resistance, or a habit that will soon be broken?

\pagebreak

\printbibliography

\end{document}

%%%%% ENVIRONMENTS %%%%%

\if{false}

\subsection{\textit{}}

\begin{outline}
	\1
\end{outline}

\fi