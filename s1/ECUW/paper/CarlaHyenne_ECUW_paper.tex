\documentclass{article}

\usepackage[utf8]{inputenc}
\usepackage[normalem]{ulem}
\usepackage[left=1.25in,right=1.25in,bottom=1.25in]{geometry}
\usepackage{setspace}\doublespacing
\setlength\parindent{0pt}
\setlength{\parskip}{1em}
\setcounter{secnumdepth}{0}
\usepackage{outlines}
\usepackage{graphicx}
\graphicspath{ {imgs} }
\usepackage{hyperref}
\usepackage{tabularx}


\usepackage[
backend=biber,
style=apa,
citestyle=authoryear,
sorting=nyt,
]{biblatex}
\addbibresource{references.bib}

\title{To what extent can (post-)socialism explain the contemporary socio-economic structure of Astana-\textit{cum}-Nur-Sultan?}
\author{Carla Hyenne}
\date{\today}

\begin{document}

\maketitle 

\section{Abstract}

Post-socialism is a concept that characterises ex-Soviet cities after independence from, and the dissolution of the Soviet Union in 1991.
Astana is the capital of Kazakhstan, and was created by authoritarian president Nazarbayev in direct response to Kazakhstan's independence. At the same time, the country was undergoing projects of nation building, state building, and economic recovery. In this sense Astana it was never a city under socialism, but still exhibits features of post-socialist cities. In this paper, we use post-socialist theory and take an empirical approach to understand to what extent Astana's contemporary socio-economic structure can be explained by post-socialism. Notably, using the marketisation of the economy and the stratification of society. 
More than thirty years after Soviet independence, it becomes apparent that cities like Astana warrant a new theory. Do neoliberal western-centric theories apply? We will discuss Astana's position as a global city and whether such mainstream categories are useful. We will conclude that neoliberal globalisation does not fully explain Astana today. The government's political and economic ambitions for the country are elsewhere, and are to be understood as an amalgam of ideologies from the east and the west.

\section{Introduction}

The dissolution of the Soviet union in 1991 left in its wake an enormous project for the ex-Soviet states in East Central Europe and Central Asia. The processes of nation building, state building, and economic reform that ensued greatly defined the countries we know today. It is particularly interesting to analyse Astana in the context of post-socialist cities, because of the way it came to be Kazakhstan's capital. 
Until 1994, the capital was Almaty, the biggest city in Kazakhstan still today. However, it made strategic sense for multiple reasons to move the capital to a more northern and central position. 

First, Almaty's location in the southeast is close to the Chinese border which poses a potential risk. It is also surrounded by mountains which makes it hard to physically expand and accommodate a larger population. Second, Almaty is a well established city, carrying cultural and architectural traditions. This clashed with first president Nazarbayev's intention to build a modern, independent and prosperous image of Kazakhstan. Third, the north of Kazakhstan was predominantly Russian. Locating the capital in the north would give the State a reason to encourage Kazakh migration to the north and promote ethnic integrity. Lastly, the president was looking for a tabula rasa to conduct his post-socialist project. Akmola, an industrial city with 200,000 habitants, in a north central location, with better connections to major Kazakh cities, surrounded by steppe, and on the Ishim river, was chosen as the ideal place to relocate the capital.

Thus, Astana is a capital city built in direct response to Kazakhstan's independence from the Soviet Union. This is why the research question, “to what extent is Astana's contemporary socio-economic structure influenced by post-socialism”, is relevant. This question is approached in two parts. First using post-socialist theory, and second using the opposite of socialism - neoliberalism - to capture how far the city has come from the former.

To find an answer, we will first provide contextual background on the city since its formation in 1994. Then, we will give an overview of literature on post-socialist urban studies, and focus on two socio-economic features of post-socialist cities: the marketisation of the economy and its effects on the social structure and inequalities within Astana. This will lead us to a discussion on whether post-socialist theory is still useful. We will conclude that Astana's contemporary socio-economic structure exhibits characteristics of post-socialism, but does not fit as neatly into this category as it might once have. However, it does not fit `conventional' western theories, either. Finally we will argue that the social, economic and political landscapes have changed and merit a new analysis. Perhaps this exists somewhere between socialism and capitalism.

\section{Creating a Capital City after Socialism}

The rebuilding of Kazakhstan after its independence in 1991 was strongly shaped by the first president, Nursultan Nazarbayev. The collapse of the USSR offered a great opportunity for politicians in the communist party to become the leaders of their independent nations. This resulted in the rise of several authoritarian regimes in Central Asian countries, whereby the leaders exercised their power similarly to that of the authoritarian communist leaders of the 20th century. However, Nazarbayev was widely accepted thanks to his success in meeting the nation's demands \parencite{isaacs2010papa}. Indeed, the post-Soviet economic prosperity and the emergence of a new middle class with disposable income helped legitimise his regime  domestically and internationally, even if prosperity was not uniform across the country. Nazarbayev stood by his motto ``economy first, then politics'', believing that only with a high standard of living would a country be stable enough to adopt democracy \parencite{kassymbekov_2020}.

Central to Kazakhstan's post-socialist success was the creation of a brand new capital, Astana. As a tabula rasa for Nazarbayev's ``pet project'' \parencite{?}, it was branded as the image of a modern, prosperous and independent Kazakhstan. It projected the future of the country's society, politics, and economy. Today, the relocation of the capital is seen as one of the most important events that cemented the post-socialist economic and political reforms \parencite{kassymbekov_2020}.

The left bank of the Ishim river is where the new administrative, monumental and recreational buildings are located. They were planned and developed on mostly virgin land, since Akmola existed on the right bank. In this sense, the urban landscape changed dramatically and quickly, which is unusual for a city. The built environment is normally slow to change, in comparison to political and economic structures which can revolutionise in a matter of days \parencite{stanilov2007post}. It played an important role in cities after socialism, especially because it was used instrumentally as a site for political representation under communism. In the case of Astana, the urban landscape not only changed in parallel with, but created a new society and economy. 

During the socialist era there was a strong segregation between Russians and Kazakhs. The post-socialist nation building project emphasised the importance of social integrity, and displayed the multi-ethnic and multi-religious composition of the country as a strength. Thus, the government had to create opportunities for the populations to mix while promoting Kazakh culture. It was important that the Russian community did not dominate the capital, so that Kazakhstan could be seen as a self-standing nation, not controlled by or relying on the imperial power. 
Astana became the place to promote (modern) Kazakh culture. Campaigns were launched to attract Kazakhs to the capital, the Kazakh language was promoted in schools, and more recently, Kazakh proficiency has become mandatory for government officials. The efforts were successful: between 1991 and 2019, Astana's Kazakh population surged from 44\% to 78.2\% \parencite{unfpa2020wekazakhstan}, a significant majority. 
Internal migration is the primary driver of population growth of the city. Nazarbayev especially served as an inspiration to many, as a self-made man himself. He came from a poor family in a rural town, worked on a steel mill, and paved a successful path for himself in politics by using his working class background and his intellect. All over Kazakhstan, people are attracted by the job opportunities and the promise of a better life that Nazarbayev embodies.

Given its history, Astana proves interesting to study as a post-socialist city precisely because it did not exist under socialism, but was built as a post-socialist achievement.

\section{Post-socialist Theory}

In the 1990s, analyses of ex-Soviet cities emerged to explain the transition from a socialist system towards a new socio-economic system. The challenge for ex-Soviet countries in East Central Europe and Central Asia in the 1990s was to rebuild the nation, the state, and the economy; they focused on processes of democratisation and marketisation \parencite{ferenvcuhova2016introduction}. Conventional urban theories, from the Chicago School, to the global cities paradigm and neo-Weberianism \parencite{haussermann2005european}, do not apply in this context.
On the contrary, it would be surprising to use western-centric theories in the context of Astana after socialism, when the city's economy and society were undergoing completely processes. Cities like Astana warrant a different analytical approach. This critiques `universal' urban theories that are in reality only applicable to cities in the global North \parencite{ferenvcuhova2016accounts} \parencite{robinson2013ordinary}. It also characterises cities on the periphery of the neoliberal, capitalist world system, as ``backwards'' and needing to catch up to western ideals. \textbf{todo: elaborate on this statement}.

Therefore, post-socialism is helpful to understand the processes that states underwent after communism. The analysis is typically done from one of two perspectives: by comparing the city to the socialist ideal \parencite{sailer1999characteristics}, or by following the `conventional' literature and comparing to the capitalist city \parencite{smith1996socialist} \parencite{haussermann1996socialist}. Although the perspectives are different, the outcomes of the analyses are similar. Coming from the second approach, Smith summarises the differences between (post-) socialist and capitalist cities by their ``general physical organisation, socio-economic differentiation and ethnic segregation'' \parencite{smith1996socialist}. Coming from the first approach, Sailer-Fliege develops a comparative model of socialist and post-socialist city ideologies in East Central Europe that overlap with Smith, even though they are coming from different angles. Furthermore, even though Sailer-Fliege developed the model specifically for East Central European cities, the overarching themes are still applicable to Central Asian cities like Astana.

\hfill
\bgroup
\def\arraystretch{1.5}
\begin{table}[h!]
\centering
\begin{tabularx}{\textwidth} {
  | >{\centering\arraybackslash}X 
  | >{\centering\arraybackslash}X |}
	\hline
	\textbf{Smith} & \textbf{Sailer-Fliege} \\ 
	\hline
	General physical organisation & 
	Compact city with slightly less homogeneity; Extensive industrial blight; new industrial areas of urban fringes \\ 
	\hline
	Socio-economic differentiation & 
	Deindustrialisation; take-off of the tertiary sector; increase of foreign direct investment; commodification of the housing sector \\ 
	\hline
	Ethnic segregation & 
	Stratified society according to efficiency; sharp rise in social inequalities; suburbanisation; luxurious housing enclaves \\
	\hline
\end{tabularx}
\caption{Post-socialist city characteristics according to two analyses coming from two different angles. Smith, from a capitalist city perspective, and Sailer-Fliege, from the socialist city ideal. The post-socialist characteristics included are not exhaustive.}
\label{table:1}
\end{table}
\egroup
\hfill

To analyse the extent to which post-socialism played a role in Astana's contemporary socio-economic structure, we will examine two post-socialist ideologies. First, the transition from a centrally planned economy with a priority given to `productive' economic sectors, towards a market economy and the take off of the tertiary sector. Second, the transition from an egalitarian society and territorial equalisation of living conditions, towards a stratified society, with rising social inequalities and growing differences in living conditions.

Afterwards, we will discuss the extent to which post-socialism is useful to understand Astana today. It makes sense for post-socialist urban theories to exist, but post-socialism should not be the end-all theory for ex-Soviet cities. This is especially important because post-socialism represents a transition to a state, somewhere between socialism and capitalism. 
We will employ the popular, western theory of global cities developed by Sassen in the 1990s, and discuss whether Astana today can be better explained with a west European perspective.

\section{Astana's contemporary socio-economic structure}

\subsection{Neoliberalisation of the economy}

The collapse of the Soviet Union created an economic crisis in Kazakhstan. The country had to pivot from a centrally planned, industrial economy trading with the Soviet Bloc, to a marketised economy opening up to the world. Today, the economy of the Akmola region where Astana is located is predominantly based on industry, agriculture and manufacturing which together make up over 50\% of the GRP since 2010. In contrast, Astana's economy is predominantly based on wholesale and retail trade, real estate activity, and professional scientific and technical activities, which together make up over 40\% of its GRP since 2010 (see figure 1).  We also find that employment in Astana is highest in wholesale and retail trade, followed by construction and education (see figure 2). Even though construction is the fourth largest contributor to Astana's GRP (5.4\% on average since 2010), we can still see the disparity between the most lucrative and most employable sectors. The conclusion is that the construction and education sectors pay relatively less than trade and real estate.

<Add figure with x axis: years, y axis: on one side gdp contributions per 5 biggest sectors in million tenge, on the other side employment numbers per 5 largest sectors>

Astana was never an industrial city, although it is embedded in an industrial region and in a country whose economy relies heavily on oil, gas and metal exports. It is in line with the post-socialist transition towards a de-industrialised economy with a growing tertiary sector.
Indeed, Astana was deliberately chosen as the location for new economic sectors and investments in technology, innovation and entrepreneurship as Kazakhstan diversified its economy. Special economic zones were created to attract foreign investors and businesses, especially in the technology industry; the number of small and medium enterprises grew fourfold between 2005 and 2020; the share of the economy (what part of the economy? grp?) that is dedicated to the tertiary sector grew from X\% to X\% since 2010; <private businesses>

To marketise and globalise the economy after socialism, the government took a Eurasian approach and looked for opportunities outside the Soviet Bloc, in Europe and Asia. Today for example, Astana has the Astana International Financial Centre (AFAIC) whose aim is to ``position itself as a global centre for business and finance'' \parencite{aifc}. 
% todo: More about international/global economy, with trade or agreements with EU/Europe. Or, more about what they can export that is not just oil/gas. 

\subsection{Commodification of Housing}

Another important feature of the post-socialist economy in Astana is the real estate market. It emerged when housing changed from a social service to a commodity, and resulted in a significant decrease of public investments in and construction of housing, the decay of the existing (Soviet) housing stock, and an increase in housing prices. 
As the government reduced its role in the provision of housing, it created conditions for private investors and individuals to develop and own homes. The beginning of the 1990s had a significant drop in housing construction compared to Soviet times. But, the neoliberalisation of the housing sector via the introduction of bank loans and mortgages led to a rapid increase in construction in the mid 2000s \parencite{unece2018housing}.

Today, the biggest participants in Kazakhstan's housing market are the real estate developers \parencite{unece2018housing}. According to national statistics, between 2005 and 2018, the volume of services related to real estate in Astana grew from 29.8 to 267.5 million tenge %\parencite{government statistics}
At the same time, the cost of housing tripled between 2001 and 2015 \parencite{seitz2021urbanization} and it has become unfeasible for most households to move to the city despite the job opportunities and higher wages. To put it into perspective, Astana is 240\% more expensive to live in than the national average \parencite{seitz2021urbanization}. The housing affordability crisis is further exacerbated by the virtual inexistence of the rental market: in 2015, over 80\% of the dwellings in Astana were owner-occupied \parencite{seitz2021urbanization}.
Commodification hasn't meant a complete abandonment of housing by the government. In 2016, the government launched the Nurly Zher programme, which aims to facilitate housing construction, provide better access to mortgages and access to housing for vulnerable populations. However, the majority of the low-middle class does not qualify as ``vulnerable'' and thus isn't eligible for social housing, which itself is extremely in Astana. \parencite{unece2018housing}.

\subsection{Social inequalities}

The neoliberalisation of the economy and the welfare state contributed to the socio-economic inequalities in Astana, a trend visible across post-socialist cities. The surge in housing prices can be explained in part by the government projections in the early 2000s that grossly underestimated the population growth. Today, Astana has more than 1 million inhabitants, and was originally projected to have 600,000 by 2030 \parencite{masterplan2001}. Contrary to post-socialist trends, Astana did not experience an urban-to-rural migration, but the opposite. Urban to rural migrations took place in East Central European cities because <todo>. In Astana, the promise of employment and a better life attracted rural Kazakhs to the new capital. However not everyone benefited equally from the structure of the economy.

The influx of construction labour created pressure within the city, and reinforced social and spatial inequalities especially between the highly educated population with high-skilled jobs, and the population with manual and low-skilled jobs.
Despite the government's promotion of ethnic mix in the 1990s, the rural Kazakh population who migrated to Astana for work (predominantly in construction) lives in the city but with a strong ``rural mentality'' \parencite{koch2014bordering}. They have a lower income, live in the right bank and in the remaining Soviet era housing complexes that escaped demolition but are sometimes in need of reconstruction.

On the other side of the river,  you will find glamorous shopping malls like Khan Shatyr, luxurious residential condos, and impressive government buildings like the Presidential Palace. The modern and consumerist amenities were built to cater to the middle class that emerged after socialism, thanks to the new economic structure. To the extreme, marketisation created a capitalist oligarchy whereby the Kazakh elites\footnote{The Kazakh elites are for the most part people in government, or related to the people in government and especially to Nazarbayev} living in Astana profit disproportionately from the wealth created by the oil and gas industries \parencite{gallo2021three}.
$\rightarrow$ Todo: more about the ultra rich in Astana and conclude section.

Nonetheless, Astana's economic success since independence has improved the standards of living and created a more equal society. The poverty gap has been closing over the last decades. Educational attainment is high, and students come from all over the country to attend Nazarbayev University that opened in 2010. The inequality index has been comparable to north European countries in the last few years. Gentrification is virtually inexistent in Astana, but this has to do with its recency more than because of social interventions.

\section{Post-Post-Socialism - a melting pot of east and west?}

% todo: Include limitations of the data

% In conclusion, socio-economic inequalities rose in Astana after socialism and were spatially patterned on either side of the Ishim. The soaring housing prices, the contrast between the large but low-paying construction sector and the lucrative real estate and wholesale and retail trade industries, and the \sout{neoliberalisation of the welfare state} are all characteristic of post-socialism. However the inequalities aren't as pronounced today as they were in the 1990s and early 2000s.

It has been thirty years since the dissolution of the Soviet Union, an era that spanned fifty-five years in the history of a nation that is thousands of years old. Post-socialism is a transitional process characterised by rapid change, and, by definition, assumes there exists a state after which the cities are no longer `post-socialist' per se.  

It follows that post-socialism will need to be phased out. The term is not future oriented, and describes a phenomenon that started decades ago. At some point, all cities will have completed the transition out of Sovietism (if not already), and the outcome of these transitions will look different across all cities and countries. Post-socialist cities will no longer fit into one single category \parencite{hirt2016conceptual}.

What does this post-post-socialist city look like? Just like Astana does not look the same as Sarajevo, there will not be a single theory to categorise all cities after post-socialism. If capitalism is the antipode of socialism and the communist regime, it can be used as a benchmark to explain how far cities have come from socialism. Global city theory lends itself readily to this analysis, since it focuses on the level to which cities are inscribed in today's global, capitalist economic network.

On the surface, Astana looks like an aspiring global city and is positioning itself strategically on a global scale. The city is becoming an autonomous force in Kazakhstan and internationally, and 2018 marked the first year it was listed as a global city on the GaWC (Globalisation and World Cities) list, if only as ``sufficient'' \parencite{gawc2018list}. But, if neoliberal capitalism were the goal for Astana, we argue it would be more obvious today in its economy, politics and society. There are multiple motivations why the government is not adopting the `global city' strategy.

First of all, Astana is a city on the periphery of the neoliberal `core', geographically, economically and symbolically.  Geographically, Astana is isolated from the world. Even though it is the most connected city in Kazakhstan by air and land, the potential to reach countries outside of Central Asia and Russia is limited.
Economically, Astana does not have the same resources as other nations do. Up to now, Kazakhstan has profited from its wealth of oil, gas and metal, but it is vulnerable to falling commodity prices \parencite{batsaikhan2017central}. Natural resources are precisely what has driven foreign investment into the country, but this industry is non-existent in Astana. The capital cannot compete with global cities in terms of innovation and technology, at least for now. This is important because what matters in the global city network is increasingly less the geographical connection, but the exchange of knowledge, ideas and capital. Astana's global potential in this sector is limited for now.
Symbolically, Astana is closer to Russia than the West. Its regime, culture and population has significant Russian influence. For example, the restriction and censorship of the internet, or local blackouts during protests and elections are reminiscent of recent Russian practices \parencite{freedomhouse2021}.

Moreover, the political landscape is not democratised like in most global cities. Nazarbayev ruled as an authoritarian leader, elections are not free by Western standards, Nazarbayev's cult of personality is visible all over Astana, and there is considerable corruption in the government. This is not to say that a city must be democratic in order to be a global city, but that the government's aspirations for Astana are not easily explained with neoliberal and globalisation discourses. Rather, the government promotes nationalist tendencies and rejects the interference of the West in politics and the economy \parencite{koch2013not}. Nazarbayev stepped down in 2018, but so far president Tokayev is going in the same direction.
Continuing on the political, the economic motivations of the government and the elites are vastly different than that of neoliberal states. Whereas neoliberal governments facilitate capitalism through laws and regulation, to encourage wealth accumulation of private companies so that they may bring prosperity to the city, the Kazakh government's participation in the neoliberal world economy is motivated by the advantages it can bring to the (political) elites.

Thus, it is clear that Astana is not driven purely by neoliberal western ideologies. Astana, and Kazakhstan as a whole, are a melting pot of east and west, reflecting their landlocked and nestled position between Europe, Russia and China, three global superpowers with vastly different socio-politico-economic <landscapes>. Post-socialism cannot explain this position alone, but neither does neoliberal western theory. The correct analysis is tending towards an amalgam of ideologies from the east and west, and distinct to Central Asia.

\section{Conclusion}

In conclusion, Astana is undeniably a post-socialist city. Imagined and built from the ground up directly following Kazakhstan's independence from the Soviet union, Astana exhibits many characteristics familiar to post-socialist cities in Europe. We have seen the marketisation of its economy, the growth of the tertiary sector, the increase in foreign direct investments, and the commodification of housing; but also the rise of socio-economic inequalities. Nazarbayev's presidency and the legacy of the Soviet union can still be experienced in Astana, through its architecture, its economy, its people, and its politics.

\sout{As the capital, it makes sense that Astana is put forth by the government as the face of modernity, progress, and wealth. It achieves this through its architecture, but also by branding itself as a city that attracts business, investments, financial institutions, skilled labour, and overall a cosmopolitan population.}

At times, it seems odd to apply post-socialist theory to a city that never existed under socialism. Plus, what is taking place today can no longer be explained purely with post-socialism. On the other hand, the mainstream theories for capitalist cities are not appropriate for Astana, either.
In the end, post-socialist urban theory can help us understand the forces that shaped the capital since the 1990s, even if the city begs for a new analysis and classification that neo-Weberian theories do not provide.
And perhaps, without a Soviet past, Astana would still be called Astana, and not Nur-Sultan - although the name still lingers, either out of habit or resistance.

\pagebreak

\printbibliography

\end{document}


