\documentclass{article}

\usepackage[utf8]{inputenc}
\usepackage[left=1.5in,right=1.5in,bottom=1in]{geometry}
\setlength\parindent{0pt}
\setlength{\parskip}{1em}
\setcounter{secnumdepth}{0}
\usepackage{outlines}
\usepackage{graphicx}
\graphicspath{ {imgs} }
\usepackage{hyperref}
\usepackage{amsmath}

\title{European Cities in an Urbanising World}
\author{Carla Hyenne }

\begin{document}

\maketitle

\tableofcontents

\pagebreak

%%%%%%%%%%%%%%%%%%%%%%%%%%%%%%%%%%%%%%%%%%%%%%%%%%%%%%%%%%%%%%%%
%																					LECTURE 1
%%%%%%%%%%%%%%%%%%%%%%%%%%%%%%%%%%%%%%%%%%%%%%%%%%%%%%%%%%%%%%%%

\section{Introduction: European Cities in an Urbanising World}
\textit{September 27th 2021, Bas Van Heur}

\begin{outline}
	\1 Theoretical and empirical introduction to the course
	\1 Discussion on globalisation debate: myths on globalisation, waves of globalisation, the problem of periodisation, the relation between globalisation and regionalisation/localisation
	\1 The position of `Europe' in the world
		\2 The changing boundaries of what and where Europe is considered to be
		\2 The imperial/colonial linkages of European countries and cities
		\2 Cities as nodes in networks of cultural and intellectual exchange
		\2 Migration movements from/to/within Europe and the position of cities
		\2 The rescaling of national state space and the relation between Europeanisation and globalisation
\end{outline}

\subsection{Periodising Globalisation}

The working definition of globalisation is:

\begin{outline}
	\1 \textit{``Globalisation is the extension, acceleration and intensification of consequential worldwide interconnections''} (Sparke, 2013, p.13). 
	\1 Places across the globe are increasingly interconnected; social relations and economic transactions increasingly occur at the intercontinental scale; the globe itself comes to be a recognisable geographic entity (Cloke et. al, 2005, p.36)
\end{outline}

This means there is an increase of interconnection of the global, through processes such as intercontinental economic and financial transactions. \textit{\textbf{Increasing}} assumes that there is a `before', but that is usually implicit. However, globalisation needs to be \textbf{situated} in time. 

We now visualise the world as one global entity, especially in ecology. Harvey (1989) describes the speeding up of globalisation, that the world always becomes faster. From horse-drawn carriages, to steam locomotive, to propeller aircraft, to jet passenger aircraft, these modes of transport have made the world a smaller place. Transport connections/network maps show that its becoming easier to connect cities to each other, rather than to the periphery. 

\subsubsection{Presentism and futurism}

Two key problems in periodising globalisation are presentism and futurism.

\begin{outline}
	\1 Globalisation is a term that is marked by its time. It emerged in the 1970s and the era of globalisation tends to be limited to post-1970s. It has a Western perspective limited to the 1990s era of Western sensibilities
	\1 This implies a \textbf{newness} and not a \textbf{continuity}: there is little sense of dialectic globalisation and de-globalisation. If something globalises, does something else de-globalise?
	\1 There is a tendency to focus on the condition, not the process and projects of globalisation: ``globalisation did this''. Such statements are impossible because globalisation is a concept. Who are the actual actors?
\end{outline}

Presentism and futurism are Western styles of thinking. 

Similarly to globalisation, the Enlightenment was not condition that could be achieved, but a process. It was a normative-societal goal. The obstacles to achieving Enlightenment are immaturity (laziness, cowardice) and the lack of freedom (people must use reason in public) (Immanuel Kant). For globalisation, immaturity means that people are not good entrepreneurs, and lack of freedom means the need to break down institutional barriers. 

Modernisation is the global ``world-making process'' of the post WWII era. It is a normative-societal goal, all is moving towards it, and little attention is paid to contradictions and tensions. 

\subsubsection{Eurocentrism}

Eurocentrism is the idea that Europe's history is the `sovereign' of all histories, that `other' histories are based on one master narrative, and that all narratives about cities are related to Europe\footnote{See `Provincialising Europe' by Dipesh Chakrabarty}.

The definition of globalisation depend on where you situate yourself in history, and also in what discipline. Definitions don't need to agree, but they \textbf{need to be explicit as to where and when they are situated}.

\subsubsection{Periodisations}

Examples of periodisations:

\begin{outline}
	\1 1820s: convergence of commodity prices across continents. The economic explanation is that local economies integrated into global systems, and the prices of goods converged. It overlooks state interventions, and is Western and colonial centric, exposing the inequality of globalisation
	\1 1571: founding of Manila as Spanish trading post. It linked trans-Atlantic and trans-Pacific trade. It was a colonial city, and decentralised Europe in the global narrative
	\1 13th century: the world system before European hegemony. There were 8 circuits in the world, that traded most commonly with each other. Not European centric, but more Asia-centric that Europe inherited
\end{outline}

\subsection{Where is Europe?}

Positioning Europe is a \textbf{metageography}, we feel the need as a society to partition the world into separate areas.
The need to position Europe is a product of European expansionism and the growth of long-distance travel and trade. We conceive the world as one, but still divide it into parts.
Overtime, new modes of categorising and labelling continents, macro-regions, areas, have emerged, and the \textbf{codification} of Europe has increased:

\begin{outline}
	\1 19th century nationalism: comparative methods, for eg. Kulturvergleich that compares distinct, territorially fixed civilisations
	\1 20th century area studies: shaped by the geopolitics of the Cold War and development studies such as modernisation
	\1 There is always a tendency to position Europe as the norm
\end{outline}

In naming large areas, there is a fair degree of uncertainty, vagueness, and politically strategic flexibility. The definition of Europe remains fluid. 

Today, Europe is represented as an integration. Historically, in the 19th-20th century various versions of `Europe' emerged. There was limited internal homogeneity regarding the versions, due to different geopolitical interests and different empires. External boundaries of Europe were fluid and/or contested.

\subsubsection{Empires and Colonies} 

Colonies had very different geographies as the colonisers, and we can think about how a city is organised based on this knowledge. This means that two neighbouring countries could have very different geographies and relationships, depending on if and by who they were colonised.

Europe could be located in the empires and their colonies, reaching far beyond the geographical `boundaries' of the land:

\begin{outline}
	\1 North Africa and the Middle East were historically part of Europe, and only in the last century did they become separate
	\1 In Russia, actors try to tie or separate it from Europe, in favour of Asia, or vice versa. It's difficult to set Russia's European boundary, because there is not a clear mark, it is a single land mass, and drawing the line is hard. Thus, creating a separate European identity in Russia is not easy.
	\1 The EU is now seen as Europe. This ignores the ``tentacles of the EU'', because there are European territories overseas that map the history of colonisation.
\end{outline}


\subsection{Different cities, different globalisations}

Cities can become strong both because of their strong and weak states. It can be estimated that if a city is big, its economy is strong; and if a city is growing, its economy is moving.

Populations have changed over the last five centuries:
\marginpar{What political, social, economic histories is your city built on?}

\begin{outline}
	\1 Up to 1500s: Mediterranean centric, North Africa and Middle East cities have the largest populations
	\1 1500-1600s:  European world-city network, hierarchical, European centric, cross-Atlantic with Potosi (Bolivia)
	\1 1500-1700s: Dutch hegemonic cycle, migration dynamics, shifting geographies, the population growth is an economic growth $\rightarrow$ a growing city is propsperous
	\1 1700-1900s: British hegemonic cycle, appearance of north and south American cities in 19th century
\end{outline}

\subsubsection{Port cities}\marginpar{Can you understand your city with these models?}

Port cities are key infrastructural hubs of globalisation. The 19th century was the golden age of port cities, due to the strong expansion of world trade. This was complemented by railroad junctions in the late 19th and early 20th century, and airports in the late 20th century.

The consequences of port cities are:

\begin{outline}
	\1 The growth of ports shapes the urban built environment
	\1 There are \textbf{diverse and flexible labour markets}
		\2 The labour force was almost exclusively male except during war
		\2 Had seasonal fluctuations, with most working on land during the summer
		\2 There was a difference of ethnicities within the work force. Liverpool had a strong influx of Irish workers, whereas Odessa, Ukraine, recruited Jews, Swiss, Germans, Greeks
	\1 There is a highly diverse social composition of the population
	\1 There is a \textbf{polarised class structure}, with port cities governed by a small regime of shipowners, bankers and merchants
\end{outline}

\subsubsection{Colonial cities}

Colonial cities are outside of Europe, but strongly connected to it through trade, military expeditions, migration networks, and knowledge exchange. They were \textbf{ruled from abroad} and the indigenous population was excluded from the governance. They acted as an interface between the imperial core, and the exploitation of its hinterland.

European architecture and planning styles were introduced. Cultural, political, economic characteristics influenced or were imposed on the colonies. 
Thus, it's possible to consider colonies as a part of Europe.

\subsubsection{Imperial cities and the post-colonial}

Imperial cities are the core, but also the counterpart, to colonial cities: strong, though unequal interactions shapes both cities.\marginpar{Think of the unequal interactions of post-soviet cities}
In general terms, wealth and growth of imperial cities depended, to a greater or lesser extent, on the exploitation of its colonial hinterlands.

Colonialism is partially visible in urban landscapes, through colonial monuments like statues, and buildings. For example, the statue of Leopold II in Brussels stands as a tribute, but the fact that he was a coloniser is sidelined.

There is an increasing contestation and criticism of imperialism at the heart of imperial cities. The `colonized' are present in the imperial core, which is what we call post-colonial urbanism. For example, the Patrice Lumumba Square was inaugurated in Brussels in 2020 to honour the first Congolese president.

\subsection{Tracing connections}\marginpar{Tracing connections makes for a better paper}

\begin{outline}
	\1 Historical connections: location matters. Different European cities are part of different histories of globlisation
	\1 Spatial connections: partially a result of these histories, different cities show different geographies of globalisation
	\1 Europe is in the world, the world is in Europe. Trace `global' spatial-historical connections within and beyond Europe as a territorially fixed entity. For example, migrants have a history external to the city, but that history is an internal part to the city
	\1 Approach globalisation not as unidirectional, but as a contested process of globalisation and de-globalisation (expansion and contraction, Osterhammel, 2011)
	\1 Focus the attention on the (often contradictory and multilayered) project of `making' European cities, not on cities as simple outcomes of global social change (Tsing 2000)
\end{outline}

\subsection{Keywords}

Globalisation
Presentism and futurism
Normative-societal goal
Eurocentrism
Periodisation

%%%%%%%%%%%%%%%%%%%%%%%%%%%%%%%%%%%%%%%%%%%%%%%%%%%%%%%%%%%%%%%%
%																					LECTURE 2
%%%%%%%%%%%%%%%%%%%%%%%%%%%%%%%%%%%%%%%%%%%%%%%%%%%%%%%%%%%%%%%%

\section{From the `European city' to comparative urbanism}
\textit{November 15th 2021, Bas van Heur}

\subsection{Max Weber - modernity and intellectual debate in early 20th century}

Max Weber, 1864-1920, was the most influential sociologist of the 20th century.
He was a \textbf{key theorist of modernisation}:

\begin{outline}
	\1 Rationalisation: historical drive towards mastering things by calculation
	\1 Zweckrationalität at the expense of Wertrationalität
	\1 Institutional rationalisation: ordered capitalism, bureaucracy, iron cage
	\1 Modernisation threatens individual agency and freedom
\end{outline}

\textbf{His methodology}
\begin{outline}
	\1 Ideal type: one-sided accentuation of empirical features into a unified analytical construct. The world is messy, and we extract elements that we deem most important for analysis (this is not the `ideal' we should all achieve)
	\1 Comparative research: ideal types enable limited generalisation about socio-spatial diversity and change
\end{outline}

\textbf{\textit{The City}}, written 1911-1914, focuses on the medieval European city and its key characteristics:
\begin{outline}
	\1 Market centre: centrality effect of local market; an exchange of goods, based on specialised productions, ie. liberal economics in the spatial sense
	\1 Stadtluft macht frei (urban vs. non-urban): greater level of freedom in the city compared to the country
	\1 Free association of `burghers': people freely organise themselves into communities, and apply rules to themselves; leads towards urban autonomy in politics, law and economy
	\1 Free but not equal: there are people with more power (upper class, military); struggle between classes only in urban life; decline of urban autonomy due to rise of modern bureaucratic state
\end{outline}

\subsection{Comparative from the start: the European city vs. the oriental city vs. the American city}

Weber is criticised to be an Orientalist, and euro-centric.
But his aim is not so much orientalist, as comparative: the ideal-typical identification of elements specific to the European city.
He increasingly compared European cities to American cities, and reflects the rise of USA as a hegemonic global power. 

\subsubsection{European cities vs. American cities}

\textbf{Convergence?} European cities will become more like American cities. Opinions can be fear: Americanisation and globalisation will lead to a loss of European urban qualities; or fascination, that American cities are the future.

\textbf{Divergence?} Or parallel development, and trust that the European city has its own trajectory because of their history, path dependency, international relations.

\subsection{The European city as an antidote to globalisation? Neo-Weberian approaches and urban design ideals}

Neo-weberian approaches. The European city as a moderate(d) version of the economic globalisation?

Häussermann and Haila (2005):\marginpar{The features can be found in post-socialist lit., can discuss the overlap?}
\begin{outline}
	\1 Public land ownership
	\1 Public services/municipal socialism: public services matter, and public money should go in to the development of local administrations that exist to serve the public
	\1 Planning instruments and administration: the city is shaped by these mechanisms, and is not a free market
	\1 National welfare states: this is new in the city (not in the medieval city), looks at how the city is positioned within their wider national context
	\1 `Burghers' and visions of the good city: citizens actively contributing and having opinions on what a good urban life is
\end{outline}

Siebel (2004)
\begin{outline}
	\1 Birthplace of the `burgher'
	\1 Place of emancipation: Stadtluft macht frei, people come to the city to be free individuals
	\1 Urban lifestyle (public/market) vs. rural lifestyle
	\1 Morphology: urban density and centrality
	\1 Socially regulated city, including economic regulation: the market is regulated, according to institutional, bureaucratic, planning rules
\end{outline}

Bagnasco and Le Galès (2000)
\begin{outline}
	\1 Urban system of small and medium-sized cities: the geography is different compared to American cities, which are made up of few large cities
	\1 Low levels of population mobility: low international migration, people tend to stay much closer to their own region compared to American places
	\1 Important role of central government: the city cannot be understood without understanding its embededness in its national state, and within the levels of multi-layered governance\marginpar{Astana?}
	\1 Middle and upper classes live in the city centres
	\1 Public services and planning administration
	\1 Cities as incomplete localised societies
	\1 Loosening of national state scale in EU context increases urban autonomy: cities have bypassed national, and directly gone to EU level if there was a blocker, and vice-versa
\end{outline}

What is missing from these perspective?

\begin{outline}
	\1 Does this apply to all European cities?
	\1 Social class element: what about non-bourgeois visions of the good city? 
	\1 Is the urban vs. rural distinction still relevant?
	\1 In what ways are the cities as localised societies `internally' unequal?
	\1 What is the real space for urban autonomy in relation to the national state and the EU?
	\1 How does the bourgeois, religious... bias in the (neo-) Weberian depiction of the urban citizenry effect in/exclusion of the other actors
\end{outline}

The European city is increasingly embraced as an urban design ideal. The design (and middle class promotion) ``often serves to ignore the reality of cities as places of inequality, tension and political difference'' (Lawton and Punch, 2014). 

Reconstructionist approaches:
\begin{outline}
	\1 Critique of modernist architecture and planning
	\1 Focus on the historical form of the city
	\1 Urban memory
	\1 Mixed use and dense urban spaces instead of functional zoning
	\1 Respect for the layout of the 19th century, with focus on individual blocks and streets
\end{outline}
It critiques Le Corbusier uniform architecture. 

Not all cities in Europe as usefully understood as `European cities'. Recap of the key elements of the `European city'

\begin{outline}
	\1 Middle and upper class live in the city centre 
	\1 Compact city: urban-rural distinction, built=up form around a focal point like church, marketplace, city hall
	\1 Public transport, public space
	\1 Strong planning administration
	\1 Small and medium-sized cities
	\1 Urban autonomy (within or without strong national welfare state)
\end{outline}

\subsection{Deconstructing the European city from within: socialist cities, industrial cities, Mediterranean cities}

\subsubsection{Socialist city}

\begin{outline}
	\1  Cities in East and Central Europe: established part of the European urban system, but industrialisation started later
	\1 Socialist period (1945-1989) was mostly built on existing pre-socialist urban structures
	\1 Limited urban autonomy: primacy of national planning and state companies
	\1 Informal local coalitions between state and companies, local state officials, and representatives of state housing associations
	\1 City center as a site for political representation
	\1 Under-urbanisation: investment priorities of the state focused on industry, less on urban infrastructure; housing shortage in urban regions necessitated long commutes
	\1 Housing and socio-spatial segregation: neglect of bourgeois heritage and housing in city centre, modernist belief in industrial/mass building; focus on panel apartment buildings, mostly city outskirts; official goal of social equality vs. reality of socio-spatial segregation due to practice of housing allocation
\end{outline}

\subsubsection{New industrial cities}

\begin{outline}
	\1 New referring to cities that only emerged during the industrial revolution
		\2 Limited centrality, no focal point (society around factories, coal mines, steel mills), sprawl not compactness
		\2 Polycentric development
		\2 Upper classes often outside of the city
	\1
\end{outline}

\subsubsection{}
\subsubsection{}

\subsection{Exporting a model: European cities across the world}


\subsection{From the `European city' to comparative urbanism}



%%%%%%%%%%%%%%%%%%%%%%%%%%%%%%%%%%%%%%%%%%%%%%%%%%%%%%%%%%%%%%%%
%																					LECTURE 3
%%%%%%%%%%%%%%%%%%%%%%%%%%%%%%%%%%%%%%%%%%%%%%%%%%%%%%%%%%%%%%%%

\section{Reading the diversity of economic structures in European cities and their dynamics (part 1)}
\date{October 18th, 2021} - \textit{Gilles Van Hamme}

\subsection{The economic space of Europe is unequal}

\begin{outline}
	\1 There is a concentration of production and affluence in the `core' of Europe, in contrast to the peripheries where densities are lower and wealth more limited. This part of Europe is described as: the core (or Centre), the ridge (``dorsale''), or the Blue Banana
	\1 This opposition is also visible in structural differences between core and peripheries: the core concentrates strategic services and knowledge- or technology-intensive production sectors
	\1 Starting at a low level in around 1800, \textbf{inequalities} explode in the 19th century, stabilises the first half of the 20th century, severley diminishes after
\end{outline}

\subsubsection{GDP}

Why is there a huge concentration of production and richness within Europe?

\textbf{GDP distribution} measures the volume of production in a given area. The dorsal axis goes from Northern England, through Benelux, West Germany, Eastern France, to Central Italy. It has a concentration of wealth, ie. GDP is high.
The concentration is in cities, mostly capital cities. Some cities outside of the dorsal axis have a high GDP: cities in east Europe, north Europe, and Spain.

\subsubsection{The geography of economic sectors}

Specialisation is the measure of added value per sector, over the total added value.

\textbf{Agriculture}: there is a general distribution of agriculture and construction within Europe, so why are there more performant regions? Because the type of agriculture is not the same everywhere. Regions have different agricultural specialisations, and some products are in higher demand than others; this has nothing to do with the quality of the produce, rather its utility.

\textbf{Machinery} is an indicator of technological skills, it requires engineering and specialised workers. Machinery is the machines to produce consumable goods (eg. cars, appliances), and not the goods themselves. It's developed in the centre of Europe, but not in Belgium/England/France, who are instead specialised in high level service industries.

\textbf{Finance} is an indicator of high level services, and financial services are present specifically in the dorsal axis, but also in cities outside the core like Madrid and Warsaw (considered second rate sectors).

\subsubsection{GDP inequality between countries}

Since the 1950s, inequality between European countries has diminished, whereas globally, inequality has rise. There was an increase in inequality in Europe in the 1970s to 1990s, which is the Eastern European trend. The economy in eastern countries under communism collapsed, and those who joined the EU recovered (Poland, Czechia). Other countries who did not (Ukraine, Russia) had a longer and harder recovery.

\subsection{Measuring economic inequalities across the European space}

GDP is a central concept to measure economic development and structure. 
Three GDP indicators are used:

\begin{outline}
	\1 GDP: economic volume of a city
	\1 GDP per head: wealth of the inhabitants of the city
	\1 Economic structure - share of the different sectors: GDP per sector / total GDP
\end{outline}

However, there are issues about the relevance, especially when measuring regional or urban economies.

\subsubsection{What is GDP?}

GDP is added value, ie. the difference between what you sell and what you buy, the sum of all added values by all economic actors in a given space. 
Added value is created by labour, and is not profit. Under capitalism, added value is distributed by paying salaries.

\begin{center}
\includegraphics[width=30em]{added_value}
\end{center}

\subsubsection{Limits of GDP}

First, added value is an imperfect indicator and doesn't include the whole economy. By economy, where understand any work that satisfies needs. For example,
\begin{outline}
	\1  ``black'' and ``informal'' economies are not included because they are unknown in data, despite efforts, eg. for drugs and prostitution
	\1 non-monetary relations are not included, eg. the domestic economy (cooking, chores, kids) represents a significant amount of working hours. Some of the domestic economy is monetised, but unequally (partially recognised in Belgium, but no Senegal)
\end{outline}

Second, \textbf{GDP per head is an average}, which hides more or less large social differences\footnote{In 2010, Poland and Brazil had the same GDP, but Brazil has more inequality than Poland (rich vs. poor distribution)}. This is more true in large cities, since they tend to be socially dual: the higher and lower classes are overrepresented. 
This interrogates the meaning of economic growth: in the US, there has been a stagnation and even decline of the poor's income, yet GDP is growing, implying a concentration of wealth with the rich. 

Third, \textbf{GDP per head doesn't reflect social wellbeing}. A higher GDP per head generally indicates higher education, longer life expectancy, thus higher social wellbeing. But beyond a certain level of GDP per head in developed countries, there is not statistical link between GDP per head and social indicators. 
For example, the US and Japan: US has a higher GDP per head than Japan, but has a lower life expectancy $\rightarrow$ there is a tipping point of wealth.

\subsubsection{Comparing GDP?}

To ensure geographical comparability of GDP,

\begin{outline}
	\1 Use GDP in PPP rather than current prices in €, to take into account the price differences (money power is not the same everywhere, even inter-regionally). But, there is not regional PPP in Europe, meaning that regions with higher prices like metropolitan areas are over-evaluated
	\1 Issue of scale: most statistics are produced in an arbitrary scale (NUTS classification). We must use coherent and comparable delimitations of urban areas, for example, \textbf{functional urban areas} that include municipalities sending 15\%+ of their active population to the core city
	
\end{outline}

\subsubsection{NUTS classification}

Hierarchical classification of regions, that enables cross-border statistical analysis within the EU. 

The way the city is delimited determines that calculation. For example, many people working in Brussels, but living outside, so Brussels is disproportionally wealthy, because the statistics don't include the regions where workers are located. This is the reason why Brussels is richer than Paris in added value. 

Therefore, the administrative level at which statistics are calculated needs to be correct, in order to make accurate comparisons. 
FUA, \textbf{functional urban areas}, are approximated using NUTS. 

\subsection{Inequalities in the European space: explanations}

\begin{outline}
	\1
\end{outline}


\subsubsection{Core-periphery model}


\begin{outline}
	\1
\end{outline}

\subsubsection{French regulation school}


\begin{outline}
	\1
\end{outline}

\subsubsection{Path dependence}


\begin{outline}
	\1
\end{outline}

\subsection{Preliminary conclusions}




%%%%%%%%%%%%%%%%%%%%%%%%%%%%%%%%%%%%%%%%%%%%%%%%%%%%%%%%%%%%%%%%
%													LECTURE 4
%%%%%%%%%%%%%%%%%%%%%%%%%%%%%%%%%%%%%%%%%%%%%%%%%%%%%%%%%%%%%%%%

\section{Reading the diversity of economic structures in European cities and their dynamics (part 2)}
\date{October 25th, 2021}

%%%%%%%%%%%%%%%%%%%%%%%%%%%%%%%%%%%%%%%%%%%%%%%%%%%%%%%%%%%%%%%%
%													LECTURE 5
%%%%%%%%%%%%%%%%%%%%%%%%%%%%%%%%%%%%%%%%%%%%%%%%%%%%%%%%%%%%%%%%

\section{Comparing European spatial planning systems and cultures}
\date{November 8th, 2021}


%%%%%%%%%%%%%%%%%%%%%%%%%%%%%%%%%%%%%%%%%%%%%%%%%%%%%%%%%%%%%%%%
%													LECTURE 6
%%%%%%%%%%%%%%%%%%%%%%%%%%%%%%%%%%%%%%%%%%%%%%%%%%%%%%%%%%%%%%%%

\section{The European Union and the Europeanisation of spatial planning and territorial development policies}
\date{November 9th, 2021}

%%%%%%%%%%%%%%%%%%%%%%%%%%%%%%%%%%%%%%%%%%%%%%%%
%													LECTURE 7
%%%%%%%%%%%%%%%%%%%%%%%%%%%%%%%%%%%%%%%%%%%%%%%%

\section{Worlds in motion: Migration and (im)mobility in a globalising Europe}
\date{November 22nd, 2021}

%%%%%%%%%%%%%%%%%%%%%%%%%%%%%%%%%%%%%%%%%%%%%%%%%%%%%%%%%%%%%%%%
%													LECTURE 8
%%%%%%%%%%%%%%%%%%%%%%%%%%%%%%%%%%%%%%%%%%%%%%%%%%%%%%%%%%%%%%%%

\section{Contextualising contemporary (post-)migration: Geographies and populations of European cities}
\date{November 29th, 2021}

%%%%%%%%%%%%%%%%%%%%%%%%%%%%%%%%%%%%%%%%%%%%%%%%%%%%%%%%%%%%%%%%
%													SEMINAR 1
%%%%%%%%%%%%%%%%%%%%%%%%%%%%%%%%%%%%%%%%%%%%%%%%%%%%%%%%%%%%%%%%

\section{Seminar: the course assignment}
\date{October 11th, 2021}

%%%%%%%%%%%%%%%%%%%%%%%%%%%%%%%%%%%%%%%%%%%%%%%%%%%%%%%%%%%%%%%%
%													SEMINAR 2
%%%%%%%%%%%%%%%%%%%%%%%%%%%%%%%%%%%%%%%%%%%%%%%%%%%%%%%%%%%%%%%%

\section{Seminar: Data and Research strategies}
\date{October 26th, 2021}



%%%%%%%%%%%%%%%%%%%%%%%%%%%%%%%%%%%%%%%%%%%%%%%%%%%%%%%%%%%%%%%%
%													SEMINAR 3
%%%%%%%%%%%%%%%%%%%%%%%%%%%%%%%%%%%%%%%%%%%%%%%%%%%%%%%%%%%%%%%%

\section{Seminar: feedback on draft paper}
\date{December 13th, 2021}

%%%%%%%%%%%%%%%%%%%%%%%%%%%%%%%%%%%%%%%%%%%%%%%%%%%%%%%%%%%%%%%%
%													READINGS
%%%%%%%%%%%%%%%%%%%%%%%%%%%%%%%%%%%%%%%%%%%%%%%%%%%%%%%%%%%%%%%%

\section{Readings}

\subsection{Introduction: European Cities in an Urbanising World}

\subsubsection{Tsing, 2000, \textit{The Global Situation}}

\begin{outline}
	\1 Discusses ``what is globalisation'', what is, is there, a globalisation era?
	\1 \textbf{Futurism}: the idea that the future will bring progress and prosperity, that globalisation is modern rather than ancient, and attributes stereotypes to the past
	\1 \textbf{Conflations}: merging of ideas/opinions/texts. Globalisation cannot be understood under a single ideological system. It is best understood by the overlaps of systems. Networks serve to connect people/ideas/movements world wide, and strengthens them. But, networks can be exclusionary
	\1 \textbf{Circulation}
	\1 Cultural anthropology is how people who share a common cultural system organise and shape their social and physical world
	\1 In anthropology, globalism is the intensification of global interconnectedness
	\1 Flow is the cultural exchange between cultures/nations, and Imagined landscapes are created by globalisation. These are too much emphasised in anthropology, rather, we should look at scale making of the global and regional
\end{outline}

\subsubsection{Osterhammel, 2011, \textit{Globalisations}}

\begin{outline}
	\1 A historian's look on a social science
	\1 \textbf{Modernity} is the development of capitalism, industrialisation, the establishment of nation states, countries, regions
	\1 \textbf{Globalisation} is the the interrelationships among countries, and how these countries `merge'
	\1 The problem with globalisation is that there are many definitions, all of them viable. Thus we should not define it but use it as a framework that contains models of change, in large spatial contexts
	\1 Discusses globalisation vs. global history, and concludes that global history encompasses globalisation
\end{outline}

\subsection{From the `European city' to comparative urbanism}

\subsubsection{Häussermann, Haila \textit{The European city: a conceptual framework and normative project}}

\textit{tldr; neo-Weberian framework for European cities; explains different schools of thoughts in last decades around cities, their definition and role in the state and the world; compares European (Occidental, post socialist) strategies with American and Asian (Oriental) political, economic, cultural strategies in the city;}

\begin{outline}
	\1 Simmel: interested in what makes ``the urban culture of big cities distinctive''; cities represent modernity because ``they are dominated by the money economy and impersonal social relations''; competition, density and heterogeneity in cities lead to economic specialisation, division of labour, and cultural diversity;
	\1 Chicago school: sees individualisation ``as endangering social integration, and not as a form of emancipation as Simmel had perceived it''; made general assumptions about cities, and assumed cities were the same across the world, which encouraged empirical studies and disproved the Chicago School's theories; main issue was that they did not take in to account politics;
	\1 Political economy perspective: Castells and Harvey ``defined cities as units of collective consumption and analyzed them in the framework of investment flows and class struggle''; used Marxist theories which were considered only applicable to capitalist cities, thus, socialist cities were not analysed with these concepts\marginpar{Can Astana be analysed with Marxist/political economic theories?}; more realistic theories than Chicago School, took into account cultural and political differences between cities, using capitalism to explain the pattern and inequalities in cities;
	\1 Global cities: refers to specific economic functions of the economy; global cities have more producer services, transnational company headquarters, and financial institutions; the social structure of global cities is shaped by these economic factors; they are spatially fragmented, segmented, and polarised;\marginpar{Is Astana a global city, or do global city attributes describe Astana in relation to other Kazakh cities?}
		\2 ``(Saski Sassen) saw global cities as dual cities: on the one hand there is a small world of high-salary elite workers active in transnational transactions, and on the other hand there are growing numbers of poor and relatively low-paid workers who produce services for the new transnational elite''\marginpar{To what extent has Astana become a (post-soviet) global city?}
	\1 There is an inclination to find a general, universal definition of cities, and ignore local cultural and political differences
	\1 Weber and the Occidental City: Weber was not interested in cities in the spatial sense, but rather, in the ``consequences of a distinct social and political institution''
		\2 A new evaluation of Weber: European cities are characterised by ``the participation of burghers in local government, rules applying to landed property, the legal status of citizens, and citizen's associations with relative freedom''; (European) nation-states are less relevant due to globalisation, there was room for ``cities to become political and economic actors'' with their own identity\footnote{Nation states are becoming less relevant in a globalised world, and cities, especially global cities (ie. the cities with the greatest political and economic power) are increasingly greater players on the global stage than the nation. If the idea for the capital of Kazakhstan was to make it a global city, who's economic, political and social structure would represent Kazakh modernity in a post-soviet era, maybe Astana was chosen as a blank slate so that the country could move away from Almaty's traditional and historical structures, and become instead a test bed for a modern, capitalist (western?) society.}
	\1 Neo-Weberian framework to cities
		\2 Features of European cities are ``city landownership, municipal socialism, the tradition of town planning, the welfare state, and visions of urban development''; the welfare state is a typical feature of European countries, to fight against poverty and social exclusion\footnote{This is a socialist concept, can it be linked to the political agenda of Kazakhstan/Astana?}
		\2 ``in recent years, these features have been challenged. Cities have begun selling their properties, what used to be public services are increasingly provided by private entrepreneurs, public-private partnerships have replaced the tradition of regulatory town planning, the welfare state is in crisis, and the plans drawn up by town planners are contested by citizens''\footnote{There is a move from a cohesive, socialist European city to an Americanised, even more capitalist city. Does Astana follow either of these city categories, and is it evolving in the same way, from a socialist to increasingly privatised city?}
	\1 Land ownership: from a model where the owners of buildings also used the buildings, to an ownership by market/financial institutions, ie. from a socialist real estate to a competitive and profit driven ownership
	\1 Conclusion: neo-Weberian approach is still relevant to understand European cities today\footnote{Can (neo) Weberian theories of Occidental (vs. Oriental) cities help explain Astana's development? How does it compare to socialist planning?}
		\2 American and Asian urban renewal programs ``displaced the urban poor'' into slums, vs. European city approach of taking responsibility for the living conditions of its citizens
\end{outline}

\subsubsection{Davis \textit{Cities in global context: a brief intellectual history}}

\begin{outline}
	\1 Primacy: when countries are dominated by one or two major cities (Berry 1961, Vapnarsky 1966)
	\1 Discusses third-world urbanism, and how/the fact that American urbanists studied them 
	\1 1970s: writings on the global context of third-world urbanization, internal colonialism, primacy, uneven development
	\1 1990s: the end of the cold war created an environment in which advanced capitalist countries competed with each other for power, and at the same time the rise of neoliberalism changed the global context. Breaking down of protectionist economic policies, acceleration and densification of world trade
	\1 In globalisation, the role of the state became less and less relevant in urban literature. But recently (1990s) re-emerged as a player in city dynamics $\rightarrow$ how does the state impact city dynamics? are there differences between states and regime-types in regards to city developments, or do (global) cities develop similarly under democratic/authoritarian/communist regimes?\marginpar{Astana has an authoritarian, democratic and post-communist regime. How does it influence it?}
	\1 `Globalisation  and Latin American Cities', Bryan Roberts (Turning to the cases, p. 102)
		\2 Distinction between globalisation from above vs. from below. Roberts argues that it is from above, ie. ``it is through a capitalist investment in real estate intensified by globalisation, owing the changing social, spatial and political conditions associated with transformations of the neoliberal state and the more globalised urban economy''
		\2 From below exists, through citizen mobilisation in response to the capitalist takeover \marginpar{This is not like Astana, where mobilisation does not exist or is reprimanded}
		\2 Examines the effect of greater dependence on global economic by measuring eg. the share of inputs/outputs in GDP, and levels of Foreign Direct Investment attracted by privatisation of state-owned companies\marginpar{Could apply to Astana}
\end{outline}

\subsection{Reading the diversity of economic structures in European cities and their dynamics}

\subsubsection{Benko and Lipietz, \textit{From the regulation of space to the space of regulation}}

\begin{outline}
	\1 The essential thing in regulation is the relationship between a structure and its elements, and specifically here between global spaces and constituent sub-spacesp
\end{outline}

\subsubsection{Martin, \textit{Rethinking Regional Path Dependence: Beyond Lock-in to Evolution}}

\textit{Keywords: path dependence, economic geopgrahy}

\begin{outline}
	\1 Argues for \textbf{change, rather than continuity}, to explain the evolution of institutions and economics. Ie., argues for the opposite of path dependency
\end{outline}

\subsubsection{Martin, \textit{National growth versus spatial equality?}}

\textit{Keywords: economic efficiency, social equity}

\begin{outline}
	\1 Traditional arguments for regional policies: 
		\2 \textbf{Economic efficiency} argument: `persisting regional disparities in economic activity are nationally inefficient'. Not using workers and production capacity to its full potential in lagging regions means that national wealth is lower than it could be. Policies to raise utilisation and productivity of human and capital resources, will raise the national wealth
		\2 \textbf{Social equity} argument: people should not be disadvantaged in job opportunities, housing, access to public services, etc, just on the basis that they live in a certain region and not another. Policies can help prevent concentration of socio-economic disadvantage, and help create social cohesion
		\2 Economic efficiency and social equity go together: 
			\3 Increasing employment and productivity in economically lagging regions brings \textbf{national efficiency}, and increases living and social welfare, thus greater \textbf{spatial equity}
			\3 Vice versa, raising living standards through social welfare policy can reduce national welfare benefits, create greater spending power, and attract business and employment
	\1 \textbf{New economic geography} argues that regional balance, spatial agglomeration, concentration of economic activity actually benefits national growth. Policies to reduce spatial inequity may actually be nationally inefficient. It is a trade-off between the two
		\2 Key features of NEG: general equilibrium model of the economy; increasing returns at firm level which leads to market structure characterised by imperfect competition; transport costs that make location matter; factory mobility; 
			\3 \textbf{Transport costs}: if they are high, firms will disperse spatially to be close to immobile markets (those who cannot travel to consume the products). If they are low, firms will concentrate spatially in order to make use of the benefits of scale economies and agglomeration, while still having consumers because immobile markets can be provisioned effectively (cheaply)
			\3 \textbf{Imperfect competition}: a condition within a market where monopoly elements allow individual producers or consumers to have some control over market prices
			\3 \textbf{Factor mobility}: the ability to move production factors (labour, capital, land) from one type of production to another
		\2 There is a point at which spatial agglomeration (ie. regional imbalance) no longer benefits the national growth: when congestion costs and negative externalities (such as market crowding effects) are too strong, the national growth starts to decline
				
\end{outline}

\subsubsection{Fujita and Krugman, \textit{The new economic geography}}

\textit{Keywords: new economic geography, spatial economy, location theory, urban economics, spatial agglomeration, scale economies}

\begin{outline}
	\1 New economic geography: a new way of thinking about economy
	\1 \textbf{Scale economies}: the advantages that come with increasing business size. For example, buying in bulk is an economy of scale because the price per unit is typically lower in larger orders, putting the firm at an advantage.
Scale economies encourages the concentration of firms and goods in one area. 
	\1 \textbf{Horizontally differentiated products}: the differentiation between products that is not based on the quality or price, the products offer the same thing at the same price. Differentiation is usually done by personal preference. For example, Pepsi/Coke, types of detergent, brand of bananas
	\1 \textbf{Constant return}: an increase in production power results in a linear increase of returns. A difference in production scale will not affect the price of the good. 
	\1 \textbf{Circular causation}: forward linkages, backward linkages
	\1 \textbf{Core-periphery pattern}: shows how cultural, political, and economic authority is dispersed across dominant core regions, and surrounding (semi-) peripheral regions.
	\1 \textbf{Agglomeration} does not need to happen, necessarily. But regions are subject to catastrophic bifurcations, where a parameter change can tip the scale such that a region becomes a strong (industrialised) core, and the other a weaker (de-industrialised) periphery.
	\1 \textbf{Agglomeration} is a self-reinforcing system. At a given point in a region's size, its original advantages (eg. port city
	\1 \textbf{Centrifugal force}: the factor that drives concentration of activity in one region. For example, the immobility of agriculture land could drive the concentration of agriculture there.
\end{outline}

\subsection{Comparing European spatial planning systems and cultures}

\subsubsection{Elinbaum and Galland, \textit{}}

\begin{outline}
	\1 Providing an analytical framework to analyse metropolitan spatial planning comparatively (compares London, Paris, Copenhagen, Barcelona), by investigation their institutional context, instrumental content, and planning processes
	\1 The term planning ``traditions'' is used to emphasise the fact that forms of planning are deeply rooted in complex historical conditions of a plan. Traditions in the case of the 4 cities:
		\2 Land-use planning tradition: corresponds with the British legal-administrative framework (London)
		\2 Comprehensive-integrated approach: corresponds to the Scandinavian legal framework, planning systems are intended to provide horizontal and vertical integration of policies across different sectors and jurisdictions (Copenhagen)
		\2 Regional-economic approach: top-down character derived from the Napoleonic legal framework (Paris)
		\2 Urbanism approach: subscribes to Napoleonic tradition but for southern Europe (Barcelona)
	\1 Proposes variables to analyse the planning strategies of metropolitan cities:
		\2 Institutional context: characterised by legal framework and participating institutions in planning; conforming vs. performing, monolithic vs. coalition
		\2 Instrumental context: general or selective in scope; legally binding or indicative form or effect; spatial models can be monocentric (eg. traditional central place model of concentric land-use patterns, with a centralised metropolitan core) or polycentric; 
		\2 Planning process: plans delivered incrementally or optionally (leaving options and paths open); planning is dependent on roles that planners adopt, technocratic vs. participative; led by public vs. private vs. public-private partnerships; plan performance as continuous vs. intermittent, depending on whether or not plans are revised and assessed;
\end{outline}

\subsubsection{Ferencuhova, \textit{Accounts From Behind the Curtain}}

\begin{outline}
	\1 Contributing a post-socialist framework to urban and cosmopolitan theory is hard, because it cannot easily be applied outside the post-socialist context
	\1 Talks about the unequal knowledge production, whereby the Western, European theories are considered universal vs. the marginal urban theories considered localised (p.115)
	\1 Demonstrates that `ordinary cities' are a response to the neo Marxist, global cities, world cities rhetoric, which is Eurocentric (p. 116)\marginpar{See p.116-117 for paper citation}
	\1 Overall discussion about the need to decentralise production of urban knowledge; gives reasons why theories from soviet block/socialist cities and states were, and are, excluded from the universal theories
\end{outline}

\subsubsection{Hirt, Ferencuhova, Tuvikene, \textit{Conceptual form: the post-socalist city}, 2016}\cite{hirt2016conceptual}
\textit{Not from the recommended readings}

\begin{outline}
	\1 Hirt
		\2 The socialist and post-socialist cities in spatial terms\marginpar{could be used to analyse Astana's planning as post-socialist}
		\2 If a post-socialist city exists, then a socialist city must have existed, too
		\2 Socialist city socio-spatial characteristics (in CEE), compared to western Europe (p.2):
			\3 More space (spatial generosity) in inner-city ceremonial public ares
			\3 Less space (spatial scarcity) in socialist mass-housing estates
			\3 Population is less segregated by class in distinct urban quarters
			\3 Less signs of urban marginality
			\3 Abundance of urban industrial land use
			\3 More uniform and standardised architectural look
		\2 These characteristics are a result of the political economy of that soviet states, which had greater power `compared to capitalist states, to control urban land, real estate, and means of production'
		\2 Using the term `socialist' and `post-socialist' assumes a common past and future for the cities that fall in these categories\marginpar{Could compare Astana to eg. a CEE capital city and show their commonality or divergence}
		\2 Multiple categories for ex-soviet states: 1, Central Europe and Baltics 2, `Russia and other formerly Soviet cities, where the public sector continues to play a very strong role in the distribution of resources through various powerful public-private-sector alliances' , and 3, Yugoslav and Balkans 
		\2 The soviet countries had really long and rich histories before sovietism. Did 50-70 years of soviet colonisation warrant them being ``post socialist'' states, and over write their previous socio, political, economic structures? Especially 30 years after independence, should they still be considered post-soviet, is that still relevant? haven't they grown into their own without needing to be linked together as post socialist countries/cities?
		\2
\end{outline}

\textit{If we want to analyse to what extent Astana is a post-socialist city, then we must assume there was a socialist city, and compare Astana's socio-spatial (or other, eg socio-economic) structure to that of a socialist city, and its antipode, the western European city in our case.}

\section{Paper}

\subsection{Feedback session}
\date{December 13th, 2021}

\subsubsection{General feedback}

What needs to be better? 

\begin{outline}
	\1 Position the city within wide European trends, and compare to other cities, put in a broader context
	\1 Explain method and data collection strategies, include methodology
	\1 Paper title that is NOT a research question; \textit{A post-socialist city between East and West}
	\1 Provide the word count
	\1 Provide literal research question
	\1 \textbf{Critically engage} with literature from the course. Skip wikipedia basic information, and repeating literature from class
	\1 Outline
		\2 Use subtitles to build argument; make subtitles interesting
		\2 Each paragraph should build up to answer the RQ
	\1 Reread sentences and replace words which can be more poignant or shorter (eg. soon vs. in the near future); don't overuse words like moreover, etc.
	\1 Explain relevance of paper for the city, and for general research purposes
\end{outline}

%%%%%%%%%%%%%%%%%%%%% ENVIRONMENTS %%%%%%%%%%%%%%%%%%%%%

\if{false}

\subsubsection{, \textit{}}

\begin{outline}
	\1
\end{outline}


\fi

\end{document}