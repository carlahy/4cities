\documentclass{article}

\usepackage[utf8]{inputenc}
\usepackage[left=1.5in,right=1.5in,bottom=1in]{geometry}
\setlength\parindent{0pt}
\setlength{\parskip}{1em}
\setcounter{secnumdepth}{0}
\usepackage{outlines}
\usepackage{graphicx}
\graphicspath{ {imgs} }
\usepackage{hyperref}
\title{European Cities in an Urbanising World}
\author{Carla Hyenne }

\begin{document}

\maketitle

\tableofcontents

\pagebreak

%%%%% LECTURE 1 %%%%%
\section{Introduction: European Cities in an Urbanising World}
\date{September 27th, 2021}

\begin{outline}
	\1 Theoretical and empirical introduction to the course
	\1 Discussion on globalisation debate: myths on globalisation, waves of globalisation, the problem of periodisation, the relation between globalisation and regionalisation/localisation
	\1 The position of `Europe' in the world
		\2 The changing boundaries of what and where Europe is considered to be
		\2 The imperial/colonial linkages of European countries and cities
		\2 Cities as nodes in networks of cultural and intellectual exchange
		\2 Migration movements from/to/within Europe and the position of cities
		\2 The rescaling of national state space and the relation between Europeanisation and globalisation
\end{outline}

\subsection{Readings}

\subsubsection{Tsing, 2000, \textit{The Global Situation}}

\begin{outline}
	\1 Discusses ``what is globalisation'', what is, is there, a globalisation era?
	\1 \textbf{Futurism}: the idea that the future will bring progress and prosperity, that globalisation is modern rather than ancient, and attributes stereotypes to the past
	\1 \textbf{Conflations}: merging of ideas/opinions/texts. Globalisation cannot be understood under a single ideological system. It is best understood by the overlaps of systems. Networks serve to connect people/ideas/movements world wide, and strengthens them. But, networks can be exclusionary
	\1 \textbf{Circulation}
	\1 Cultural anthropology is how people who share a common cultural system organise and shape their social and physical world
	\1 In anthropology, globalism is the intensification of global interconnectedness
	\1 Flow is the cultural exchange between cultures/nations, and Imagined landscapes are created by globalisation. These are too much emphasised in anthropology, rather, we should look at scale making of the global and regional
\end{outline}

\subsubsection{Osterhammel, 2011, \textit{Globalisations}}


\begin{outline}
	\1 A historian's look on a social science
	\1 \textbf{Modernity} is the development of capitalism, industrialisation, the establishment of nation states, countries, regions
	\1 \textbf{Globalisation} is the the interrelationships among countries, and how these countries `merge'
	\1 The problem with globalisation is that there are many definitions, all of them viable. Thus we should not define it but use it as a framework that contains models of change, in large spatial contexts
	\1 Discusses globalisation vs. global history, and concludes that global history encompasses globalisation
\end{outline}


\subsection{}

%%%%% LECTURE 2 %%%%%
\section{From the `European city' to comparative urbanism}
\date{November th, 2021}

%%%%% LECTURE 3 %%%%%
\section{Reading the diversity of economic structures in European cities and their dynamics (part 1)}
\date{October 18th, 2021}


%%%%% LECTURE 4 %%%%%
\section{Reading the diversity of economic structures in European cities and their dynamics (part 2)}
\date{October 25th, 2021}

%%%%% LECTURE 5 %%%%%
\section{Comparing European spatial planning systems and cultures}
\date{November 8th, 2021}


%%%%% LECTURE 6 %%%%%
\section{The European Union and the Europeanisation of spatial planning and territorial development policies}
\date{November 9th, 2021}

%%%%% LECTURE 7 %%%%%
\section{Worlds in motion: Migration and (im)mobility in a globalising Europe}
\date{November 22nd, 2021}

%%%%% LECTURE 8 %%%%%
\section{Contextualising contemporary (post-)migration: Geographies and populations of European cities}
\date{November 29th, 2021}

%%%%% SEMINAR 1 %%%%%
\section{Seminar: the course assignment}
\date{October 11th, 2021}

%%%%% SEMINAR 2 %%%%%
\section{Seminar: Data and Research strategies}
\date{October 26th, 2021}



%%%%% SEMINAR 3 %%%%%
\section{Seminar: feedback on draft paper}
\date{December 13th, 2021}







\begin{outline}
	\1
\end{outline}








\end{document}