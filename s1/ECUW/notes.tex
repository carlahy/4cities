\documentclass{article}

\usepackage[utf8]{inputenc}
\usepackage[left=1.5in,right=1.5in,bottom=1in]{geometry}
\setlength\parindent{0pt}
\setlength{\parskip}{1em}
\setcounter{secnumdepth}{0}
\usepackage{outlines}
\usepackage{graphicx}
\graphicspath{ {imgs} }
\usepackage{hyperref}
\title{European Cities in an Urbanising World}
\author{Carla Hyenne }

\begin{document}

\maketitle

\tableofcontents

\pagebreak

%%%%% LECTURE 1 %%%%%
\section{Introduction: European Cities in an Urbanising World}
\date{September 27th, 2021}

\begin{outline}
	\1 Theoretical and empirical introduction to the course
	\1 Discussion on globalisation debate: myths on globalisation, waves of globalisation, the problem of periodisation, the relation between globalisation and regionalisation/localisation
	\1 The position of `Europe' in the world
		\2 The changing boundaries of what and where Europe is considered to be
		\2 The imperial/colonial linkages of European countries and cities
		\2 Cities as nodes in networks of cultural and intellectual exchange
		\2 Migration movements from/to/within Europe and the position of cities
		\2 The rescaling of national state space and the relation between Europeanisation and globalisation
\end{outline}

\subsection{Readings}

\subsubsection{Tsing, 2000, \textit{The Global Situation}}

\begin{outline}
	\1 Discusses ``what is globalisation'', what is, is there, a globalisation era?
	\1 \textbf{Futurism}: the idea that the future will bring progress and prosperity, that globalisation is modern rather than ancient, and attributes stereotypes to the past
	\1 \textbf{Conflations}: merging of ideas/opinions/texts. Globalisation cannot be understood under a single ideological system. It is best understood by the overlaps of systems. Networks serve to connect people/ideas/movements world wide, and strengthens them. But, networks can be exclusionary
	\1 \textbf{Circulation}
	\1 Cultural anthropology is how people who share a common cultural system organise and shape their social and physical world
	\1 In anthropology, globalism is the intensification of global interconnectedness
	\1 Flow is the cultural exchange between cultures/nations, and Imagined landscapes are created by globalisation. These are too much emphasised in anthropology, rather, we should look at scale making of the global and regional
\end{outline}

\subsubsection{Osterhammel, 2011, \textit{Globalisations}}


\begin{outline}
	\1 A historian's look on a social science
	\1 \textbf{Modernity} is the development of capitalism, industrialisation, the establishment of nation states, countries, regions
	\1 \textbf{Globalisation} is the the interrelationships among countries, and how these countries `merge'
	\1 The problem with globalisation is that there are many definitions, all of them viable. Thus we should not define it but use it as a framework that contains models of change, in large spatial contexts
	\1 Discusses globalisation vs. global history, and concludes that global history encompasses globalisation
\end{outline}

\subsection{Periodising Globalisation}

The working definition of globalisation is:

\begin{outline}
	\1 \textit{``Globalisation is the extension, acceleration and intensification of consequential worldwide interconnections''} (Sparke, 2013, p.13). 
	\1 Places across the globe are increasingly interconnected; social relations and economic transactions increasingly occur at the intercontinental scale; the globe itself comes to be a recognisable geographic entity (Cloke et. al, 2005, p.36)
\end{outline}

This means there is an increase of interconnection of the global, through processes such as intercontinental economic and financial transactions. \textit{\textbf{Increasing}} assumes that there is a `before', but that is usually implicit. However, globalisation needs to be \textbf{situated} in time. 

We now visualise the world as one global entity, especially in ecology. Harvey (1989) describes the speeding up of globalisation, that the world always becomes faster. From horse-drawn carriages, to steam locomotive, to propeller aircraft, to jet passenger aircraft, these modes of transport have made the world a smaller place. Transport connections/network maps show that its becoming easier to connect cities to each other, rather than to the periphery. 

\subsubsection{Presentism and futurism}

Two key problems in periodising globalisation are presentism and futurism.

\begin{outline}
	\1 Globalisation is a term that is marked by its time. It emerged in the 1970s and the era of globalisation tends to be limited to post-1970s. It has a Western perspective limited to the 1990s era of Western sensibilities
	\1 This implies a \textbf{newness} and not a \textbf{continuity}: there is little sense of dialectic globalisation and de-globalisation. If something globalises, does something else de-globalise?
	\1 There is a tendency to focus on the condition, not the process and projects of globalisation: ``globalisation did this''. Such statements are impossible because globalisation is a concept. Who are the actual actors?
\end{outline}

Presentism and futurism are Western styles of thinking. 

Similarly to globalisation, the Enlightenment was not condition that could be achieved, but a process. It was a normative-societal goal. The obstacles to achieving Enlightenment are immaturity (laziness, cowardice) and the lack of freedom (people must use reason in public) (Immanuel Kant). For globalisation, immaturity means that people are not good entrepreneurs, and lack of freedom means the need to break down institutional barriers. 

Modernisation is the global ``world-making process'' of the post WWII era. It is a normative-societal goal, all is moving towards it, and little attention is paid to contradictions and tensions. 

\subsubsection{Eurocentrism}

Eurocentrism is the idea that Europe's history is the `sovereign' of all histories, that `other' histories are based on one master narrative, and that all narratives about cities are related to Europe\footnote{See `Provincialising Europe' by Dipesh Chakrabarty}.

The definition of globalisation depend on where you situate yourself in history, and also in what discipline. Definitions don't need to agree, but they \textbf{need to be explicit as to where and when they are situated}.

\subsubsection{Periodisations}

Examples of periodisations:

\begin{outline}
	\1 1820s: convergence of commodity prices across continents. The economic explanation is that local economies integrated into global systems, and the prices of goods converged. It overlooks state interventions, and is Western and colonial centric, exposing the inequality of globalisation
	\1 1571: founding of Manila as Spanish trading post. It linked trans-Atlantic and trans-Pacific trade. It was a colonial city, and decentralised Europe in the global narrative
	\1 13th century: the world system before European hegemony. There were 8 circuits in the world, that traded most commonly with each other. Not European centric, but more Asia-centric that Europe inherited
\end{outline}

\subsection{Where is Europe?}



\subsection{Different cities, different globalisations}



\subsection{Tracing connections}



\subsection{Keywords}

Globalisation
Presentism and futurism
Normative-societal goal
Eurocentrism
Periodisation

%%%%% LECTURE 2 %%%%%
\section{From the `European city' to comparative urbanism}
\date{November th, 2021}

%%%%% LECTURE 3 %%%%%
\section{Reading the diversity of economic structures in European cities and their dynamics (part 1)}
\date{October 18th, 2021}


%%%%% LECTURE 4 %%%%%
\section{Reading the diversity of economic structures in European cities and their dynamics (part 2)}
\date{October 25th, 2021}

%%%%% LECTURE 5 %%%%%
\section{Comparing European spatial planning systems and cultures}
\date{November 8th, 2021}


%%%%% LECTURE 6 %%%%%
\section{The European Union and the Europeanisation of spatial planning and territorial development policies}
\date{November 9th, 2021}

%%%%% LECTURE 7 %%%%%
\section{Worlds in motion: Migration and (im)mobility in a globalising Europe}
\date{November 22nd, 2021}

%%%%% LECTURE 8 %%%%%
\section{Contextualising contemporary (post-)migration: Geographies and populations of European cities}
\date{November 29th, 2021}

%%%%% SEMINAR 1 %%%%%
\section{Seminar: the course assignment}
\date{October 11th, 2021}

%%%%% SEMINAR 2 %%%%%
\section{Seminar: Data and Research strategies}
\date{October 26th, 2021}



%%%%% SEMINAR 3 %%%%%
\section{Seminar: feedback on draft paper}
\date{December 13th, 2021}







\begin{outline}
	\1
\end{outline}








\end{document}