\documentclass[12pt]{article}

\usepackage{setspace}
\usepackage{parskip}
\usepackage[normalem]{ulem}
\onehalfspacing
\usepackage[footnotesep=0.5cm]{geometry}
\setcounter{secnumdepth}{0}

% Margins
\topmargin=-0.45in
\evensidemargin=0in
\oddsidemargin=0in
\textwidth=6.5in
\textheight=9.0in
\headsep=0.25in

\title{Astana, KZ: A City in an Urbanising World}
\author{Carla Hyenne}
\date{}

\begin{document}

\maketitle 

\section{Research}

\subsection{Geography}

\begin{itemize}
  \item Akmoly settlement started in 1830, it's advantageous geography was clear. Roads connected to the East/West/South areas
  \item Large urban areas, favourable geographical position, proximity to the major economic centers of the region, considerably demographic capacity, good transportation facilities, relatively favourable climate
  \item 1971, Tselinograd becomes center of the oblast
  \item Former capital Almaty is still largest city in the country (what are the differences in social, political, economic powers between Almaty and Astana?)
  \item Located on the Ishim River, very flat steppe region, very spacious landscape, in between the north of KZ and the very sparsely settled center, because of the river. 
  \item Astana is divided by the river: the nort has the old boroughs, and the south has the new boroughs
\end{itemize}

\subsection{Economy}

\begin{itemize}
  \item Served as a route to transport equipments during war, industries supplied WWII material (to Soviets). After WWII Akmoly was beacon of economic revival in Soviet Union
  \item Became part of the Virgin Lands Campaign by Nikita Khrushchev, in order to boost agricultural (grain) production to help with the food shortages in the Soviet Union
  \item Based on trade, industrial production (mainly of building material, foodstuff, mechanical engineering), transport, communication, construction
  \item Astana International Financial center opened in 2018 to become a hub for financial services in Central Asia
  \item Headquarters for state-owned corporations
  \item The shift of the capital was an economic boost for Astana, and the economic development attracted foreign investors. The New City "special economic zone" was created to develop industry and make the city attractive to investors
\end{itemize}

\subsection{Politics}

\begin{itemize}
  \item Is, or was, an oblast: a type of federal subject of Russian Federation. What did this mean for Akmoly? 
  \item Renamed from Akmoly to Tselinograd (1961) to represent city's role in the Virgin Lands Campaign, it means "city of virgin lands"
  \item 1991, Kazakhstan gains its independence after dissolution of Soviet union, name is restored to Akmola.
  \item 1994, decree "on the transfer of the capital of Kazakhstan" which is officially moved from Almaty to Akmoly in 1997, and renamed to Astana in 1998 (Why did the name change?). A major driver for the movement of the capital was the growing Kazakh population in Astana, which was done on purpose to alleviate burden on Almaty which was running out of space for expansion. Almaty is also located on earthquake prone land, and near Chinese border
  \item 1999, Astana is awarded medal of City of Peace by UNESCO
  \item 2019, Nur Sultan resigns and city is renamed to him, even if there is resistance by residents (out of habit? opposition?) who continue to call it Astana
  \item Platform for high profile diplomatic talks, and summits on global issues
  \item KZ is rich with oil money, yet a lot of it has just gone into enriching the capital and the buildings to project power and influence
  \item KZ needs to navigate Russia, China and US 
\end{itemize}


\subsection{City Scape/Development}

\begin{itemize}
  \item 1960s completely transformed Tselinograd: three high-rise housing districts began, new monumental public buildings (including Virgin Land Palace, Palace of Youth, House of Soviets, new airport, sports venues).
  \item Divided into four districts
  \item In 1998, the KZ government launched a competition for renowned architects and urban planners to design the new capital. The winner, Kurokawa, decided to preserve and redevelop the existing city, and create a new city at the south and east of the river. 
  \item Given the above, the North of the rail way is the industrial part and the poor residential areas. Between the railway and the river is the city center, where intense building is happening. South is the new area with government administrations, diplomatic quarters, government buildings.
  \item Centrally planned city
\end{itemize}

\subsection{Population}

\begin{itemize}
  \item Astana has 1.1M residents, mostly Kazakhs (80\%), Russians (17\%), and other central Asians representing <1\% each. The majority Kazakh population is a recent phenomenon, and shift in the last decades
  \item There was a drive to attract Kazakhs north-ward from Almaty, due to density limits of Almaty. this was key in shifting the capital
  \item Attracts migrant workers, legal and illegal
  \item Attracts young professionals
\end{itemize}


- spectacular urbanism, what is behind this image?

%Research question


\section*{Part a}

Lorem ipsum dolor sit amet, consectetur adipiscing elit, sed do eiusmod tempor incididunt ut labore et dolore magna aliqua. Ut enim ad minim veniam, quis nostrud exercitation ullamco laboris nisi ut aliquip ex ea commodo consequat. Duis aute irure dolor in reprehenderit in voluptate velit esse cillum dolore eu fugiat nulla pariatur. Excepteur sint occaecat cupidatat non proident, sunt in culpa qui officia deserunt mollit anim id est laborum


\section*{Part b}

Lorem ipsum dolor sit amet, consectetur adipiscing elit, sed do eiusmod tempor incididunt ut labore et dolore magna aliqua. Ut enim ad minim veniam, quis nostrud exercitation ullamco laboris nisi ut aliquip ex ea commodo consequat. Duis aute irure dolor in reprehenderit in voluptate velit esse cillum dolore eu fugiat nulla pariatur. Excepteur sint occaecat cupidatat non proident, sunt in culpa qui officia deserunt mollit anim id est laborum

\end{document}


