\documentclass{article}

\usepackage[utf8]{inputenc}
\usepackage[left=1.5in,right=1.5in,bottom=1in]{geometry}
\setlength\parindent{0pt}
\setlength{\parskip}{1em}
\setcounter{secnumdepth}{0}
\usepackage{outlines}
\usepackage{graphicx}
\graphicspath{ {imgs} }
\usepackage{hyperref}
\title{Urban Economic Geography - Summarised Notes}
\author{Carla Hyenne }

\newcommand{\alignedmarginpar}[1]{%
        \marginpar{\raggedright\small #1}
    }

\begin{document}

\maketitle

\tableofcontents

\pagebreak

\section{The city as a social product}

Emphasis on the city as a product dynamic relationships of societies with urban space, and not a set of necessary attributes that define `the city'. 

\textbf{Urban triumphalism}: the city is a site of progress and a way to prosperity. The major contemporary challenges are urban challenges, and the urban is where solutions can be found. Eg. green, smart, productive, participative, etc., city \alignedmarginpar{Glaeser}

\textbf{Critique of urban triumphalism}: it is a purely pro-growth perspective, which contradicts its sustainability goals. It is about finding solutions to pre-defined challenges (how to...) which turns urban issues into techno-management issues where copy/paste can be found and match city leader's views.

\textbf{Breaking with urban triumphalism}:  the city should not be seen as an actor but as dynamic space, where actors with different resources, interests, inspirations, interact in diverse ways (conflict, resistance, mobilisation, collaboration, solidarity...)\alignedmarginpar{Marcuse}

\textbf{P. Marcuse, \textit{The City as a Perverse Metaphor}, 2005}: seeing the city as an actor excludes a set of the population who does not share the same interests. Not all the city is international, competitive, etc., even if some firms and some people are.

\textbf{Critical urban studies}: contra-naturalising, there is nothing natural about cities and they are not living organisms; contra techno-managerial, focus instead on tensions and contradictions, not solutions to standardised challenges.\alignedmarginpar{Lefebvre}

\textbf{Production of urban space}: urban space is organised to reflect and support a particular society (capitalist, soviet, post-socialist, etc.)\alignedmarginpar{LA vs. Moscow}. Cities do not have a set of attributes necessary to be a city, but they are the product of dynamic relationships of societies with urban space.

\textbf{Creative destruction}: the production of space is always a work in progress, with inherited socio-spatial configurations reshaped by new logics.\alignedmarginpar{Place de Brouckère, Senne}

\textbf{New epistemology of the urban}: concentrated urbanisation, extended urbanisation, differential urbanisation \alignedmarginpar{Brenner and Schmidt}

\textbf{Concentrated urbanisation}: spatial clustering of population, mean of transportation, infrastructure, investment, ie. agglomeration

\textbf{Extended urbanisation}: activation and transformation of places, territories, landscapes in relation to agglomeration processes; uneven thickening and stretching of urban fabric across the planet

\textbf{Differential urbanisation}: creative destruction of implosion/explosion of socio-spatial organisation; production of new urban 'potentials' for the appropriation of urban configurations and for the production of radically new forms of urban space


\textbf{}

\textbf{}

\textbf{}

\textbf{}

\pagebreak
\section{L2}


\pagebreak
\section{L3}


\pagebreak
\section{L4}


\pagebreak
\section{L5}


\pagebreak
\section{L6}


\pagebreak
\section{L7}


\pagebreak
\section{L8}


\pagebreak
\section{L9}


\pagebreak
\section{L10}

\section{Glossary}

\textbf{Accumulation by dispossession}:

\textbf{Internalisation of capitalism}: our self worth is directly defined by our productivity

\section{Ordinary Urbanisations}

\textbf{Ordinary urbanisations}

\textbf{Subaltern urbanism}

\textbf{Popular urbanisation}

\textbf{Urbanisation of neoliberalism}

\textbf{Neoliberalisation of urbanism}

\textbf{Popular centrality}

\textbf{Ordinary urbanisation}


\textbf{}


\textbf{}
\end{document}
