\documentclass{article}

\usepackage[utf8]{inputenc}
\usepackage[left=1.5in,right=1.5in,bottom=1in]{geometry}
\setlength\parindent{0pt}
\setlength{\parskip}{1em}
\setcounter{secnumdepth}{0}
\usepackage{outlines}
\usepackage{graphicx}
\graphicspath{ {imgs} }
\usepackage{hyperref}
\title{Urban Economic Geography - Summarised Notes}
\author{Carla Hyenne }

\newcommand{\alignedmarginpar}[1]{%
        \marginpar{\raggedright\small #1}
    }

\begin{document}

\maketitle

\tableofcontents

\pagebreak
\section{The course logic}

\textbf{Who produces the city, and how?} $\rightarrow$ Lefebvre

\textbf{Spatial practices of capital}: market-led urbanism, on the role of profit-oriented economic actors, centred on commodification, growth and accumulation $\rightarrow$ Neo-Marxism, regulation theory, Harvey

\textbf{Territorial regulations}: state-led urbanism, on the role of urban governance frameworks, centred on development, control and legality $\rightarrow$ Neo-institutionalism, governance theory, urban regime theory, Brenner

\textbf{Everyday life}: popular urbanism, on the role of `ordinary' city dwellers, centred on livelihood, habitat, community building $\rightarrow$ gentrification studies, post-colonial urban theory, N. Smith, J. Robinson, A. Roy

The \textbf{neoliberal city} is the alliance of market-led and state-led urbanism, the urbanisation of neoliberalism, and the neoliberalisation of urbanism. It interacts with \textbf{urbanisation from below}, through processes of disinvestments and dispossession of \textbf{ordinary urbanisation}, and its resistance to forces of the neoliberal city.

\pagebreak
\section{The city as a social product}

\textit{tldr; emphasis on the city as a product dynamic relationships of societies with urban space, and not a set of necessary attributes that define `the city', and not an actor. The city is not only material, but the set of all forces within it (political, economic, social,...)}

\textbf{Urban triumphalism}: the city is a site of progress and a way to prosperity. The major contemporary challenges are urban challenges, and the urban is where solutions can be found. Eg. green, smart, productive, participative, etc., city \alignedmarginpar{Glaeser}

\textbf{Critique of urban triumphalism}: it is a purely pro-growth perspective, which contradicts its sustainability goals. It is about finding solutions to pre-defined challenges (how to...) which turns urban issues into techno-management issues where copy/paste can be found and match city leader's views.

\textbf{Breaking with urban triumphalism}:  the city should not be seen as an actor but as dynamic space, where actors with different resources, interests, inspirations, interact in diverse ways (conflict, resistance, mobilisation, collaboration, solidarity...)\alignedmarginpar{Marcuse}

\textbf{P. Marcuse, \textit{The City as a Perverse Metaphor}, 2005}: seeing the city as an actor excludes a set of the population who does not share the same interests. Not all the city is international, competitive, etc., even if some firms and some people are.

\textbf{Critical urban studies}: contra-naturalising, there is nothing natural about cities and they are not living organisms; contra techno-managerial, focus instead on tensions and contradictions, not solutions to standardised challenges.\alignedmarginpar{Lefebvre}

\textbf{Production of urban space}: urban space is organised to reflect and support a particular society (capitalist, soviet, post-socialist, etc.)\alignedmarginpar{LA vs. Moscow}. Cities do not have a set of attributes necessary to be a city, but they are the product of dynamic relationships of societies with urban space.

\textbf{Creative destruction}: the production of space is always a work in progress, with inherited socio-spatial configurations reshaped by new logics.\alignedmarginpar{Place de Brouckère, Senne}

\textbf{Henri Lefebvre}: any political economy implies a spatial order. Urban space produces a certain type of society and vice versa. The urban is site for social struggles, which must appropriate space.

\textbf{New epistemology of the urban}: concentrated urbanisation, extended urbanisation, differential urbanisation \alignedmarginpar{Brenner and Schmidt}

\textbf{Concentrated urbanisation}: spatial clustering of population, mean of transportation, infrastructure, investment, ie. agglomeration

\textbf{Extended urbanisation}: activation and transformation of places, territories, landscapes in relation to agglomeration processes; uneven thickening and stretching of urban fabric across the planet

\textbf{Differential urbanisation}: creative destruction of implosion/explosion of socio-spatial organisation; production of new urban 'potentials' for the appropriation of urban configurations and for the production of radically new forms of urban space

\pagebreak
\section{The production of urban space under capitalism}

\textbf{Harvey's theoretical project}: ``to integrate an understanding of processes of urbanisation and built environment formation into the general theory of the laws of motion of capital''. Understand how urbanisation helps us understand capitalism

\textbf{Capitalism}: a system of economic production based on the circulation/exchange of privately-owned capital, geared towards capital accumulation. An ever expanding system because capitalism must grow. It has a geography: it isn't the same everywhere and is embedded in social, cultural and institutional configs. It has a history: it evolved over time.

\textbf{Capitalism and urbanisation}: cities were not created by capitalism, but the rise and extension of capitalism is deeply linked to urbanisation processes.

\textbf{Crises of capitalism}: a crisis prone system that is resilient. Accumulation is at risk when surplus capital has no profitable outlet because the circulation of capital cannot continue.\alignedmarginpar{China's Evergrande}

\textbf{Technological fix}: reinvesting capital surplus in new or improved production capacities, ie. sectors fuelled by tech innovations.

\textbf{Financial fix}: reinvesting capital surplus into financial assets. The `financialisation' of capitalism

\textbf{Spatial fix}: reinvesting capital surplus into the production of profitable built environments. Eg. new spaces of industrial production, of transport and logistics, of consumption, of rent speculation, new urban markets like AirBnB

\textbf{Contradictions of capitalism}: capitalism produces space as an outlet for capital surplus, only to later destroy this space so that it can again produce it and ensure capital circulation\alignedmarginpar{Harvey}

\textbf{Any fix is short lived}: as soon as the capital is invested, the fix is consumed. It will only be profitable once again when the space is destroyed and reinvested\alignedmarginpar{Paris' Haussmanisation, post-was capitalism and urbanisation}

\textbf{Surburbanisation}: a lifestyle change that had social, economic and political implications. Social because it created a new lifestyle and new demands; economic to create and meet those demands (new products like home equipment); political because middle class demands shifted focus from community to private ownership and defence of private space, suburban votes became conservative republicans.\alignedmarginpar{American suburbs, Belgian suburbs}

\textbf{Theory of urbanisation under capitalism}: capitalism has an insatiable addiction to the production of profitable space, because this overcomes capitalism's internal contradictions. Under capitalism: the mobilisation of urban change for the sake of capital accumulation is permanent, and is at the forefront in times of crisis; any spatial fix is short-lived, because the profitable way out of a crisis paves the way for the next crisis; fixing space has ramifications on the building environment, modes of consumption, mobility patterns, cultural and political subjectivities; any spatial fix comes with patterns of uneven development.

\textbf{Uneven spatial development}: under capitalism, investment in some places result in disinvestment elsewhere. The disinvested areas will in the future be sites for investment once again. The spatial fix is cyclic: capitalism does not solve a crisis, but moves them geographically. Creative destruction displaces communities who will not be able to return afterwards.\alignedmarginpar{Detroit}

\pagebreak
\section{The Neoliberal City}

\textit{tdlr; zooming in to the present conjuncture of the production of space under capitalism, ie. urbanisation under neoliberal capitalism, or the ‘neoliberal’ city. Urban spaces are sites and vehicles for neoliberalism: sites because that is where the accumulation takes places, and vehicles because of the financialisation of the real estate market, and it’s speculation. The 21st century city is a city of rents.}

\textbf{Neoliberalism}: intellectual and political proposals that extend market mechanisms and ethics of competition to an ever-wider spectrum of social activities, based on strong state intervention (Pinson, 2020). A class project to remove restrictions in the way of accumulation (neomarxism, Harvey)

\textbf{Neoliberal turn}: since the 1980s there is a growing gap between the profit rate (return on productive capacities) and accumulation rate (share of profit reinvested in productive capacities). Capital has grown, but the surplus hasn't been accumulated (reinvested in productive capacities).

\textbf{Trans-nationalisation}: surplus capital being invested externally. Companies globalised, off-shored and outsourced, there were mergers and acquisitions, FDI (investors establishing long-lasting interests in foreign enterprises)

\textbf{Financialisation}: surplus capital going into financial markets. Results in an increasing concentration of capital in the hands of the financiers. Financialisation uses asset valuation strategies, the purpose is to valorise assets and speculate to maximise rental yields or make capital. gains on resale. The largest asset managers (eg. BlackRock, Vanguard) have greater wealth than that of countries like Germany, France, Switzerland (based on GDP)

\textbf{Cities as critical sites capital accumulation}: urban space is where the accumulation of capital takes place. Multiple frameworks illustrate this: global cities who control ever-more of the world's networks and flows (Sassen); worlding cities emerging as global cities (A. Roy, A. Ong); globalisation from below, is the production and consumption associated to transnational communities beyond political borders (A. Portes); inconspicuous cities beyond global cities that global cities are investing in, thus are crucial for capital accumulation (Choplin, Pliez); planetary urbanisation whereby places far from the urban core and peripheries are strongly linked to urbanism and urbanisation (Brenner and Schmid)

\textbf{Cities as vehicles for capital accumulation}:\alignedmarginpar{Battersea Power Station} urban space is financialised through the real estate market and its speculation. Assetisation turns real estate into financial assets. Urbanisation becomes speculative, land is mobilised for the maximisation of rent extraction, and the 21st century metropolis becomes a city of rent\alignedmarginpar{Charnock et. al}.

\textbf{21st century city of rent}: neoliberal urbanisation visible by: the rise of corporate landlords and buy-to-let housing markets; speculative projects tailored to meet investors' requirements first; increased tendency of urban governments to rely on real estate valorisation to finance public utilities; growing un-affordability as a system feature, because profit is necessary. $\Rightarrow$ uneven geography with investments rising in one place (city growth) leaving other places behind (shrinking cities).

\pagebreak
\section{New Urban Policies: Urban Governance and the Entrepreneurial City}

\textit{tldr; capitalism isn’t the only force shaping urban space. Urban governance and the entrepreneurial city (neoliberalisation of urbanism) play an increasing role, visible through the rescaling of Statehood, new urban governance, and urban entrepreneurialism}

\textbf{Statehood}: an abstract idea, represents public power, public authorities, all forms (actions, mechanisms) through which public power is exercised. Compared to the State which is an institution.

\textbf{Rescaling of Statehood}: more importance is given to other scales than the national one\alignedmarginpar{(N. Brenner, 2004)}: upscaling (supra-national tiers of government), downscaling (subnational tiers), and outsourcing (towards private and civil society actors).
The new Statehood is multi-layered, less national centric, the urban/regional scales are more relevant for urban/regional policy.\alignedmarginpar{Transnational networks of city governments; Brussels' North Quarter}

\textbf{Urban regime theory}: no single actor in the city has all resources/capital necessary to produce urban policies and projects. The ``capacity to govern''\alignedmarginpar{(Stone, 1989)} and produce urban space is distributed among actors, at different scales, bringing different resources (legislative, political, economic, symbolic, etc.).

\textbf{Urban politics}: assembling the actors (stakeholders, institutions, etc.) and building a consensus in order to compose an effective capacity to govern with a certain stability.

\textbf{Urban regime}: the set of (in)formal arrangements that enable multi-stakeholder coalitions. A network of people in space. The stakeholders have a common interest of growing the economy.\alignedmarginpar{Growth coalition, governance coalition} Channels that exist to assemble urban governance frameworks: PPPs, citizen participation frameworks, strategic planning

\textbf{Public-private partnerships}: legally binding, contractual arrangement between a private actor (contractor) and a public authority, for the development of an equipment of public interest over a fixed period of time. Contracts are typically DBFM-O, and public authorities contribute eg. a yearly fee, land allocation, tax cuts, etc. Limits indebtedness and operating costs, but is actually more costly than a bank loan, expertise is lost, it grows the influence of private actors, confidential PPP contracts make them less democratic. 

\textbf{Citizen's participation frameworks}: urban politics taking becoming participatory, in order to bring policy-makers closer to the people, and to be open and responsive. Limitations include biases and over/under-representation, gap of information and knowledge between experts and citizens, and not focusing on the wider picture. Plus, how much power is actually transferred to citizen through participatory planning? (Arnstein, 1969)

\textbf{Strategic planning}: project-based urban planning, a consensus building tool to mobilise diverse stakeholders around how the city should be spatially developed.

\textbf{The entrepreneurial city}: the mainstream in urban/regional policy, views cities like businesses. A successful city must have business-friendly environments, in order to be competitive and to grow. There are three dimensions: inter-urban competition whereby cities adopt a competitive and entrepreneurial stance on urban policies; supply side economics a development strategy whereby investments in the quality of a place will raise urban competitiveness and economic growth, and create trickle-down effects; and symbolic policies like city branding.

\pagebreak
\section{Culture as Urban Regenerator?}

\textit{tldr; to what extent can culture act as an urban regenerator, and what are the consequences? Does it work, for whom, and in whose interest? Multiple dimensions of culture as an urban regenerator, including powerful success stories; culture as urban entrepreneurialism; and the creative class.}

\alignedmarginpar{Kanal Pompidou, Roubaix}

\textbf{Culture-led urban regeneration}:\alignedmarginpar{Roubaix} investing locally in cultural externalities (consumption facilities, production facilities, cultural amenities) and expect a wide range of beneficial impacts (economic, urban, symbolic, social). Popular with policy-makers, but little evidence that it achieves expected results. 

\textbf{Power of success stories}: success stories propose accessible `policy fixes', through `mobile policies' that are one size fits all.

\textbf{Guggenheim effect}: using a flagship cultural brand like Guggenheim, as part of a broad urban regeneration strategy. The objective is to boost a transition towards a service-based economy, focusing on tourism and advanced business services. The strategy is to use public investments in the first place, in order to attract flagship brands. Impacts include a tourist boom, a concentration of budget into a single cultural development, poor results in business attraction, quasi-privatisation of urban development, a reliance by public authorities on land valorisation to pay for investments.

\textbf{Culture as urban entrepreneurialism}: \alignedmarginpar{European Capital of Culture} cities investing in culture, in order to develop an identity and a cultural economy. Cities are places of promotion, and should showcase themselves. Budgets are concentrated in a selection of places and institutions, at the expense of others, and there are uncertain longterm impacts regarding the use of equipment and infrastructure, and tourism visits.

\textbf{The art of rent}: (Harvey, 1989), globalisation and less protectionism have diminished the potential for monopoly rents. But since capital requires monopoly powers, thus it employs culture to reassert the monopoly, through uniqueness and authenticity. 

\textbf{Creative class theory}: the presence of a `creative class' is what makes urban and regional economies thrive. The create class must be attracted to the city, by creating a ``quality of place'' or a ``people's climate''. This is achieved by Tolerance (towards diverse, cosmopolitan individuals), Talent (who needs to be attracted) and Technology (high tech and modernity). This is the reverse of the conventional strategy whereby businesses attract workers, here, the creative class (workers) attract firms. Critiques are: creativity is restricted to commodification; social classes are not mentioned, but lower classes excluded; there is little evidence to support that soft factors attract people to a city more than hard factors (jobs, income, education, etc.); boosts gentrification and displaces people. The creative class theory is popular amongst policy makers because it focuses on improving the quality of life (but for who?) and anticipating trickle-down effects, contra investments in local welfare.

\textbf{Creative class theory alternatives}: artivism \alignedmarginpar{Hamburg}, paying closer attention to the diverse landscape of art and culture that is below the radar of policy makers, and pushing for non-entrepreneurial projects, whose cultural goals are detached from place competitiveness and attractiveness. Use culture and creativity as vehicles for emancipation, not place-marketing.

\pagebreak
\section{(Contra) Gentrification}

\textit{tldr; gentrification takes many forms, is present globally, and is always associated through dispossessions (in different forms). It can be market let, state led, culture led, but always involves a profit-seeking actor. Resistance exists to maintain popular spaces.}

\textbf{Origins of gentrification}: \alignedmarginpar{Ruth Glass, 1964} a residential process (rehabilitation of housing displacing incumbent residents), sporadic and limited (restricted to certain neighbourhoods), an unusual direction of neighbourhood change (against Chicago School), more than a descriptor of a particular process of urban change (criticising the wealthy). The term is politically loaded, as opposed to ``urban revitalisation'' and has a critical intent built in.

\textbf{Gentrification as a generic concept}: not bound to a time and place. It is taking place globally, is rooted in the political economy of contemporary capitalism, and states (neoliberal or authoritarian) play a key role $\Rightarrow$ \textbf{planetarisation of state and capital gentrification strategies}

\textbf{Gentrification}: any form of reconfiguration of popular spaces to the advantage of socially advantaged groups, resulting in various forms of dispossession of ordinary users of these spaces\alignedmarginpar{Lees, Shin, Lopez-Morales}. Gentrification is a form of class power over space; a power over the production of space; to remake popular spaces.

\textbf{Rent gap theory}: \alignedmarginpar{Neil Smith, 1979} explains how investors choose neglected neighbourhoods. Criticises the `back to the city movement' which explains gentrification as a change in middle-class lifestyle. It is the difference between the capitalised ground rent under present use, and the potential ground rent under the `highest and best' market use\alignedmarginpar{Airbnb}. It also requires collective social action, to bring together necessary actors to actually bring investments, produce and close the rent gap.

\textbf{State-led gentrification}: \alignedmarginpar{Brussels' Canal} gentrification is part of the political economy of urban development and involves a combination of diverse policies and state interventions. State interventions include material interventions (installing equipment), regulatory interventions (allowing more floors to be built), and symbolic interventions (marketing campaigns, place promotion). Gentrification policies do not have a name, like zero-tolerance policing, mixed-community policy, creative city policies \alignedmarginpar{Lees, Shin, Lopez-Morales}, and are stealthy \alignedmarginpar{Bridge, Butler, Lees}

\textbf{Material and symbolic remaking of popular spaces}: a new-middle class moving into gentrifying areas contribute to adjusting material and symbolic characteristics of working-class neighbourhoods to cater to new middle-class taste. Creates a new class identity\alignedmarginpar{Ley}, whereby landscapes of ``authentic'' or ``alternative'' consumption are appropriated to form a distinct social identity.

\textbf{Dispossessions}: dispossession is built in to the concept of gentrification. They can take different forms through various mechanisms, such as direct displacement (people who are displaced by economic, physical and legal motives), indirect and exclusionary displacement (not necessarily the displaced but those who are not allowed in), neighbourhood resource displacement (losing a sense of place or of a supportive local government), community displacement (loss of influence of incumbent communities).

\textbf{Contra-gentrification}: resistance to gentrification holds existing socio-spatial configurations in their present state. It includes everyday, ordinary resistances, it takes different forms like collective mobilisations, claiming the right to stay put, etc\alignedmarginpar{Annunziata, Rivas Alonzo}. Not to confuse with resilience, which is how space adapts to change from unquestioned external forces.

\pagebreak
\section{Ordinary Urbanisms}

\textit{tldr; how inhabitants produce space, alongside capital and state actors; inhabitants contribute to the production of space in their own right, through their ordinary uses of urban space and their spatial practices as city users}

\textbf{Learning from above and from below}: \alignedmarginpar{Ananya Roy, 2009} the political economy is inclined to look at cities and urban change from above, ie. the perspective of the state. Learning from below is coming from the perspective of the governed, rather than from the perspective of those in public/private governing seats.

\textbf{Ordinary cities}: \alignedmarginpar{Jennifer Robinson, 2006}based on the fact that many large cities are left off maps like that of global cities, it is a new heuristic that would apply to all cities, as opposed to a discrete category. A call for \textbf{cosmopolitanising or provincialising} to take inspiration from a wide range of cities. Forms the basis of post-colonial studies: moving away from EuroAmerican hegemony, focusing on ``off the map'' places, paying attention to informality and everyday lives of urban poor.

\textbf{Epistemologies of the South}: \alignedmarginpar{Boaventura de Sousa Santos, 2014} how do we learn and produce knowledge, where do we look to? The global south is not a geographical concept, but a metaphor for the suffering caused by capitalism and colonialism. The global South is present in the northern hemisphere.

\textbf{Subaltern urbanism}: \alignedmarginpar{Ananya Roy, 2009} is about the situatedness of urban knowledge, and acknowledging that all cities can be places of theory-making. \textbf{Urban informality}\alignedmarginpar{Cureghem} is a key dimension of subaltern urbanism. It is a mode of urbanisation that can take multiple forms, and is always distanced from the formal spatial planning rules or land regulations.

\textbf{Popular urbanisation}: \alignedmarginpar{Streule et al.} \alignedmarginpar{Geçekondu} the collective agency of subaltern populations in the production of space, through collective experiences, including mobilisations (eg. claiming land, getting access to services denied/not offered by public or private institutions). Popular urbanisations are often excluded by policies like slum-free cities and large-scale redevelopment. 

\textbf{The city seen from below}: \alignedmarginpar{Rosa Bonheur, 2019}\alignedmarginpar{Roubaix} a socio-ethnographic approach to look at what the working-class build, exchange, and produce to ensure their livelihoods, rather than focusing on their precarity and domination. What the working-class does is \textbf{subsistence work}

\textbf{Subsistence work}: the economic activities of the working-class that ensures their livelihoods (home production of goods and services, low-budget self-renovations, shopping works, paper works, economic activities meeting local, low-budget demands). Subsistence work is \textbf{spatial} because it relies on resourceful spaces, and it helps to reproduce these spaces. 

\textbf{Popular centrality}: a set of resourceful spaces made by and for populations in socially dominated positions, and those spaces are absolutely central to the daily lives of these populations. Spaces of popular centrality are resistance spaces vis-à-vis gentrification\alignedmarginpar{Marolles}. The production of popular centralities is inscribed in the logics of \textbf{unequal spatial development}. It is because capital has withdrawn from certain cities, and because urban policies are turning away from the needs of the working class, that subsistence work has a strong impact on local space.

\pagebreak
\section{On Urban Alternatives}

\textit{tldr; urban alternatives sit between state-led and popular urbanism; they offer an alternative to neoliberal city, driven by capital growth and accumulation, but often have to work with political frameworks/urban strategies whose ambitions may clash with alternative urbanism. The Right to the City is a theory as an antidote to the mainstream urban entrepreneurialism.}

\textbf{Urban alternatives characteristics}: initiatives coming from the bottom up\alignedmarginpar{Grassroots, citizen-led} but not excluding relationships or support from the state; having alternative value systems, by being distanced from market-growth rationalities\alignedmarginpar{NGO} and from neoliberal approaches that do no address problems of inhabitants directly but anticipate trickle-down effects; and, having a purpose of alleviating or counteracting inequalities in the urban environment.

\textbf{Community Land Trusts}: an initiative between state-led and popular urbanism, it is a non-profit, membership-based organisation that holds and manages the land as a trust on behalf of a given community. The land is taken out of the market logic for a leasing period (eg. 50 years) and a monthly fee (eg. 10€), which reduces the cost of the housing.

\textbf{CLT Critiques}: the Brussels government is supporting CLTs, put show extremely limited ambitions in the production of social housing. Creates a residualisation of social housing, where publicly-subsidised housing creates a safety net only for low(est)-income households. CLTs are developed in areas otherwise targeted as zones for 'urban revitalisation', aimed at increasing attractiveness towards investors and middle-class.

\textbf{Critical lessons on urban alternatives}: two proposed conditions for urban alternatives are 1, a clear challenge to the mainstream `neoliberal cities', and 2, a macro ambition aiming at changes to the many. An alternative label is not a guarantee of immunity against the neoliberal mainstream\alignedmarginpar{alter-washing}, and a collection of alternative niches do not make a comprehensive alternative politics. There is a need for a wider, alternative politics of the city.

\textbf{Right to the City}: is the power of producing urban space for the sake of livelihood, rather than for the sake of capital accumulation, urban growth or social control\alignedmarginpar{Missing from CLT}. The Right to the City could be a prospect for a wider, alternative politics of the city. However, the RttC is often integrated by transnational institutions as a right to access and use existing urban services, it is depoliticised into urban ``quality'' or ``innovation'', and risks installing dominant actors to co-opt and repress selected grass-roots initiatives over others.

\textbf{Fare-Free Public Transport}: an alternative transport initiative to urban entrepreneurialism, between state-led and popular urbanism. A system where fares are fully subsidised by an entity, on publicly governed, if not owned, transport infrastructures and networks.

\textbf{Perspectives on FFPT}: they are uneconomical and irrational (neo-classicism), by decreasing the value of transport and network revenue which creates financial instability. They fail to increase service quality and to generate a modal shift away from cars. 

\textbf{Critical perspectives on FFPT}: are they really tackling transport-related inequalities, and changing the capitalist paradigm of transport planning?

\textbf{Analysing alternative policies}: alternatives need to be studied on a case-by-case basis in order to understand: who is actually affected (age groups, income levels, occupations, etc.); the participation and power relations of passengers and workers with the policy; whether they resist state violence\alignedmarginpar{Fare controls targeting undocumented migrants}; inter-municipal solidarity and territorial identity; the privatisation of public transport and reduction of costs.

\if{False}
\pagebreak
\section{Key words}

\textbf{Accumulation by dispossession}:

\textbf{Internalisation of capitalism}: our self worth is directly defined by our productivity

\textbf{Ordinary urbanisations}

\textbf{Subaltern urbanism}

\textbf{Popular urbanisation}

\textbf{Urbanisation of neoliberalism}

\textbf{Neoliberalisation of urbanism}

\textbf{Popular centrality}

\textbf{Ordinary urbanisation}

\textbf{}

\textbf{}
\fi
\end{document}
