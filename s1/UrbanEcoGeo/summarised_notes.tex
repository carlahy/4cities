\documentclass{article}

\usepackage[utf8]{inputenc}
\usepackage[left=1.5in,right=1.5in,bottom=1in]{geometry}
\setlength\parindent{0pt}
\setlength{\parskip}{1em}
\setcounter{secnumdepth}{0}
\usepackage{outlines}
\usepackage{graphicx}
\graphicspath{ {imgs} }
\usepackage{hyperref}
\title{Urban Economic Geography - Summarised Notes}
\author{Carla Hyenne }

\newcommand{\alignedmarginpar}[1]{%
        \marginpar{\raggedright\small #1}
    }

\begin{document}

\maketitle

\tableofcontents

\pagebreak

\section{The city as a social product}

\textit{tldr; emphasis on the city as a product dynamic relationships of societies with urban space, and not a set of necessary attributes that define `the city', and not an actor. The city is not only material, but the set of all forces within it (political, economic, social,...)}

\textbf{Urban triumphalism}: the city is a site of progress and a way to prosperity. The major contemporary challenges are urban challenges, and the urban is where solutions can be found. Eg. green, smart, productive, participative, etc., city \alignedmarginpar{Glaeser}

\textbf{Critique of urban triumphalism}: it is a purely pro-growth perspective, which contradicts its sustainability goals. It is about finding solutions to pre-defined challenges (how to...) which turns urban issues into techno-management issues where copy/paste can be found and match city leader's views.

\textbf{Breaking with urban triumphalism}:  the city should not be seen as an actor but as dynamic space, where actors with different resources, interests, inspirations, interact in diverse ways (conflict, resistance, mobilisation, collaboration, solidarity...)\alignedmarginpar{Marcuse}

\textbf{P. Marcuse, \textit{The City as a Perverse Metaphor}, 2005}: seeing the city as an actor excludes a set of the population who does not share the same interests. Not all the city is international, competitive, etc., even if some firms and some people are.

\textbf{Critical urban studies}: contra-naturalising, there is nothing natural about cities and they are not living organisms; contra techno-managerial, focus instead on tensions and contradictions, not solutions to standardised challenges.\alignedmarginpar{Lefebvre}

\textbf{Production of urban space}: urban space is organised to reflect and support a particular society (capitalist, soviet, post-socialist, etc.)\alignedmarginpar{LA vs. Moscow}. Cities do not have a set of attributes necessary to be a city, but they are the product of dynamic relationships of societies with urban space.

\textbf{Creative destruction}: the production of space is always a work in progress, with inherited socio-spatial configurations reshaped by new logics.\alignedmarginpar{Place de Brouckère, Senne}

\textbf{Henri Lefebvre}: any political economy implies a spatial order. Urban space produces a certain type of society and vice versa. The urban is site for social struggles, which must appropriate space.

\textbf{New epistemology of the urban}: concentrated urbanisation, extended urbanisation, differential urbanisation \alignedmarginpar{Brenner and Schmidt}

\textbf{Concentrated urbanisation}: spatial clustering of population, mean of transportation, infrastructure, investment, ie. agglomeration

\textbf{Extended urbanisation}: activation and transformation of places, territories, landscapes in relation to agglomeration processes; uneven thickening and stretching of urban fabric across the planet

\textbf{Differential urbanisation}: creative destruction of implosion/explosion of socio-spatial organisation; production of new urban 'potentials' for the appropriation of urban configurations and for the production of radically new forms of urban space

\pagebreak
\section{The production of urban space under capitalism}

\textbf{Harvey's theoretical project}: ``to integrate an understanding of processes of urbanisation and built environment formation into the general theory of the laws of motion of capital''. Understand how urbanisation helps us understand capitalism

\textbf{Capitalism}: a system of economic production based on the circulation/exchange of privately-owned capital, geared towards capital accumulation. An ever expanding system because capitalism must grow. It has a geography: it isn't the same everywhere and is embedded in social, cultural and institutional configs. It has a history: it evolved over time.

\textbf{Capitalism and urbanisation}: cities were not created by capitalism, but the rise and extension of capitalism is deeply linked to urbanisation processes.

\textbf{Crises of capitalism}: a crisis prone system that is resilient. Accumulation is at risk when surplus capital has no profitable outlet because the circulation of capital cannot continue.\alignedmarginpar{China's Evergrande}

\textbf{Technological fix}: reinvesting capital surplus in new or improved production capacities, ie. sectors fuelled by tech innovations.

\textbf{Financial fix}: reinvesting capital surplus into financial assets. The `financialisation' of capitalism

\textbf{Spatial fix}: reinvesting capital surplus into the production of profitable built environments. Eg. new spaces of industrial production, of transport and logistics, of consumption, of rent speculation, new urban markets like AirBnB

\textbf{Contradictions of capitalism}: capitalism produces space as an outlet for capital surplus, only to later destroy this space so that it can again produce it and ensure capital circulation\alignedmarginpar{Harvey}

\textbf{Any fix is short lived}: as soon as the capital is invested, the fix is consumed. It will only be profitable once again when the space is destroyed and reinvested\alignedmarginpar{Paris' Haussmanisation, post-was capitalism and urbanisation}

\textbf{Surburbanisation}: a lifestyle change that had social, economic and political implications. Social because it created a new lifestyle and new demands; economic to create and meet those demands (new products like home equipment); political because middle class demands shifted focus from community to private ownership and defence of private space, suburban votes became conservative republicans.\alignedmarginpar{American suburbs, Belgian suburbs}

\textbf{Theory of urbanisation under capitalism}: capitalism has an insatiable addiction to the production of profitable space, because this overcomes capitalism's internal contradictions. Under capitalism: the mobilisation of urban change for the sake of capital accumulation is permanent, and is at the forefront in times of crisis; any spatial fix is short-lived, because the profitable way out of a crisis paves the way for the next crisis; fixing space has ramifications on the building environment, modes of consumption, mobility patterns, cultural and political subjectivities; any spatial fix comes with patterns of uneven development.

\textbf{Uneven spatial development}: under capitalism, investment in some places result in disinvestment elsewhere. The disinvested areas will in the future be sites for investment once again. The spatial fix is cyclic: capitalism does not solve a crisis, but moves them geographically. Creative destruction displaces communities who will not be able to return afterwards.\alignedmarginpar{Detroit}

\pagebreak
\section{The Neoliberal City}

\textit{tdlr; zooming in to the present conjuncture of the production of space under capitalism, ie. urbanisation under neoliberal capitalism, or the ‘neoliberal’ city. Urban spaces are sites and vehicles for neoliberalism: sites because that is where the accumulation takes places, and vehicles because of the financialisation of the real estate market, and it’s speculation. The 21st century city is a city of rents.}

\textbf{Neoliberalism}: intellectual and political proposals that extend market mechanisms and ethics of competition to an ever-wider spectrum of social activities, based on strong state intervention (Pinson, 2020). A class project to remove restrictions in the way of accumulation (neomarxism, Harvey)

\textbf{Neoliberal turn}: since the 1980s there is a growing gap between the profit rate (return on productive capacities) and accumulation rate (share of profit reinvested in productive capacities). Capital has grown, but the surplus hasn't been accumulated (reinvested in productive capacities).

\textbf{Trans-nationalisation}: surplus capital being invested externally. Companies globalised, off-shored and outsourced, there were mergers and acquisitions, FDI (investors establishing long-lasting interests in foreign enterprises)

\textbf{Financialisation}: surplus capital going into financial markets. Results in an increasing concentration of capital in the hands of the financiers. Financialisation uses asset valuation strategies, the purpose is to valorise assets and speculate to maximise rental yields or make capital. gains on resale. The largest asset managers (eg. BlackRock, Vanguard) have greater wealth than that of countries like Germany, France, Switzerland (based on GDP)

\textbf{Cities as critical sites capital accumulation}: urban space is where the accumulation of capital takes place. Multiple frameworks illustrate this: global cities who control ever-more of the world's networks and flows (Sassen); worlding cities emerging as global cities (A. Roy, A. Ong); globalisation from below, is the production and consumption associated to transnational communities beyond political borders (A. Portes); inconspicuous cities beyond global cities that global cities are investing in, thus are crucial for capital accumulation (Choplin, Pliez); planetary urbanisation whereby places far from the urban core and peripheries are strongly linked to urbanism and urbanisation (Brenner and Schmid)

\textbf{Cities as vehicles for capital accumulation}:\alignedmarginpar{Battersea Power Station} urban space is financialised through the real estate market and its speculation. Assetisation turns real estate into financial assets. Urbanisation becomes speculative, land is mobilised for the maximisation of rent extraction, and the 21st century metropolis becomes a city of rent\alignedmarginpar{Charnock et. al}.

\textbf{21st century city of rent}: neoliberal urbanisation visible by: the rise of corporate landlords and buy-to-let housing markets; speculative projects tailored to meet investors' requirements first; increased tendency of urban governments to rely on real estate valorisation to finance public utilities; growing un-affordability as a system feature, because profit is necessary. $\Rightarrow$ uneven geography with investments rising in one place (city growth) leaving other places behind (shrinking cities).

\pagebreak
\section{New Urban Policies: Urban Governance and the Entrepreneurial City}

\textbf{}

\textbf{}

\textbf{}

\textbf{}

\textbf{}

\textbf{}

\textbf{}

\textbf{}

\textbf{}

\textbf{}

\textbf{}

\pagebreak
\section{L5}

\textbf{}

\textbf{}

\textbf{}

\textbf{}

\textbf{}

\textbf{}

\textbf{}

\textbf{}

\textbf{}

\textbf{}

\textbf{}

\pagebreak
\section{L6}

\textbf{}

\textbf{}

\textbf{}

\textbf{}

\textbf{}

\textbf{}

\textbf{}

\textbf{}

\textbf{}

\textbf{}

\textbf{}

\pagebreak
\section{L7}

\textbf{}

\textbf{}

\textbf{}

\textbf{}


\textbf{}

\textbf{}

\textbf{}

\textbf{}

\textbf{}

\textbf{}

\textbf{}

\pagebreak
\section{L8}


\textbf{}

\textbf{}

\textbf{}

\textbf{}

\textbf{}

\textbf{}

\textbf{}

\textbf{}

\textbf{}

\textbf{}

\textbf{}

\section{Glossary}

\textbf{Accumulation by dispossession}:

\textbf{Internalisation of capitalism}: our self worth is directly defined by our productivity

\section{Ordinary Urbanisations}

\textbf{Ordinary urbanisations}

\textbf{Subaltern urbanism}

\textbf{Popular urbanisation}

\textbf{Urbanisation of neoliberalism}

\textbf{Neoliberalisation of urbanism}

\textbf{Popular centrality}

\textbf{Ordinary urbanisation}


\textbf{}


\textbf{}
\end{document}
