
\documentclass[12pt]{article}

\usepackage[utf8]{inputenc}
\usepackage[margin=1in]{geometry}
\usepackage{setspace} \onehalfspacing
\setlength\parindent{0pt}
\setlength{\parskip}{1em}
\setcounter{secnumdepth}{0}
\usepackage{outlines}
\usepackage{graphicx}
\usepackage{caption}
\captionsetup{justification=centering, width=5in}
\graphicspath{ {imgs} }
\usepackage[normalem]{ulem}
\usepackage{hyperref}
\usepackage{color,soul}
\normalsize	

\usepackage{comment}
\pagenumbering{gobble}

\usepackage[
backend=biber,
style=apa,
citestyle=authoryear,
sorting=nyt,
]{biblatex}
\addbibresource{exam.bib}

\title{Urban Economic Geography Exam}
\author{Carla Hyenne\\[3ex]4 CITIES}
\date{January 14th, 2022}

\begin{document}

\maketitle

Question 1: In capitalist societies, dynamics of urbanisation structurally entail the production of spatial inequalities.

Question 2: TODO

\pagebreak
\singlespacing

If every political economy implies a spatial order \parencite{Lefebvre}, then capitalist societies exist by design. Since the public and private sectors have the most influence on the production of space, their strategies directly reproduce a capitalist society, which is by definition unequal. Accumulation on one side of the equation, causes dispossession on the other.

Thus, all profit-driven urbanisation dynamics, whether publicly or privately driven, systematically produce spatial inequalities. 
On the side of the market, private actors are looking for opportunities to invest surplus capital. In the public sector, governments play an important role by supporting laws and regulation that facilitate the growth of private enterprises. Their support can include anything from promoting projects that anticipate trickle-down effects \parencite{TODO}, to value capture financing, and providing tax breaks and subsidies.
The assessment of profitability can be based on speculation (a result of the financialisation of real estate) or on the potential rent to be extracted from new land uses \parencite{Neil Smith}.

Because capital accumulation is a fundamental law of capitalism, surplus capital must always be reinvested. However, circulation is obstructed when capital is accumulated beyond investment capacities, thereby causing a boom and bust cycle \parencite{TODO}. This is one internal contradiction of capitalism, as articulated by Harvey \citeyear{TODO}.

With regards to urbanisation, this deficiency is overcome with a spatial fix, which systematically creates socio-spatial inequalities \parencite{Harvey}.
A spatial fix is the investment into the production of space, when capital has no other outlet. 
This creates a mechanism of creative destruction, where the space in question will be neglected to the benefit of another, more profitable space, but will at a future date regain its profitable appeal, and be built anew when capital once again uses the urban as a profitable outlet.
The contradiction here is that this spatial fix is consumed as soon as the capital is invested.

Such profit-making urbanisation strategies create spatial inequalities on different scales. Locally, investments upscale certain neighbourhoods but neglect others, causing their decline. Regionally, certain cities like capitals or global cities attract investments, depreciating cities. Globally, capital is concentrated in the hands of wealthy financiers, predominantly in the global North, extracting capital from cities in the global South, which becomes a metaphor for the ``suffering caused by capitalism’’ \parencite{de2015epistemologies}.

Capitalism’s internal contradictions, where surplus is generated beyond the means of investment, but must nonetheless be placed, is accomplished by investing into the production of space. This systematically creates spatial inequalities by means of disinvestments. This is, until another crisis comes along and disinvested places again deemed profitable.

Nonetheless, there are forces that resist spatial inequalities systematically created by profit-making projects. For one, the neoliberal state still plays an important role in regulating capitalism, and cities under capitalism could be much worse. Then, not-for profit projects that serve disadvantaged populations, such as social housing, directly address those who are most affected by the spatial inequalities. Other urban alternatives?


\pagebreak

Question 2 TODO

\printbibliography 

\end{document}
