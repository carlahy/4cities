
\documentclass[12pt]{article}

\usepackage[utf8]{inputenc}
\usepackage[margin=1in]{geometry}
\usepackage{setspace} \onehalfspacing
\setlength\parindent{0pt}
\setlength{\parskip}{1em}
\setcounter{secnumdepth}{0}
\usepackage{outlines}
\usepackage{graphicx}
\usepackage{caption}
\captionsetup{justification=centering, width=5in}
\graphicspath{ {imgs} }
\usepackage[normalem]{ulem}
\usepackage{hyperref}
\usepackage{color,soul}
\normalsize	

\usepackage{comment}
\pagenumbering{gobble}

\usepackage[
backend=biber,
style=apa,
citestyle=authoryear,
sorting=nyt,
]{biblatex}
\addbibresource{exam.bib}

\title{Urban Economic Geography Exam}
\author{Carla Hyenne\\[3ex]4 CITIES}
\date{January 14th, 2022}

\begin{document}

\maketitle

\pagebreak
\singlespacing

\textbf{In capitalist societies, dynamics of urbanisation structurally entail the production of spatial inequalities.}

If every political economy implies a spatial order \parencite{lefebvre2009state}, then capitalist societies exist by design. Since the public and private sectors have the most influence on the production of space, it is their strategies that directly reproduce a capitalist society. By definition, this society is unequal. Accumulation on one side of the equation, causes dispossession on the other.

Thus, all profit-driven urbanisation dynamics, whether publicly or privately driven, systematically produce spatial inequalities. 
On the side of the market, private actors are looking for opportunities to invest surplus capital. In the public sector, governments play an important role by supporting laws and regulation that facilitate the growth of private enterprises. Their support can include anything from promoting projects that anticipate trickle-down effects, to value capture financing, and providing tax breaks and subsidies.
The assessment of profitability can be based on speculation (a result of the financialisation of real estate) or on the potential rent to be extracted from new land uses \parencite{smith1979toward}.

Because capital accumulation is a fundamental law of capitalism, surplus capital must always be reinvested. However, circulation is obstructed when capital is accumulated beyond investment capacities, thereby causing a boom and crisis cycle. This is one of the internal contradictions of capitalism \parencite{harvey2014seventeen}.

With regards to urbanisation, this deficiency is overcome with a spatial fix \parencite{harvey2001globalization}.
A spatial fix is the investment into the production of space, when capital has no other outlet. 
It creates a mechanism of creative destruction, where space is destroyed to welcome a new, profitable build.
The contradiction here is that the `fix' is consumed as soon as the capital is invested. This systematically creates spatial inequalities by means of disinvestments. With time, the space in question will be neglected to the benefit of another with more profit potential. But, the disinvested space will in the future regain its profitable appeal, and be built anew when capital again needs a profitable outlet.
In this sense, geographical inequalities are necessary for capitalism to ``periodically reinvent itself’’ \parencite[p.157]{harvey2014seventeen}.

Profit-making urbanisation strategies create spatial inequalities on different scales. Locally, investments upscale certain neighbourhoods but neglect others, causing their decline. Regionally, capitals and global cities attract investments, depreciating other cities and peripheries. Globally, capital is concentrated in the hands of wealthy financiers, predominantly in the global North, who expand their investments internationally and focus on the world regions that generate the most profit. Or, they loan money for public infrastructure projects to developing countries, with high interest rates, indefinitely indebting them. 

Nonetheless, there are forces that resist spatial inequalities systematically created by profit-making projects. For one, the neoliberal state still plays an important role in regulating capitalism - capitalist cities \textit{could} be much worse. And, not-for-profit projects that serve disadvantaged populations, such as social housing, directly address those who are most affected by the spatial inequalities.

\pagebreak

\textbf{From software engineering to critical urbanism}

Coming from the field of software engineering, where space for criticality is limited - what is binary is not up for debate - I find nuances in urban studies extremely interesting. The fact that no solution is proposed, even by the most prominent scholars like Harvey who make a point of not having answers, shows the complexity of our societies. 

Unfortunately, I found that the popular, non academic literature on cities I had access to prior to this semester did not often offer a critical view of cities (\textit{Smart Cities} \parencite{townsend2013smart}, \textit{Happy City} \parencite{montgomery2013happy}...). 

Before this course, I couldn’t eloquently articulate what I wanted from urban studies. I was interested in `the urban’ at large, as a complex environment with multiple forces (economic, political, and especially social) more than the built environment as a material space. Hence why I chose ‘urban studies’ and not ‘development’, or ‘planning’.
I also wanted the necessary keys to debate urban issues from a different perspective than the mainstream, which often doesn’t have a social aspect. I didn’t know it at the time, but I can now label the mainstream with triumphalist, neoliberal, and entrepreneurial scripts.

Thus, the criticality of the course opened my eyes and gave me the perspective I was looking for, but couldn’t articulate. 

What I mean by criticality isn’t only studying Harvey and Lefebvre, and criticising Florida\footnote{Criticising him is not hard to do, but I truly wonder what I would have thought of Florida prior to this Masters. It’s impossible for me to know now.}. But more importantly, that we can be critical of all projects for the city - and ask, for whom? Using Lefebvre’s framing of the urban as first and foremost a \textit{social} space, helps me keep this question in mind.
As we discussed, even successful, anti-capitalist, pro-social alternatives that aim to resist the systems we deconstruct, can be critiqued. For example, Community Lands Trusts and fair-free public transport, which on the surface seem to be a net positive for the city and the people, but in reality still are not perfect.

I also wonder what I would have thought about the urban regeneration projects in Molenbeek, or the Kanal Centre Pompidou before the course. If I’m honest with myself, I probably would have thought it beneficial for the neighbourhood. I would not have thought about the (in)direct displacement of residents, or the fact that a contemporary art centre would not address pressing socio-economic issues. This is almost embarrassing to admit. 

Therefore, criticality is what will stay with me during my time as an urban studies student, and beyond. In tech, there is a tendency to go with the fastest, working solution, and to ``fail fast, fail often’’. Unfortunately, this does not translate well to the city, where changes are much slower than in software, and affect people’s lives in a more significant way than technology. Software that works is great, but an affordable home is better, and there is no quick fix to the housing issue.
I appreciate the time we took to take a critical lens, challenge mainstream theories, and deconstruct alternatives.

\printbibliography 

\end{document}
