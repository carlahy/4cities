\documentclass{article}


\usepackage[utf8]{inputenc}
\usepackage[left=1in,right=1in,bottom=1in]{geometry}
\usepackage{outlines}
\setlength\parindent{0pt}
\setlength{\parskip}{1em}
\setcounter{secnumdepth}{0}

\title{Urban Analysis I - Exercises}
\author{Group F}

\begin{document}

\maketitle

\tableofcontents

\section{Exercise I}

Checkout the statistical figures and maps of your neighbourhood with your group on monitoringdesquartiers.brussels. Discuss the demographic profile of your neighbourhood, for example housing, ethnic composition, age structure, educational or employment level, etc. What is particularly striking about your neighbourhood? Amongst the various documents gathered for the portfolio, select one figure that demonstrates the particularity of your neighbourhood. 

\begin{itemize}
	\item Think about how to represent this figure
	\item How would you explain what you see in the figure?
	\item think about locating the neighbourhood's social characteristics across space (within the city) and across time
\end{itemize}

Question to answer: \textbf{Did social characteristics of the neighbourhood change over time? How do the characteristics of the neighbourhood differ from other neighbourhoods in Brussels?}

\subsection{Notes}

- relatively wealthy (low number -18yos live in a home with no labour income, higher taxable income 15-18K, ), this is a trend with seconde couronne

- more female than male

- higher density of >65yo, this is a trend in the deuxieme couronne and generally when you go from centre to outskirts. This shapes the dwellings, because older people are less likely to move often or rent. Thus dwellings are houses, large, expensive (compared to shared flats, houses split in multiple homes, cheaper)

- homes are generally smaller than in the country. Within Brussels, the centre has smallest households and the seconde couronne the highest. Still, this is lower than national average. To understand this, cross-reference with the number of single households under 30yo, for eg. It makes sense that Dries has larger households since it is an older population, who is likely to have kids and be well off

- education: low number of kids in Brussels go to school in/near their neighbourhood, true for Dries (parents choose schools based on reputation, rather than proximity)

- population is mostly European, both from EU or outside it (there is a clear contrast of European population concentrated in the south-east and quartier europeen, compared to rest of Brussels). Interesting that neighbourhoods directly to the east of Dries (Watermael Boifort) have really low numbers non-EU Europeans

- low number of African (north and subsaharan) compared to rest of Brussels. Again, trend with rest of seconde couronne/south east 

- type of housing mostly private houses (not apartments), with a high number of rooms per person (2+) and often lived in by owner (trends in the seconde couronne)

\subsection{Redaction}

One particularly striking fact of our neighbourhood is the amount of social housing it has compared to the surrounding south-east neighbourhoods of the `seconde couronne'. 

The possible explanations for this are:

\begin{outline}
	\1 There was unbuilt, available land in Dries that the municipality could use to build social housing, as the demand continues to increase. This would be verifiable by looking at the date of the buildings, and confirming whether there was anything there previously; also looking at surrounding neighbourhoods, and whether they have available land or not; if yes, perhaps policies made it easier in Dries to built social housing?
	\1
\end{outline}













\end{document}