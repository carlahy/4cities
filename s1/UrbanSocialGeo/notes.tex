\documentclass{article}


\usepackage[utf8]{inputenc}
\usepackage[left=1in,right=1in,bottom=1in]{geometry}
\setlength\parindent{0pt}
\setlength{\parskip}{1em}
\setcounter{secnumdepth}{0}

\title{Urban Social Geography}
\author{Carla Hyenne }

\begin{document}

\maketitle

\tableofcontents

\pagebreak

\section{Urban Geographical Traditions}
\date{September 27th, 2021}

Draw a general outline of how urban geography has been practiced in the last 50 years, and how it interrogates some of urban geography's main concepts and approaches.

\subsection{What is "the urban"?}

\subsubsection{Urban Age}

We are living in the Urban Age, where the population growth is proportional to urban growth. Since circa 2005, the majority (more than 50\%) lives in urban areas, or cities, as opposed to rural areas.

Urbanization is a \textbf{global phenomenon}. It started with the global North (USA, Western Europe) in the 19th century, spread to South America, USSR, Oceania in the 1950s, and Africa, Asia in the 1990s. Urbanisation intensified in the 1990s, and as of the 2000s, Asia dominates the economic growth and urbanisation.

Urbanization isn't equal throughout the world. For example, in 2018, ~40\% of Africa's population was urban, ~50\% in Asia, ~75\% in Europe, ~80\% in Latin America and North America, and ~70\% in Oceania.

Is urbanization really "global"? Are we really in an "urban age"? How we measure the urban depends on statistics and national definitions and categories that diverge. It is impossible to find one definition of the urban. What is understood as "the urban" is a \textit{chaotic abstraction}, and doesn't neatly overlay cities in the spatial sense (boundaries). 

This makes the distinction of the urban vs. the rural, and the categorisation of space in either or, a black box. Is it important to think about the rural? What is difference between the urban and the city?

\subsubsection{What is the difference between the urban and the city?}

The urban is a phenomenon, a process, and elements of the urban can exist outside of the city. 
It is a set of values which direct how we might organise a city.

The city is a built environment, made of concrete. It is a marker in space, it is a physical manifestation of the urban. 
It has boundaries and is political - cities have mayors and elections, for example.

Thus, the urban is greater, theoretically, than the city.

\subsubsection{Urban Conceptualisations}

\begin{itemize}
  \item A distinctive way of life, which can take place in cities but also outside of the city (suburbs, rural, slums)
  \item It epitomises a particular society (capitalist, industrial, fordist, modern, classist...). The urban can therefore be categorised differently depending on the context and time period. 
  \item Projects symbolic power (see city conceptualisations)
\end{itemize}

\subsection{What is "the city"?}

\subsubsection{City Conceptualisations}

\begin{itemize}
  \item A lump of material, the \textbf{built environment}
  \item The non-city is... what? the rural? It is hard to say where the city ends. For examples, large boulevards connecting the city to "outside" spaces may have shops along them, which could be characteristic of the city. But are they still part of the city?
  \item A complex division of labour, with increasing efficiency and surplus, but also inequality
  \item Projects symbolic power, through skyscrapers and impressive architecture that reminds the world of the city's social/political/economic dominance. Think of the CCTV tower in Beijing, World Trade Center in NYC, Burj Khalifa in Dubai. But what is behind this image of spectacular urbanism? Is it a facade?
  \item Is administrative, with administrative boundaries
\end{itemize}

\subsection{The Urban as a Process: Urbanisation}

Urbanisation is:

\begin{enumerate}
  \item \textbf{Demographic process} in which cities gain more and a wider variety of residents, with an increased density
  \item It speaks to the increasing \textbf{globalisation of urban economic, political and cultural influence}
  \item It helps us consider \textbf{how space is organised} through processes of uneven development
\end{enumerate}

Urbanism is:

\begin{enumerate}
  \item Narrowly defined as \textbf{urban design}
  \item Gaining an \textbf{urban attribute}, a psychological and sociological feeling, giving particular meaning to urban space. \textit{Flaneurs, dandys}
\end{enumerate}

Planning is:

\begin{enumerate}
  \item A future-oriented activity, where actors of various types engage to govern how development will take place
\end{enumerate}

\subsection{What is "geography"?}

Geography is the terrain, the typologies, the interconnectedness of space and 'something' within that space; the \textbf{social and physical processes within the context of space}.

Geography is defined by "how" we study, rather than "what", with an emphasis on space. There are different concepts of space: territory, scale, place, and networks.

The \textbf{territory} defines boundaries, and sovereignty of a space (the Brussels capital has sovereignty within Belgium). 

The \textbf{scale} defines the sensitivity of processes (teaching is a small scale, commuting is a large scale, the Brussels capital scale is greater than its territory). 

The \textbf{network} defines hubs and leaks beyond the territory, towards micro-networks (Brussels connected to other cities via a transport network, but also by people living in eg. Ghent and commuting to Brussels to work. This is a leak of taxes and money from the capital to Flanders/Wallonia)

The \textbf{place} is the attachement of meaning, sentiment, to a place (how Brussels is represented in Flemish vs. Wallonia media)

\subsection{What is being "critical"?}

To be critical is to be aware of your own biases, and not cherry-picking your information. To be critical is to fact-check and question, to be reflexive about your own positionality. You should bring up new concepts, and take seriously the experience and position of others. Critical research should be socially relevant and politically engaged. The Chicago School would define it as \textit{"reducing the illusions in society itself"}.

Brenner and Schmidt are critical authors of the Urban Age.

\subsubsection{Epistemological Rules of Thumb}

\begin{enumerate}
  \item There is no universal theory of anything
  \item Every theory has birthmarks: what were the questions, situated in time and space, that gave rise to formulating a search question/theory in a particular way?
  \item Reflexivity on birthmarks is required if you want to be critical, therefore all theories need to be provincialised
  \item Can theories "speak" across contexts?
  \item Engaged pluralism might allow for inter=theoretical conversation and comparison
\end{enumerate}

\subsection{Foundational Approaches}

After WWII, geography moved from purely territorial to plural definitions, from regional (urban) geography (descriptive, map-oriented, idiographic) to a spacial science (nomothetic, method-driven, applied orientation).

The \textbf{materialist approach} has materialist frameworks, is concerned with distribution and social-justice, and agenda-setting: Marxist geography, Structuralist geography, Critical geography, Radical geography, Feminist geography, Critical realist geography

The \textbf{humanistic approach} is about the experienced city, issues of representation and discourse, uses qualitative methods, and is about giving voice: Humanist geography, Cultural urban geography, Post-structuralist geography, Post-colonial geography, Queer geography

\section{Theories of world-city formation}
\date{Octobre 3rd, 2021}

The first session on world and global cities aims to:
\begin{itemize}
  \item Provide a framework to conceptualise and analyse cities under (economic) globalisation
  \item Give an overview of key theories of world-city formation dating back to the 1970s-1980s and onwards
  \item Introduce model-based approaches to the world-city network as a heuristic to map out how cities are positioned on global flows of capital, knowledge, and people
  \item Discuss a number of key critiques of the alleged "world/global cities paradigm", voiced from social constructivist and post-colonial positions
\end{itemize}

\section{Polarization in world/global cities}

\section{Urban segregation: patterns and causes}

\section{Neighbourhood effects and living with diversity}

\section{Cultures of urban research}

\section{Urban cultures}

\section{Transport and cities: a historical hegemony}

\section{Critical perspectives on urban transport}

\end{document}