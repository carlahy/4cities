\documentclass{article}

\linespread{1.1}
\usepackage[utf8]{inputenc}
\usepackage[left=1.5in,right=1.5in,bottom=1in]{geometry}
\setlength\parindent{0pt}
\setlength{\parskip}{1em}
\setcounter{secnumdepth}{0}
\usepackage{outlines}
\usepackage{graphicx}
\graphicspath{ {imgs} }
\usepackage{hyperref}
\usepackage{color,soul}

\usepackage{comment}
\specialcomment{topicsen}{\begingroup\bfseries\scriptsize}{\endgroup}
%\excludecomment{topicsen}

\usepackage[
backend=biber,
style=apa,
citestyle=authoryear,
sorting=nyt,
]{biblatex}
\addbibresource{proposals_refs.bib}

\title{Statements of Interest for Thesis}
\author{Carla Hyenne}
\date{}

\begin{document}

\maketitle

\textbf{Topic \#1 (preferred) City swimming: can reclaiming bodies of water make cities greener, healthier, and more democratic?}

\textbf{Potential supervisors:} Francesc Baró, Nicola da Schio

\textit{Keywords: public space, leisure space, urban sustainability, urban environmental justice}

Some cities are known for their urban swimming, especially in Switzerland or Scandinavia. Basel, Oslo, and Stockholm have fantastic infrastructure that makes it safe, accessible and free for anyone to take part, any time of day and even year. Although urban swimming can be taken for granted in these cities, it wasn't always possible. 
In 20th century Switzerland, swimming was banned because of the diseases and pollution in the water \parencite{ammann_2017}. 
After successful campaigning by inhabitants, rivers were transformed back into swimmable spaces. The initiative made the cities cleaner (better sewage systems, no waste dumped into rivers) and the population healthier (less diseases originating from contaminated water, access to free and fun exercise).

The reason I find urban swimming a worthwhile topic to study, is because it can provide accessible leisure spaces for people in any socio-economic situation; it reclaims natural public space in dense urban environments; it contributes to making cities more resilient against heat; and overall, it improves the quality of urban life.

My approach would be to explore different practices of cities regarding urban swimming. Many cities are experimenting with turning rivers back into swimmable environments, on different scales. What are the goals of these projects? What makes some more ``successful'' than others? What do/can they achieve ? What are the unexpected/negative consequences?
I would find it interesting to compare cases from cities in distinct geographical and cultural situations, like Copenhagen and Madrid, for example.

\textbf{Topic \#2 How do environmentally-minded food movements like veganism shape cities?}

\textbf{Potential supervisors}: Francesc Baró, Nicola da Schio, Mathieu Van Criekingen

\textit{Keywords: green gentrification, environmental justice, food sustainability}

Given our pressing environmental concerns, food movements like organic, local, zero-waste, and vegan/vegetarian have become popular. Veganism has gained popularity in the last decade or so. People adhere to it for a variety of reasons: environment, ethics, health, culture.
Although a personal choice, the vegan `trend' has wider consequences. There is a new industry selling plant-based alternatives, which are often accessible only to middle and upper classes. As such, vegan businesses become associated with (newly) hipster neighbourhoods, like Hackney in London. What is particularly interesting is that the movement comes from the intention to reduce one's impact on the environment and improve animal welfare - but is criticised as a moralising, white, middle-class lifestyle unaffordable to many, eroding culture and traditions.

The reason I find this subject worthwhile to study is three-fold.
\\First, the plant-based movement is growing and spreading across cities (predominantly in the global North). 
\\Second, agriculture one of the biggest global GHG emitters, along with transport. But, since agriculture doesn't take place in the urban, there is less research into urban food sustainability. What exists focuses on eg. food miles and community gardens, rather than the carbon footprint of food products.
%urban food research is generally centred around reducing transport distance, urban food production, community gardens,... and not on the carbon footprint of the food products
\\Third, if veganism prides itself as an environmentally sustainable lifestyle, what can be said about social sustainability, if it is inaccessible to working classes?

Thus, I would like to study the environmental but most importantly the social impact of veganism in cities, and understand the underlying consequences of the movement. I expect this to revolve around gentrification, resistance to veganism and thus gentrification, and  injustices due to the moralising and inaccessible dimensions of veganism.

%Currently, there is limited research on veganism and the city. I would draw on research in green gentrification \parencite{kocisky2021towards}, 

%Most of the research in this area is in the US ... on how organic farmer's markets (eg. Whole Foods) open in low income, black neighbourhoods.

What I think would be interesting, is to study and compare two cities in two distinct political, social and/or cultural contexts. Scandinavian cities like Copenhagen are known to have a progressive, health and environment oriented (food) culture, and overall wealthier population than a city like Madrid, who's culture (food and other) relies more heavily on animals.

% (animal products have a significant impact on the environment compared to plant-based products), ethical (the belief that humans do not have the right to consume animals simply because they can), health (plant-based diets have health merits), cultural (it is a lifestyle, can be spiritual and align with religious beliefs, and create community). 
%The morality of a movement rooted in ethics and environmental concerns, in contrast with the working class, communities of colour which it indirectly displaces, through farmers' markets and whole foods stores.

%What would also be interesting is to look at the implications of the movement beyond the urban, where food is produced. 

\begin{comment}
\textbf{Topic}: Overprotecting: how do regulations restricting the engagement of people with the natural or built environment in the city affect our experience
Maybe this culminates in the pandemic, where the State mandated people to stay home - using public space and infrastructure like benches, parks, playgrounds... 
\end{comment}

\begin{comment}
\textbf{Topic \# 3 Smart cities for citizens - resisting big tech but embracing progress}
\textbf{Potential supervisor}: Corentin Debailleul
Smart cities is a concept that makes city officials buzz. Unfortunately, smart solutions are often implemented by private investors seeking a profit, without the proper research, without consulting local residents, and with the Silicon Valley mindset of `fail fast, fail often' which is not ethical in an urban environment where everyday people are both testers and users. At best, smart solutions don't address the issues they intend, and at worst, they are dangerous - reinforcing racist or sexist biases, putting our physical persons in danger, or breaching our privacy. 
Without wanting to sound cliché, and as software engineer myself, I will say that technology is coming to our cities whether we want it or not. 
It is therefore relevant to understand how smart cities and popular 
And, technology has great power. Ideally, 
\end{comment}

\printbibliography

\end{document}
