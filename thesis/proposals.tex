\documentclass{article}

\linespread{1.1}
\usepackage[utf8]{inputenc}
\usepackage[left=1.5in,right=1.5in,bottom=1in]{geometry}
\setlength\parindent{0pt}
\setlength{\parskip}{1em}
\setcounter{secnumdepth}{0}
\usepackage{outlines}
\usepackage{graphicx}
\graphicspath{ {imgs} }
\usepackage{hyperref}
\usepackage{color,soul}

\usepackage{comment}
\specialcomment{topicsen}{\begingroup\bfseries\scriptsize}{\endgroup}
%\excludecomment{topicsen}

\title{Thesis proposals}
\author{Carla Hyenne}
\date{}

\begin{document}

\maketitle

\textbf{Topic}: How does veganism, a cultural, environmental and ethical movement shape the city?

\textbf{Potential supervisor}: Mathieu Van Criekingen 

\textit{Keywords}: environmental gentrification, veganism, whiteness, 

Veganism is a movement that has gained popularity in the last decade or so. People adhere to it for a variety of reasons: environmental (animal products have a significant impact on the environment compared to plant-based products), ethical (the belief that humans do not have the right to consume animals simply because they can), health (plant-based diets have health merits), cultural (it is a lifestyle, can be spiritual and align with religious beliefs, and create community). 

Although a personal choice, the vegan `trend' has wider consequences. With the rise in popularity came new products, businesses, and restaurants selling plant-based alternatives which can be expensive. Vegan shops and restaurants are often associated with hipster, trendy neighbourhoods. 
What is particularly interesting is that the movement comes from positive intentions - to reduce one's impact on the environment and animal cruelty - but is criticised as a middle-class lifestyle unaffordable to many, is accused of eroding cultural heritage and traditions, and ???. Given that veganism is spreading across cities in the global North, I want to explore its impact on cities.

The morality of a movement rooted in ethics and environmental concerns, in contrast with the working class, communities of colour which it indirectly displaces, through farmers' markets and whole foods stores.

Veganism, or vegetarianism (a less extreme lifestyle) can also be a part of religious or cultural practices predominant in Asia. 

I could approach this from the perspective of ???


What would also be interesting is looking at two different European cities, for example Copenhagen and Madrid. 


\textbf{Topic}: Overprotecting: how do regulations restricting the engagement of people with the natural or built environment in the city affect our experience

\textbf{Potential supervisor}: TODO


\textbf{Topic}: Smart cities for citizens - resisting big tech but embracing progress

\textbf{Potential supervisor}: Corentin Debailleul

Smart cities is a concept that makes city officials buzz. Unfortunately, smart solutions are often implemented by private investors seeking a profit, without the proper research, without consulting local residents, and with the Silicon Valley mindset of `fail fast, fail often' which is not ethical in an urban environment where everyday people are both testers and users. At best, smart solutions don't address the issues they intend, and at worst, they are dangerous - reinforcing racist or sexist biases, putting our physical persons in danger, or breaching our privacy. 

Without wanting to sound cliché, and as software engineer myself, I will say that technology is coming to our cities whether we want it or not. 
It is therefore relevant to understand how smart cities and popular 
And, technology has great power. Ideally, 

\end{document}
