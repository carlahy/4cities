\documentclass{article}

\linespread{1.15}
\usepackage[utf8]{inputenc}
\usepackage[left=1.5in,right=1.5in,bottom=1in]{geometry}
\setlength\parindent{0pt}
\setlength{\parskip}{1em}
\setcounter{secnumdepth}{0}
\usepackage{outlines}
\usepackage{graphicx}
\graphicspath{ {imgs} }
\usepackage{hyperref}
\usepackage{color,soul}
\usepackage[normalem]{ulem}

\usepackage[
backend=biber,
style=apa,
citestyle=authoryear,
sorting=nyt,
]{biblatex}
\addbibresource{refs.bib}

\usepackage{comment}
\specialcomment{topicsen}{\begingroup\bfseries\scriptsize}{\endgroup}
%\excludecomment{topicsen}

\newcommand{\alignedmarginpar}[1]{%
        \marginpar{\raggedright\small #1}
    }

\title{Urban Analysis 3}
\author{Carla Hyenne}

\begin{document}

\maketitle

\tableofcontents

\pagebreak

%%%%%%%%%%%%%%%%%%%%%%%%%%%%%%%%%%%%%%%%%%%%%%%%%%%%%%%%%%%%
%											STATEMENT OF PURPOSE
%%%%%%%%%%%%%%%%%%%%%%%%%%%%%%%%%%%%%%%%%%%%%%%%%%%%%%%%%%%%
\section{Statement of Purpose}

Draft 1:

\textbf{I am studying} the (re) conversion of inland, (natural) urban blue spaces into swimmable environments\\
\textbf{because I want to find out} what impact blue spaces can have on a city's social and environmental sustainability\\
\textbf{in order to help my reader understand} the potential that water can have in their city.

Draft 2:

\textbf{I am studying} the (re) conversion of inland, (natural) urban blue spaces into swimmable environments\\
\textbf{because I want to find out} how usable blue spaces can improve environmental justice in the city \\
\textbf{in order to help my urban planners understand} the importance of incorporating blue spaces in urban planning and development.\\

\textit{So what} if we don't know how reconverting blue spaces into swimmable environments can increase environmental justice? If we don't know what the impact that planning for usable blue spaces can have on people, \sout{and the environment}?

What are the \textit{consequences} if the question is not answered?

This is a \textit{conceptual problem} because solving it will not change anything in the world, but will help us better understand it (\cite{booth2003craft} (p. 53)). Steps to formulating the research question:

\begin{outline}
	\1 1. the topic
	\1 2. the condition of the problem that I do not understand $\rightarrow$ question 1
	\1 3. the indirect question, that is wider and more significant than the first one (the condition) $\rightarrow$ question 2
\end{outline}
	
The research is \textit{applied} because it has practical consequences (\cite{booth2003craft}, p. 59)

Notes:

\begin{outline}
	\1 Doesn't include people's health
	\1 I want to focus on public open space --> environmental justice? privatisation of public space taking away opportunities for people to connect to nature and to each other, which is environmentally unjust, and thus open, public blue spaces play an active role in reducing environmental injustice? 
\end{outline}

%%%%%%%%%%%%%%%%%%%%%%%%%%%%%%%%%%%%%%%%%%%%%%%%%%%%%%%%%%%%
%											GENERAL IDEAS
%%%%%%%%%%%%%%%%%%%%%%%%%%%%%%%%%%%%%%%%%%%%%%%%%%%%%%%%%%%%
\section{General ideas for the research}

\begin{outline}
	\1 Methodology
		\2 Quantitative: survey data... would need a significant part of the population which will be hard
		\2 Qualitative: contact organisation, NGOs, etc. who work with the blue spaces
		\2 Interview questions:
			\3 Exposure: time of the day/year; 
	\1 Frameworks
		\2 What are the surrounding environments of the blue space? Do certain conditions make the blue space ``better'' (in terms of environmental benefits, usage...), like the facilities, the local environment, the wildlife? (ref. \parencite{garrett2019urban})
		\2 How do we organise people's usage of blue space? For example, the accessibility of the space (incl. distance from home, transport access), indirect/incidental/intentional exposure (ref. \parencite{garrett2019urban})
		\2 Do the type of activities carried out by the users matter, or does simple exposure matter (eg. type of activity, duration of exposure, direct contact with water...)? (ref. \parencite{garrett2019urban})
	\1 Sustainability
		\2 SDG 11: Sustainable cities and communities; Making cities and human settlements inclusive, safe, resilient and sustainable
			\3 11.7 target: ``By 2030, provide universal access to safe, inclusive and accessible, \textbf{green and public spaces}, in particular for women and children, older persons and persons with disabilities''
			\3 11.7 indicator ``Average share of the built-up area of cities that is open space for public use for all, by sex, age and persons with disabilities'' $\rightarrow$ what about blue spaces?
		\2 SDG 3: Good health and wellbeing; Ensure healthy lives and promote well being for all ages
			\3 3.4 Target ``By 2030, reduce by one third premature mortality from non-communicable diseases through prevention and treatment and promote mental health and well-being''
			\3 3.4 Indicator: ``Mortality rate attributed to cardiovascular disease, cancer, diabetes or chronic respiratory disease''
	\1 Urban planning
		\2 How can the learnings on blue spaces be used in urban planning and development frameworks? How can spaces be tailored for their users, in terms of age, or income, for example?
		\2 Is environmental justice inscribed in urban planning/development practices?
		\2 How, or is, the European Union's Green Infrastructure Strategy involved with the city's planning in any way?
		\2 Does the urban planning culture of a city effect how blue spaces are designed or prioritised?
\end{outline}

%%%%%%%%%%%%%%%%%%%%%%%%%%%%%%%%%%%%%%%%%%%%%%%%%%%%%%%%%%%%
%											READINGS
%%%%%%%%%%%%%%%%%%%%%%%%%%%%%%%%%%%%%%%%%%%%%%%%%%%%%%%%%%%%
\subsection{Readings}

\subsubsection{Garrett et al., \textit{Urban blue spaces and health and wellbeing in Hong Kong: Results from a survey of older adults}, 2019} \parencite{garrett2019urban}

\begin{outline}
	\1 What are the potential health benefits of being near to, seeing, using blue spaces? The study looks are the perceived well-being of elderly in Hong-Kong
	\1 The benefits of green spaces in urban environments are known. Dense, stressful, polluted urban environments can cause physical and mental illnesses, and green spaces can help reduce these symptoms. Three main ways in which green and health are linked (according to Markevych et al., (p. 100)):
		\2 ``Reducing environmental harms (eg. mitigating noise pollution)'' $\rightarrow$ environment
		\2 ``Supporting emotion regulation and the restoration of depleted cognitive capacities (eg. through stress alleviation)'' $\rightarrow$ social
		\2 ``Building capacities (eg. through supporting physical activity)'' $\rightarrow$ social
	\1 Blue spaces provide similar benefits, like reducing stress, heart diseases, making people more physically active, etc. But to what extent do you need to frequent the blue spaces to gain these benefits? How much blue/green space is required in order to get the benefits
	\1 Three research questions: (p. 101)
		\2 ``To what extent is self-reported general health and wellbeing in Hong Kong related to an individuals exposure to the city's blue spaces?''
		\2 ``Which environmental factors predict blue space visit frequency in Hong Kong?''. They looked at the safety, presence of wildlife, clean/free from litter, and good facilities like footpaths or toilets
		\2 ``Are some visit and environmental characteristics associated with better short-term recalled wellbeing outcomes?''
	\1 The conclusion, indirect and intentional exposure were associated with good health and high level of wellbeing; good facilities and wildlife are related to intentional usage; safety and wildlife are related to higher levels of wellbeing and g when visiting the blue space; visible blue space from home is related to better perceived health; those who regularly visit/see from home blue spaces are more likely to have good mental health and good general health.
	
	\1 $\Rightarrow$ This paper analyses the perceived health benefits that blue spaces can have on the population (it focuses on elderly residents). It defines three categories of blue space exposure: indirect (view from home), incidental (on commute route), and intentional (purposeful visit), and addresses different factors that could influence people's perception and use of the blue space, like safety, cleanliness, facilities, accessibility. They find a positive relation between blue space and overall wellbeing, including good health and good mental health.
	It recognises the particularities of Hong Kong: it is surrounded by water, has high quality and safe public space, a fantastic public transit system (which makes it easy for people to reach blue spaces if they don't live nearby). I think it is necessary to acknowledge these particularities in a city, because they can influence how people interact with blue (or green) spaces.
	
	Why is it relevant to me? It provides a categorisation of blue space usage, and a framework for analysing the benefits and quality of blue spaces. It focuses on the effect that blue spaces have on the physical wellbeing of people, and this falls under social sustainability. 
\end{outline}

\subsubsection{Gascon et al., \textit{Outdoor blue spaces, human health and well-being: A systematic review of quantitative studies}, 2017} \parencite{gascon2017outdoor}

\begin{outline}
	\1 Summarises the quantitative evidence from existing research on the positive effects of blue spaces on people's health, and concludes that there is a correlation between blue space exposure and benefits in mental health and in wellbeing. Says there is too few and too much heterogeneity in the studies
	\1 There are claims of the positive effect of blue space on people's health and wellbeing, either by being \textbf{proximally or distally/virtually} (being in, on or near/being able to see, hear or sense water). The effects of blue spaces could be similar to that of green spaces, which include ``stress reduction, increased physical activity, promotion of positive social contacts, increased place attachement and the reduction of extreme temperatures'' (p. 1212)
	\1 Differentiates outdoor blue space type into inland (rivers, lakes, ponds, streams, rivulets, wetlands, freshwaters) and non-inland (coast, beach, salt waters)
	
	\1 $\Rightarrow$ This paper reviews quantitative studies on blue space and associated health benefits. It reports the type of blue space, the environment of the blue space, and how the health outcomes were evaluated.
	What I found useful was the differentiation between inland and non-inland spaces, which I will use in my research. I also realised I don't want to focus specifically on the relation of blue space to human health (eg. general health, mental health and wellbeing, physical activity, other morbidities) because there is already good research in this area, especially in Europe. I still want to focus on the social aspect of blue spaces, but perhaps more on blue spaces as open, public and free spaces. Also, quantitative research will be hard because I don't think I would be able to reach enough participants (I assume in the scale of hundreds?) in both case studies
\end{outline}

\subsubsection{Raymond et al., \textit{Integrating multiple elements of environmental justice into urban blue space planning using public participation geographic information systems}} \parencite{raymond2016integrating}

\begin{outline}
	\1 Uses PPGIS (Public Participation Geographic Information System) method for ``spatially identifying and assessing multiple elements of environmental justice in urban blue space''. Using Finland as a case study, it looks for:
		\2 ``Diversity and spatial distribution of clusters base don the activities undertaken in urban blue space''. A cluster is an area defined by the type of activity carried out there
		\2 ``Diversity of users in each cluster, representing a composite measure of income, age and family income'', also race, gendre, disabilities. How facilities and services influence who uses the space
		\2 ``Extent of perceived problems and unpleasant experiences in each cluster'' 

	\1 Environmental justice: a principle that claims that all people have a right to be protected from environmental pollution, and live in a clean and healthy environment. 
		\2  PPGIS enables urban planning to tailor blue spaces to specific types of activities and users. It is used within the context of environmental justice, because it helps analyse user diversity (the mix of users) and perceived problems and unpleasant experience (PPUE, the perceived negative qualities of a place). PPGIS can help analyse how ``different elements of environmental justice are spatially distributed across the landscape. Understanding environmental justice from multiple perspectives is crucial to ensuring that urban settings are designed in ways that contribute to a range of place-based experiences, including social interactions between diverse user groups, as well as provide possibilities for connection to nature. '' (p. 199)
		\2 Environmental justice resonates with SDG 11 to ``provide universal access to safe, inclusive and accessible, green and public spaces''
		\2 Privatisation of urban space is taking away opportunities for people to connect with each other, and with nature
		\2 \textbf{Multiplicity as a principle in urban inclusion} may create conflict between different user types and PPUE, based for example on the types of activities offered, and can lead to feelings of exclusion, discomfort or fear\alignedmarginpar{Inclusivity in public space\\Right to blue space}
	\1 Helsinki and surrounding regions as case study: the context is that Helsinki residents are never more than 10km away from the shore
		\2 Survey asked participants about PPUE (encompasses cost, inclusivity, facilities, environment, safety...), usage based on time of year, activity, socio-demographic features (age, income, ethnicity, family situation...)
		\2 Income level was a strong determinator of accessibility
	\1 Environmental justice is multidimensional: the paper takes into account user diversity, activity diversity, and PPUE
		\2 Using four-quadrant framework for combining user diversity and activity diversity (high/low user diversity, high/low activity diversity matrix), they classify areas (``clusters''), for eg. Helsinki shorelines are mostly hi user/lo activity diversity, and lo user/hi activity diversity ]
		\2 By classifying areas into these quarters, they could see what type of blue space existed (in terms of users and activities), and find some characteristics of the quarters (eg. more nature activities in one area, maybe because of the environment?)
	\1 Most cited PPUE was ``certain group of people or use method bothers me'', ``the environment is littered or not cared for'', ``the water quality is poor'', ``the location is crowded''
	\1 $\Rightarrow$ This paper focuses on environmental justice which I think is a great term to help frame my research interest, because it incorporates principles like sustainability, inclusivity, and right to open blue spaces. The paper looks at the diversity of users, diversity of activities, and the PPUE (perceived problems and unpleasant experience) of the blue spaces, to demonstrate the multidimensionality of environmental justice. It also made me wonder whether I should focus on a certain socio-demographic group (by age, income levels, gender, ethnicity, etc.), and  how that group is affected by the presence, uses, and/or design of blue spaces, or lack thereof. 

\end{outline}


\printbibliography

\end{document}
