\documentclass{article}

\linespread{1.5}
\usepackage[utf8]{inputenc}
\usepackage[left=1.5in,right=1.5in,bottom=1in]{geometry}
\setlength\parindent{0pt}
\setlength{\parskip}{1em}
\setcounter{secnumdepth}{0}
\usepackage{outlines}
\usepackage{graphicx}
\graphicspath{ {imgs} }
\usepackage{hyperref}
\usepackage{color,soul}
\usepackage[normalem]{ulem}

\usepackage[
backend=biber,
style=apa,
citestyle=authoryear,
sorting=nyt,
]{biblatex}
\addbibresource{refs.bib}

\usepackage{comment}
\specialcomment{topicsen}{\begingroup\bfseries\scriptsize}{\endgroup}
%\excludecomment{topicsen}

\newcommand{\alignedmarginpar}[1]{%
        \marginpar{\raggedright\small #1}
    }

\title{Urban Analysis 3}
\author{Carla Hyenne}

\begin{document}

\maketitle

\tableofcontents

\pagebreak

\section{Statement of Purpose}

Draft:

\textbf{I am studying} the (re) conversion of inland, (natural) urban blue spaces into swimmable environments\\
\textbf{because I want to find out} what impact blue spaces can have on a city's social and environmental sustainability\\
\textbf{in order to help my reader understand} the importance of incorporating blue spaces in urban planning and development.\\

Notes:

\begin{outline}
	\1 Doesn't include people's health
\end{outline}

\section{General ideas for the research}

\begin{outline}
	\1 Methodology
	\1 Frameworks
		\2 What are the surrounding environments of the blue space? Do certain conditions 
\end{outline}


\subsection{Readings}

\subsubsection{Garrett et al., \textit{Urban blue spaces and health and wellbeing in Hong Kong: Results from a survey of older adults}, 2019} \parencite{garrett2019urban}

\begin{outline}
	\1 What are the potential health benefits of being near to, seeing, using blue spaces? The study looks are the perceived well-being of elderly in Hong-Kong
	\1 The benefits of green spaces in urban environments are known. Dense, stressful, polluted urban environments can cause physical and mental illnesses, and green spaces can help reduce these symptoms. Three main ways in which green and health are linked (according to Markevych et al., (p. 100)):
		\2 ``Reducing environmental harms (eg. mitigating noise pollution)'' $\rightarrow$ environment
		\2 ``Supporting emotion regulation and the restoration of depleted cognitive capacities (eg. through stress alleviation)'' $\rightarrow$ social
		\2 ``Building capacities (eg. through supporting physical activity)'' $\rightarrow$ social
	\1 Blue spaces provide similar benefits, like reducing stress, heart diseases, making people more physically active, etc. But to what extent do you need to frequent the blue spaces to gain these benefits? How much blue/green space is required in order to get the benefits
	\1 Three research questions: (p. 101)
		\2 ``To what extent is self-reported general health and wellbeing in Hong Kong related to an individuals exposure to the city's blue spaces?''
		\2 ``Which environmental factors predict blue space visit frequency in Hong Kong?''
		\2 ``Are some visit and environmental characteristics associated with better short-term recalled wellbeing outcomes?''
	\1 Quality of the infrastructure, wildlife, and general environment around the blue space
	\1 This paper analyses the perceived health benefits that blue spaces can have, and focuses on an elderly population. It defines three categories of blue space usage: indirect (view from home), incidental (on commute route), and intentional (purposeful visit). 
	It also recognises the particularities of Hong Kong: it is surrounded by water, has high quality and safe public space, a fantastic public transit system (which makes it easy for people to reach blue spaces if they don't live nearby). I think it is necessary to acknowledge these particularities in a city, because they can influence how people interact with blue (or green) spaces. 
	Why is it useful? Provides a categorisation of blue space usage; a framework for analysing the benefits of blue spaces, and also the quality of blue space; 
\end{outline}


\subsubsection{Gascon et al., \textit{Outdoor blue spaces, human health and well-being: A systematic review of quantitative studies}, 2017} \parencite{gascon2017outdoor}

\begin{outline}

\end{outline}


\subsubsection{Raymond et al., \textit{Integrating multiple elements of environmental justice into urban blue space planning using public participation geographic information systems}} \parencite{raymond2016integrating}

\begin{outline}

\end{outline}


\printbibliography

\end{document}
