\documentclass{article}

\linespread{1.5}
\usepackage[utf8]{inputenc}
\usepackage[left=1.5in,right=1.5in,bottom=1in]{geometry}
\setlength\parindent{0pt}
\setlength{\parskip}{1em}
\setcounter{secnumdepth}{0}
\usepackage{outlines}
\usepackage{graphicx}
\graphicspath{ {imgs} }
\usepackage[hyphens]{url}
\usepackage{hyperref}
\usepackage{color,soul}
\usepackage[normalem]{ulem}

\usepackage[
backend=biber,
style=apa,
citestyle=authoryear,
sorting=nyt,
]{biblatex}
\addbibresource{refs.bib}

\usepackage{comment}
\specialcomment{topicsen}{\begingroup\bfseries\scriptsize}{\endgroup}
%\excludecomment{topicsen}

\newcommand{\alignedmarginpar}[1]{%
        \marginpar{\raggedright\small #1}
    }
    
\newcommand{\bisection}[1]{\textbf{\textit{#1}}}

\DeclareCiteCommand{\citeyear}
    {}
    {\bibhyperref{\printdate}}
    {\multicitedelim}
    {}

\title{Research Proposal}
\author{Carla Hyenne}
\date{}

\begin{document}

\maketitle

\tableofcontents 

%%%%%%%%%%%%%%%%%%%%%%%%%%%%%%%%%%%%%%%%%%%%%%
%				ABSTRACT (MAX 500 WORDS)
%%%%%%%%%%%%%%%%%%%%%%%%%%%%%%%%%%%%%%%%%%%%%%

\section{Abstract}

\hl{TODO}

%%%%%%%%%%%%%%%%%%%%%%%%%%%%%%%%%%%%%%%%%%%%%%
%								INTRO (???)
%%%%%%%%%%%%%%%%%%%%%%%%%%%%%%%%%%%%%%%%%%%%%%

\section{Introduction}

Most large cities are located near water, either inland like rivers, lakes or harbours, or on the coast like salt water. Water is important for the city. It serves as a life source, a communication channel, a motor for industry and the economy, for recreation and exercise, for cultural and spiritual practices, and for community forming and belonging. 

With urbanisation, water in cities became polluted or transformed by industries \parencite{kampa_langaas_anzaldua_2016}. 
Waterfronts became unattractive and unsafe to swim, thereby disrupting the relationship people had with water and its surrounding environment.
Starting in the 20th century, post-industrial European cities started considering waterfronts as strategic opportunities to revitalise the city \parencite{del2021dismantling}. In the last few decades, the importance of water in the city has entered public consciousness due in part to climate concerns. Most recently, the COVID-19 pandemic made people acutely aware of the benefits of urban blue-green spaces \parencite{kohsaka2021urban}.
% develop more on government reasons for investing in waterfront redevelopment, maybe ref. \parencite{del2021dismantling} as an example
Through local, national and international political pressure, public demand and (neoliberal) urban renewal, governments have invested significant resources in revitalising waterfronts into attractive natural places for people. 

\hl{TODO: explain the problem}

The aim of this research is to understand the extent to which urban waterfronts support the uses, perceptions and meanings of wide range of people and communities

Towards this objective, I use the concept environmental justice and recognition justice, which articulate the ways in which people's perceptions and values of nature are represented in the city.

I explore this concept in the context of Copenhagen, a city . Specifically, I focus on \hl{three?} spaces with distinct characteristics: BS1 which has received a lot of public investment and is used by the city to promote liveability and sustainability; BS2 which ...; and BS3, which... .

Finally, the research unites the concepts of ?recognition justice and place attachement? to help illustrate the diversity of perceptions that exist in the city, and how blue spaces can allow, or not, different people and groups to express themselves. 

This is intended to give visibility into how different identities make sense of blue space, and how blue spaces give people the opportunity to express themselves...

%%%%%%%%%%%%%%%%%%%%%%%%%%%%%%%%%%%%%%%%%%%%%%
%								LITERATURE REVIEW (2-3 PAGES)
%%%%%%%%%%%%%%%%%%%%%%%%%%%%%%%%%%%%%%%%%%%%%%

\section{Literature review}

% Making a body of water swimmable is a large public investment: the water itself has to be cleaned, which involves identifying and dealing with polluting sources; the waterbed and surrounding environments have to be rehabilitated to safely welcome people; and the waterfront has to be equipped with infrastructure for swimming, sitting, walking, and more. Since public space is a highly valued commodity in the city, reconverting blue spaces is a great way to take advantage of unused areas and proves to be valuable for a multitude of reasons.

% define blue spaces, and blue interventions: eg. in \parencite{anguelovksi2021green}, a green interventions as nature-based infrastructures and amenities; 
% blue spaces are all surface water in the city
% maybe add a sentence that introduces what is coming in the review, eg. 1. many benefits "for the people", but actually 2. for whom??

This section reviews the academic literature on blue spaces, and incorporates literature from wider concepts such as nature-based solutions and greening. First, it reviews the main benefits offered by blue spaces. Then, it critically analyses the consequences that blue urban renewal can have on people and the environment, and raises important questions that need to be addressed in order for cities to be inclusive, sustainable and just.

\subsection{The benefits of blue spaces}

Blue spaces have undeniable positive effects in cities, which Gascon et al. (\citeyear{gascon2017outdoor}) summarise as ``stress reduction, increased physical activity, promotion of positive social contacts, increased place attachment and the reduction of extreme temperatures'' \parencite{gascon2017outdoor}. The literature on the potential of blue spaces centres around three main topics relating to people and the environment.

\bisection{Climate adaptation and mitigation}
In the context of climate change, blue carbon ecosystems can naturally alleviate pollution, heat stress, flooding or drought, and increase the climate resiliency of cities (\cite{lin2020water}, \cite{manteghi2015water}, \cite{o2021international}). 
This is because blue spaces have a surface temperature lower than green space, high evaporation and thermal capacity which creates a cooling effect, and a spillover mechanism which extends the cooling effect inland \parencite{lin2020water}.
However, the potential of blue spaces for climate mitigation is not well known relative to green spaces. As a result, less urban planning recommendations exist, not to mention that they are more difficult to provide and maintain (\cite{manteghi2015water}, \cite{volker2013evidence}).
% === Blue and green spaces can be used in complementary ways. For example, the density and morphology of blue spaces determine the extent to which they cool the surrounding area. The more uniform and compact they are - like lakes and reservoirs, unlike long and undulating rivers - the greater their cooling effect \parencite{lin2020water}. Therefore, inland blue spaces like rivers and canals need to be supplemented with green spaces in order to mitigate heat islands. //// Green areas like parks or forests can be used to cool inland heat islands.

\bisection{Health and well-being}
The best known way in which blue space has a positive impact is on people's physical and psychological health. Being exposed to water makes people feel better, happier, and be more active. 
There is an extensive repertoire of quantitative studies demonstrating these effects on people's health and well-being (\cite{gascon2017outdoor}, \cite{britton2020blue}).
Qualitative studies also show that exposure to water improves mental health, regardless of how people interact with it (\cite{garrett2019urban}, \cite{van2021urban}).
In deprived neighbourhoods in particular, people tend to have poorer mental health and lower life satisfaction compared to wealthier areas. Projects that increase access to high quality blue spaces in socio-economically disadvantaged neighbourhoods have the potential to significantly influence the well-being of their residents \parencite{van2021urban}.

\bisection{Connection and community}
Lastly, waterfronts give people with the opportunity to connect with each other and with nature. People develop a sense of place attached to water, and even polluted or degraded waterfronts can play a central role in the community, providing refuge, connection, entertainment, and even food \parencite{toomey2021place}. 
Furthermore, blue space revitalisation projects can be an opportunity to create community bonds by engaging residents in the design and building process. During the ``urban acupuncture'' intervention conducted by BlueHealth in a deprived area of Plymouth, UK, residents who participated in the project reported a greater sense of well-being and life satisfaction due to feelings of community belonging and safety \parencite{van2021urban}.

% === emphasise the multifunctionality and the co-benefits of blue spaces: they are good for climate, health, community... many functions and benefits (ref. \parencite{o2021international})

% === read duke2020

\subsection{Social and environmental repercussions of urban renewal}

Despite the undeniable benefits of water in the city, transforming waterfronts into high-quality public spaces can have harmful consequences on people and the environment. They can cause environmental damage and loss of biodiversity, disrupt human and nature relationships, displace people by way of gentrification, and reinforce social inequalities by excluding individuals and communities on the basis of socio-economic differences, or racist and sexist practices. The following section explores three of these avenues.

% === \bisection{Environmental damage} 

\bisection{Exclusion through discrimination} Due to historical but ongoing racial, sexist or ethnic discrimination, some groups may choose to avoid public spaces altogether. They may feel more vulnerable and less safe in public, especially in blue spaces that are most popular in the warmer months and where people have less coverage.
When public space is designed for the mainstream, it discourages a wider range of people from using the space. Alternative experiences are invisible and not normalised because they happen privately. Ultimately, the right to the city is not exercised, further discriminating vulnerable groups.

\bisection{Disrupting community} 
In stark contrast to the social bonds that can be fostered when residents are involved in revitalisation projects, when the local community's perceptions and values are not understood by planners, changes can disrupt human-to-human or human-to-water connections \parencite{toomey2021place}. This is susceptible to happen when a community's social practices do not fit with the social norms \parencite{wessells2014urban}. 
Moreover, marginalised or stigmatised communities may find it hard or impossible to communicate their experience to the mainstream because they lack the words to articulate their reality. And vice-versa: wealthy, white, males may not be capable of understanding the experience of `others'.
This of course does not mean that minority communities do not attach meaning to place - on the contrary, people are inextricably linked to place, and disrupting this relationship threatens their belonging or even existence.

% === Toomey et al. (\citeyear{toomey2021place}) propose using language like ``place-disruption'' and ``place-protection'' to promote mutual understanding and avoid privileging the values of mainstream groups over those of marginalised communities.

\bisection{Neoliberal urban renewal} 
In perverse cases, cities prioritise growth over well-being and community. 
City officials are exploiting nature-based solutions to brand their cities as green and liveable\footnote{For example, Madrid promoting the Madrid Río project on the official tourism website \parencite{madridrio}, or Oslo advertising its new urban waterfront promenade along which ``you find yourself surrounded by some of Oslo's world-renowned architectural gems'' \parencite{visitoslo}.}, and to promote greening as a win-win strategy where ``no one is left behind by the trickle-down of benefits from green infrastructure'' \parencite{anguelovski2021green}.
With ``glitzy green'' renewal projects (ibid.), cities try to attract a new creative class rather than addressing public blue space as a common good and prioritising the concerns of existing residents (\cite{wessells2014urban}, \cite{anguelovski2020expanding}, \cite{del2021dismantling} on revitalising the Amsterdam Noord waterfront).
These strategies perpetuate inequalities by privileging the values of new groups over existing residents, who risk facing displacement. White, environmentally privileged upper classes who can afford to live near nature will move in and price out residents who will be forced to move to neighbourhoods with worse access to attractive natural space. This phenomenon challenges the well-being benefits observed in visitors of the waterfront - do they feel happier and healthier because they are using blue spaces, or because they can afford to and choose to live near them?

\subsection{Achieving equity and justice in urban blue spaces}

Promoting nature-based solutions for the sake of environmental sustainability ignores wider social, economic, and cultural needs which need representation if a city is to be `sustainable'. By ignoring the consequences of urban renewal projects, cities are increasing inequalities and marginalising socio-economically disadvantaged groups. 

Thus, how can cities provide equal access to high-quality urban blue spaces, so that everyone in the city can share their benefits?
How can marginal voices be actively included in planning? How can natural spaces be designed for the existing communities who need it most, rather than for the privileged few who already have unhindered access to nature?

% If cities are to reduce inequalities and improve access to blue space, the social and environmental trade-offs of 
% For one, blue urban renewal must be accompanied by ``bold, progressive social policies'' \parencite{anguelovski2021green}. 
% === And, those in power need to want to listen to and understand alternative voices, and distinct groups need a common language to communicate their experience. 

% overcoming barriers to justice and equity --> ``questioning the social exclusion and white privilege embedded in new greening landscapes and how those social dynamics reshape security and vulnerability'' \parencite{anguelovski2021green}
% - even in cities with relatively strong social welfare policies, urban renewal projects can cause displacement and exclusion

%%%%%% COMMENTS %%%%%%
\begin{comment}
- Importance of situating ??
- Right to the city enabled through public blue space (in connection to recognition justice)
	1. Public blue spaces are a shared, common good, that everyone should be able to access \parencite{wessells2014urban}	
	2. cite more general right to the city literature
	3. Sharing the benefits that waterfronts have on people and the environment
- individuals and communities are inextricably linked to place, and disrupting place/not providing it threatens their existence/belonging
\end{comment}
%%%%%%%%%%%%%%%%%%%

%%%%%%%%%%%%%%%%%%%%%%%%%%%%%%%%%%%%%%%%%%%%%%
%								PROBLEM STATEMENT AND RQ (1 PAGE)
%%%%%%%%%%%%%%%%%%%%%%%%%%%%%%%%%%%%%%%%%%%%%%
\pagebreak
\section{Problem statement and research question}

% Ideal- people have equal opportunities to access urban blue spaces
If designed well and distributed equitably, blue space offers many benefits for people and the environment. As such, providing equal access to public blue spaces in the city helps to combat social inequalities. Since public space is a highly valued commodity in the city, reconverting blue spaces is a great way to take advantage of unused areas. 

% Real- physical and psychological barriers can restrict access to blue spaces
Although there is no direct economic barrier to public space (there is not entrance fee), rarely are public blue spaces fairly accessible to everyone. There exists both physical and psychological barriers which can prevent individuals, or whole communities, from enjoying urban waterfronts.

%With regards to physical accessibility, those who live or work in the vicinity of blue space are more likely to visit it. The availability of public transport, bicycle infrastructure, and the quality of the roads and footpaths around and in the space, also influence who visits. 
%Psychological barriers refer to how welcomed and safe people feel. Blue spaces, and public spaces in general, can be more or less welcoming to certain people. Gender, age, ethnicity, income, or preferences in aesthetics or activities, amongst other factors, all influence who feels comfortable, and who doesn't. 
%Discriminatory practices, such as the pressure to consume in order to fit in, excludes people.

% Consequence- increasing social and environmental inequalities (not everyone benefits fairly from blue/natural spaces), discrimination (racism, sexism...), displacement (green/blue gentrification)
Therefore, even if a blue space is freely open to the public, this does not mean that people use and benefit from them equitably. Understanding this phenomenon is important because public spaces are places of community, identity, attachment, and well-being, and ignoring experiences that differ from the mainstream increases social inequalities, discrimination, and displacement.

Given the above, my research aims to answer the following question: \textbf{to what extent are the public blue spaces in Copenhagen (un)fairly accessible, and who benefits from them?}

What meaning do users attach to public blue spaces, and to what extent do the blue spaces in the city represent the values of all communities?

to what extent do urban blue spaces support 

what values and perceptions do urban blue spaces support, and what does this tell us about


%%%%%%%%%%%%%%%%%%%%%%%%%%%%%%%%%%%%%%%%%%%%%%
%							RESEARCH DESIGN (4-6 PAGES)
%%%%%%%%%%%%%%%%%%%%%%%%%%%%%%%%%%%%%%%%%%%%%%

\section{Research design}

To answer my research question, I will use the following sub-questions:

\begin{enumerate}
	\item Who are the users of the space, how do they perceive it and how do they feel in it?
	\item Why do they choose to use this space?
	\item How diverse are the users... 
	\item How 
\end{enumerate}

My research will be explanatory, because I aim to explain why the phenomenon of unequal access to blue space takes place (or not) based on principles of recognition justice. I will take an inductive approach, whereby my theory will emerge from data I will collect on people’s experiences, perceptions and preferences of a blue space. This data is both subjective and spatial. My approach to collecting the data will be to visit the blue space(s) and interview users, for which I will need a set of interview questions as well as a way to record spatial data like their home, or the blue/green spaces they frequent. These requirements lend themselves well to public participatory GIS (PPGIS), a map-based survey method linking qualitative and GIS data. Using PPGIS to study people’s relationship to green and blue space is recommended as research method which “might uncover local spatial knowledge and perceptions” (Anguelovski et al. 2020). It has been used to study people’s interactions with and the distribution of blue spaces in Helsinki metropolitan area (Raymond et al. 2016), and also by BlueHealth to “uncover spatial aspects of people’s relationships with blue spaces” (BlueHealth n.d.).

\subsection{Case study} 

\bisection{What type of case study? what unit of analysis?}
Every blue space, neighbourhood and city will have a different set of conditions (social, political, economic, cultural, environmental, etc.) which explain who uses the space, how they feel in it, and why they use it. In order to uncover these conditions, the unit of analysis should be scoped to a specific location on the water where people linger and have the opportunity to swim; this spot should have been rehabilitated by the city, and be public; and, it would be particularly interesting to study two rehabilitated blue spaces within one city, in order to compare the social consequences of investments in different neighbourhoods.

\bisection{Specific context of Copenhagen, similar to Hong Kong}
Given these considerations, Copenhagen makes for an interesting case for the following four reasons. First, Copenhagen is located on the Kattegat strait and has 92 km of coastline \parencite{comertler2017greens}, therefore water features prominently in the urban landscape. Second, due to the amount of shoreline in Copenhagen and the city trying to position itself as a world leader in sustainability, there have been many blue space rehabilitation projects since 2002. Today there are four harbour baths (Island Brygge, Fisketorvet, Sandkaj and Sluseholmen) and various urban beaches (Amager Strandpark, Svanemølle) \parencite{visitcopenhagen_baths}. Third, Copenhagen today is experiencing an increase in poverty and ethnic segregation (Moller and Larsen 2015), as well as a growing racist discourse in the media and politics. For example, through the classification of some neighbourhoods as ‘ghettos’  \parencite{simonsen2008practice}. This evolving socio-economic landscape and its surrounding discourse make it important to understand who feels included in blue spaces, and might not. Finally, Copenhagen’s reputation as “the most liveable city” \parencite{visitdenmark_2021}, due in part to the swimming spots in the harbour, begs the question - for whom is the city liveable?

\hl{TODO:decide on specific blue spaces in Copenhagen}

Ultimately, unequal access to blue space is an environmental injustice and this framework will lead my research. Environmental justice is a multi-faceted concept which brings together social and environmental concerns. Amongst other things, it advocates for the equitable access to environmental benefits. The dimension of justice that I will focus on is recognition. Recognition justice deals with people and groups’ perceptions, values and preferences which may influence the ways in which they interact, or not, with a blue space \parencite{anguelovski2020expanding}.

\subsection{Theoretical framework}

\bisection{Environmental justice}
Environmental justice (EJ) provides a lens through which to understand social and environmental inequalities related to blue spaces. By bringing together social and environmental concerns, the environmental justice paradigm advocates for the equal access to the benefits offered by natural spaces; and, in turn, sharing environmental burdens. 
% TODO: argue that sustainability not being only enviro but also eco and social, and \parencite{agyeman2016trends}
% introduce the history of environmental justice, starting in the USA (emphasis on racism), but also in Europe ofc (eg. amsterdam noord, ref. \parencite{del2021dismantling})
% talk about environmental privilege, which is the disproportionate access and benefit that white, upper class residents have because of their proximity to green/blue/open spaces \parencite{anguelovski2021green} and Park and Pellow
EJ is traditionally broken down into three dimensions: distributional justice, procedural justice, and recognition justice \parencite{todo:cite schlosberg}.

Distributional justice focuses on where blue spaces are situated in the city and whether they address social, economic, racial or ethnic inequalities by striving to``avoid displacement and new negative green, ecological, climate and health effects'' \parencite{anguelovski2020expanding}.

Procedural justice deals with questions of discrimination in public participation and decision making situations, even if ideal participation doesn't prevent spaces from being captured by gentrifiers\footnote{urban community gardens} \parencite{anguelovski2020expanding}. 

Finally, recognition justice addresses individuals' and communities’ perceptions, values and preferences which may influence how they interact, or not, with public space. This dimension is the most relevant to understand the subjective experiences of the users of the space.

\bisection{Recognition}
Recognising the experience of those who do not fit into the `norm' also means acknowledging that some practices take place in the private and not public sphere, because of historical racial, sexist, ethnic discrimination. For example, women who disproportionally carry out domestic and care work have a different daily pattern which does not match that of the average 9-5 worker, and the spaces and mobility options should be adapted for them to reach blue spaces with ease. Or, they may feel more vulnerable and less safe in public, and prefer private spaces \parencite{wessells2014urban}. How can blue spaces be inclusive of a diversity of people, carrying out a diversity of activities at all times of the day, week, or year? 
% Language like ``place-disruption'' and ``place-protection'' can promote mutual understanding, and avoid privileging the values of the mainstream over the values of marginalised communities \parencite{toomey2021place}.

% To understand injustices suffered by those who do not fit in to the white, heteronormative, upper-class cateogy, Anguelovski et al. (\citeyear{anguelovski2020expanding}) propose two goals to advance the concept of justice: 1, uncovering the material and immaterial power, and 2, advancing new principles for equity in urban greening. - TODO explain these concepts?

\section{Conclusion}

TODO

\printbibliography

\end{document}

%%%%%%%%%%%%%%%%%%%%%%%%%%%%%%%%%%%%
%				COMMENTS
%%%%%%%%%%%%%%%%%%%%%%%%%%%%%%%%%%%%
\begin{comment}
 

\end{comment}