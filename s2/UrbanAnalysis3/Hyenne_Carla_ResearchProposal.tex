\documentclass{article}

\linespread{1.5}
\usepackage[utf8]{inputenc}
\usepackage[left=1.5in,right=1.5in,bottom=1in]{geometry}
\setlength\parindent{0pt}
\setlength{\parskip}{1em}
\setcounter{secnumdepth}{0}
\usepackage{outlines}
\usepackage{graphicx}
\graphicspath{ {imgs} }
\usepackage[hyphens]{url}
\usepackage{hyperref}
\usepackage{color,soul}
\usepackage[normalem]{ulem}

\usepackage[
backend=biber,
style=apa,
citestyle=authoryear,
sorting=nyt,
]{biblatex}
\addbibresource{refs.bib}

\usepackage{comment}
\specialcomment{topicsen}{\begingroup\bfseries\scriptsize}{\endgroup}
%\excludecomment{topicsen}

\newcommand{\alignedmarginpar}[1]{%
        \marginpar{\raggedright\small #1}
    }
    
\newcommand{\bisection}[1]{\textbf{\textit{#1}}}

\DeclareCiteCommand{\citeyear}
    {}
    {\bibhyperref{\printdate}}
    {\multicitedelim}
    {}

\title{Research Proposal}
\author{Carla Hyenne}
\date{}

\begin{document}

\maketitle

\tableofcontents 

%%%%%%%%%%%%%%%%%%%%%%%%%%%%%%%%%%%%%%%%%%%%%%
%				ABSTRACT (MAX 500 WORDS)
%%%%%%%%%%%%%%%%%%%%%%%%%%%%%%%%%%%%%%%%%%%%%%

\section{Abstract}

% max 500 words
- A lot of studies on geographic accessibility/proximity

- However, accessibility is a multidimensional and complex concept which cannot be reduced to the spatial distribution of natural spaces. 

- Particularly interesting with natural blue spaces is that they are fairly immobile. It is harder to move a river or lake, than it is to create a public green park. Given that users visits are determined most importantly by perceived accessibility compared to geographical distribution and proximity to the space, it is worthwhile to study people's perceptions and experiences of blue space visits

\begin{comment}
\section{Introduction}

% GENERAL INTRO ON BLUE SPACE

Most large cities are located near water, either inland like rivers, lakes or harbours, or on the coast like salt water. Water is important for the city. It serves as a life source, a communication channel, a motor for industry and the economy, for recreation and exercise, for cultural and spiritual practices, and for community forming and belonging. 

With urbanisation, water in cities became polluted or transformed by industries \parencite{kampa_langaas_anzaldua_2016}. 
Waterfronts became unattractive and unsafe to swim, thereby disrupting the relationship people had with water and its surrounding environment.
Starting in the 20th century, post-industrial European cities started considering waterfronts as strategic opportunities to revitalise the city \parencite{del2021dismantling}. In the last few decades, the importance of water in the city has entered public consciousness due in part to climate concerns. Most recently, the COVID-19 pandemic made people acutely aware of the benefits of urban blue-green spaces \parencite{kohsaka2021urban}.
% develop more on government reasons for investing in waterfront redevelopment, maybe ref. \parencite{del2021dismantling} as an example
Through local, national and international political pressure, public demand and (neoliberal) urban renewal, governments have invested significant resources in revitalising waterfronts into attractive natural places for people. 
And since public space is a highly valued commodity in the city, reconverting blue spaces is a great way to take advantage of unused areas.

% THE PROBLEM

The aim of this research is to understand the extent to which urban waterfronts support the uses, perceptions and meanings of wide range of people and communities

Towards this objective, I use the concept environmental justice and recognition justice, which articulate the ways in which people's perceptions and values of nature are represented in the city.

I explore this concept in the context of Copenhagen, a city . Specifically, I focus on \hl{three?} spaces with distinct characteristics: BS1 which has received a lot of public investment and is used by the city to promote liveability and sustainability; BS2 which ...; and BS3, which... .

Finally, the research unites the concepts of environmental justice and perceived accessibility to help illustrate ?the diversity of perceptions that exist in the city, and how blue spaces can allow, or not, different people and groups to express themselves.?

This is intended to give visibility into how different identities make sense of blue space, and how blue spaces give people the opportunity to express themselves...
\end{comment}

%%%%%%%%%%%%%%%%%%%%%%%%%%%%%%%%%%%%%%%%%%%%%%
%								LITERATURE REVIEW (2-3 PAGES)
%%%%%%%%%%%%%%%%%%%%%%%%%%%%%%%%%%%%%%%%%%%%%%
\pagebreak
\section{Literature review}

This section reviews the academic literature on blue spaces, and incorporates literature from wider concepts such as greening. It also introduces concepts that are central in understanding equity with regards to urban blue space, namely environmental justice and accessibility. These two concepts will be central to the research.

% Context and relevant background info - 2 paragraphs
\subsubsection{The benefits of urban blue spaces}

In an urban context, blue spaces have undeniable positive effects which Gascon et al. (\citeyear{gascon2017outdoor}) summarise as ``stress reduction, increased physical activity, promotion of positive social contacts, increased place attachment and the reduction of extreme temperatures''. The benefits fall under three categories.
First, being exposed to water makes people feel better, happier, and be more active. There is an extensive repertoire of quantitative studies demonstrating these effects on people's health and well-being (\cite{gascon2017outdoor}, \cite{britton2020blue}).
Qualitative studies also show that exposure to water improves mental health, regardless of how people interact with it (\cite{garrett2019urban}, \cite{van2021urban}).
Second, waterfronts give people the opportunity to connect with each other and with nature. Waterfront revitalisation projects can be an opportunity to create community bonds by engaging residents in the design and building process. For example, the ``urban acupuncture'' intervention conducted in a deprived area of Plymouth, UK, showed that participating residents reported a greater sense of well-being and life satisfaction due to feelings of community belonging and safety \parencite{van2021urban}.
Lastly, in the context of climate change, blue carbon ecosystems can naturally alleviate pollution, heat stress, flooding or drought, and increase the climate resiliency of cities (\cite{lin2020water}, \cite{o2021international}). 

Given the potential of water, and that public space is a highly valued commodity in the city, revitalising blue spaces into usable, attractive environments is a great way to take advantage of unused areas.
 
% Critical review, comparison, summary of literature - 1.5 pages
\subsubsection{The social and environmental consequences of blue urban renewal}

Despite the undeniable benefits of water in the city, transforming waterfronts into high-quality public space can have harmful consequences on people and the environment.
Two mechanisms of action are exclusionary planning, and neoliberal urban renewal \sout{which can displace people by way of gentrification.} These reinforce socio-spatial inequalities by discriminating against people on the basis of socio-economic and cultural differences, or by way of racist and sexist practices.

% disrupting relationships
First, in stark contrast to the social bonds that can be fostered when residents are involved in revitalisation projects, when the local community’s perceptions are not understood by planners, changes can disrupt human-to-human or human-to-water connections \parencite{toomey2021place}. This is particularly susceptible to happen when a community's social practices do not fit with the social norms \parencite{wessells2014urban}. 
Moreover, marginalised or stigmatised communities may find it hard or impossible to communicate their experience to the mainstream because they lack the words to articulate their reality. And vice-versa: wealthy, white, males may not be capable of understanding the experience of `others' \parencite{anguelovski2020expanding}. To this end, Toomey et al. (\citeyear{toomey2021place}) propose using language like ``place-disruption'' and ``place-protection'' to promote mutual understanding and avoid privileging the values of mainstream groups over those of marginalised communities.

% neoliberal urban renewal
Second, cities are prioritising economic growth over well-being and community. Local governments are exploiting nature-based solutions to brand their cities as green and liveable\footnote{For example, Madrid promoting the Madrid Río project on the official tourism website \parencite{madridrio}, or Oslo advertising its new urban waterfront promenade along which ``you find yourself surrounded by some of Oslo's world-renowned architectural gems'' \parencite{visitoslo}.}, and to promote greening as a win-win strategy where ``no one is left behind by the trickle-down of benefits from green infrastructure'' \parencite{anguelovski2021green}.
Anguelovski et al. (\citeyear{anguelovski2021green} explain that with ``glitzy green'' renewal projects, cities try to attract a new creative class rather than addressing public blue space as a common good and prioritising the concerns of existing residents (\cite{wessells2014urban}, \cite{anguelovski2020expanding}).
These strategies perpetuate inequalities by privileging the values of white, environmentally privileged upper classes who can afford to live near nature, thereby pricing out residents who will be displaced to neighbourhoods with less attractive nature.

\subsubsection{The environmental justice principle}

To articulate the phenomenon whereby natural spaces provide social and environmental benefits but at the same time discriminate against vulnerable populations, scholars have used the concept of environmental justice.
Environmental justice is based on the principle that everyone should have equal opportunities to access clean, healthy, unpolluted spaces, and in turn, share environmental burdens. As Agyeman et al. explain (\citeyear{agyeman2016trends}), it started as a social movement in the US in the 1980s at a time when it became obvious that ethnic minority and low-income populations were disproportionately exposed to polluted and degraded land.
Since then, environmental justice has concretised into an academic discourse and is typically broken down into three categories: distributional justice, procedural justice, and recognition justice.

When applied in the context of urban public blue and green space, distributional justice focuses on where these are situated in the city, and whether they address social, economic, racial or ethnic inequalities by striving to``avoid displacement and new negative green, ecological, climate and health effects'' \parencite{anguelovski2020expanding}.
Procedural justice deals with questions of discrimination in public participation and decision making. 
Finally, recognition justice addresses individual and community perceptions and preferences which may influence how people interact, or not, with the space.

\hl{LINK environmental justice and accessibility}

It follows that public space is where environmental (in)justice takes place. Thus, it is important to understand 

%Limitations of previous studies and gaps in knowledge - 1-2 paragraphs
\subsubsection{Geographical vs. perceived accessibility}

To date, studies that evaluate the degree to which people can make use of urban blue-green space have focused on measuring geographical accessibility, such as spatial distribution and proximity to people’s homes. % TODO cite studies?
However, this ignores the fact that accessibility is a multidimensional concept which cannot be reduced to purely a physical dimension \parencite{wang2015physical}. Perceived access is also important to consider when studying social benefits of blue-green space. Are people happier and healthier because they live near nature, or because they can afford to?
As Anguelovski et al. (\citeyear{anguelovski2020expanding}) put forward, environmental justice must go further in understanding ``how [...] people’s experiences of place shape their perception of access’’.

To this end, Wang et al. (\citeyear{wang2015physical}) suggest focusing on perceived accessibility, ie. ``the quality, diversity, and size of the green spaces or socio-personal characteristics including age, income, safety, and cultural concerns''. Comparing this approach to geographical accessibility, researchers studying two neighbourhoods with differing socio-economic status in Brisbane, Australia, concluded that perceived accessibility was better suited to explain park-use than their proximity to home \parencite{wang2015comparison}.
This shows that in the context of environmental justice, recognition can be more influential than distributional justice in detecting unequal access to nature. 

\subsubsection{Research question}

%Knowledge that your research will add - 1 paragraph
Although there is substantial evidence showing that perceived accessibility is significant in determining use of green space, there are limited studies that translate this idea to blue spaces.
However, blue spaces are particularly interesting because natural water bodies like rivers or lakes are relatively immobile and cannot be planned in the same way as public parks. 
In this context, distributional justice as a measure of accessibility becomes less relevant compared to recognition when it comes to providing equal opportunities for people to access waterfronts. Thus, it is worthwhile to explore subjective experiences relating to blue space which may or may not encourage usage.


My research will therefore focuse on perceived accessibility. This is important to study in order to uncover what immaterial hurdles are in the way of creating environmentally just cities when it comes to blue space.

% Research question
Given the above, my research aims to answer the following question: \textbf{to what extent do subjective experiences shape how (un)fairly accessible high quality, public blue spaces in the city are, and what does this mean for the environmental just city?}

%%%%%%%%%%%%%%%%%%%%%%%%%%%%%%%%%%%%%%%%%%%%%%
%								PROBLEM STATEMENT AND RQ (1 PAGE)
%%%%%%%%%%%%%%%%%%%%%%%%%%%%%%%%%%%%%%%%%%%%%%
\pagebreak
\section{Problem statement and research question}

% IDEAL - people have equal opportunities to access urban blue spaces
When blue spaces are well designed and distributed across the city, they offer many benefits for people and the environment. As such, making sure that a wide range of people from diversity of socio-economic backgrounds have equal access to public blue spaces helps to combat social inequalities. 

% REAL - physical and psychological barriers can restrict access to blue spaces
Although there is no direct economic barrier to public space (there is not entrance fee), rarely are public blue spaces fairly accessible to everyone. There exists both physical and psychological barriers which can prevent individuals, or whole communities, from enjoying waterfronts.

% CONSEQUENCES - increasing social and environmental inequalities (not everyone benefits fairly from blue/natural spaces), discrimination (racism, sexism...), displacement (green/blue gentrification)
Therefore, even if a blue space is freely open to the public, this does not mean that people use and benefit from them fairly. Understanding this phenomenon is important because public spaces are places of community, identity, attachment, and well-being, and ignoring experiences that differ from the mainstream increases social inequalities, discrimination, and displacement.

% RQ
Given the above, my research aims to answer the following question: \textbf{to what extent are high quality, public blue spaces in the city (un)fairly accessible, and for whom?}

% EXPLAIN THE CONCEPTS
% The research question is broken down as follows:
% \bisection{The extent} \hl{TODO}
% \bisection{Fairness} \hl{TODO}
% \bisection{Accessibility}  \hl{TODO} . What kind of experiences and perceptions shape people's perceived accessibility of a park? The factors can be both physical and non-physical.
%Non-physical barriers refer to how welcomed and safe people feel. Blue spaces, and public spaces in general, can be more or less welcoming to certain people. Gender, age, ethnicity, income, or preferences in aesthetics or activities, amongst other factors, all influence who feels comfortable, and who doesn't. 
%With regards to physical accessibility, those who live or work in the vicinity of blue space are more likely to visit it. The availability of public transport, bicycle infrastructure, and the quality of the roads and footpaths around and in the space, also influence who visits. 

% \bisection{For whom?} \hl{TODO}

%Discriminatory practices, such as the pressure to consume in order to fit in, excludes people.

%%%%%%%%%%%%%%%%%%%%%%%%%%%%%%%%%%%%%%%%%%%%%%
%							RESEARCH DESIGN (4-6 PAGES)
%%%%%%%%%%%%%%%%%%%%%%%%%%%%%%%%%%%%%%%%%%%%%%

\section{Research design}

To answer my research question, I will use the following sub-questions:

\begin{enumerate}
	\item Who are the users of the space, how do they perceive it and how do they feel in it?
	\item Why do they choose to use this space?
	\item How diverse are the users... 
	\item How 
\end{enumerate}

\subsection{Methodology}

Data on perceived accessibility is a subjective measure, based on qualitative data. It cannot be quantified in the same way as physical proximity

My research will be explanatory, because I aim to explain why the phenomenon of unequal access to blue space takes place (or not) based on principles of recognition justice. I will take an inductive approach, whereby my theory will emerge from data I will collect on people’s experiences, perceptions and preferences of a blue space. This data is both subjective and spatial. My approach to collecting the data will be to visit the blue space(s) and interview users, for which I will need a set of interview questions as well as a way to record spatial data like their home, or the blue/green spaces they frequent. These requirements lend themselves well to public participatory GIS (PPGIS), a map-based survey method linking qualitative and GIS data. Using PPGIS to study people’s relationship to green and blue space is recommended as research method which “might uncover local spatial knowledge and perceptions” (Anguelovski et al. 2020). It has been used to study people’s interactions with and the distribution of blue spaces in Helsinki metropolitan area (Raymond et al. 2016), and also by BlueHealth to “uncover spatial aspects of people’s relationships with blue spaces” (BlueHealth n.d.).


\subsection{Theoretical frameworks}

\bisection{Environmental justice}
Environmental justice (EJ) provides a lens through which to understand social and environmental inequalities related to blue spaces. By bringing together social and environmental concerns, the environmental justice paradigm advocates for the equal access to the benefits offered by natural spaces; and, in turn, sharing environmental burdens. 
% TODO: argue that sustainability not being only enviro but also eco and social, and \parencite{agyeman2016trends}
% introduce the history of environmental justice, starting in the USA (emphasis on racism), but also in Europe ofc (eg. amsterdam noord, ref. \parencite{del2021dismantling})
% talk about environmental privilege, which is the disproportionate access and benefit that white, upper class residents have because of their proximity to green/blue/open spaces \parencite{anguelovski2021green} and Park and Pellow
EJ is traditionally broken down into three dimensions: distributional justice, procedural justice, and recognition justice \parencite{todo:cite schlosberg}.

Distributional justice focuses on where blue spaces are situated in the city and whether they address social, economic, racial or ethnic inequalities by striving to``avoid displacement and new negative green, ecological, climate and health effects'' \parencite{anguelovski2020expanding}.

Procedural justice deals with questions of discrimination in public participation and decision making situations, even if ideal participation doesn't prevent spaces from being captured by gentrifiers\footnote{urban community gardens} \parencite{anguelovski2020expanding}. 

Finally, recognition justice addresses individuals' and communities’ perceptions, values and preferences which may influence how they interact, or not, with public space. This dimension is the most relevant to understand the subjective experiences of the users of the space.

% explain that this is most relevant for my research...
Recognising the experience of those who do not fit into the `norm' also means acknowledging that some practices take place in the private and not public sphere, because of historical racial, sexist, ethnic discrimination. For example, women who disproportionally carry out domestic and care work have a different daily pattern which does not match that of the average 9-5 worker, and the spaces and mobility options should be adapted for them to reach blue spaces with ease. Or, they may feel more vulnerable and less safe in public, and prefer private spaces \parencite{wessells2014urban}. How can blue spaces be inclusive of a diversity of people, carrying out a diversity of activities at all times of the day, week, or year? 
% Language like ``place-disruption'' and ``place-protection'' can promote mutual understanding, and avoid privileging the values of the mainstream over the values of marginalised communities \parencite{toomey2021place}.

% To understand injustices suffered by those who do not fit in to the white, heteronormative, upper-class cateogy, Anguelovski et al. (\citeyear{anguelovski2020expanding}) propose two goals to advance the concept of justice: 1, uncovering the material and immaterial power, and 2, advancing new principles for equity in urban greening. - TODO explain these concepts?

\bisection{Accessibility}

\subsection{Case study}

\bisection{What type of case study? what unit of analysis?}
Every blue space, neighbourhood and city will have a different set of social, political, economic, cultural, and environmental conditions which shape who uses the space, how they feel in it, and why they use it. 
In order to uncover these conditions, the unit of analysis should be scoped to a specific location on the water where people linger and have the opportunity to swim; this spot should have been rehabilitated by the city, and be public; and, it would be particularly interesting to study two rehabilitated blue spaces within one city, in order to compare the social consequences of investments in different neighbourhoods.

\bisection{Specific context of Copenhagen}
Given these considerations, Copenhagen makes for an interesting case for the following four reasons. First, Copenhagen is located on the Kattegat strait and has 92 km of coastline \parencite{comertler2017greens}, therefore water features prominently in the urban landscape. Second, due to the amount of shoreline in Copenhagen and the city trying to position itself as a world leader in sustainability, there have been many blue space rehabilitation projects since 2002. Today there are four harbour baths (Island Brygge, Fisketorvet, Sandkaj and Sluseholmen) and various urban beaches (Amager Strandpark, Svanemølle) \parencite{visitcopenhagen_baths}. Third, Copenhagen today is experiencing an increase in poverty and ethnic segregation (Moller and Larsen 2015), as well as a growing racist discourse in the media and politics. For example, through the classification of some neighbourhoods as ‘ghettos’  \parencite{simonsen2008practice}. This evolving socio-economic landscape and its surrounding discourse make it important to understand who feels included in blue spaces, and might not. Finally, Copenhagen’s reputation as “the most liveable city” \parencite{visitdenmark_2021}, due in part to the swimming spots in the harbour, begs the question - for whom is the city liveable?

- Mention similarity to Hong Kong, and perhaps limitations of that study

\hl{TODO:decide on specific blue spaces in Copenhagen}

Ultimately, unequal access to blue space is an environmental injustice and this framework will lead my research. Environmental justice is a multi-faceted concept which brings together social and environmental concerns. Amongst other things, it advocates for the equitable access to environmental benefits. The dimension of justice that I will focus on is recognition. Recognition justice deals with people and groups’ perceptions, values and preferences which may influence the ways in which they interact, or not, with a blue space \parencite{anguelovski2020expanding}.

\subsection{Feasibility?}

\section{Conclusion}

TODO

\printbibliography

\end{document}

%%%%%%%%%%%%%%%%%%%%%%%%%%%%%%%%%%%%
%				COMMENTS
%%%%%%%%%%%%%%%%%%%%%%%%%%%%%%%%%%%%
\begin{comment}
 

\end{comment}