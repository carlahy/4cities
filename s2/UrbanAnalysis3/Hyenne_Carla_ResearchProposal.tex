\documentclass{article}

\linespread{1.5}
\usepackage[utf8]{inputenc}
\usepackage[left=1.5in,right=1.5in,bottom=1in]{geometry}
\setlength\parindent{0pt}
\setlength{\parskip}{1em}
\setcounter{secnumdepth}{0}
\usepackage{outlines}
\usepackage{graphicx}
\graphicspath{ {imgs} }
\usepackage[hyphens]{url}
\usepackage{hyperref}
\usepackage{color,soul}
\usepackage[normalem]{ulem}

\usepackage[
backend=biber,
style=apa,
citestyle=authoryear,
sorting=nyt,
]{biblatex}
\addbibresource{refs.bib}

\usepackage{comment}
\specialcomment{topicsen}{\begingroup\bfseries\scriptsize}{\endgroup}
%\excludecomment{topicsen}

\newcommand{\alignedmarginpar}[1]{%
        \marginpar{\raggedright\small #1}
    }

\title{Research Proposal}
\author{Carla Hyenne}
\date{}

\begin{document}

\maketitle

\tableofcontents 

%%%%%%%%%%%%%%%%%%%%%%%%%%%%%%%%%%%%
%				ABSTRACT (MAX 500 WORDS)
%%%%%%%%%%%%%%%%%%%%%%%%%%%%%%%%%%%%

\section{Abstract}


%%%%%%%%%%%%%%%%%%%%%%%%%%%%%%%%%%%%
%				LITERATURE REVIEW (2-3 PAGES)
%%%%%%%%%%%%%%%%%%%%%%%%%%%%%%%%%%%%
\pagebreak
\section{Literature review}

Most large cities are located around water, either inland water like rivers, lakes, ponds, or harbours, or salt water on the coasts.
Water is important to the city. It serves as a life source, a communication channel, a motor for industry and the economy, for recreation and exercise, for cultural and spiritual practices, and for community forming and belonging.
%- \parencite{toomey2021place} on community identity at a polluted/degraded blue space
%- \parencite{van2021urban} on community engagement, feelings of belonging, safety

Through urbanisation, water in cities became polluted or transformed by industries \parencite{kampa_langaas_anzaldua_2016}, disrupting the relationship of people with waterfronts and their natural environments. The waterfront became unattractive and unsafe to swim.
In the last few decades, the rising environmental and climate concerns have sensibilised people to the importance of water in the city. With (international) political pressure and public demand, governments have invested significant resources in rehabilitating waterfronts into attractive natural places for people to enjoy.

Making a body of water swimmable is a large public investment: the water itself has to be cleaned, which involves identifying and dealing with polluting sources; the waterbed and surrounding environments have to be rehabilitated to safely welcome people; and the waterfront has to be equipped with infrastructure for swimming, sitting, walking, and more. Since public space is a highly valued commodity in a city, reconverting blue spaces is a great way to take advantage of unused areas, and revitalising waterfronts into high quality spaces prove to be valuable for a multitude of reasons. 

Firstly, in the context of climate change, blue spaces play a crucial role in mitigating pollution, heat stress, etc., similarly to green spaces

	1. Blue spaces act as heat sinks \parencite{lin2020water}

	2. Blue and green spaces can work alongside each other, and in tandem

	3. Urban planning projects to revitalise, rehabilitate, reconnect urban water and waterfronts- includes  ``pollution mitigation, coastal restoration, climage change adaptation, rezoning, economic development, and increased public access to waterfront areas'' \parencite{toomey2021place}

Secondly, blue space has the ability to improve people's physical and psychological wellbeing. Being exposed to water makes people feel better, happier, and be more active \parencite{gascon2017outdoor}, much like green spaces. 
There is an extensive repertoire of quantitative studies looking into the health and wellbeing effects of blue spaces, which can be summarised as ``stress reduction, increased physical activity, promotion of positive social contacts, increased place attachment and the reduction of extreme temperatures'' \parencite{gascon2017outdoor}. Qualitative studies have also shown that exposure to water improves mental health, regardless of whether or not people intentionally interact with the water or just experience it from their apartment window \parencite{garrett2019urban}.
In deprived areas in particular, people tend to have poorer mental health and lower life satisfaction compared to wealthier areas \parencite{van2021urban}. Revitalising blue spaces in these areas to increase access to high quality waterfronts has the potential to have an even greater influence on the wellbeing of those living in socio-economically disadvantaged neighbourhoods.

Thirdly, waterfronts provide people with the opportunity to connect with each other and with nature. People develop a strong sense of place attached to water. Even polluted or degraded waterfronts can play a central role in the community, providing refuge, connection, entertainment, and even food \parencite{toomey2021place}. 
Furthermore, waterfront revitalisation projects can be an opportunity to create community bonds by engaging residents in the design and building process. In the``urban acupuncture'' intervention done by BlueHealth in a deprived area of Plymouth, UK, residents who participated in the project reported a greater sense of wellbeing and life satisfaction due to feelings of community belonging and safety \parencite{van2021urban}.

Despite the undeniable benefits of urban blue spaces, revitalisation projects can have harmful consequences on people and the environment.
In a stark contrast to the social bonds that can be fostered when residents are involved in revitalising a waterfront, when the local community's perceptions and values are not understood by the planners, changes can disrupt human-to-human or human-to-water connections \parencite{toomey2021place}. In perverse cases, cities prioritise growth rather than wellbeing and community. With``flashy glitzy green (blue)'' (todo:ref) infrastructure like the XPROJECT in CITYX, cities try to attract a new creative class rather than addressing public blue spaces as a common good, and prioritising the concerns of existing residents (\cite{wessells2014urban}, \cite{anguelovski2020expanding}). 
Such projects privilege the values of new groups over the values of existing residents, who risk facing displacement. Those who can afford to live near blue or green spaces will move in and price out the lower classes, who will be forced to move to neighbourhoods with worse access to attractive blue space. On one hand, quality waterfronts reduce social and environmental inequalities, but at the same time cause gentrification \parencite{todo:REF OF GENTRIFICATION PAPER}

How do we make sure that everyone in the city benefits equally from high quality blue spaces? Given that cities promote their waterfronts (TODO:give example of promotion/slogan), and the numerous qualitative and quantitative studies on the benefits that water has on people, the qualities of blue spaces are well known by governments, planning offices and academics. On the other hand, the social and environmental consequences of revitalisation projects are not always understood or taken into account by planners. If cities are to reduce inequalities regarding access to natural spaces, blue space interventions need to understand social and environmental consequences and trade-offs attached.

The inequalities surrounding access to natural spaces is articulated through environmental justice, a multi-faceted concept which brings together social and environmental concerns. Amongst other things, it advocates for the equal access to the benefits offered by natural spaces. The justice paradigm is typically broken down into three dimensions: distributional justice, procedural justice, and recognition justice \parencite{todo:cite schlosberg}.

	2. Distributional justice focuses where blue spaces are situated in the city and whether they address social, economic, racial or ethnic inequalities by attempting to``avoid displacement and new negative green, ecological, climate and health effects'' \parencite{anguelovski2020expanding} on disadavantaged populations.

	3. Procedural justice deals with questions of discrimination in public participation and decision making situations. Nonetheless, even ideal participation doesn't prevent spaces from being captured by gentrifiers\footnote{urban community gardens}.

	4. Recognition justice addresses individuals and communities’ perceptions, values and preferences which may influence the ways in which they interact, or not, with public space \parencite{anguelovski2020expanding}

Marginalised or stigmatised communities may find it hard or impossible to communicate their experience to the mainstream because they lack the words to articulate their reality. And vice-versa, white, heteronormative societies may not be capable of appreciating the experience of others. This of course does not mean that minority communities do not attach meaning to place, but that groups with distinct value systems need a common language to communicate. To this end, Toomey et al. propose using language like``place-disruption'' and``place-protection'' to promote mutual understanding and avoid privileging the values of mainstream groups over those of marginalised communities.

Recognising the experience of those who do not fit into a heternormative, white society also means acknowledging that some practices take place in the private and not public sphere due to historical discrimination. For example, women carry out domestic and care work and their daily pattern may not match that of the average 9-5 worker; they may feel more vulnerable and less safe in public, thus preferring private spaces \parencite{wessells2014urban}. How can blue spaces be inclusive of a diversity of people, carrying out a diversity of activities at all times of the day, week, or year?

	5. These three dimensions of environmental justice have been extensively used by scholars to understand (in)justices linked to blue green spaces. Taking this traditional framework of environmental justice further, Anguelovski et al. propose two goals to advance the concept of justice: 1, uncovering the material and immaterial power, and 2, advancing new principles for equity in urban greening.  ==explain these concepts==
	
	6. Importance of situating ??
	

Right to the city enabled through public blue space (in connection to recognition justice)

	1. Public blue spaces are a shared, common good, that everyone should be able to acces \parencite{wessells2014urban}	

	2. todo:cite more general right to the city literature

	3. Sharing the benefits that waterfronts have on people and the environment
	
!!!!!! individuals and communities are inextricably linked to place, and disrupting place/not providing it threatens their existence / belonging


%%%%%%%%%%%%%%%%%%%%%%%%%%%%%%%%%%%%
%				PROBLEM STATEMENT AND RQ (1 PAGE)
%%%%%%%%%%%%%%%%%%%%%%%%%%%%%%%%%%%%

\section{Problem statement and research question}

Ideal- everyone can access blue spaces equally in the city, this is important because interacting with blue spaces is good for mental and physical health 
Real- physical and psychological barriers can restrict access to blue spaces ()
Consequence- , ultimately leading to environmental injustices (not everyone benefits fairly from blue/natural spaces), discrimination (racism, sexism...), displacement (green/blue gentrification)

Reviewed RQ: to what extent are the perceptions and values of Copenhagen's communities represented in the blue spaces?


%%%%%%%%%%%%%%%%%%%%%%%%%%%%%%%%%%%%
%				RESEARCH DESIGN (4-6 PAGES)
%%%%%%%%%%%%%%%%%%%%%%%%%%%%%%%%%%%%

\section{Research design}

- specific context of copenhagen, similar to hongkong 

\printbibliography

\end{document}

%%%%%%%%%%%%%%%%%%%%%%%%%%%%%%%%%%%%
%				COMMENTS
%%%%%%%%%%%%%%%%%%%%%%%%%%%%%%%%%%%%
\begin{comment}
 

\end{comment}