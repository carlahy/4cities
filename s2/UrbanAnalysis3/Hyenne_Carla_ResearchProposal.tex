\documentclass{article}

\linespread{1.5}
\usepackage[utf8]{inputenc}
\usepackage[left=1.5in,right=1.5in,bottom=1in]{geometry}
\setlength\parindent{0pt}
\setlength{\parskip}{1em}
\setcounter{secnumdepth}{0}
\usepackage{outlines}
\usepackage{graphicx}
\graphicspath{ {imgs} }
\usepackage[hyphens]{url}
\usepackage{hyperref}
\usepackage{color,soul}
\usepackage[normalem]{ulem}

\usepackage[
backend=biber,
style=apa,
citestyle=authoryear,
sorting=nyt,
]{biblatex}
\addbibresource{refs.bib}

\usepackage{comment}
\specialcomment{topicsen}{\begingroup\bfseries\scriptsize}{\endgroup}
%\excludecomment{topicsen}

\newcommand{\alignedmarginpar}[1]{%
        \marginpar{\raggedright\small #1}
    }
    
\newcommand{\bisection}[1]{\textbf{\textit{#1}}}

\title{Research Proposal}
\author{Carla Hyenne}
\date{}

\begin{document}

\maketitle

\tableofcontents 

%%%%%%%%%%%%%%%%%%%%%%%%%%%%%%%%%%%%
%				ABSTRACT (MAX 500 WORDS)
%%%%%%%%%%%%%%%%%%%%%%%%%%%%%%%%%%%%

\section{Abstract}


%%%%%%%%%%%%%%%%%%%%%%%%%%%%%%%%%%%%
%				LITERATURE REVIEW (2-3 PAGES)
%%%%%%%%%%%%%%%%%%%%%%%%%%%%%%%%%%%%
\pagebreak
\section{Literature review}

Water is important for the city. It serves as a life source, a communication channel, a motor for industry and the economy, for recreation and exercise, for cultural and spiritual practices, and for community forming and belonging. As such, most large cities are located near water, either inland like rivers, lakes or harbours, or on the coast like salt water.

Through urbanisation, water in cities became polluted or transformed by industries \parencite{kampa_langaas_anzaldua_2016}, disrupting the relationships people had with waterfronts and the surrounding environment. Waterfronts were unattractive, and the water unsafe to swim.
Starting in the 20th century, post-industrial European cities started considering waterfronts as strategic opportunities to revitalise the city \parencite{del2021dismantling}, and in the last few decades, the importance of water in the city has entered public consciousness due in part to climate concerns.
% develop more on government reasons for investing in waterfront redevelopment, maybe ref. \parencite{del2021dismantling} as an example
Through local and international political pressure, public demand and (neoliberal) urban renewal, governments have invested significant resources in revitalising waterfronts into attractive natural places for people. 

Making a body of water swimmable is a large public investment: the water itself has to be cleaned, which involves identifying and dealing with polluting sources; the waterbed and surrounding environments have to be rehabilitated to safely welcome people; and the waterfront has to be equipped with infrastructure for swimming, sitting, walking, and more. Since public space is a highly valued commodity in the city, reconverting blue spaces is a great way to take advantage of unused areas and proves to be valuable for a multitude of reasons. 

% define blue spaces, and blue interventions: eg. in \parencite{anguelovksi2021green}, a green interventions as nature-based infrastructures and amenities
% maybe add a sentence that introduces what is coming in the review, eg. 1. many benefits "for the people", but actually 2. for whom??

\subsection{The benefits of blue spaces}

\bisection{Climate adaptation}
In the context of climate change, blue spaces have a role to play in mitigating pollution, heat stress, XXX and XXX. 
Blue spaces are ecosystems that naturally alleviate climate change impacts, such as urban heat islands. They have a lower surface temperature than green spaces, evaporative effects and high heat capacity, and a spillover mechanism which extends the cooling effect up to 200 meters \parencite{lin2020water}.

However, blue spaces are more complicated to design and maintain than green spaces, and as a result aren't as prevalent in urban planning as green space \parencite{TODO:find ref in lin2020water}. 

- act in both a similar and complementary way to green spaces//blue spaces can work in tandem with green spaces. for example: the morphology of blue spaces determines the extent to which they cool the surrounding area, and the larger and more compact a space, the greater the impact. thus, lakes/oceans/reservoirs have a greater impact, and rivers less impact; they need to be combined with green islands; the spillover effect is only 200m, after which green space is necessary to provide cooling effect in uhi 	\parencite{lin2020water}

\bisection{Health and well-being}
Blue space has the ability to improve people's physical and psychological health. Being exposed to water makes people feel better, happier, and be more active \parencite{gascon2017outdoor}. 
There is an extensive repertoire of quantitative studies looking into the health and well-being effects of blue spaces, which can be summarised as ``stress reduction, increased physical activity, promotion of positive social contacts, increased place attachment and the reduction of extreme temperatures'' \parencite{gascon2017outdoor}. Qualitative studies have also shown that exposure to water improves mental health, regardless of whether people intentionally interact with the water, or just experience it from their apartment window \parencite{garrett2019urban}.
In deprived neighbourhoods in particular, people tend to have poorer mental health and lower life satisfaction compared to wealthier areas \parencite{van2021urban}. Projects that increase access to high quality blue spaces in socio-economically disadvantaged neighbourhoods have the potential to greatly influence the well-being of their residents.

\bisection{Connection and community}
Waterfronts give people with the opportunity to connect with each other and with nature. People develop a sense of place attached to water. Even polluted or degraded waterfronts can play a central role in the community, providing refuge, connection, entertainment, and even food \parencite{toomey2021place}. 
Furthermore, blue space revitalisation projects can be an opportunity to create community bonds by engaging residents in the design and building process. In the ``urban acupuncture'' intervention conducted by BlueHealth in a deprived area of Plymouth, UK, residents who participated in the project reported a greater sense of well-being and life satisfaction due to feelings of community belonging and safety \parencite{van2021urban}.

\subsection{Social and environmental repercussions of blue space revitalisation}

\bisection{Disrupting community}
Despite the undeniable benefits of urban blue spaces, interventions can have harmful consequences on people and the environment.
In stark contrast to the social bonds that can be fostered when residents are involved in revitalisation projects, when the local community's perceptions and values are not understood by planners, changes can disrupt human-to-human or human-to-water connections \parencite{toomey2021place}. In perverse cases, cities prioritise growth over well-being and community. With ``glitzy green'' \parencite{anguelovski2021green} renewal projects like the Amsterdam Noord waterfront, cities try to attract a new creative class rather than addressing public blue spaces as a common good and prioritising the concerns of existing residents (\cite{wessells2014urban}, \cite{anguelovski2020expanding}, \cite{del2021dismantling}).
Such projects privilege the values of new groups over that of existing residents, who risk displacement. Those who can afford to live near green or blue spaces will move in and price out the lower classes, who will be forced to move to neighbourhoods with worse access to attractive natural spaces. 
Paradoxically, quality waterfronts appear to reduce social and environmental inequalities but also cause gentrification. 

% \bisection{Disrupting natural environments}

Thus, how do we make sure that everyone in the city benefits equally from high quality blue spaces? The benefits are well known by governments, planning offices and academics; we see this through the promotion of waterfronts (eg. Madrid Río \parencite{madridrio}, or Oslo's urban waterfront promenade \parencite{visitoslo} on the city's tourism website), as well as by the numerous qualitative and quantitative studies on health and well-being or climate adaptation.
On the other hand, the social and environmental consequences of revitalisation projects are not systematically taken into account by planners. If cities are to reduce inequalities associated to access to natural spaces, the social and environmental consequences and trade-offs must be understood, and greening (or blueing) projects must be accompanied by ``bold, progressive social policies'' \parencite{anguelovski2021green}.

% even in cities with relatively strong social welfare policies, urban renewal projects

% gentrification and the green rent gap

% trickle down effects, greening/blueing as a win-win strategy  `` Often underpinned by a list of ecological, health, social and economic benefts, when residents listen to public offcials, planners, investors and even health and ecology experts, there are told that greening is a win-win strategy. No one is left behind by the trickle-down of benefts from green infrastructure. Green increases value, helps you connect with your neighbors, encourages physi- cal activity and recreation and fulflls climate goals. It’s a 100\% fail-proof pack- age, a drumbeat that seems to have cities across the globe marching in unison.''

\subsection{Achieving equity and justice in urban blue spaces}

\bisection{Environmental justice}
Environmental justice provides a lens through which to understand these inequalities. By bringing together social and environmental concerns, the environmental justice paradigm advocates for the equal access to the benefits offered by natural spaces; and, in turn, sharing environmental burdens. 
\hl{talk about sustainability not being only enviro but also eco and social}
% introduce the history of environmental justice, starting in the USA (emphasis on racism), but also in Europe ofc (eg. amsterdam noord, ref. \parencite{del2021dismantling})
% talk about environmental privilege, which is the disproportionate access and benefit that white, upper class residents have because of their proximity to green/blue/open spaces \parencite{anguelovski2021green} and Park and Pellow
The environmental justice paradigm is traditionally broken down into three dimensions: distributional justice, procedural justice, and recognition justice \parencite{todo:cite schlosberg}.

Distributional justice focuses on where blue spaces are situated in the city and whether they address social, economic, racial or ethnic inequalities by striving to``avoid displacement and new negative green, ecological, climate and health effects'' \parencite{anguelovski2020expanding}.

Procedural justice deals with questions of discrimination in public participation and decision making situations. Nonetheless, even ideal participation doesn't prevent spaces from being captured by gentrifiers\footnote{urban community gardens} \parencite{anguelovski2020expanding}.

Finally, recognition justice addresses individuals' and communities’ perceptions, values and preferences which may influence the ways in which they interact, or not, with public space \parencite{anguelovski2020expanding}.
Marginalised or stigmatised communities may find it hard or impossible to communicate their experience to the mainstream because they lack the words to articulate their reality. And vice-versa: white, heteronormative societies may not be capable of understanding the experience of `others'. This of course does not mean that minority communities do not attach meaning to place, but that two or more groups with distinct value systems need a common language to communicate. To this end, Toomey et al. propose using language like ``place-disruption'' and ``place-protection'' to promote mutual understanding and avoid privileging the values of mainstream groups over those of marginalised communities \parencite{toomey2021place}.

\bisection{Recognition}
Recognising the experience of those who do not fit into the `norm' also means acknowledging that some practices take place in the private and not public sphere. This is due to historical racial, sexist, ethnic discrimination. For example, women who disproportionally carry out domestic and care work have a different daily pattern which does not match that of the average 9-5 worker, and the spaces and mobility options should be adapted for them to reach blue spaces with ease. Or, they may feel more vulnerable and less safe in public, and prefer private spaces \parencite{wessells2014urban}. How can blue spaces be inclusive of a diversity of people, carrying out a diversity of activities at all times of the day, week, or year? 

%These three dimensions of environmental justice have been extensively used by scholars to understand (in)justices linked to blue and green spaces. Taking this traditional framework of environmental justice further, Anguelovski et al. propose two goals to advance the concept of justice: 1, uncovering the material and immaterial power, and 2, advancing new principles for equity in urban greening.  --> todo: explain these concepts

% overcoming barriers to justice and equity --> ``questioning the social exclusion and white privilege embedded in new greening landscapes and how those social dynamics reshape security and vulnerability'' \parencite{anguelovski2021green}

\begin{comment}
	6. Importance of situating ??


Right to the city enabled through public blue space (in connection to recognition justice)

	1. Public blue spaces are a shared, common good, that everyone should be able to acces \parencite{wessells2014urban}	

	2. todo:cite more general right to the city literature

	3. Sharing the benefits that waterfronts have on people and the environment
	
!!!!!! individuals and communities are inextricably linked to place, and disrupting place/not providing it threatens their existence / belonging

\end{comment}


%%%%%%%%%%%%%%%%%%%%%%%%%%%%%%%%%%%%
%				PROBLEM STATEMENT AND RQ (1 PAGE)
%%%%%%%%%%%%%%%%%%%%%%%%%%%%%%%%%%%%

\section{Problem statement and research question}

Ideal- everyone can access blue spaces equally in the city, this is important because interacting with blue spaces is good for mental and physical health 
Real- physical and psychological barriers can restrict access to blue spaces ()
Consequence- , ultimately leading to environmental injustices (not everyone benefits fairly from blue/natural spaces), discrimination (racism, sexism...), displacement (green/blue gentrification)

Reviewed RQ: to what extent are the perceptions and values of Copenhagen's communities represented in the blue spaces?


%%%%%%%%%%%%%%%%%%%%%%%%%%%%%%%%%%%%
%				RESEARCH DESIGN (4-6 PAGES)
%%%%%%%%%%%%%%%%%%%%%%%%%%%%%%%%%%%%

\section{Research design}

- specific context of copenhagen, similar to hongkong 

\printbibliography

\end{document}

%%%%%%%%%%%%%%%%%%%%%%%%%%%%%%%%%%%%
%				COMMENTS
%%%%%%%%%%%%%%%%%%%%%%%%%%%%%%%%%%%%
\begin{comment}
 

\end{comment}