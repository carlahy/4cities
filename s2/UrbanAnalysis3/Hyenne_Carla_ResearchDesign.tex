\documentclass{article}

\linespread{1.5}
\usepackage[utf8]{inputenc}
\usepackage[left=1.5in,right=1.5in,bottom=1in]{geometry}
\setlength\parindent{0pt}
\setlength{\parskip}{1em}
\setcounter{secnumdepth}{0}
\usepackage{outlines}
\usepackage{graphicx}
\graphicspath{ {imgs} }
\usepackage[hyphens]{url}
\usepackage{hyperref}
\usepackage{color,soul}
\usepackage[normalem]{ulem}

\usepackage[
backend=biber,
style=apa,
citestyle=authoryear,
sorting=nyt,
]{biblatex}
\addbibresource{refs.bib}

\usepackage{comment}
\specialcomment{topicsen}{\begingroup\bfseries\scriptsize}{\endgroup}
%\excludecomment{topicsen}

\newcommand{\alignedmarginpar}[1]{%
        \marginpar{\raggedright\small #1}
    }

\title{Research Design}
\author{Carla Hyenne}
\date{}

\begin{document}

\maketitle

%%%%%%%% INTRO %%%%%%%%

Much like green spaces, blue spaces in the city have a positive impact on people's perceived mental and physical wellbeing - being near water makes people feel better and happier. Many cities have public natural spaces like rivers, lakes, harbours or canals, but they aren't always adapted for leisure. To be attractive to people, waterfronts need to be clean, safe, welcoming, accessible, offer a diversity of uses for a diversity of users, amongst other factors. 

In the last centuries, water was polluted by industries, and became a place unsafe for swimming. Through public pressure or environmental concerns, cities have invested resources in cleaning and restoring the water and waterfronts into places where people would want to spend time lounging, swimming or other. This restoration is not a simple feat. 

The reconversion/rehabilitation projects require investment and planning from the city. The water itself has to be cleaned, which involves identifying and dealing with polluting sources; the water bed and the surrounding environment have to be rehabilitated to safely welcome people; and the waterfront has to be fitted with infrastructure for swimming, sitting, walking, and other activities.

TODO: explain more why swimming
%Within these spaces, I am focusing specifically on areas that are made swimmable, because this requires a greater effort and investment than only upgrading the waterfront with spaces for walking, sitting or playing.

Additionally, blue space rehabilitation projects are not profit-oriented. Natural water bodies in European cities are usually public spaces that are freely accessible - you do not need to pay to use them. Without a monetary incentive, the city's discourse with regards to blue space rehabilitation is for the benefit of people and not directly a money-making mechanism.

In reality, despite blue space being designed for people, it may not be fairly accessible for everyone. The needs of some may be more represented than others, and more vulnerable communities might not be able to access the space, and thus do not enjoy the same benefits. Differences in experience are based on a multitude of socioeconomic factors - gender, age, physical or mental abilities, income, ethnicity, education - and also on practical matters such as the distance between home and water, itself often based on socioeconomic differences.

The reasons why it is important to understand inequalities in access to high quality blue space are two-fold. 
First, those who do not have the opportunity to interact with the space (directly or indirectly) are not able to benefit from the positive effects blue spaces have on wellbeing. % Vulnerable communities become even more disadvantaged compared to those with more privilege, who gain valuable natural public space.
Second, when we ignore the fact that reconverted blue space raises the attractiveness of the neighbourhood or the city, we ignore the problems that the working and middle classes will face, such as displacement caused by a rise in rental and housing prices. This deepens inequalities by constraining less-privileged households to move to neighbourhoods deprived of natural spaces, and where they cannot benefit from the wellbeing effects.

Given the above, my research aims to answer the following question: \textbf{why do natural, urban swimming spaces hinder environmental justice in the city?}

\sout{to what extent does a natural blue space, rehabilitated into a swimming spot, contribute to environmental justice in the city?}

\textbf{what can access tell us about the ways in which natural, urban blue space developed into a swimming spot contribute, or not, to environmental justice int he city?}

To answer this, I will address the following questions:

\begin{enumerate}
	\item Who are the users of the space, how do they perceive it and how do they feel in it?, which will help to understand
	\item Who benefits from the space, and what are the factors that make the space accessible to them?, which will help to understand
	\item How accessible is the space to a diversity of people (based on socioeconomic factors like income, education, age, immigration background, etc. \parencite{baro2021school})?, which will help to understand
	\item Why are blue spaces unfairly accessible, available and attractive, thereby hindering environmental justice?
\end{enumerate}

%%%%%%%% GUIDING FRAMEWORKS %%%%%%%%

These concerns align with the concept of urban environmental justice, which advocates for equitably sharing the benefits of natural spaces  in the city. Specifically, my research will focus distributive justice. Distributive environmental justice with regards to blue urban spaces refers to whether all communities have equal access to, availability of and attractiveness of the blue spaces near where they live \parencite{kronenberg2020environmental}. 
``Blueing'' cities undeniably makes ...
 on how the location of blue space rehabilitation influences who uses it, and how.
 
 Explanatory -> why do this mechanism/phenomenon happen? 

% The concept of justice can be further broken down into three categories: distributional justice, participatory justice, and recognition justice. 

Thus, my research will be inscribed in the framework of environmental justice, and more specifically in <distributional justice?>
Therefore, it is particularly interesting to study the rehabilitation of natural bodies of water in a city 

To conduct my research, I will take an inductive approach. 


%%%%%%%% THEORETICAL FRAMEWORKS %%%%%%%%




%%%%%%%% METHODS %%%%%%%%



%%%%%%%% CASE STUDY %%%%%%%%

The city I will focus my research on is Copenhagen.
The reasons Copenhagen makes for an interesting case with regards to my research are as follows. Firstly, the city is surrounded by water and there are many natural inland water features ("residents are never farther than x kilometres from the shoreline"). 
Second, due to the amount of shoreline in Copenhagen, there have been many blue space rehabilitation projects in the last decades. The XX, XX, and XX harbours have all been made accessible as swimming spaces, and the XX is currently undergoing a transformation.
Thirdly, the socioeconomic landscape of the city makes it worthwhile to evaluate distributional justice. Since the early 2010s, the rate of immigration and the number of working-class households have increased, at the same time as racist and xenophobic discourses in the media and politics. 
Finally, Copenhagen's reputation as ``the most liveable city'' \parencite{visitdenmark_2021}, due in part to the swimming spots in the harbour, begs the question - for whom is the city liveable?



\printbibliography

\end{document}
