\documentclass{article}

\linespread{1.5}
\usepackage[utf8]{inputenc}
\usepackage[left=1.5in,right=1.5in,bottom=1in]{geometry}
\setlength\parindent{0pt}
\setlength{\parskip}{1em}
\setcounter{secnumdepth}{0}
\usepackage{outlines}
\usepackage{graphicx}
\graphicspath{ {imgs} }
\usepackage[hyphens]{url}
\usepackage{hyperref}
\usepackage{color,soul}
\usepackage[normalem]{ulem}

\usepackage[
backend=biber,
style=apa,
citestyle=authoryear,
sorting=nyt,
]{biblatex}
\addbibresource{refs.bib}

\usepackage{comment}
\specialcomment{topicsen}{\begingroup\bfseries\scriptsize}{\endgroup}
%\excludecomment{topicsen}

\newcommand{\alignedmarginpar}[1]{%
        \marginpar{\raggedright\small #1}
    }

\title{Research Design}
\author{Carla Hyenne}
\date{}

\begin{document}

\maketitle

%%%%%%%% INTRO & THEORETICAL FRAMEWORKS %%%%%%%%

TODO: explain more why swimming

%%% blue spaces increase people's perceived sense of wellbeing
Urban blue spaces have a positive influence on people's perceived mental and physical wellbeing - being near water makes people feel better and happier \parencite{gascon2017outdoor}.
%%% people living in deprived areas benefit relatively more from blue spaces compared to more privileged neighbourhoods
In particular, blue spaces have the potential to have a greater impact on the wellbeing of those living in socio-economically disadvantaged neighbourhoods. This is due to low socio-economic status being linked to poorer mental health and lower life satisfaction \parencite{van2021urban}.

%%% blue spaces exist in cities but they aren't always attractive
Many cities have natural water like rivers, lakes, harbours or canals. Unfortunately, over the last centuries, water bodies were polluted by industries and became unsafe to swim \parencite{???}. Through public pressure or for environmental concerns \parencite{???}, cities have invested resources into cleaning and restoring water and waterfronts into attractive natural places for people to enjoy.
%%% efforts are put into turning them into attractive places- quality, public blue spaces where you can swim require investments from the city
Making a body of water swimmable is a large public investment: the water itself has to be cleaned, which involves identifying and dealing with polluting sources; the waterbed and surrounding environments have to be rehabilitated to safely welcome people; and the waterfront has to be equipped with infrastructure for swimming, sitting, walking, and more.
Since public space is a highly valued commodity in a city, reconverting blue spaces is a great way to take advantage of unused areas.

%%% blue spaces are not equally accessible by everyone - by analysing blue spaces with a environmental justice lens, we see that they are unfairly accessible and the most vulnerable populations do not get to feel the benefits

% in reality, blue spaces are not equally accessible to everyone in a city- physical accessibility, psychological accessibility, ???
In reality, despite public blue spaces being almost always free to access - you do not need to pay to use them - they may not be fairly accessible to everyone. 
Firstly, physical accessibility. Those who are most likely to frequent a blue space are those living or working in the vicinity \parencite{???}. The availability of public transport, bicycle infrastructure, the quality of the roads and footpaths around and in the space, also influence who visits.

Second, psychological accessibility. Blue spaces (and public spaces in general) can be more or less welcoming to certain groups of people. Specifically, gender, age, ethnicity, or general preferences in aesthetics or activities influence who feels welcomed or attracted to a space. Discriminatory and exclusionary practices, such as being made to feel like  spending money is necessary to fit in socially in the space, can make people feel unwelcomed \parencite{???}.

% Third, ??? TODO: framework for these statements

Thus, even if a blue space is freely open to the public, this does not mean that people use and benefit from them fairly. 
This phenomenon is important to understand because because unequal access to blue space is an environmental injustice.

% explain environmental justice

Environmental justice is a multi-faceted concept which brings together social and environmental concerns. Amongst other things, it advocates for the equitable access to environmental benefits, like the increased wellbeing that exposure to blue space provides.

% explain justice 
Environmental justice itself can be broken down into three categories: distributional justice, procedural justice, and recognition justice. 
Distributional justice focuses on social consequences based on the location of blue space interventions in the city, such as environmental gentrification.
Procedural justice deals with inclusionary and exclusionary <practices> in participation and decision making.
Recognition justice is concerned with people and communities' perceptions and values attached to a space \parencite{anguelovski2020expanding}.

...

Public spaces are places of community, identity and attachment, and everyone in the city should have equal rights to them \parencite{agyeman2016trends}. When someone's access to public space is restricted, this restriction is an act against the self-determination of the person's community and identity.


% when we ignore the fact that reconverted blue space raises the attractiveness of the neighbourhood or the city, we ignore the problems that the working and middle classes will face, such as displacement caused by a rise in rental and housing prices. This deepens inequalities by constraining less-privileged households to move to neighbourhoods deprived of natural spaces, and where they cannot benefit from the wellbeing effects.

Given the above, my research aims to answer the following question: \textbf{to what extent are the natural urban swimming spaces in Copenhagen (un)fairly accessible, and who benefits from them?}
%and what does that tell us about environmental justice in the city?

This will be approached with the following sub-questions:

\begin{enumerate}
	\item Who are the users of the space, how do they perceive it and how do they feel in it?, which will help understand
	\item What are the factors that make the space feel accessible to them?, which will help understand
	\item How accessible is the space to a diversity of people (based on socioeconomic factors like income, education, age, immigration background, etc. \parencite{baro2021school})?, which will help understand
	\item How fairly accessible are the blue spaces, and why?
\end{enumerate}

%%%%%%%% GUIDING FRAMEWORKS %%%%%%%%

My research will be explanatory, because I aim to explain why the phenomenon of unequal access to blue space happens (or not), 

I will take an inductive approach. whereby my theory will emerge from the data.

TODO: explain about perception/values of space; about perception/recognition justice; ie. general frameworks for guiding the research
 
\parencite{kronenberg2020environmental}. 


%%%%%%%% METHODS %%%%%%%%

To answer my research question, I will need to collect data on people's experiences and perceptions of a blue space, which is subjective information, and link it to spatial data. To this end, I will use public participatory GIS (PPGIS) which allows for exactly this: linking qualitative data to objective GIS data. This method has successfully been used by BlueHealth to understand people's relationships to blue spaces (see \url{https://bluehealth2020.eu/projects/softgis/}) 

%%%%%%%% CASE STUDY %%%%%%%%

The city I will focus my research on is Copenhagen.
The reasons Copenhagen makes for an interesting case with regards to my research are as follows. 
Firstly, because of Copenhagen's location on the Kattegat strait, water features prominently in the city and is an important part of the urban landscape ("residents are never farther than x kilometres from the shoreline"). 
Second, due to the amount of shoreline in Copenhagen, there have been many blue space rehabilitation projects in the last decades. The XX, XX, and XX harbours have all been made accessible as swimming spaces, and the XX is currently undergoing a transformation.
Thirdly, the socioeconomic landscape of the city makes it worthwhile to evaluate distributional justice. Since the early 2010s, the rate of immigration and the number of working-class households have increased, at the same time as racist and xenophobic discourses in the media and politics. 
Finally, Copenhagen's reputation as ``the most liveable city'' \parencite{visitdenmark_2021}, due in part to the swimming spots in the harbour, begs the question - for whom is the city liveable?

Then, Copenhagen, as a city in northern Europe, prides itself in its low-tolerance for inequalities - but really?


\printbibliography

\end{document}


%%%%%%%%%%%%%%%%%%%%%%%%%%%%%%%%%%%%
%				COMMENTS
%%%%%%%%%%%%%%%%%%%%%%%%%%%%%%%%%%%%
\begin{comment}
 
 Explanatory -> why do this mechanism/phenomenon happen? 
 
 
These concerns align with the concept of urban environmental justice, which advocates for equitably sharing the benefits of natural spaces  in the city.
 
\end{comment}