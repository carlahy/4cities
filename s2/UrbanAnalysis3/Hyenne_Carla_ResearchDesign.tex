\documentclass{article}

\linespread{1.5}
\usepackage[utf8]{inputenc}
\usepackage[left=1.5in,right=1.5in,bottom=1in]{geometry}
\setlength\parindent{0pt}
\setlength{\parskip}{1em}
\setcounter{secnumdepth}{0}
\usepackage{outlines}
\usepackage{graphicx}
\graphicspath{ {imgs} }
\usepackage[hyphens]{url}
\usepackage{hyperref}
\usepackage{color,soul}
\usepackage[normalem]{ulem}

\usepackage[
backend=biber,
style=apa,
citestyle=authoryear,
sorting=nyt,
]{biblatex}
\addbibresource{refs.bib}

\usepackage{comment}
\specialcomment{topicsen}{\begingroup\bfseries\scriptsize}{\endgroup}
%\excludecomment{topicsen}

\newcommand{\alignedmarginpar}[1]{%
        \marginpar{\raggedright\small #1}
    }

\title{Research Design}
\author{Carla Hyenne}
\date{}

\begin{document}

\maketitle

%%%%%%%% INTRO %%%%%%%%

Much like green spaces, blue spaces in the city have a positive impact on people's perceived mental and physical wellbeing - being near the water makes people feel better and happier. Natural spaces like rivers, lakes, harbours or canals are often free to access but aren't always adapted as places of leisure. To be attractive to people, waterfronts need to be clean, safe, welcoming, accessible, and offer a diversity of uses for a diversity of users. 

In the last centuries, water bodies in cities were polluted by industries, and became unsafe to swim. Through public pressure or environmental concerns, cities have invested into cleaning the water and restoring it into places where people would want to spend time, lounging, swimming or other. This is not a simple feat. 

These reconversion/rehabilitation projects require investment and planning from the city. The water itself has to be cleaned, which involves identifying and dealing with polluting sources; the water bed and the surrounding environment have to be rehabilitated to safely welcome people; and the waterfront has to be fitted with infrastructure for swimming, sitting, walking, and other activities.
This requires a lot of money and planning from the city.

Additionally, blue space rehabilitation projects are not profit-oriented. Usually, natural bodies of water in European cities are public spaces that are freely accessible - you do not need to pay to use them. Without a monetary incentive, the city's discourse with regards to rehabilitation blue spaces is purely for the benefit of the current residents, or attracting new residents.

In reality, despite a waterfront being designed to welcome a diversity of people, it may not benefit everyone fairly. The needs of some may be more represented than others, and more vulnerable communities might not be able to access the space as liberally or enjoy the same benefits. Differences in experience can be based on a multitude of socioeconomic factors such as gender, age, physical or mental abilities, income, ethnicity, education, and also on practical matters such as the distance between home and water (which can itself be based on socioeconomic differences ). 

The reasons why understanding inequalities in accessing rehabilitated blue space is important, are two-fold. First, those who do not interact with the space (either directly or indirectly) are not able to experience its positive effects on wellbeing, which increases inequality. Vulnerable communities become even more disadvantaged compared to those with more privilege, who gain valuable natural public space.

Second, when we ignore the fact that rehabilitated blue space raises the attractiveness of the neighbourhood or the city, we ignore the problems that the working and middle classes will face, such as displacement caused by a rise in rental and housing prices. 

These concerns align with the concept of urban environmental justice, which advocates for equitable access to the benefits that natural spaces offer in the city.

Thus, my research will be inscribed in the framework of environmental justice, and more specifically in <distributional justice?>
Therefore, it is particularly interesting to study the rehabilitation of natural bodies of water in a city 

In reality, not everyone enjoys the benefits of a public waterfront. <explain why>
In particular, the most vulnerable communities are disproportionally excluded from the spaces <explain why>

This is important when it comes to reducing inequalities in the city.


%%%%%%%% FRAMEWORKS %%%%%%%%



%%%%%%%% METHODS %%%%%%%%


\printbibliography

\end{document}
