\documentclass{article}

\linespread{1.5}
\usepackage[utf8]{inputenc}
\usepackage[left=1.5in,right=1.5in,bottom=1in]{geometry}
\setlength\parindent{0pt}
\setlength{\parskip}{1em}
\setcounter{secnumdepth}{0}
\usepackage{outlines}
\usepackage{graphicx}
\graphicspath{ {imgs} }
\usepackage[hyphens]{url}
\usepackage{hyperref}
\usepackage{color,soul}
\usepackage[normalem]{ulem}

\usepackage[
backend=biber,
style=apa,
citestyle=authoryear,
sorting=nyt,
]{biblatex}
\addbibresource{refs.bib}

\usepackage{comment}
\specialcomment{topicsen}{\begingroup\bfseries\scriptsize}{\endgroup}
%\excludecomment{topicsen}

\newcommand{\alignedmarginpar}[1]{%
        \marginpar{\raggedright\small #1}
    }

\title{Research Design}
\author{Carla Hyenne}
\date{}

\begin{document}

\maketitle

%%%%%%%% INTRO & THEORETICAL FRAMEWORKS %%%%%%%%

%%% blue spaces increase people's perceived sense of wellbeing
Urban blue spaces have a positive impact on people's perceived mental and physical wellbeing - being near water makes people feel better and happier, and be more active \parencite{gascon2017outdoor}. 
%%% people living in deprived areas benefit relatively more from blue spaces compared to more privileged neighbourhoods
In particular, blue spaces have the potential to have an even greater influence on the wellbeing of those living in socio-economically disadvantaged neighbourhoods, due to low socio-economic status being linked to poorer mental health and lower life satisfaction \parencite{van2021urban}.

%%% blue spaces exist in cities but they aren't always attractive
Most European cities have at least one natural blue space like a river, lake, harbour or canal. Due to urbanisation, water bodies have been polluted or transformed by industries and became unsafe to swim \parencite{kampa_langaas_anzaldua_2016}. Through public pressure or for environmental concerns, cities have invested resources into cleaning and restoring water and waterfronts into attractive natural places for people to enjoy.
%%% efforts are put into turning them into attractive places- quality, public blue spaces where you can swim require investments from the city
Making a body of water swimmable is a large public investment: the water itself has to be cleaned, which involves identifying and dealing with polluting sources; the waterbed and surrounding environments have to be rehabilitated to safely welcome people; and the waterfront has to be equipped with infrastructure for swimming, sitting, walking, and more.
Since public space is a highly valued commodity in a city, reconverting blue spaces is a great way to take advantage of unused areas.

% in reality, blue spaces are not equally accessible to everyone in a city- physical accessibility, psychological accessibility, and ???
Moreover, public spaces do not have an entrance fee, which removes an economic barrier to access. However, this doesn't imply that they are fairly accessible to everyone.
The first reason why some people may not be able to use a blue space is physical accessibility. Those who are most likely to frequent a blue space are those living or working in the vicinity. The availability of public transport, bicycle infrastructure, the quality of the roads and footpaths around and in the space, also influence who visits.
The second reason is psychological accessibility. Blue spaces (and public spaces in general) can be more or less welcoming to certain groups of people. Gender, age, ethnicity, income, or preferences in aesthetics or activities, amongst other factors, influence who feels welcomed or attracted to a space. Discriminatory practices exclude people and groups; for example, being made to feel like spending money is necessary to fit in excludes people based on their socio-economic class \parencite{wessells2014urban}.
% Third, ??? TODO: framework for these statements

Therefore, even if a blue space is free and open to the public, this does not mean that people use and benefit from them fairly. Understanding this phenomenon is important because public spaces are places of community, identity and attachment, and everyone in the city should have equal rights to them \parencite{agyeman2016trends}. Every blue space, neighbourhood and city will have a different set of conditions (social, political, economic, cultural, environmental, etc.) which explain who uses the space, how they feel in it, and why they use it. In order to uncover these conditions, the unit of analysis should be scoped to a specific location on the water where people linger and have the opportunity to swim; this spot should have been rehabilitated by the city, and be public; and, it would be particularly interesting to study two rehabilitated blue spaces within one city, in order to compare the social consequences of investments in different neighbourhoods.

Given these considerations, Copenhagen makes for an interesting case for the following four reasons.
First, Copenhagen is located on the Kattegat strait and has 92 km of coastline \parencite{comertler2017greens}, therefore water features prominently in the urban landscape.
Second, due to the amount of shoreline in Copenhagen and the city trying to position itself as a world leader in sustainability, there have been many blue space rehabilitation projects since 2002. Today there are four harbour baths (Island Brygge, Fisketorvet, Sandkaj and Sluseholmen) and various urban beaches (Amager Strandpark, Svanemølle) \parencite{visitcopenhagen_baths}.
Third, Copenhagen today is experiencing an increase in poverty and ethnic segregation \parencite{moller2015socioeconomic}, as well as a growing racist discourse in the media and politics. For example, through the classification of some neighbourhoods as `ghettos' \parencite{simonsen2008practice}. This evolving socio-economic landscape and its surrounding discourse make it important to understand who feels included in blue spaces, and might not. 
Finally, Copenhagen's reputation as ``the most liveable city'' \parencite{visitdenmark_2021}, due in part to the swimming spots in the harbour, begs the question - for whom is the city liveable?\footnote{Note: I still need to decide which which specific harbour baths in Copenhagen I will study}

%%%%%%%% THEORETICAL FRAMEWORKS %%%%%%%%
Ultimately, unequal access to blue space is an environmental injustice and this framework will lead my research.
% Intro to environmental justice
Environmental justice is a multi-faceted concept which brings together social and environmental concerns. Amongst other things, it advocates for the equitable access to environmental benefits. % like the increased wellbeing that exposure to blue space provides.
% Three dimensions of justice
%Justice in general can be broken down into three categories: distributional justice, procedural justice, and recognition justice. 
%With regards to blue space, distributional justice focuses on social consequences based on the location of blue space interventions in the city, such as environmental gentrification.
%Procedural justice deals with questions of discrimination in public participation and decision making situations.
The dimension of justice that I will focus on is \textit{recognition}. Recognition justice deals with people and groups' perceptions, values and preferences which may influence the ways in which they interact, or not, with a blue space \parencite{anguelovski2020expanding}.

Given the above, my research aims to answer the following question: \textbf{to what extent are the natural urban swimming spaces in Copenhagen (un)fairly accessible, and who benefits from them?} %and what does that tell us about environmental justice in the city?

This will be approached with the following sub-questions:

\begin{enumerate}
	\item Who are the users of the space, how do they perceive it and how do they feel in it?
	\item What are the factors that make the space feel accessible to them?
	\item How accessible is the space to a diversity of people? % (based on socioeconomic factors like income, education, age, immigration background, etc. \parencite{baro2021school})?
	\item How fairly accessible are the blue spaces, and why?
\end{enumerate}

%%%%%%%% RESEARCH FRAMEWORKS, METHODS %%%%%%%%
My research will be explanatory, because I aim to explain why the phenomenon of unequal access to blue space takes place (or not) based on principles of recognition justice.
I will take an inductive approach, whereby my theory will emerge from data I will collect on people's experiences, perceptions and preferences of a blue space. This data is both subjective and spatial. My approach to collecting the data will be to visit the blue space(s) and interview users, for which I will need a set of interview questions as well as a way to record spatial data like their home, or the blue/green spaces they frequent.
These requirements lend themselves well to public participatory GIS (PPGIS), a map-based survey method linking qualitative and GIS data.
Using PPGIS to study people's relationship to green and blue space is recommended as research method which ``might uncover local spatial knowledge and perceptions'' \parencite{anguelovski2020expanding}. It has been used to study people's interactions with and the distribution of blue spaces in Helsinki metropolitan area \parencite{raymond2016integrating}, and also by BlueHealth to ``uncover spatial aspects of people’s relationships with blue spaces'' \parencite{bluehealthsoftgis}.

\printbibliography

\end{document}


%%%%%%%%%%%%%%%%%%%%%%%%%%%%%%%%%%%%
%				COMMENTS
%%%%%%%%%%%%%%%%%%%%%%%%%%%%%%%%%%%%
\begin{comment}
 
Explanatory -> why do this mechanism/phenomenon happen? 

When we ignore the fact that reconverted blue space raises the attractiveness of the neighbourhood or the city, we ignore the problems that the working and middle classes will face, such as displacement caused by a rise in rental and housing prices. This deepens inequalities by constraining less-privileged households to move to neighbourhoods deprived of natural spaces, and where they cannot benefit from the wellbeing effects.

https://web.archive.org/web/20210509064844/https://stateofgreen.com/en/partners/state-of-green/news/from-industrial-harbour-to-urban-harbour-bath/

\end{comment}