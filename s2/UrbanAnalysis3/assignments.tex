\documentclass{article}

\linespread{1.15}
\usepackage[utf8]{inputenc}
\usepackage[left=1.5in,right=1.5in,bottom=1in]{geometry}
\setlength\parindent{0pt}
\setlength{\parskip}{1em}
\setcounter{secnumdepth}{0}
\usepackage{outlines}
\usepackage{graphicx}
\graphicspath{ {imgs} }
\usepackage{hyperref}
\usepackage{color,soul}
\usepackage[normalem]{ulem}

\usepackage[
backend=biber,
style=apa,
citestyle=authoryear,
sorting=nyt,
]{biblatex}
\addbibresource{refs.bib}

\usepackage{comment}
\specialcomment{topicsen}{\begingroup\bfseries\scriptsize}{\endgroup}
%\excludecomment{topicsen}

\newcommand{\alignedmarginpar}[1]{%
        \marginpar{\raggedright\small #1}
    }

\title{UA3 Assignments}
\author{Carla Hyenne}

\begin{document}

\maketitle

\tableofcontents

\pagebreak

\section{Statement of Purpose Draft}

Draft 1:

\textbf{I am studying} the (re) conversion of inland, (natural) urban blue spaces into swimmable environments\\
\textbf{because I want to find out} what impact blue spaces can have on a city's social and environmental sustainability\\
\textbf{in order to help my reader understand} the potential that water can have in their city.

Draft 2:

\textbf{I am studying} the (re) conversion of inland, (natural) urban blue spaces into swimmable environments\\
\textbf{because I want to find out} how usable blue spaces can improve environmental justice in the city \\
\textbf{in order to help my urban planners understand} the importance of incorporating blue spaces in urban planning and development.\\

\pagebreak

\section{Literature review}

\subsubsection{Garrett et al., \textit{Urban blue spaces and health and wellbeing in Hong Kong: Results from a survey of older adults}, 2019}

This paper analyses the perceived health benefits that blue spaces can have on the population (it focuses on elderly residents). It defines three categories of blue space exposure: indirect (view from home), incidental (on commute route), and intentional (purposeful visit), and addresses different factors that could influence people's perception and use of the blue space, like safety, cleanliness, facilities, accessibility. They find a positive relation between blue space and overall wellbeing, including good health and good mental health.

It recognises the particularities of Hong Kong: it is surrounded by water, has high quality and safe public space, a fantastic public transit system (which makes it easy for people to reach blue spaces if they don't live nearby). I think it is necessary to acknowledge these particularities in a city, because they can influence how people interact with blue (or green) spaces.
	
Why is it relevant to me? It provides a categorisation of blue space usage, and a framework for analysing the benefits and quality of blue spaces. It focuses on the effect that blue spaces have on the physical wellbeing of people, and this falls under social sustainability. 


\subsubsection{Gascon et al., \textit{Outdoor blue spaces, human health and well-being: A systematic review of quantitative studies}, 2017}

This paper reviews quantitative studies on blue space and associated health benefits. It reports the type of blue space, the environment of the blue space, and how the health outcomes were evaluated.

What I found useful was the differentiation between inland and non-inland spaces, which I will use in my research. I also realised I don't want to focus specifically on the relation of blue space to human health (eg. general health, mental health and wellbeing, physical activity, other morbidities) because there is already good research in this area, especially in Europe. I still want to focus on the social aspect of blue spaces, but perhaps more on blue spaces as open, public and free spaces. Also, quantitative research will be hard because I don't think I would be able to reach enough participants (I assume in the scale of hundreds?) in both case studies.

\subsubsection{Raymond et al., \textit{Integrating multiple elements of environmental justice into urban blue space planning using public participation geographic information systems}}

This paper focuses on environmental justice which I think is a great term to help frame my research interest, because it incorporates principles like sustainability, inclusivity, and right to open blue spaces. The paper looks at the diversity of users, diversity of activities, and the PPUE (perceived problems and unpleasant experience) of the blue spaces, to demonstrate the multidimensionality of environmental justice. It also made me wonder whether I should focus on a certain socio-demographic group (by age, income levels, gender, ethnicity, etc.), and  how that group is affected by the presence, uses, and/or design of blue spaces, or lack thereof. 

\end{document}
