\documentclass{article}

\linespread{1.5}
\usepackage[utf8]{inputenc}
\usepackage[left=1.5in,right=1.5in,bottom=1in]{geometry}
\setlength\parindent{0pt}
\setlength{\parskip}{1em}
\setcounter{secnumdepth}{0}
\usepackage{outlines}
\usepackage{graphicx}
\graphicspath{ {imgs} }
\usepackage{hyperref}
\usepackage{color,soul}

\usepackage[
backend=biber,
style=apa,
citestyle=authoryear,
sorting=nyt,
]{biblatex}
\addbibresource{exam.bib}

\usepackage{comment}
\specialcomment{topicsen}{\begingroup\bfseries\scriptsize}{\endgroup}
%\excludecomment{topicsen}

\newcommand{\alignedmarginpar}[1]{%
        \marginpar{\raggedright\small #1}
    }

\title{Socio-Spatial Urban Diversity}
\author{Carla Hyenne}

\begin{document}

\maketitle

\tableofcontents

\pagebreak

\section{Introduction}

\subsection{}

\section{}

\section{}

\section{}

\section{Readings}

\subsection{Introduction}

\subsubsection{R. Cavicchia, R, Cucca, \textit{Densification and School Segregation: The Case of Oslo}, 2020}

\begin{outline}
	\1 \textit{tldr;} urban densification is promoted as a desirable thing in cities, making them more sustainable, better social mix, better conditions to live together. However, they can create undesirable residential patterns like segregation and gentrification, and empirical evidence shows that it benefits the richer population. Paper studies residential and specifically school segregation in Oslo, focusing on neoliberal planning approaches
	\1 Why Oslo: growing economically and demographically, has promoted urban densification since the 1980s to prevent urban sprawl, it has an egalitarian social welfare state reflected in the education policies, and school segregation reflects residential segregation because kids are assigned to schools by proximity (`catchment area') > where children live determines where they go to school, thus residential patterns are crucial for understand school segregation in general
	\1 RQ: ``how are the densification developments of the past two decades associated with changes in the distribution of children with different backgrounds in Oslo?''
	\1 Densification policies have to be complemented with policies that counter segregation/gentrification/social inequality that follows
	\1 Oslo
		\2 Strong socio-spatial segregation in Oslo and is considered a `dual-city'. Recent increase in immigration show that immigrants settle in the Eastern, lower income neighbourhoods
		\2 Housing policies shifted since the 1980s, from social to neoliberal, privatised housing markets
		\2 Oslo densification policies are two-fold: from inner to outer city, and along transport lines
\end{outline}

\subsubsection{Dahinden, Fischer, Menet (2021) \textit{Knowledge production, reflexivity, and the use of categories in migration studies: tackling challenges in the field}}

\begin{outline}
	\1 A turn where terms like society, culture, migration, are revised, and `knowledge production' in migration studies is being given more attention. There is a risk that knowledge production may perpetuate ``particular hegemonic power relations and concomitant forms of social and political exclusion'', ie. reproduce power structures that should no longer exist, and in the worst case, neo-colonial reasoning
	\1 To analyse knowledge production, it uses: the reflexive turn in anthropology and postcolonial scholarship
\end{outline}

\subsubsection{Steven Vertovec, \textit{Talking around super-diversity}, 2019}

\begin{outline}
	\1 How different genders experience migration (transnational, Eastern European) differently, with regards to entrepreneurial activities (eg. opening a business). Through their work, women gain prestige in the eyes of their family and friends back home
	\1 Migrant women: typically depicted as invisible, ``uneducated, illiterate and passive'' (p. 591) but this article explores their agency
	\1 RQ: ``how do transnational migrant entrepreneurs utilise symbolic capital within their entrepreneurial activities in the UK? What role does gender play in this process?''
	\1 Symbolic capital: resources (ie. capital) that is made available to individuals as a result of honour, prestige, recognition
		\2 There is a Soviet concept still present in post-socialist societies (including in Eastern European migrants in the UK) where one uses their personal connections to get ahead, and another concept where people identify others as `one of us' and form communities
\end{outline}


\end{document}

\begin{comment}

\subsubsection{\textit{}}

\begin{outline}

\end{outline}

\end{comment}

