\documentclass{article}

\linespread{1.2}
\usepackage[utf8]{inputenc}
\usepackage[left=1.5in,right=1.5in,bottom=1in]{geometry}
\setlength\parindent{0pt}
\setlength{\parskip}{1em}
\setcounter{secnumdepth}{0}
\usepackage{outlines}
\usepackage{graphicx}
\graphicspath{ {imgs} }
\usepackage{hyperref}
\usepackage{color,soul}
\usepackage[normalem]{ulem}

\usepackage[
backend=biber,
style=apa,
citestyle=authoryear,
sorting=nyt,
]{biblatex}
\addbibresource{exam.bib}

\usepackage{comment}
\specialcomment{topicsen}{\begingroup\bfseries\scriptsize}{\endgroup}
%\excludecomment{topicsen}

\newcommand{\alignedmarginpar}[1]{%
        \marginpar{\raggedright\small #1}
    }

\title{Socio-Spatial Urban Diversity}
\author{Carla Hyenne}

\begin{document}

\maketitle

\tableofcontents

\pagebreak

%%%%%%%%%%%%%%%%%%%%%%%%%%%%%%%%%%%%%%%%%%%%%%%%%%%%%%%%%%%%%%%%
%					LECTURE 1
%%%%%%%%%%%%%%%%%%%%%%%%%%%%%%%%%%%%%%%%%%%%%%%%%%%%%%%%%%%%%%%%
\section{Introduction}
\textit{Fieldwork at AAKH}

\subsection{Where do you see diversity?}

Where do you see diversity? Where do you \textbf{not} see diversity? When do socio-spatial effects emerge from ``observed diversity''? Urban diversity in Vienna in comparison to what?

\textit{A photo of a park in Vienna}

There is diversity in the \textbf{built environment} through buildings, public space; in \textbf{functions}, like residential, school, playground, transport; in \textbf{architecture}, through differences in height, density, older vs. modern architecture, balconies and winter gardens, the orientation of the roof (sloped or flat, new builds have flat roofs because they are more affordable to build, increase the potential rent of the top floor, and snow isn't a problem in Vienna anymore); in \textbf{living standards}, social and public vs. private housing; in \textbf{green infrastructure}, trees and grass.

What we do not see are a diversity of people, but could be due to the time of day and year; any \textbf{informal or illegal practices}, meaning anything against pre-defined functional uses, like sleeping rough, graffiti, dancing or performing which could feel unusual; \textbf{ethnic diversity}, there are no restaurants or shops representing minority populations and ethnic communities.

We are asking, \textbf{what is there, or isn't there, and why?}

What processes are causing the diversity, or lack thereof? \textbf{Exclusionary processes}, driven by neoliberal policies and the financialisation of the housing market; 
\textbf{lock-in effects} of the housing market, whereby older social housing buildings and apartments are much more affordable than newly built ones (est. up to twice as less), so even if by European standards rents are still affordable, moving into a newer and smaller (public) apartment can be very expensive. To get social housing, you must have lived in Vienna for some time, thus students and refugees are not eligible;
\textbf{migration} is a global driver of socio-demographic change (both domestic and international), caused by young people moving to the city, in Vienna, a low fertility rate means that migration is necessary to maintain/grow the population. Migration in Vienna is first and foremost domestic, joining the EU in 1995 opened migration to Germany, Eastern Europe, the Balkans, ex-Yugoslavian countries like Serbia Bosnia-Herzegovina, and circa 2015 Syria, Afghanistan, Iraq; 
\textbf{segregation}, which exists in a socially-mixed city like Vienna, even if the `borders' are not so drastic.

\subsection{Super-Diversity}

\textit{```Super-diversity' is proposed as a summary term. Whatever we choose to call it, there is much to be gained by a multidimensional perspective on diversity, both in terms of moving beyond `the ethnic group as either the unit of analysis or sole object of study' (Schiller et al., 2006) and by appreciating the coalescence of factors which condition people's lives''} (Steven Vertovec, 2007)

Ethnic communities are not homogeneous, and ethnicity as a category to speak about diversity is not enough. We require a multidimensional perspective.

\textit{``The basic argument advanced for coining the term and developing the concept is that it describes changing patterns of global migration flows of the post-World War II decades that have entailed the movement of people from more varied national, ethnic linguistic, and religious backgrounds, who occupy more varied legal statuses, and who bring a wide range of human capital (education, work skills, and experience)''} (Foner et al., 2019)

\subsection{Change, everyday life, socio-spatial disparities}

\textbf{Change} what dimensions does ``demographic change'' include in the context of super-diversity?

\textbf{Everyday life} how are practices of living together formed in urban neighbourhoods?

\textbf{Socio-spatial diversity} how do consequences and outcomes manifest in everyday?

Brainstorming \textbf{research questions} based on the reading - use why, what (analytical), how (descriptive)...

\begin{outline}
	\1 Can socio-spatial disparities ever have a positive impact on minority groups? $\rightarrow$ socio-spatial disparities
	\1 When comparing refugees from the Ukrainian war in 2022 and refugees from the Middle-East in 2015 (themselves a highly diverse, non homogenous group), how do public policies and public discourse influence these people's experience and integration into Vienna?
		\2 Change:
		\2 Everyday life:
		\2 Socio-spatial disparities:
\end{outline}


%%%%%%%%%%%%%%%%%%%%%%%%%%%%%%%%%%%%%%%%%%%%%%%%%%%%%%%%%%%%%%%%
%					LECTURE 2
%%%%%%%%%%%%%%%%%%%%%%%%%%%%%%%%%%%%%%%%%%%%%%%%%%%%%%%%%%%%%%%%
\section{Housing Market}

\textit{Guest lecturer Celine Janssen}


%%%%%%%%%%%%%%%%%%%%%%%%%%%%%%%%%%%%%%%%%%%%%%%%%%%%%%%%%%%%%%%%
%					LECTURE 3
%%%%%%%%%%%%%%%%%%%%%%%%%%%%%%%%%%%%%%%%%%%%%%%%%%%%%%%%%%%%%%%%
\section{Gentrification}

\textit{Fieldwork at XXX}

%%%%%%%%%%%%%%%%%%%%%%%%%%%%%%%%%%%%%%%%%%%%%%%%%%%%%%%%%%%%%%%%
%					LECTURE 4
%%%%%%%%%%%%%%%%%%%%%%%%%%%%%%%%%%%%%%%%%%%%%%%%%%%%%%%%%%%%%%%%
\section{Arrival Space \& the Role of Public Space}

\textit{Fieldwork at XXX}


%%%%%%%%%%%%%%%%%%%%%%%%%%%%%%%%%%%%%%%%%%%%%%%%%%%%%%%%%%%%%%%%
%					LECTURE 5
%%%%%%%%%%%%%%%%%%%%%%%%%%%%%%%%%%%%%%%%%%%%%%%%%%%%%%%%%%%%%%%%
\section{Social Innovation}

\textit{Guest lecturer XXX}


%%%%%%%%%%%%%%%%%%%%%%%%%%%%%%%%%%%%%%%%%%%%%%%%%%%%%%%%%%%%%%%%
%					READINGS
%%%%%%%%%%%%%%%%%%%%%%%%%%%%%%%%%%%%%%%%%%%%%%%%%%%%%%%%%%%%%%%%
\section{Readings}

\subsection{Introduction}

\subsubsection{R. Cavicchia, R, Cucca, \textit{Densification and School Segregation: The Case of Oslo}, 2020}

\begin{outline}
	\1 \textit{tldr;} urban densification is promoted as a desirable thing in cities, making them more sustainable, better social mix, better conditions to live together. However, they can create undesirable residential patterns like segregation and gentrification, and empirical evidence shows that it benefits the richer population. Paper studies residential and specifically school segregation in Oslo, focusing on neoliberal planning approaches
	\1 Why Oslo: growing economically and demographically, has promoted urban densification since the 1980s to prevent urban sprawl, it has an egalitarian social welfare state reflected in the education policies, and school segregation reflects residential segregation because kids are assigned to schools by proximity (`catchment area') > where children live determines where they go to school, thus residential patterns are crucial for understand school segregation in general
	\1 RQ: ``how are the densification developments of the past two decades associated with changes in the distribution of children with different backgrounds in Oslo?''
	\1 Densification policies have to be complemented with policies that counter segregation/gentrification/social inequality that follows
	\1 Oslo
		\2 Strong socio-spatial segregation in Oslo and is considered a `dual-city'. Recent increase in immigration show that immigrants settle in the Eastern, lower income neighbourhoods
		\2 Housing policies shifted since the 1980s, from social to neoliberal, privatised housing markets
		\2 Oslo densification policies are two-fold: from inner to outer city, and along transport lines
	\1 Why does densification lead to segregation?
		\2 
\end{outline}

\subsubsection{Dahinden, Fischer, Menet \textit{Knowledge production, reflexivity, and the use of categories in migration studies: tackling challenges in the field}, 2021}

\begin{outline}
	\1 How are categories, and categorisation, of migrants formulated? And, how can they help reproduce or perpetuate racist and marginalisation discourses? This paper argues we should take a reflexive approach to categories in migration studies
	\1 A turn where terms like society, culture, migration, are revised, and `knowledge production' in migration studies is being given more attention. There is a risk that knowledge production may perpetuate ``particular hegemonic power relations and concomitant forms of social and political exclusion'', ie. reproduce power structures that should no longer exist, and in the worst case, neo-colonial reasoning
	\1 To analyse knowledge production, it uses: the reflexive turn in anthropology and postcolonial scholarship
	\1 Reflexivity is defined here as ``a process of ``decentring'' by distancing one's research from well-established ideas while developing alternative ones'' (p. 536)
		\2 
\end{outline}

\subsubsection{Steven Vertovec, \textit{Talking around super-diversity}, 2019}

\begin{outline}
	\1 \textit{tldr; analyses how different articles or researchers have used and interpreted the term super-diversity, and why the term is widely used}
	\1 The term focuses on migration, and on the ``new social patterns, forms and identities arising from migration-driven diversification'' (abstract). Vertovec believes it has gained popularity because social scientists are looking for ``ways of describing and talking about increasing and intensifying complexities in social dynamics and configurations at the neighbourhood, city, national and global levels'' (p. 135), sometimes through migration patterns, but also through other ``complex social developments''
		\2 It started from the observation in UK migration statistics that migrants came from more countries of origin, thus from a greater diversity of languages, ethnicities, religions... which have created shifts in things like legal statuses, labour market realities, gender and age experiences, residential patterns and segregation, etc. $\rightarrow$ new patterns\alignedmarginpar{change, socio-spatial disparities, everyday life}
		\2 New patterns of inequality, racism, segregation, creolisation (selective blending of cultures)
	\1 Super-diversity has been used in policy circles and by NGO
	\1 A typology of uses of super-diversity 
		\2 as very much diversity
		\2 as a backdrop to a study ie. as a new condition or context, perhaps painting super-diversity as a happy new normal and proving the opposite, and that class and racism still matter
		\2 as more ethnicity, more ethnic groups in a place
		\2 as multidimensional reconfiguration, whereby multiple dimensions need to be taken into consideration when measuring diversity
		\2 as moving beyond ethnicity which is not the optimal unit for analysis, and needing to consider many more aspects in and outside of the ethnic group
		\2 as new or other complexities with regards to migrants, for eg. their place of origin, reasons for migration, careers, sociocultural and linguistic features that can't be assumed; not assuming a direct relation between ethnicity, citizenship, residence, origin, language, profession (p. 132); new social formations, hierarchies, power relations within the migrant group, diverse networks;
	\1 Is super-diversity similar to intersectionality? No, because intersectionality doesn't deal with migration patterns and outcomes, and mostly focuses on race, gender and class
	\1 One limitation in studies and research on ethnicity, are the statistics, data, census collected by States, and the categories they presuppose for people and groups. The studies begin with a set of given categories, rather than looking for better fitting categories
	\1 Change:
	\1 Everyday life:
	\1 Socio-spatial disparities:
\end{outline}

Additional reading: Vershinina, Rodgers, \textit{Symbolic capital within the lived experiences of Eastern European migrants: a gendered perspective}, 2019

\begin{outline}
	\1 How different genders experience migration (transnational, Eastern European) differently, with regards to entrepreneurial activities (eg. opening a business). Through their work, women gain prestige in the eyes of their family and friends back home
	\1 Migrant women: typically depicted as invisible, ``uneducated, illiterate and passive'' (p. 591) but this article explores their agency
	\1 RQ: ``how do transnational migrant entrepreneurs utilise symbolic capital within their entrepreneurial activities in the UK? What role does gender play in this process?''
	\1 Symbolic capital: resources (ie. capital) that is made available to individuals as a result of honour, prestige, recognition
		\2 There is a Soviet concept still present in post-socialist societies (including in Eastern European migrants in the UK) where one uses their personal connections to get ahead, and another concept where people identify others as `one of us' and form communities
\end{outline}

\subsubsection{\textit{}}

\begin{outline}

\end{outline}



\end{document}

\begin{comment}

\subsubsection{\textit{}}

\begin{outline}

\end{outline}

\end{comment}

