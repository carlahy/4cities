\documentclass{article}

\linespread{1.5}
\usepackage[utf8]{inputenc}
\usepackage[left=0.5in,right=1.25in,bottom=0.75in,top=0.75in]{geometry}
\setlength\parindent{0pt}
\setlength{\parskip}{1em}
\setcounter{secnumdepth}{0}
\usepackage{outlines}
\usepackage{graphicx}
\graphicspath{ {imgs} }
\usepackage[hyphens]{url}
\usepackage{hyperref}
\usepackage{color,soul}
\usepackage[normalem]{ulem}
\usepackage{chronology}
\usepackage{tabularx}
\usepackage{setspace}

\newcommand{\foo}{\hspace{-2.3pt}$\bullet$ \hspace{5pt} }


\usepackage[
backend=biber,
style=apa,
citestyle=authoryear,
sorting=nyt,
]{biblatex}
\addbibresource{refs.bib}

\usepackage{comment}
\specialcomment{topicsen}{\begingroup\bfseries\scriptsize}{\endgroup}
%\excludecomment{topicsen}

\newcommand{\bisection}[1]{\textbf{\textit{#1}}}

\newcommand{\alignedmarginpar}[1]{%
        \marginpar{\raggedright\small #1}
    }

\title{Principles of Urbanism and Planning - Revisions}
\author{Carla Hyenne}

\begin{document}

\maketitle

\tableofcontents

\pagebreak

%%%%%%%%%%%%%%%%%%%%%%%%%%%%%%%%%%%%%%%%%%%%
\pagebreak
\section{Planning the Ancient City}
%%%%%%%%%%%%%%%%%%%%%%%%%%%%%%%%%%%%%%%%%%%%

\bisection{What is a city?}\alignedmarginpar{Childe 10 traits} population size, division of labour, agriculture surplus, monumnental buildings, existence of ruling class, existence of writing and numbers, predictive sciences, existence of artisan class, long distance trade, `organic solidarity' based on interdependency

\bisection{Mesopotamia}\alignedmarginpar{Fertile Crescent} some of the earliest urban settlements from 3500BCE. Oldest cities (Ur) have an oval form and are located near water for boats before wheels were invented. Newer cities (Borsippa) have a rectangular form for road traffic, are larger, with innovations and social segregation Cities have an orientation (due to winds), walls, a palace on the river (zikkurat), court houses (became larger and more regular over time)

\bisection{Greek urban planning}\alignedmarginpar{Athens, Olympia} Archaioteros tropos until 480BC, a growing together of earlier settlements, creating irregular layouts but paying attention to views, on a hill. Neoteros tropos until 335BC is a grid-pattern layout, with main streets North-South, strict zoning, plots completely built over, court houses, obsession with correct proportions, move to valleys when secure fortifications developed

\bisection{Athens' culture} first large Polis but golden age is short-lived. Cosmopolitan cultural was crucial (meltingpot of connections with trade and colonies), introduced democracy, individualism but also collective civic goals, written history, philosophy, rationalism and utilitarianism,  held debates about how city should be organised

\bisection{Athens' urban structure} public buildings (Akropolis, Parthenon) were important, some water pipelines, sewers, public baths. Life was very public, homes tiny and primitive, with no gardens but many with court-yards. Trading happened around the Agora, used money, and primitve ways of financing (no budgets)

\bisection{The rise and fall of Ancient Greece}\alignedmarginpar{P. Hall} a unique ethnic and cultural melting pot, but the Golden Age depended on exploitation. There were huge sums of tribute payments, it was an aristocratic society with abundant time for leisure, and metics or resident `aliens' are running the economy (half inside, half outside mainstream society)

\bisection{Rome urban planning} an unplanned, on a flood-prone hill, gradually improving with aqueducts. The city walls and important public buildings (Forum, Colloseum) were built and expanded by different emperors. Tenement housing was socially mixed, building heights regulated (3-5 stories), but unstable due to additional stories. Ground floor of residential buildings was shops and storage rooms, with people living above

\bisection{Dealing with sewage, bringing water, feeding people} first priority in order to avoid epidemics. The first sewage system, Cloaca Maxima, built to remove city waste into river Tiber. Second priority was to bring water; aqueducts brought 1 billion litres/day. There was no water storage or taps, and it ran continuously, cleaning the streets between raised pavements. Third priority was to bring food; daily food doles were distributed including meat and bread

\bisection{Rome's Golden Age} lasted 50BCE-150CE - why? Limited innovations/planning besides aqueducts and sewers mean that population couldn't be upheld; a highly unequal society, where the emperor was receiving goods/slaves from the poor and colonies, and when this flow was interupted the cities couldn't stand

%\bisection{Pompeii vs. Novaesium} Pompeii had an unplanned old town, with a planned new town; Novaesium was a legioner's camp, built extremely regularly and always for soldiers to recognise 

%%%%%%%%%%%%%%%%%%%%%%%%%%%%%%%%%%%%%%%%%%%%
\pagebreak
\section{Planning the Medieval City}
%%%%%%%%%%%%%%%%%%%%%%%%%%%%%%%%%%%%%%%%%%%%

\bisection{From roman empire to middle ages} the Dark Ages (500-1000AD) was a period during which many former powers disappeared - the Roman empire disintegrated which cut off trade routes and information flows (except for oriental merchants), technologies were lost, cities shrank and disappeared. Then started the Middle Ages

\bisection{Early middle ages} the Ancient world was fragmented into small-scale, manorial feudalist places, a rural system protected by fortifications and castles\alignedmarginpar{Tower of London, Ljubljana castle}. The manor was an (self-sufficient) economic, judicial, military unit

\bisection{High middle ages} cities in this period have 6 origins:
1) former Roman cities blossoming anew, as residences of Emperors, Archbishops, Dukes (Aachen, Cologne, Vienna)
2) monastery settlements
3) castles of the Principality (``bourg'') combined with settlements for craftsmen and merchants (``faubourg'', Graz, Ljubljana)
4) commercial settlements of free merchants and craftsmen
5) free manors and market-towns
6) newly-founded mining towns

\bisection{Characteristics of medieval cities} city walls to protect against barbaric attacks, and castle on a hill; ditches around the city (`graben'); narrow streets with burger houses; church (christian endeavour); central market square to feed a growing population. Almost all European cities date from the high High Middle Ages

\bisection{High medieval city} 1200AD, new towns founded, ancient towns renewed; strict separation of urban and rural functions, with two types of urban places - natural settlements as local-market based central places, and systemic settlements as long-distance-trade based, with varying degrees of freedom from feudal powers. Flanders and North Italy are two hearths of Medieval urbanism. They are `trade-originating' Europe and systematic settlements, compared to the rest as `trade-supporting' Europe with natural settlements that depend on their role as central places. Black death pandemics (1348) more than halve populations of cities

\bisection{Urbanism in High Medieval Cities} unlike Ancient cities, housing combines work and home because space is scarse; built with local building materials and with vernacular architecture; ancient infrastructure like amphitheatres are recycled into new uses or their material reused. Two basic types of cities are the Mediterranean (an uninterrupted urban tradition) and the North of the Alps (new foundations and many new settlers)

\bisection{Mediterranean city} a city of factions, building upon and with Roman remains; towers palaces surrounded by henchmen quarters, mix of Renaissance palaces, earliest tenement houses, and individual houses, many arcades

\bisection{North of the Alps} a city of guilds, there are no city-states bordering each other, but Free Cities surrounded by feudal countryside, stark contrast between urban-rural society. Two settlement systems are ``central-place'' and ``mercantile'', there's a functional segregation based on guilds. Merchants are clustering at ports, nobilty remaining in countryside castles. There are tall house-shops, narrow plots, tight city walls, apprenticeships and journeymen, and ``quarters of tolerance'' for the outsider merchants or students

\bisection{Timeline of Medieval Cities} from geomorphic and unplanned evolving into a geometric planned city: unplanned cities (1100) $\rightarrow$ rebirth of the planned city $\rightarrow$ cross-shaped market towns $\rightarrow$ towns with long market-streets $\rightarrow$ towns with ladder-type streets $\rightarrow$ rebirth of the grid city $\rightarrow$ the grid-shaped town (13th century)

%%%%%%%%%%%%%%%%%%%%%%%%%%%%%%%%%%%%%%%%%%%%
\pagebreak
\section{Planning the Absolutist City: Renaissance}
%%%%%%%%%%%%%%%%%%%%%%%%%%%%%%%%%%%%%%%%%%%%

\textit{Renaissance marks the transition between the Middle Ages to modernity in 15th-16th century. A period of demographic and economic recovery after the Black Death, when the royal powers consolidated and absolutist regimes emerged. Medieval localism is replaced by Baroque centralism, focusing on capital/residential cities of unifying nation states. Geometry and aesthetics are privileged above all else, as well as fortifications - baroque city plans were a military conquest of space. This period marks the beginning of colonial empires.}

\bisection{Medieval dissolution} the Black Death created social disorganisation, medieval communal life faded and power came into the hands of the those who controlled armies, trade routes, or a great accumulation of capital. There was a new economy based on merchant capitalism; a new political framework with central despots and oligarchs, embodied in a nation state; and a new ideology, derived from mechanistic physics elaborated by the army

\bisection{Baroque city} the ills of the Medieval city became unbearable (crowded city walls, disorganised streets, unhygienic, prone to crime) and the walled city couldn't expand horizontally but only in height and density. Absolutist cities grew to sizes far beyond Medieval cities. The street is the unit of planning, planners bring clarity and simplicity by taking down city walls, destroying old sheds and houses, rebuilding crooked alleys into straight streets and rectangular squares, created uniformity with repetition of elements (eg. doors, straight roofs)

\bisection{Ideal cities} there are no Renaissance cities per se, only ideal designs and architectural proposals that were never built (eg. utopias), with innovative ideas like underground goods transported by boat, and street-level pedestrian traffic (Leonardo Da Vinci)

\bisection{Fortified cities} the city is treated as a military support and involes engineers in fortified city planning. The city needs huge military installments (barracks, parade grounds, avenues, arsenals) and a military population. This grew the expertise in administration and accounting, creating demand for mass literacy and basic maths. Public space was supersized and used by wheeled vehicles of the rich requiring stables and mews (poor continue to walk), but many slums, lack of space and air, which defied the high-aesthetic principles

\bisection{Baroque housing} Baroque cities are the origin of tenement housing, new inventions like WC improved the hygiene standards which were still very poor despite the luxurious Baroque lifestyle of the court. Palaces were built to house servants, furniture was a thing of display rather than function
% Conversion to rental housing???

\bisection{New avenues, squares, buildings}\alignedmarginpar{Place des Vosges} a commitment to open space, utmost importance of avenues, designed for wheeled vehicles and army parade grounds. New public attractions like zoos, pleasure gardens, carrousels, museums, with dominant aesthetics and a focus on geometric figures (star-shaped designs, oriented South) 

\bisection{New (countryside) castles and parks}\alignedmarginpar{Versailles, Schönbrunn, Belvedere} new urban quarters and residential cities for royalty. There were giant palaces and gardens equipped with the latest gadgets (steam pump fountains til then only used in industry)

\bisection{Limitations of baroque planning} no concern for the neighbourhood as a unit, for family housing, no conception of ordering of business and industry as a necessary part of any larger achievement of urban order

%%%%%%%%%%%%%%%%%%%%%%%%%%%%%%%%%%%%%%%%%%%%
\pagebreak
\section{Summary of Ancient, Medieval and Absolutist Cities}
%%%%%%%%%%%%%%%%%%%%%%%%%%%%%%%%%%%%%%%%%%%%

\begin{tabular}{r |@{\foo} p{0.9\textwidth}}
 & \bisection{Mesopotamia} one of the earliest urban settlements \\
500-400BCE & \bisection{Athens} Golden Age of the first metropolis \\
480BCE  & \bisection{Archaioteros tropos} old-style, a growing together of earlier settlements, irregular layouts, attention paid to views, built on a hill for natural fortification \\
335BCE  & \bisection{Neotero tropos} new-style, grid layout, main streets, strict zoning, court houses, obsession with correct proportions, able to move to valleys because of innovations in fortifications \\
50BCE-150CE & \bisection{Roman Empire} the Golden Age of Rome \\
800 & \bisection{Dissolution of Roman Empire} \\
6th-11th century & \bisection{Dark Ages} contraction of cities and emergence of a new rural economy: feudalism and manorialism, central-places \\
12th century & \bisection{High Middle Ages} break from feudalism and emergence of cities based on trade and manufacturing (economic driver, not military and administrative), from natural to systemic settlements with more freedoms; cities of factions factions (South) and guilds (North) \\
14th century & \bisection{Black Death} wipes out up to half the population of cities \\
15th-16th century & \bisection{Renaissance} the transition between Middle Ages and Modernity \\
17th century & \bisection{Baroque, Fortification cities} characterised by heavy military infrastructure, straight avenues and aesthetics of geometry privileged above the city as a social unit \\
18th century & \bisection{Revolutions} the American, English and French removes the absolute powers, overthrows feudal estates, secularises the state, removes restrictive regulations imposed by guilds and municipalities \\
1835 & \bisection{Industrial cities} the first and greatest industrial cities appear, manufacturing raw materials and textiles, and with polytechnics \\
1848 & \bisection{British Health Act} is the first attempt to regulare the industrial city \\
1800s & \bisection{Reforms} in London: Metropolitan policy, prison reform, poor law, asylyms, public health acts, water supply, sewers  \\
\end{tabular}

%%%%%%%%%%%%%%%%%%%%%%%%%%%%%%%%%%%%%%%%%%%%
\pagebreak
\section{Early Capitalism: Industrial City Miseries, Early Reforms}
%%%%%%%%%%%%%%%%%%%%%%%%%%%%%%%%%%%%%%%%%%%%

\textit{Up to this point, cities are not built for industry but for handicrafts. The industrial era is one with little urban planning, which is partly why there were so many consequences. This chapter is focused mainly on England and Scotland.}

\bisection{Industrial cities} Circa 1835, the first and greatest industrial cities appear, manufacturing raw materials (water, coal, iron, salt, etc.). Innovations in textile industry include flying shuttle, jenny, steam-powered mules. Polytechnics are substitutes for Universities.

\bisection{Manchester, England}\alignedmarginpar{first innovative milieu}1760-1830, the centre of synergy, with an egalitarian class structure; psychological freedom; incentives to innovate; modest formal education and limited capital required; intelligence network for trading and engineering. BUT it is unprepared for the new industry

\bisection{Conditions of the working class in England}\alignedmarginpar{Friedrich Engels} there are no urban problems as such, only social problems that require revolutionary changes in society to be made. There are deplorable living conditions: starvation, bad sanitation leading to cholera epidemics, homelessness and slums out of sight of middle-upper classes, dense and centralised population. No revolution even though the working class outnumber the bourgeoisie more than 2:1

\bisection{Planning and urbanism} small and irregular courts, lanes, back alleys. Back-to-back housing as an efficient housing model, is the first urbanistic answer to the industrial city but has no aeration

\bisection{British Health Act} 1848 excludes London, Scotland, Ireland. It is the first attempt at regulating the industrial city. Includes: sewage and drainage; refuse removal; sanitary conditions; slaughterhouse regulations; ventilation and hygiene of tenement housing; street pavement and maintenance; public gardens and parks; water supply; funeral services; specific taxes and levies

\bisection{London reforms} in the 1820s, the Metropolitan Police is established; Prison Reforms; Poor Law reforms taking up Benthamian ideas of workhouses for the poor, uniforms, severe discipline, separation by gender and age; Asylums Act; Public Health Act; Water Supply; Sewers

\bisection{Private philanthropist reforms}\alignedmarginpar{Familistère, Bournville} private investments into housing and social conditions, `utopias'

\bisection{Glasgow, Scotland} 1770-1890, a city on the European periphery benefitting from the Trans-Atlantic economy; nearby coal and iron deposits grows industry and economy; steamboats first used on inland waters then able to reach coastal waters and open sea, and iron ships;   international exhibitions in Glasgow (1888, 1901); firms have welfare programmes including housing, but no tenement housing in Scotland compared to terraced/row housing in England

\bisection{Tenement housing} housing shared by multiple dwellings (ie. apartments); pure grid pattern without facilities like toilets or running water; vertical segregation of classes living in the same building (Paris); thin and tall structures (NYC)

%%%%%%%%%%%%%%%%%%%%%%%%%%%%%%%%%%%%%%%%%%%%
\pagebreak
\section{Summary Urban Planning and Urbanism Practices 19th-21st century}
%%%%%%%%%%%%%%%%%%%%%%%%%%%%%%%%%%%%%%%%%%%%

{\setstretch{1.15}
\begin{tabularx}{1.15\textwidth} { 
  | >{\raggedright\arraybackslash}X
  | >{\raggedright\arraybackslash}X 
  | >{\raggedright\arraybackslash}X 
  | >{\raggedright\arraybackslash}X
  | >{\raggedright\arraybackslash}X 
  | >{\raggedright\arraybackslash}X | }
  \hline
  Chapters
  & \textbf{Urban Engineering}
   & \textbf{Reformist} 
   & \textbf{Modernist} 
   & \textbf{Urban Development Planning}
   & \textbf{Urban Governance} \\
  \hline
  \textbf{Timeline}
  & 1860-1900s
  & 1900s-1940s
  & 1950s-1960s
  & 1960-1980s
  & 1980s-today \\ 
  \hline
  \textbf{Planning Types (Suitner)}
  & Pre-WWI Civil Engineering and Urban Design
   & Inter-war Reformist Urbanism, Social Planning
   & Post-WWII Modernist Expert Planning
   & Urban Development Plans or Comprehensive Planning
   & Strategic Management, Collaborative Planning \\
  \hline
  \textbf{Planning Philosophies (Selle)}
   & Urban Engineering
   & Catchment Planning
   & Catchment Planning
   & Urban Development Plans
   & Perspective Planning \\
      \hline
  \textbf{Tools (Albers)}
  & Alignment plans
  & Zoning
  & Zoning
  & Development plans
  & Projects \\
      \hline
  \textbf{Themes}
  &  Averting risks and adaptation planning to avoid health and natural disasters; straightening of streets and regulations of heights and widths; the rest is left to the market
   & Reformist urbanism of the 1920s; early years of local welfare state in Europe; garden city and settlers' movement; public and social housing programs; new standards in zoning legislation and building codes; beginnings of regional planning 
   & Modernity on both sides of the Iron Curtain; comparing welfare capitalist and socialist European city of the 1950s-60s; reconstruction and urban development post-WWII, under different welfare and housing regimes; urban lifecycles and suburbanisation; socialist cities
   & Golden age of welfare in 1970s European cities; comprehensive planning, public coordination of all kinds of urban developments; hey-days of mass social housing; state as a pioneer of urban renewal
   & Restructuring and resistence; commodification of urban development since the 1980s; EU integration and enlargement; European cities in competition; urban entrepreneurialism and project planning; challenge of social cohesion and sustainability; urban marketing and branding \\
  \hline
  \textbf{Goals}
  & Risk aversion and safety as main concerns, engineering foundations for healthy \& growing population
  & Not only ordering but also regulating uses behind the walls
  & Ambitions to plan everything related to the urban, a `super-zoning' approach using project management at all levels of planning
  & Coordination of many actors for urban planning for state-driven renewal
  & Planning is a cooperation of many actors heavily influenced by the market \\
   \hline
   \textbf{Urban Social Policy}
  & Sanitary legislation; asylums 
  & Pioneer welfare services and social housing in cities 
  & Emerging variants/types of national welfare states
  & Golden age of comprehensive national welfare states
  & Downsizing and rescaling the welfare state \\
   \hline
   \textbf{Urban Economic Policy}
  & Liberalism
  & Communalisation of city services
  & Nationalisation of heavy/basic industry
  & Regional \& urban location policies and agencies
  & Neoliberalism; city competition and marketing \\
   \hline
\end{tabularx}}

%%%%%%%%%%%%%%%%%%%%%%%%%%%%%%%%%%%%%%%%%%%%
\pagebreak
\section{Urban Engineering and Fin-de-Siècle Urbanism}
%%%%%%%%%%%%%%%%%%%%%%%%%%%%%%%%%%%%%%%%%%%%

\textit{The beginning of `serious' planning. Technical and engineering challenges are emerging in growing cities (mid to late 19th century), and this period is focusing on massive infrastructure to make cities safe and enable their expansion. The goal is to avoid health crises (eg. cholera) or natural disasters (eg. flooding). It is about averting risk, adaptation planning, and laying the technical foundations which enabled later social and political phases (eg. Red Vienna).}

\bisection{Planning culture} the main tool is alignment plans, that provide safety and avert risks in the city through the ordering streets on a grid system, height and width regulations, and unassuming but important engineering works. These are accompanied by beautification of the city's image. The rest is left to the market

\bisection{Undramatic technical works}\alignedmarginpar{Vienna} the city carried out engineering works to accomodate a growing population in a healthy and safe environment. These included public services, flooding protection, water, electricity and gas supply, public health systems and public hospitals

\bisection{Beautification}\alignedmarginpar{Ringstraße} of the city image is at the forefront of planning, considered important to project the cultural and political ideologies. New buildings of high-culture (museums, theatres, operas, universities) are built in the styles according to their function (Renaissance as a liberal style), and are a contrast to the Baroque palaces and Gothic cathedrals of the old city

\bisection{Building codes}\alignedmarginpar{Vienna 1829} new/first building codes to regulate building heights and width streets (improves airflow and hygiene), making sewage lines obligatory, and other elementary works

\bisection{Grid pattern} originating from NYC and became a planning norm

\bisection{Urban engineering elements} technical infrastructure elements are: height zoning, outer suburbs incorporation, trees planted along boulevards, gas lights in streets, electricity and electric street cars, bridges for metropolitan railroad, granite pavements, new sewers

\bisection{Urban design elements}\alignedmarginpar{Ringstrasse} beautification accompanies urban engineering through parks, green belts, cultural and educational institutions, tenement palaces, world exhibition (serve to spread ideas), urban furniture. Some works like public hospitals can fall in both engineering and beautification categories

\bisection{Tenement housing} luxury tenement palaces make up most of the Ringstraße, they are built in new styles (neo-gothic, neo-baroque, neo-renaissance)

\bisection{Transporting ideas, patents, technologies}\alignedmarginpar{corrugated iron} world exhibitions are mega-events which drive innovation by spreading ideas across countries/continents. Starchitects recycle ancient architectural styles to beautify cities\alignedmarginpar{Athens in Vienna}

\bisection{Deficiencies of urban engineering}

%%%%%%%%%%%%%%%%%%%%%%%%%%%%%%%%%%%%%%%%%%%%
\pagebreak
\section{Reformist Urbanism Pre-WWII}
%%%%%%%%%%%%%%%%%%%%%%%%%%%%%%%%%%%%%%%%%%%%

\textit{Cities are growing and need some organisation and order, and a reconfiguration of places and borders, to respond to the ills of the industrial city and urbanisation. This period is the origin of social welfare and democracy, when there are qualitative improvements in education (schools), health system (hospitals), social work (orphenages, social housing, public pools, libraries). Nonetheless previous elements of planning, like engineering (pipelines, electricity, sewers) are still around.}

\bisection{Planning culture} the tool is zoning, and the goal of urban order replaces pure safety goals

\bisection{Political and scientific framing} democracy arrives, with general voting rights and not graded according to income and wealth. It is the origin of the welfare administration, with qualitative improvements in education, health and housing. The scientific method is employed and democratic participation formally granted

\bisection{New towns and garden cities}\alignedmarginpar{Lechtsworth, 1903; Le Corbusier Villes Radieuses} starting in the early 1900s as a solution to mass housing, it follows the principles of separating uses and homesteading, where there is healthy living and working. Garden cities emerge as `new towns' in the UK (Howard). The idea internationalises through conferences, spreading through Europe, North Am., Russia, etc.

\bisection{Settlers' movement}\alignedmarginpar{Otto Wagner hospital vs. allotment gardens vs. garden city} during the subsistence wartime economy, people are given plots of land to grow food. After WWI, these allotment gardens are turned into cooperative, grassroots movement creating spontaneous unplanned suburbs, a contrast to the well-planning Garden Cities 

\bisection{WWI housing problems}\alignedmarginpar{Harloe} cessation of housing investment; collapse of new housing finance (and hyperinflation destroyed loans); rapid rent increase resisted by tenants (because of unregulated rental market, caused evictions); large urban industrial centers particularly affected by in-migration; rural areas with industry unable to house workers because no housing stock.
The economic and social instability after WWI created a desirable condition for state housing, as a guarantee against a social revolution which transformed into a lasting reform

\bisection{Social housing} social housing emerged from socialist states as way to regulate housing market. States pro-actively provide housing (except in USA with repression of socialism, market went to suburbs). Introduced rent freeze and social housing, an upgrade from tenement housing with more green space, inner courtyard, kindergartens

\bisection{Zoning} previously only focused on building height and street width regulations but allowed almost all land use types (19th century). Early 1900s introduced land-use zoning, like green belts and industrial zones, in a concentric model, and mixed uses are gradually reduced over decades. In 1930s, first high-rises arrive in Europe

\bisection{Regional planning}\alignedmarginpar{conurbation\\Greater London\\Greater Berlin} resistance to incorporation of more and more land by cities into `greater' cities (19th century) creates a need for regional planning, because urban regions need coordination for transport systems, access recreation areas, etc. Happening at a similar time than suburbanisation, and creates urban cores (where people work) and commuter zones (where people live). Settlement looks continuous, no visible rural area per se

%%%%%%%%%%%%%%%%%%%%%%%%%%%%%%%%%%%%%%%%%%%%
\pagebreak
\section{Modernist Urbanism Post-WWII}
%%%%%%%%%%%%%%%%%%%%%%%%%%%%%%%%%%%%%%%%%%%%

\textit{19th century had limited social policies (only insurances for elite) and urban economic policies are non-existant - investors are free to do what they want. In the 20th century, the private supply of elementary infrastructure is questionned as inefficient, unreliable, expensive. In the inter-war period emerged pioneering welfare systems providing social health, education, housing. Social policies popularise post-1945, with a new type of welfare system under capitalism, impinging on planning. The state focused on building heavy industry (gas, metal), until 1970s\alignedmarginpar{UN City Vienna} when the state starts attracting investors and international political institutions through locationalised policies}

\bisection{Garden cities}\alignedmarginpar{New Towns} expanded after 1945. Cities had to be reconstructed, should be built and ordered by its function (zoning). The debate emerged on housing density - high-rise (less land) or low-rise (spills into rural areas)?

\bisection{Social housing}\alignedmarginpar{Policy after 1945} Golden age, a mass production of social housing improving in quality (eg. size, bathroom, heating), becoming the biggest share of the housing supplies in some countries; State production subsidies for building (`aide à la pierre'), housing morgages, consumption subsidies (`aide à la personne'); end of rent control led to liberalisation phasing out affordable housin; social housing production peaked in 1970s, leading to over-production and lack of demand

\bisection{Zoning} Continued, improved, and extended to the whole urban area. Includes land-use regulations, careful planning of what uses should be allowed next to each other, and the end of mixed-zoning

\bisection{Regional planning} Continued resistance to incorporation especially in new democratic times, as such regional planning remains rare and difficult. But it is needed, so regions find other ways such as coordinating cycling paths, public transport systems with uniform tarrif system, planning recreational and protected areas, etc

\bisection{Welfare regimes} Nation States become ambitious and create many welfare policies, including housing, education, health. Different regimes organise welfare systems differently\alignedmarginpar{Esping Anderson welfare triangle}: liberal regime around market; social democratic regime around state supply; conservative regime around family supply and informal connections. Two models are Bismarck, based on work-related insurance for maintenance of employment, and Beveridge, based on universal provision for preventing poverty

\bisection{Housing regimes} public and private markets influence each other to various degrees: on one hand, dualist model with private market characterised by high rents and for-profit, and social housing reserved for low-income, private and public not competing. On the other, a unitary model where non-profit is competing with for-profit housing, which forces rents to stay low, social housing open to a broader class and makes up large share of housing stock. Divergence is persisting, and there is a convergence towards residualisation of social housing

\bisection{Stages of urban development} Dynamics of urban/suburban/desuburban/reurban-isation change population of core and ring zones, and are dynamics to be tamed and managed with planning

\bisection{Socialist and post-socialist urbanism} Political systems such as communism impact urban development, with socialist cities having different characteristics and urban dynamics

%%%%%%%%%%%%%%%%%%%%%%%%%%%%%%%%%%%%%%%%%%%%
\pagebreak
\section{Urban Development or Comprehensive Planning}
%%%%%%%%%%%%%%%%%%%%%%%%%%%%%%%%%%%%%%%%%%%%

\textit{Starting in 1960s-70s. The urban development plans are ambitious plans for everything `urban', but also embedded in a broader, fully-developed national welfare state (not just the embrionic welfare city of the 1920s). Development plans coordinate a multitude of public actors/investors and developments, and don't refrain from steering the market. The state is leading urban renewal, and not private investors. Today, urban development plans take many, many forms, is more or less flexible and precise, and includes private actors and isn't only state dominated}

\bisection{Planning culture} tool is urban development plans (massive planning documents), using project management at all levels (from individual buildings to entire city) for coordination

\bisection{Golden age of welfare}\alignedmarginpar{Sweden's distinct regime} large investments from state into pillars of the welfare state: housing, health, education. Urban renewal (of city centre and new suburbs) are state-led, and not from private investors

\bisection{Housing welfare} myriad of land banking systems; direct housing subsidies where State gives money to developers and individuals to construct dwellings (`aide à la pierre', object subsidies), or money given directly to individuals proportional to income (`aide à la personne'), as well as indirect subsidies with tax rebates

\bisection{General plans}\alignedmarginpar{Sweden as inspiration, fingerplans} started in 1920s, popularised post-WWII. Planning expanded, development axes growing towards new suburban districts/satellite towns along transport stops, opening of subways, pedestrian and cycling paths, redevelopment of run-down city centres; all state led

\bisection{Urban development planning} plans are comprehensive, area-wide, project oriented, influencing markets. The goal is to: maximise choice in welfare supply,\alignedmarginpar{both public \textit{and} private transport, subway and highway}improve efficiency of welfare administration, more mixed zoning, city planning coordinates all sectoral planning, focus on technical infrastructure, deepen academic involvement (sociologists), new planning culture involves more than experts (public meetings, debates)

\bisection{Coordination} urban development plans coordinate public actors actors, such as: government departments, local/regional departments, appointed agencies, private enterprises, community interests. Also coordinate developments like public transport systems

\bisection{City developments}\alignedmarginpar{Donau City} high rise buildings appear in 70s, as offices and council housing. Require strong integration with urban development planning because they require an agreement from the city: requires access to public-transport, no obstruction of protected views

\bisection{Mass social housing} considered a pillar of the welfare state, social housing production reached its peak in 1970s; it is not only a social policy but also regional planning, energy policy, economic policy

\bisection{State-driven urban renewal}\alignedmarginpar{Sweden's distinct housing regime} the state focused on inner city renewal, created historic preservation zones, lowered densities in certain districts, introduced parking management. The rediscovering inner city centre and the gentrification movement were (inadvertedly?) started by the state, and regional and urban agencies try to attract firms and influence the market. General plans extend along axes and get broader and broader

%%%%%%%%%%%%%%%%%%%%%%%%%%%%%%%%%%%%%%%%%%%%
\pagebreak
\section{Urban Governance and Collaborative Planning}
%%%%%%%%%%%%%%%%%%%%%%%%%%%%%%%%%%%%%%%%%%%%

\textit{Planning coordinates many actors, is strategic and competitive. Economic forces lead urban development, the welfare state is restructured, the EU is making steps to integration and enlargements by funding large urban projects, there is inter-city competition and urban branding, and sustainability integrated in projects.}

\bisection{Welfare restructuring/retrenchment} following the economic crisis of the mid-1970s, the Golden Age of welfare is over and social spending is cut. States become more greedy and distributes benefits only after proof of need, and these requirements are more and more restrictive

\bisection{Planning culture} based on strategic projects, planning is locally focused on one area, and thus pays less attention to the rest of the city/region (from regional to project planning). Planners grasp chances when they can (anywhere in the world) and take initiatives

\bisection{Collaborations} increasing  collaboration and complexity between markets, states and households, which makes development viable by acounting for many interests and finding compromises. ASID used to analyse relationships between agency/structure/institutions/discourse in urban/regional development and their socio-economic dev

\bisection{Urban development plans}\alignedmarginpar{Zielgebiet Gürtel} UDP, target areas setting priorities for specific areas. Gradually change in character to become more flexible, an instrument for urban governers, collaborative planners, PPPs, to handle planning with no precision and subject to change

\bisection{Growth coalitions and machines}\alignedmarginpar{Molotch, GaWC} include local businesses, politicians, local media, urban services, unis, cultural institutions. Cities try to attract businesses and wealthy inhabitants to feed government funds, through local boosterism that broadcasts attractity of the city, using rankings of investment climate, attractive transport connections, instrumentalisation of cultural events, urban branding, visionary plans - bypassing other cities

\bisection{Neoliberalisation, commodification, financialisation} tendency towards secrecy, less democratic, more elite-drive priorities to avoid disagreements (non-public agreements bypassing local assemblies). Projects helped by marketing, selective deregulation\alignedmarginpar{urban development corporations, PPP contracts}, shift from social to economic policy, with growth coalitions and machines. Some areas like urban development corporations\alignedmarginpar{Docklands} are excluded from regulations as SEZ

\bisection{Entrepreneurialism}\alignedmarginpar{Harvey 1989} emerging type of urban governance, where comprehensive UDP is taken over by planned urban `fragments'. Inter-urban competition reduces autonomy of local states, fosters growth coalitions/machines and PPPs, competition makes it necessary to enter in EU/global market. Urban entrepreneurs are making use of dense networks of cities for tourism and attraction of new classes, and urban spectacles are important

\bisection{Urban branding and imagineering}\alignedmarginpar{Suitner 2014} ideas of cities are transported through media and social media, who are communicating optimism. Contributing to this is the festivalisation of urban developments

\bisection{Festivalisation of UDP}\alignedmarginpar{Seestadt festival, Donau City, Bilbao effect}  mega-events starting with world exhibition to cultural capitals, project kick-offs with sports events, cultural events instrumentalised, flagship projects, starchitecture

%%%%%%%%%%%%%%%%%%%%%%%%%%%%%%%%%%%%%%%%%%%%
% 						READINGS
%%%%%%%%%%%%%%%%%%%%%%%%%%%%%%%%%%%%%%%%%%%%

\section{Readings}

\subsubsection{HALL Greece}

\subsubsection{HALL Ancient Rome}

\begin{outline}
	\1 Ancient Rome was the first mega city, estimated at 1 million people. Although it wasn’t extremely technologically advanced, its success (in terms of population, and power I guess) is in part due to its urban infrastructure and developments.
	\1 It was an extremely unequal society, and had features we recognise in modern cities: dense living spaces with (abu- sive) landlords, courts, theatres, basilicas, large gathering spaces (colosseum), forums, sewage systems,...
	\1 Two main and complex challenges for Rome: to feed and water its large population. It managed to do both, even though it might not have been done efficiently
	\1 Rome had colonies who produced grain, three quarters of the grain consumption must have come from outside of the city vicinity, usually by river and sea from other (today Italian) regions and countries, from public (imperial) and private merchants
	\1 Administration and public order: the emperor and the rich needed to keep poor people, the plebs, content, if they
didn’t want them to revolt, this meant policing
		\2 When the city became too large to effectively rule by one body, it was divided in 14 boroughs
		\2 The taxation system financed doles, games, public services, with money from the wealthier populations
	\1 Rome ended when the capital was moved to Constantinople. Interestingly, Rome didn’t have many technological innovations, but perhaps this is due to its large pool of slave labour
\end{outline}

\subsubsection{VANCE Feudalism}

\begin{outline}
	\1 \textbf{Roman Empire}: cities must put every effort on bringing goods and slaves from surrounding areas/colonies, in order to support political system; when this flow was interrupted, cities and the political economy suffered
		\2 \textbf{Collapse of Roman Empire}: the collapse is due not to the conditions that allowed barbarians to enter cities, but because it allowed the strongly integrated city networks to weaken. There were also internal failures, like: capacity to build military roads from Italy to Scotland but these roads were not wide enough for proper military carts; trade was considered a lowly occupation and not prioritised over less lucratic agriculture; and there were no economies of scale in production of goods and manufacturing. ``The failure of the Roman system its paratism on the countrysidem and its critical dependence on the network of urbanisation brought the functional structure down'' (p. 84)
		\2 \textbf{Charlemagne's crowning} (800) marks the division from classicle to medieval times. Creates a rival power to Byzantine empire and the establishment of Western civilisation as we know it today
	\1 \textbf{Walled cities}: in the feudal times in the Middle Ages, protective walls were built to provide reasonable security, even around the smallest spaces. This is different from the Roman Empire when only cities on the edge were walled (although more cities were walled when the Roman Empire declined)
	\1 \textbf{Dark Ages}: there is a \textit{contraction} of cities to near extinction, due to economic stagnation that followed the breaking of the Mediterranean trade link because of arrival of Muslims; beginning of \textit{parochialisation} where important staples are produced locally
		\2 Two new institutions in Dark Ages: \textbf{contracted trade}, where goods were given to foreign merchants (Orientals) able to trade in a multi-national and multi-religious market; and a \textbf{new rural economy} to care for localised production and demand, ie. feudalism and the manor, shaping the feudal economy and pushing towards autarky (local self-sufficiency)
		\2 There were no `cities' in the Dark Ages, because there were no places where people devoted themselves to work other than agriculture, and no places with a distinct legal and political system; but there were places that were fortified and administrated
	\1 \textbf{The Medieval Church} came to dominate (shrinking) cities and became a powerful presence, trying to befriend foreigners (other religions),  trading and expand cities beyond the purpose of `central-places'
	\1 \textbf{Feudalism}: a new order was set up in the countryside, so basically a rural system. Land was owned by a sovereign (eg. nobility, vassal) and distributed to people (eg. soldiers) in exchange for their support
		\2 \textbf{Manorialism} a closed economic system based on landownership, ``the purpose was to maximise sufficiency of local provision by undertaking to grow as many necessary crops as possible, to fashion its own tools, to weave and make its own clothing, and otherwise to create a closed economic system''. The Lord of the manor had every incentive to close off his subjects' access to trade to anyone else but himself. There were restrictions on freedom of movement, thus people traded only locally
		\2 \textbf{Central-place theory}: land is defensively kept by a sovereign, and organised to be self-sufficient, based on ``excise taxation'', ie. imposed levies on transport/import/export/storage/sale of goods in defined geographical markets, and a right to engage in trade as a way to tax entry and participation in trade. This encouraged the splitting up of land into manors, so lords could gain more money from trading with each other
		\2 \textbf{Bourg}: on a greater scale than the manor, barons could build fearsome castles (or walls of exceptional strength) to dominate an area and withstand conquests, and could dominate others who lacked protection
		\2 \textbf{Distribution} of manors and castles (ie. private estates) strategically across the land (eg. in England) was done to decentralise support for the central authority, and to maintain a national economy (emergence of nation-states created new tax systems)
	\1 \textbf{High Middle Ages}: the city emerges in the 12th century, when cities emancipate from the repression of the rural feudalism system, and the traditional function of urban places as the cradle of change was restored. Contrast to rural system where wealth is built in physical resources and property, in the city wealth was built on transferable capital and its reinvestment
\end{outline}

\subsubsection{VANCE Medieval city}

\begin{outline}
	\1 \textbf{Medieval city} emerges in 13th century and has completely new qualities both socially and morphologically. There is an effective separation of the city from the countryside, and urbanisation happens not under military and administrative control but by economic drivers - medieval towns are founded to engage in trade and manufacturing, not military and administrative activities
	\1 Two types of settlements in the Middle Ages, in competition with one another
		\2 \textbf{Natural settlements} that operate within closed political-economic systems, with local market places situated around human and not economic forces, they are repeatedly produced by feudal order. The size and shape are determined not by location of customers/economic motives, but by how it can stand as a defensible and administrable unit
		\2 \textbf{Systematic settlements} based on long-distance trade, had independence from the constraining feudal order that produced natural settlements. Grew from the separation from the feudal system, the cities extended geographically beyond the central market domain and found trade links, and were more free because of their economic power
	\1 Three realms are \textbf{trade originating} cities, that are large and prosperous due to manufacturing from skilled artisans and trade supported by clever traders travelling all over Europe; and \textbf{trade supporting} cities, that were fewer but of considerable size and importance to keep trade going; and lastly \textbf{feudal} Europe, antipathic to cities and almost outside of trade
	\1 \textbf{Social urbanisation}: cities were founded in places that had potential for trade, and were populated by social movement. People migrated to cities because they could not own property in the feudal system, and they wanted more freedom and prosperity in the city. Institutions to aculturate rural new-comers were required, as well as jobs, to keep rebelions at bay
	\1 \textbf{Medieval houses}: reflecting the times, buildings housed both workers and their work. Medieval towns were founded to engage in productive activities of trade and manufacture
	\1 \textbf{Market place} was at the centre of the city, the town hall was located there to show importance of the city although built long after foundation of the city; good sized towns had many markets, for different goods
	\1 \textbf{Separation of functions}: starting with a vertical separation of functions, and then specialised buildings and landuses like the castle, church, abbey, monastery, guildhall, city hall, market halls...; social divisions are important (especially in Southern Europe) as well as functional segregation (especially in Northern Europe)
		\2 \textbf{Mediterranean city}: discrete and definite social quarters, creating a \textbf{city of factions}, or internal enemies. This created an urban form that afforded protection for groups against their neighbours. The tower palace was the urban power structure, and it was surrounded by the residences (quarters) of the supporters of the faction (henchmen)
		\2 \textbf{Northern city}: place of well-perceived occupational quarters, creating a city of guilds, with guildhalls as the focal urban building. The city had a concentration of trades in particular streets/quarters
	\1 15th century, provision of housing for workers was based on rent and rent-paying abilities
\end{outline}

\subsubsection{MUMFORD Baroque Power}

\begin{outline}
	\1 \textbf{Medieval dissolution}: a shift in authority and power at the end of the Middle Ages meant that medieval life gradually faded and blended with new times. The medieval institutions (feudalism, manorialism) reorganised into military organisations, new `renascence' buildings were built on top of the medieval city plan, within the medieval city walls, by craftsmen and guilds organised on medieval lines
	\1 \textbf{New urban complex}: a reogranisation of society
		\2 \textbf{New economy} based on merchant capitalism (moving goods from one market to another where they are more valuable)
		\2 \textbf{New political framework} with a central despot or oligarch, usually embodied in a nation state
		\2 \textbf{New ideology} derived from mechanistic physics, previously elaborated by the army and the monestary
		\2 The Black Death in the 14th century wiped out up to half the population of towns. This was accompanied by a social disorganisation (similar to post war), medieval communal life disintegrated and power came into the hands of those who controlled armies, trade routes, and a great accumulation of capital. Academic (universities, scientists, philosophers) and spiritual freedoms (witchcraft, paganism) were repressed by the ruling body
		\2 There was an open desire from rulers to grow their kingdom: ``to produce and display wealth, to seize and extend power, became the universal imperatives; they had long been practiced, but they were now openly avowed, as guiding principles for a whole society'' (p. 346)
		\2 ``From medieval universality to baroque uniformity: from medieval localism to baroque centralism: from the absolutism of God and the Holy Catholic Church to the absolutism of the temporal sovereign and the national state, as both a source of authority and an objective of collective worship'' 
	\1 \textbf{Baroque cities}: more of a purification than a rebirth, that open up and modify the structure of the medieval city. The disorganisation of medieval cities becbame unbearable (unhygienic, crowded, prone to crime) and new planners and builders took down walls, toor down sheds/booths/old houses, rebuilt crooked alleys into straight streets or rectangular squares
		\2 Clarity and simplicity built by straight streets, unbroken horizontal roof, round arch, repetition of uniform elements (eg. doors), 
		\2 \textbf{Baroque plans}: new urban quarters, new residential cities for royalty
	\1 \textbf{Fortified towns}: necessity to upgrade city walls which no longer served to protect the city (because of new technologies). All efforts went into militarisation of the city, and recruiting soldiers to defend. Suburbs extended out of the fortified cities, only accessible by the wealthy because they had to be reached by horse (the poor was always on foot)
		\2 Emphasis on war in this period; all law is martial law, and anyone capable of financially supporting the military effort could become the master of the city or gain power
	\1 Avenues were the most important symbol of the city, after that institutions and buildings, and only after that the city as a social unit
		\2 \textbf{Avenues} designed for wheeled vehicles, as if the city's purpose was to facilitate for traffic and transport
\end{outline}


\subsubsection{MUMFORD Court Palace Capital}

\begin{outline}
	\1 The \textbf{Baroque court} is living lavishly, doing as it pleases with no restrain, and controlled the city in nearly every aspect of life
		\2 From the urban, the palace had rent, tribute, taxes, command of the army, control of the state; from the rural side came well-built, well-excercised, well-fed men and women who made up the court and received honours
		\2 Palaces housed the servants and hundreds of horses
	\1 Pleasures of the palace: pleasure gardens, zoos, museums, art galleries, all were created to serve the pleasure of the court and wealthy, with eating and drinking in expensive restaurants and cafes
		\2 Furniture became a thing of display rather than function, special attention was paid to the aesthetics of home
	\1 \textbf{Limitations of baroque planning} no concern for the neighbourhood as a unit, for family housing, no conception of ordering of business and industry as a necessary part of any larger achievement of urban order
\end{outline}

\subsubsection{ENGELS Great Towns}

\begin{outline}
	\1 The `great towns' of Scotland, Ireland and England were a colossal centralisation on people in cities (millions), creating commercial capitals of the world (Birmingham, Leeds, Manchester, Dublin, Glasgow, London). They are inhabited chiefly by the working class, who work in factories, and who outnumber the bourgeoisie 2:1
	\1 \textbf{Quality of life} is sacrificed for the industrial and urban innovations in the city, there is intense individualisation and disregard for others, and people with power (capitalists) exploit others (the poor) for profit
		\2 The poor are an `inconvenience' to the bourgeoisie, and live in deplorable situations. They have no property of their own
		\2 \textbf{Dwellings} are badly planned and badly built. The working class is crowded in slums, with bad ventilation, as little space as possible, with no furniture. Building density is high - dwellings and people are piled on top of another in ill-ventilated and badly drained streets. Risks of flooding and overflow of the bad sewer systems. There are thousands homeless.
		\2 \textbf{Diseases} and epidemics breakout, eg. cholera
		\2 \textbf{Clothing} the working class had single set of clothes, in which they live, work, sleep; many barefoot
		\2 \textbf{Food} is of poor quality and old, many died of starvation or food poisoning from expired or altered food-stuffs; the poor are ripped off as much as possible and the poor are the first victimes of fraud
	\1 \textbf{New factory towns} planning of new towns for industry, with factories, outside of town. The working class districts were separated from the rest, to hide them from the wealthier bourgeoisie
	\1 \textbf{Rise of capitalists} who are driven by profit and exploiting the working class
\end{outline}

\subsubsection{SCHORSKE Ringstrasse}

\begin{outline}
	\1 \textbf{Undramatic technical works} carried out by the city of Vienna in the 19th century, to accomodate a growing population in a healthy and safe environment. These included public services, the Danube flood protection, a superb water supply, public hospitals (the city now assuming responsibility of medical provision instead of the charitable church), a public health system
		\2 Expanding Vienna still retained the Baroque commitment to open space, but parks were conceived in a more organic manner and not pure geometry
		\2 Planning practice: based on a grid system, with regulations only on heights and widths of buildings
	\1 \textbf{Politics}: after the army was defeated by France and Prussia in 1859 and 1866, the military lost power and liberals took over. There was a new ruling class, expressing new values that were reflected in the built environment (eg. Ringstrasse plans)		\2 Late defortification/destruction of city walls, then enabled incorporation of (previously feudal) suburbs into the administration of the city
		\2 Liberal buildings: University, Rathaus, Parlament
	\1 \textbf{Beautification} of the city's image as a driver of planning, pluralism of architectural styles whereby each building was constructed in the architectural style deemed appropriate to the function: Renaissance style University as a symbol of liberal culture; Baroque Burgtheater; 
		\2 Style as the expression of a cultural and political era, is considered more important for a building than function
		\2 Importance is accorded to the location and the orientation of the Ringstrasse buildings, eg. the Parlament is directly facing the Hofburg Palace and concretising the liberal politics of the time
		\2 \textbf{Priority to monumental buildings}, of which Votivkirche was the first. The contrast of old Medieval/Baroque styles is visible between the old inner city (Baroque palaces and residence, Gothic cathedral) and the Ring area (centres of higher culture like museums, university, theatres, opera)
\end{outline}


\subsubsection{HARLOE Social Housing}

\begin{outline}
	\1 Describes social housing during the period of 1914 to late 1920s, in a post-WWI and interwar context
	\1 \textbf{Pre WWI liberalism} was reflected in the social housing provision: it was limited, capitalist preferred expanding their own schemes rather than building housing, incursions on private property rights and private housign market were marginal
	\1 WWI redefined the state/economy relationship, the private market collapsed (supply and affordability crisis) and there was social unrest which created the ideal conditions for state intervention. These were seen as a temporary change from the pre-WWI situation
		\2 The embryonic social housing methods and institutional structures started in the inter-war period, were built upon and solidified post-WWII
	\1 WWI on housing and industry: prior to WWI, there had been no involvement from the state or industry into the matters of the workers, who organised in unions. However, a relationship between state, capital and labour (`corporatism') was necessary in order to manage consequences of the war on society. State guaranteed earnings and control of inflation to labour; state and capital collaborated and there was little control over porfit margins from war production (`war time profiteering'); income inequalities were exacerbated even though unemployement and extreme poverty diminished
		\2 Inflation and loss in income, profiteering by industry while labour was paying the cost, consumer shortages including of housing and food, strains in the state/capital/labour relationship, radicalised the working class and led to strikes - this ``seemed to threaten'' the social order
	\1 Post-WWI economic instability, inflation, working-class militancy, meant that governments and elites felt a social revolution coming and made substantial efforts to respond to working-class demands - social housing was used as a garantee against revolution
		\2 There were common developments: 1. the cessation of housing investment, 2. collapse of housing finance, 3. rapidly inflating rent levels, eviction of tenants unable to pay more, resentment of profiteering landlords, 4. problem particularly acute in industrial centres where demand was overinflated by in-migration of war-workers, 5. acute housing shortage in rural and small areas, where industries were located but with no working-class housing stock	
			\3 Rise of suburban movement: suburban lotissements in Paris, creating the communist `red belt'
			\3 Increase in housing standards: eg. along Garden City ideals (in the UK) which raised the standards for social housing
			\3 Middle-class conceptions of how housing should be were incorporated into social housing for workers, eg. segregation of space, not allowed to work inside the home, not able to sublet or double up, small gardens only suitable for recreation
		\1 The rent controls and housing subsidies were supposed to be temporary measures in times of crisis, to prevent a workers revolution. Instead, it led to a persisting social reform. After the war time crises were resolved (inflation, uncontrolled rents, economic instability, risk of worker uprising), the social housing policies stayed around ``because underlying social and economic conditions prevented a politically convincing, legitimised return to the `solution' of private market provision'' (p. 143)
		\1 One critique of the mass social housing provision of the 1920s is that it did not have much impact on providing dwellings for the less well-paid, unemployed, or inactive population, and was provided mostly for working and middle class households
\end{outline}

\subsubsection{MATZNETTER MUNDT Welfare Regimes and Housing}

\begin{outline}
	\1 Esping Anderson's welfare triangle classifies ``qualitatively different arrangements between state, market and the family''
		\2 Previously, public policy research used quantitative measures like the share of social spending against state budget or GDP. This made it seem like states were converging on social policies, depending on socio-economic development. A similar observation could be made on the convergence of housing towards privatisation
		\2 In contrast, the triangular typology was multi-dimensional, and applied to a limited set of social policies: health insurance, unemployment insurance, pension systems. EA measured the deommodication of the services and the resulting stratification
	\1 Three types of welfare regimes:
		\2 \textbf{Liberal welfare state}: the welfare policies are only distributed to those who meet requirements, subsidies controlled on a regular basis and lifted when income exceeds certain amount. Market provides the welfare services. Residualist policies focusing on specific individuals
		\2 \textbf{Social-democratic welfare state}: the welfare policies address the population as a whole without many restrictions. Welfare provision from the state and not the market, goods and services are decommodified. As a corollary taxes are high but welfare services are of good quality. Universalist policies focusing on entire population
		\2 \textbf{Conservation-corporatist welfare state}:  graded welfare policies, according to class, status and gender, and reproduces income and wealth differentials (inequalities). Welfare is provided mostly by the family and disproportionately women
	\1 Path dependency and path changes of welfare regimes: we observe that important past decisions curtail the options for future development similar to Bismarck-Beverdige divide
		\2 Bismarck: social policies are based on work-related insurance and on earning-related benefits for employees; this model is about income maintenance for employees
		\2 Beveridge: social policies are based on universal provision, need, based on residence entitlement, benefits are flat rate and taxes high; this model is about prevention of poverty; eg. NHS, national insurance service
		\2 Lib-lab regime: strategic integration of Labour parties and unions into governing Liberal coalitions
	\1 More welfare regimes: EA's welfare triangle doesn't account for Mediterranean welfare states; it should be extended with a gender discrimination index to account for women's informal care work
	\1 Housing systems and welfare regimes: post-WWII housing needs decreased in quantitative and later qualitative terms; subsidies are shifted from the supply side to demand side
		\2 Housing systems are dependent on welfare regimes: in liberal regimes, developers rely on land speculation rather than the building profits (eg. France, Britain); in social-democratic regimes, the reverse happens, because the land supply is tightly regulated by the state developers rely on building profits (eg. Sweden)
		\2 How important is housing for the welfare state?	Kemeny distinguishes dualist and unitary housing systems related to welfare regimes. Differences in rental sector emerged from the degree of privatisation vs. collectivisation
		\2 \textbf{Dualism}, or dual rental markets: there is both private and public housing provision, and the state excludes private market from social housing for low-income. Private market characterised by high rents, there is no competition between private and public sectors
		\2 \textbf{Unitary}: non-profit public housing is allowed to compete with for-profit private market. In this way, non-profit housing is regulating the private market and forcing rents to stay low. Social housing makes up a large share of housing stock and is open to a broad class of people
	\1 \textbf{Home ownership} is considered a privatisation of one of the pillars of the welfare society, a contrast to home ownership being considered an ideal of today's society
\end{outline}

\subsubsection{HARVEY From Managerialism to Entrepreneurialism}

\begin{outline}
	\1 Urban governance shift from providing local services, facilities and benefits (managerialism), to encouraging local development and employment growth (entrepreneurialism)
	\1 1980s reorientation of urban governance towards entrepreneurialism: cities realise that they must play a role in sustaining employement and businesses in their area, by being innovative and entrepreneurial, to ensure prosperity for the population
	\1 1970s: local governments are successful in taking over from central governments in supporting small firms, building ties between public and private sector, promoting local areas to attract new business. There is also declining power of the state to control multinational money flows, because investments are taking places in negotiations between internationl finance capital and local powers maximising their attractiveness to investors
		\2 Civic boosterism and entrepreneurialism in US
	\1 Urban governance is not a government. Rather, it is a broader coalition of political and economic forces, made up of eg. mayors, local and international firms.
		\2 `Cities' are not active agents, but mere things; urbanisation is a spatially bound, social process with many interacting actors. There are formations of coalitions and alliances, and in some places politicians (eg. mayors) have stronger power and in other, firms do.
	\1 PPP at the center of entrepreneurialism, are speculative in execution and design and therefor prone to the difficulties and dangers attached to speculative developments, as opposed to rationally planned and coordinated development	
\end{outline}

\subsubsection{MOULAERT ASID}

\begin{outline}
	\1 Analyses the relationship between Agency, Structure, Institutions, Discourse in urban and regional development, in order to assess socio-economic development at different spatial scales
	\1 \textbf{Agency} meaningful human behaviour, individual or collective, that makes a difference in natural and/or social worlds. Can include: appropriation or transformation of nature; creation or variation of identities, subjectivities, social standing; design, building, arrangement, creative destruction, dismantaling of institutions; re-articulation of discourses
	\1 \textbf{Structure} the natural and/or social realities defined in specific spatial context, that will not change in the short-medium term. The structure shapes the emergent properties of interactions between social agents, it facilitates or constrains them
		\2  Morphology, demography, political economic system, change from authoritarian monarchy to democratic republic, urban networks
	\1 \textbf{Institution} a set of more or less coherent, interconnected routines, organisational practices, conventions, rules, sanctioning mechanisms, governing practices over some kind of actions; significance/reach can be local to global
	\1 \textbf{Discourse} production of intersubjective sense and meaning
		\2 Debates, ideals, discourses, images
\end{outline}

%%%%%%%%%%%%%%%%%%%%%%%%%%%%%%%%%%%%%%%%%%%%
% 						NOTES
%%%%%%%%%%%%%%%%%%%%%%%%%%%%%%%%%%%%%%%%%%%%

\begin{comment}
\begin{chronology}[10]{1900}{2010}{100ex}[\textwidth]
\event{1930}{Bismarckian (pioneers)}
\event{1960}{Beveridgian (divergence)}
\event{1997}{Convergence}
\end{chronology}
\end{comment}

\printbibliography

\end{document}
