\documentclass{article}

\linespread{1.5}
\usepackage[utf8]{inputenc}
\usepackage[left=0.5in,right=1.25in,bottom=0.75in,top=0.75in]{geometry}
\setlength\parindent{0pt}
\setlength{\parskip}{1em}
\setcounter{secnumdepth}{0}
\usepackage{outlines}
\usepackage{graphicx}
\graphicspath{ {imgs} }
\usepackage[hyphens]{url}
\usepackage{hyperref}
\usepackage{color,soul}
\usepackage[normalem]{ulem}

\usepackage[
backend=biber,
style=apa,
citestyle=authoryear,
sorting=nyt,
]{biblatex}
\addbibresource{refs.bib}

\usepackage{comment}
\specialcomment{topicsen}{\begingroup\bfseries\scriptsize}{\endgroup}
%\excludecomment{topicsen}

\newcommand{\bisection}[1]{\textbf{\textit{#1}}}

\newcommand{\alignedmarginpar}[1]{%
        \marginpar{\raggedright\small #1}
    }

\title{Principles of Urbanism and Planning - Revisions}
\author{Carla Hyenne}

\begin{document}

\maketitle

\tableofcontents

\pagebreak

%%%%%%%%%%%%%%%%%%%%%%%%%%%%%%%%%%%%%%%%%%%%
\pagebreak
\section{Planning the ancient city}
%%%%%%%%%%%%%%%%%%%%%%%%%%%%%%%%%%%%%%%%%%%%

\textit{Authors}:

\textit{Influential figures}:

\subsection{Greek urban planning}

\subsection{Roman urban planning}


\bisection{}

\bisection{}

\bisection{}

\bisection{}

\bisection{}

\bisection{}

\bisection{}

\bisection{}

\bisection{}

\bisection{}

\bisection{}


%%%%%%%%%%%%%%%%%%%%%%%%%%%%%%%%%%%%%%%%%%%%
\pagebreak
\section{Planning the medieval city}
%%%%%%%%%%%%%%%%%%%%%%%%%%%%%%%%%%%%%%%%%%%%

\subsubsection{From roman empire to middle ages}

\subsubsection{Early middle ages}

\subsubsection{High middle ages}

\subsubsection{Medieval cities}

\bisection{High medieval city}

\bisection{Mediterranean city}

\bisection{North of the Alps}

%%%%%%%%%%%%%%%%%%%%%%%%%%%%%%%%%%%%%%%%%%%%
\pagebreak
\section{Renaissance: Planning the absolutist city}
%%%%%%%%%%%%%%%%%%%%%%%%%%%%%%%%%%%%%%%%%%%%

\bisection{Ideal cities}

\bisection{Fortification towns}

\bisection{Tenement housing}


\bisection{}

\bisection{}

\bisection{}

\bisection{}

\bisection{}

\bisection{}

\bisection{}

\bisection{}

\bisection{}

\bisection{}

\bisection{}

%%%%%%%%%%%%%%%%%%%%%%%%%%%%%%%%%%%%%%%%%%%%
\pagebreak
\section{Early capitalism}
%%%%%%%%%%%%%%%%%%%%%%%%%%%%%%%%%%%%%%%%%%%%

\textit{Authors}:

\textit{Influential figures}:

\bisection{Baroque}

\bisection{Manchester, Glasgow}

%%%%%%%%%%%%%%%%%%%%%%%%%%%%%%%%%%%%%%%%%%%%
\pagebreak
\section{Urban engineering and fin-de-siècle urbanism}
%%%%%%%%%%%%%%%%%%%%%%%%%%%%%%%%%%%%%%%%%%%%
Urban Planning Type I

\textit{Authors}:

\textit{Influential figures}:

\bisection{}

%%%%%%%%%%%%%%%%%%%%%%%%%%%%%%%%%%%%%%%%%%%%
\pagebreak
\section{Reformist urbanism pre-WWII}
%%%%%%%%%%%%%%%%%%%%%%%%%%%%%%%%%%%%%%%%%%%%
Urban Planning Type IIa 

\bisection{New towns and garden cities}:\alignedmarginpar{Lechtsworth, 1903}

\bisection{Social housing}

\bisection{Zoning}

\bisection{Regional planning}

\bisection{Welfare and housing regimes}


%%%%%%%%%%%%%%%%%%%%%%%%%%%%%%%%%%%%%%%%%%%%
\pagebreak
\section{Modernist urbanism post-WWII}
%%%%%%%%%%%%%%%%%%%%%%%%%%%%%%%%%%%%%%%%%%%%

\textit{The emergence of welfare states after 1945 means that the concepts from the inter-war reformist period has superseeded. Planning is reconstructed in parallel to the new democratic times, and infused by economic and social policies}

\bisection{Garden cities} The idea expanded after 1945 into New Towns. Cities had oto be reconstructed, and debate emerged on housing density - should housing be high-rise (requiring significantly less land, more dense) or low-rise (housing spills into outer rural areas, less dense). The city should be build and ordered by its function

\bisection{Social housing} Golden age of social housing, with a mass production of social housing improving in quality (eg. size, bathroom, heating). Public housing was the biggest share of the housing supplies in some countries. State offered subsidies for building housing and housing morgages. Rent liberalisation phased out affordable housing

\bisection{Zoning} Continued, improved, and extended to the whole urban area. Includes land-use regulations, careful planning of what uses should be allowed next to each other, and mixed-zoning is over

\bisection{Regional planning} Continued resistance to incorporation especially in new democratic times, as such regional planning remains rare and difficult. But it is needed, so regions find other ways such as coordinating cycling paths, public transport systems with uniform tarrif system, planning recreational and protected areas, etc

\bisection{Urban social policy} In 19th century, there were limited social policies, only insurances for the elite. In 20th century, in the inter-war period, pioneering systems start providing social health, education, housing. Social policies popularise post-1945 with a new type of welfare system under capitalism, impinging on planning

\bisection{Urban economic policy} In the 19th century, urban economic policies are non-existant and investors are free to do what they want. In 1920s, private supply of elementary infrastructure is questionned as inefficient, unreliable, expensive. Post-1945, State focuses on building heavy industry (public gas, metal), until 1970s when States start attracting investors and international political institutions through locationalised policies\alignedmarginpar{UN City Vienna}

\bisection{Welfare and housing regimes} Nation States become ambitious and create many welfare policies, including housing, education, health. Different regimes organise their welfare systems differently\alignedmarginpar{Welfare triangle}. Liberal regime centered on the market; social democratic regime centered around state supply; conservative regime centered aroundfamily supply and informal connections. In housing sectors, unitary vs. dualist regimes where public and private markets influence each other to various degrees

\bisection{Stages of urban development} Dynamics of urban/suburban/desuburban/deurban-isation change population of core and ring zones, and are dynamics to be tamed and managed with planning

\bisection{Socialist and post-socialist urbanism} Political systems such as communism impact urban development, with socialist cities having different characteristics and urban dynamics.

%%%%%%%%%%%%%%%%%%%%%%%%%%%%%%%%%%%%%%%%%%%%
\pagebreak
\section{Urban development plans}
%%%%%%%%%%%%%%%%%%%%%%%%%%%%%%%%%%%%%%%%%%%%

Urban Planning Type III 

\bisection{}

\bisection{}

\bisection{}

\bisection{}

\bisection{}

\bisection{}

\bisection{}

\bisection{}

\bisection{}

\bisection{}

\bisection{}



%%%%%%%%%%%%%%%%%%%%%%%%%%%%%%%%%%%%%%%%%%%%
\pagebreak
\section{Urban management, collaborative planning}
%%%%%%%%%%%%%%%%%%%%%%%%%%%%%%%%%%%%%%%%%%%%

Urban Planning Type IV

\textit{Authors}:

\textit{Influential figures}:


\bisection{}

\bisection{}

\bisection{}

\bisection{}

\bisection{}

\bisection{}

\bisection{}

\bisection{}

\bisection{}

\bisection{}

\bisection{}


%%%%%%%%%%%%%%%%%%%%%%%%%%%%%%%%%%%%%%%%%%%%
\pagebreak
\section{Planning the sustainable city}
%%%%%%%%%%%%%%%%%%%%%%%%%%%%%%%%%%%%%%%%%%%%

Urban Planning Type V

\textit{Authors}:

\textit{Influential figures}:


\bisection{}

\bisection{}

\bisection{}

\bisection{}

\bisection{}

\bisection{}

\bisection{}

\bisection{}

\bisection{}

\bisection{}

\bisection{}


\printbibliography

\end{document}
