\documentclass{article}

\linespread{1.5}
\usepackage[utf8]{inputenc}
\usepackage[left=0.5in,right=1.25in,bottom=0.75in,top=0.75in]{geometry}
\setlength\parindent{0pt}
\setlength{\parskip}{1em}
\setcounter{secnumdepth}{0}
\usepackage{outlines}
\usepackage{graphicx}
\graphicspath{ {imgs} }
\usepackage[hyphens]{url}
\usepackage{hyperref}
\usepackage{color,soul}
\usepackage[normalem]{ulem}
\usepackage{chronology}
\usepackage{tabularx}

\usepackage[
backend=biber,
style=apa,
citestyle=authoryear,
sorting=nyt,
]{biblatex}
\addbibresource{refs.bib}

\usepackage{comment}
\specialcomment{topicsen}{\begingroup\bfseries\scriptsize}{\endgroup}
%\excludecomment{topicsen}

\newcommand{\bisection}[1]{\textbf{\textit{#1}}}

\newcommand{\alignedmarginpar}[1]{%
        \marginpar{\raggedright\small #1}
    }

\title{Principles of Urbanism and Planning - Revisions}
\author{Carla Hyenne}

\begin{document}

\maketitle

\tableofcontents

\pagebreak

%%%%%%%%%%%%%%%%%%%%%%%%%%%%%%%%%%%%%%%%%%%%
\pagebreak
\section{Planning the ancient city}
%%%%%%%%%%%%%%%%%%%%%%%%%%%%%%%%%%%%%%%%%%%%

\textit{todo}

\bisection{Greek urban planning}

\bisection{Roman urban planning}

\bisection{}

\bisection{}

\bisection{}

\bisection{}

\bisection{}

\bisection{}

\bisection{}


%%%%%%%%%%%%%%%%%%%%%%%%%%%%%%%%%%%%%%%%%%%%
\pagebreak
\section{Planning the medieval city}
%%%%%%%%%%%%%%%%%%%%%%%%%%%%%%%%%%%%%%%%%%%%

\bisection{From roman empire to middle ages} the Dark Ages (500-1000AD) was a period during which many former powers disappeared - the Roman empire disintegrated which cut off trade routes and information flows (except for oriental merchants), technologies were lost, cities shrank and disappeared. Then started the Middle Ages

\bisection{Early middle ages} the Ancient world was fragmented into small-scale, manorial feudalist places, protected by fortifications and (medieval) castles\alignedmarginpar{Tower of London, Ljubljana castle}. The manor was an economic, judicial, military unit

\bisection{High middle ages} cities in this period have 6 origins:
1) former Roman cities blossoming anew, as residences of Emperors, Archbishops, Dukes (Aachen, Cologne, Vienna)
2) monastery settlements
3) castles of the Principality (``bourg'') combined with settlements for craftsmen and merchants (``faubourg'', Graz, Ljubljana)
4) commercial settlements of free merchants and craftsmen
5) free manors and market-towns
6) newly-founded mining towns

\bisection{Characteristics of medieval cities} city walls to protect against barbaric attacks, and castle on a hill; ditches around the city (`graben'); narrow streets with burger houses; church (christian endeavour); central market square to feed a growing population. Almost all European cities date from the high High Middle Ages

\bisection{High medieval city} 1200AD, new towns founded, ancient towns renewed; strict separation of urban and rural functions, with two types of urban places - natural settlements as local-market based central places, and systemic settlements as long-distance-trade based, with varying degrees of freedom from feudal powers. Flanders and North Italy are two hearths of Medieval urbanism. They are `trade-originating' Europe and systematic settlements, compared to the rest as `trade-supporting' Europe with natural settlements that depend on their role as central places. Black death pandemics (1348) more than halve populations of cities

\bisection{Urbanism in High Medieval Cities} unlike Ancient cities, housing combines work and home because space is scarse; built with local building materials and with vernacular architecture; ancient infrastructure like amphitheatres are recycled into new uses or their material reused. Two basic types of cities are the Mediterranean (an uninterrupted urban tradition) and the North of the Alps (new foundations and many new settlers)

\bisection{Mediterranean city} a city of factions, building upon and with Roman remains; house-towers, mix of Renaissance palaces, earliest tenement houses, and individual houses, many arcades

\bisection{North of the Alps} a city of guilds, there are no city-states bordering each other, but Free Cities surrounded by feudal countryside, stark contrast between urban-rural society. Two settlement systems are ``central-place'' and ``mercantile'', there's a functional segregation based on guilds. Merchants are clustering at ports, nobilty remaining in countryside castles. There are tall house-shops, narrow plots, tight city walls, apprenticeships and journeymen, and ``quarters of tolerance'' for the outsider merchants or students

\bisection{Timeline of Medieval Cities} from geomorphic and unplanned evolving into a geometric planned city: unplanned cities (1100) $\rightarrow$ rebirth of the planned city $\rightarrow$ cross-shaped market towns $\rightarrow$ towns with long market-streets $\rightarrow$ towns with ladder-type streets $\rightarrow$ rebirth of the grid city $\rightarrow$ the grid-shaped town (13th century)

%%%%%%%%%%%%%%%%%%%%%%%%%%%%%%%%%%%%%%%%%%%%
\pagebreak
\section{Renaissance: Planning the absolutist city}
%%%%%%%%%%%%%%%%%%%%%%%%%%%%%%%%%%%%%%%%%%%%

\bisection{Renaissance} Renaissance marks the transition between middle ages to modernity in 15th-16th century. A period of demographic and economic recovery after Black Death pandemic. Royal powers consolidated and absolutist regimes emerge. Medieval localism is replaced by Baroque centralism, focusing on capital/residential cities of unifying states. Beginning of colonial empires.

\bisection{Renaissance/Baroque city} there are no Renaissance cities, only ideal designs, absolutist cities grow to sizes far beyond Medieval cities. Origins of office buildings and tenement housing. Military innovations involve engineers in fortified city planning; walled cities cannot expand horizontally, only in height and density

\bisection{Ideal cities} never built, only architectural proposals (eg. utopias), with innovative ideas like underground good transported by boat, and street-level pedestrian traffic (Leonardo Da Vinci)

\bisection{Fortification towns} huge military installments (barracks, parade grounds, avenues, arsenals) and military population. A growing expertise in administration and accounting, creating demand for mass literacy and basic maths. Public space is supersized and used by wheeled vehicles of the rich requiring stables and mews, poor continue to walk. Giant palaces and gardens, including latest gadgets (steam pump fountains til then only used in industry). Public attractions like zoos, pleasure gardens, carrousels, museums, with outward aesthetic dominating and focus on geometric figures

%\bisection{Conversion to rental housing} 

\bisection{New avenues and squares} 

\bisection{New (countryside) castles and parks} Versailles, Paris; Schönbrunn, Vienna; Edinburgh New Town; Karlsruhe centered on Schloss (a Baroque new town). All very geometric, star-shaped designs, oriented ideally to face South

%%%%%%%%%%%%%%%%%%%%%%%%%%%%%%%%%%%%%%%%%%%%
\pagebreak
\section{Early capitalism: industrial city miseries, early reforms}
%%%%%%%%%%%%%%%%%%%%%%%%%%%%%%%%%%%%%%%%%%%%

\textit{Up to this point, cities are not built for industry but for handicrafts. The industrial era is one with little urban planning, which is partly why there were so many consequences. This chapter is focused mainly on England and Scotland.}

\bisection{Industrial cities} Circa 1835, the first and greatest industrial cities appear, manufacturing raw materials (water, coal, iron, salt, etc.). Innovations in textile industry include flying shuttle, jenny, steam-powered mules. Polytechnics are substitutes for Universities.

\bisection{Manchester, England}\alignedmarginpar{first innovative milieu} is the centre of synergy, with an egalitarian class structure; psychological freedom; incentives to innovate; modest formal education and limited capital required; intelligence network for trading and engineering. BUT it is unprepared for the new industry

\bisection{Conditions of the working class in England}\alignedmarginpar{Friedrich Engels} there are no urban problems as such, only social proclems that require revolutionary changes in society to be made. There are deplorable living conditions: starvation, bad sanitation creating cholera epidemics, homelessness and slums out of sight of middle class, dense and centralised population. No revolution even though the working class outnumber the bourgeoisie more than 2:1.

\bisection{Planning and urbanism} small and irregular courts, lanes, back alleys. Back-to-back housing as a efficient housing is the first urbanistic answer to the industrial city, but has no aeration. 

\bisection{British Health Act} 1848 excludes London, Scotland, Ireland. It is the first attempt at regulating the industrial city. Includes: sewage and drainage; refuse removal; sanitary conditions; slaughterhouse regulations; ventilation and hygiene of tenement housing; street pavement and maintenance; public gardens and parks; water supply; funeral services; specific taxes and levies.

\bisection{London reforms} in the 1820s, the Metropolitan Police is established; Prison Reforms; Poor Law reforms taking up Benthamian ideas of workhouses for the poor, uniforms, severe discipline, separation by gender and age; Asylums Act; Public Health Act; Water Supply; Sewers

\bisection{Private philanthropist reforms}\alignedmarginpar{Familistère, Bournville} private investments into housing and social conditions, `utopias'

\bisection{Glasgow, Scotland} a city on the European periphery benefitting from the Trans-Atlantic economy; nearby coal and iron deposits grows industry and economy; steamboats first used on inland waters then able to reach coastal waters and open sea, and iron ships;   international exhibitions in Glasgow (1888, 1901); firms have welfare programmes including housing, but no tenement housing in Scotland compared to terraced/row housing in England

\bisection{Tenement housing} housing shared by multiple dwellings (ie. apartments); pure grid pattern without facilities like toilets or running water; vertical segregation of classes living in the same building (Paris); thin and tall structures (NYC);

%%%%%%%%%%%%%%%%%%%%%%%%%%%%%%%%%%%%%%%%%%%%
\pagebreak
\section{Summary Urban Planning and Urbanism Practices 19th-21st century}
%%%%%%%%%%%%%%%%%%%%%%%%%%%%%%%%%%%%%%%%%%%%

\begin{center}
\begin{tabularx}{\textwidth} { 
  | >{\raggedright\arraybackslash}X 
  | >{\raggedright\arraybackslash}X 
  | >{\raggedright\arraybackslash}X | }
  \hline
  Urban Planning and Urbanism & Urban Social Policy & Urban Economic Policy \\ 
        \hline
  Type IV: New Urban Management, Collaborative Planning
  & Downsizing and rescaling the welfare state
  & Neoliberalism; city competition and marketing \\ 
      \hline
  Type III: Urban Development Planning
  & Golden age of comprehensive national welfare states
  & Regional and urban location policies and agencies \\ 
      \hline
  Type IIb: Post-WWII Modernist Urbanism 
  & Emerging variants/types of national welfare states
  & Nationalisation of heavy/basic industry \\ 
    \hline
  Type IIa: Inter-War Reformist Urbanism 
  & Pioneer welfare services and social housing in cities 
  & Communalisation of city services \\ 
  \hline
  Type I: pre-WW1 Civil Engineering and Beautification 
  & Sanitary legislation; asylums 
  & Liberalism \\ 
  \hline
\end{tabularx}
\end{center}

\begin{tabularx}{1.2\textwidth} { 
  | >{\raggedright\arraybackslash}X
  | >{\raggedright\arraybackslash}X 
  | >{\raggedright\arraybackslash}X 
  | >{\raggedright\arraybackslash}X
  | >{\raggedright\arraybackslash}X 
  | >{\raggedright\arraybackslash}X | }
  \hline
  Chapters
  & \textbf{Urban Engineering}
   & \textbf{Reformist} 
   & \textbf{Modernist} 
   & \textbf{Urban Development Planning}
   & \textbf{Urban Governance} \\
  \hline
  Timeline
  & 1860-1900s
  & 1900s-1940s
  & 1950s-1960s
  & 1960-1980s
  & 1980s-today \\ 
  \hline
  Themes
  & 
   & Reformist urbanism of the 1920s; early years of local welfare state in Europe; garden city and settlers' movement; public and social housing programs; new standards in zoning legislation and building codes; beginnings of regional planning 
   & Modernity on both sides of the Iron Curtain; comparing welfare capitalist and socialist European city of the 1950s-60s; reconstruction and urban development post-WWII, under different welfare and housing regimes; urban lifecycles and suburbanisation; socialist cities
   & Golden age of welfare in 1970s European cities; comprehensive planning, public coordination of all kinds of urban developments; hey-days of mass social housing; state as a pioneer of urban renewal
   & Restructuring and resistence; commodification of urban development since the 1980s; EU integration and enlargement; European cities in competition; urban entrepreneurialism and project planning; challenge of social cohesion and sustainability; urban marketing and branding \\
  \hline
  Planning Types (Suitner)
  & Pre-WWI Civil Engineering and Urban Design
   & Inter-war Reformist Urbanism, Social Planning
   & Post-WWII Modernist Expert Planning
   & Urban Development Plans or Comprehensive Planning
   & Strategic Management, Collaborative Planning \\
  \hline
  Planning Philosophies (Selle)
   & Urban Engineering
   & Catchment Planning
   & Catchment Planning
   & Urban Development Plans
   & Perspective Planning \\
      \hline
  Tools (Albers)
  & Alignment plans
  & Zoning
  & Zoning
  & Development plans
  & Projects \\
      \hline
  Goals
  & Safety as main concern, planning straighten streets and regulate heights and widths; rest is left to the market
  & Not only ordering but also regulating uses behind the walls
  & Ambitions to plan everything related to the urban, a `super-zoning' approach using project management at all levels of planning
  & 
  &  \\
   \hline
\end{tabularx}

%%%%%%%%%%%%%%%%%%%%%%%%%%%%%%%%%%%%%%%%%%%%
\pagebreak
\section{Urban engineering and fin-de-siècle urbanism}
%%%%%%%%%%%%%%%%%%%%%%%%%%%%%%%%%%%%%%%%%%%%

\textit{The beginning of `serious' planning. Technical and engineering challenges are emerging in growing cities (mid to late 19th century), and this period is focusing on massive infrastructure to make cities safe and enable their expansion. The goal is to avoid health crises (eg. cholera) or natural disasters (eg. flooding). It is about averting risk, adaptation planning, and laying the technical foundations which enabled later social and political phases (eg. Red Vienna).}

\bisection{Planning culture} tool is alignment plans that provide safety, by ordering streets, heights and widths, and the rest is left to the market

\bisection{Building codes} new building codes to regulate height of buildings and minimum street width; this improves airflow and hygiene.

\bisection{Grid pattern} originating from NYC and became a planning norm

\bisection{Urban engineering elements} technical infrastructure elements height zoning, outer suburbs incorporated, trees planted along boulevards, gas lights in streets, electricity and electric street cars, bridges for metropolitan railroad, granite pavements, new sewers

\bisection{Urban design elements}\alignedmarginpar{Ringstrasse} beautification accompanies urban engineering through parks, green belts, cultural and educational institutions, tenement palaces, world exhibition (serve to spread ideas), urban furniture. Some works like public hospitals can fall in both engineering and beautification categories

\bisection{Tenement housing} luxury tenement palaces open, built in different styles (neo-gothic, neo-baroque, neo-renaissance)

\bisection{Transporting ideas, patents, technologies}\alignedmarginpar{corrugated iron} world exhibitions are mega-events which drive innovation by spreading ideas across countries/continents. Starchitects recycle ancient architectural styles to beautify cities\alignedmarginpar{Athens in Vienna}

%%%%%%%%%%%%%%%%%%%%%%%%%%%%%%%%%%%%%%%%%%%%
\pagebreak
\section{Reformist urbanism pre-WWII}
%%%%%%%%%%%%%%%%%%%%%%%%%%%%%%%%%%%%%%%%%%%%

\textit{Cities are growing and need some organisation and order, and a reconfiguration of places and borders, to respond to the ills of the industrial city and urbanisation. This period is the origin of social welfare and democracy, when there are qualitative improvements in education (schools), health system (hospitals), social work (orphenages, social housing, public pools, libraries). Nonetheless previous elements of planning, like engineering (pipelines, electricity, sewers) are still around.}

\bisection{Planning culture} tool is zoning

\bisection{New towns and garden cities}\alignedmarginpar{Lechtsworth, 1903; Le Corbusier Villes Radieuses} starting in the early 1900s as a solution to mass housing; follow the principles of separating uses and homesteading, where there is healthy living and working; emerge as `new towns' in the UK; the idea internationalises through conferences, spreading through Europe, North Am., Russia, etc.

\bisection{Settlers' movement}\alignedmarginpar{Otto Wagner hospital vs. allotment gardens vs. garden city}people are given land to grow food during the war as a subsistence wartime economy; after WWI, these allotment gardens are turned into cooperative, grassroots movement creating spontaneous unplanned suburbs, contrast to garden cities 

\bisection{Social housing} during WWI investments in housing stopped, hyperinflation destroyed loans, rapid rent increase because of unregulated rental market, war industry affect industrial and rural areas; social housing emerged from socialist states as way to regulate housing market. Introduced rent freeze and social housing, an upgrade from tenement housing - has more green space, inner courtyard, kindergarten

\bisection{Zoning} previously only focusing on height regulation of buildings (19th century) but allowed almost all land use types. Early 1900s introduced green belts, industrial zones, in a concentric model, disallowed mixed uses. In 1930s, first high-rises arrive in Europe

\bisection{Regional planning}\alignedmarginpar{conurbation} resistance to incorporation of more and more land by cities into `greater' cities (19th century) creates a need for regional planning because urban regions need coordination for transport systems, recreation areas. Happening at a similar time than suburbanisation, and creates urban cores (where people work) and commuter zones (where people live). Settlement looks continuous, and no visible rural area per se

%%%%%%%%%%%%%%%%%%%%%%%%%%%%%%%%%%%%%%%%%%%%
\pagebreak
\section{Modernist urbanism post-WWII}
%%%%%%%%%%%%%%%%%%%%%%%%%%%%%%%%%%%%%%%%%%%%

\textit{The emergence of welfare states after 1945 means that the concepts from the inter-war reformist period has superseeded. Planning is reconstructed in parallel to the new democratic times, and infused by economic and social policies}

\bisection{Garden cities} The idea expanded after 1945 into New Towns. Cities had to be reconstructed, and debate emerged on housing density - should housing be high-rise (requiring significantly less land, more dense) or low-rise (housing spills into outer rural areas, less dense). The city should be build and ordered by its function

\bisection{Social housing} Golden age of social housing, with a mass production of social housing improving in quality (eg. size, bathroom, heating). Public housing was the biggest share of the housing supplies in some countries. State offered subsidies for building housing and housing morgages. Rent liberalisation phased out affordable housing

\bisection{Zoning} Continued, improved, and extended to the whole urban area. Includes land-use regulations, careful planning of what uses should be allowed next to each other, and mixed-zoning is over

\bisection{Regional planning} Continued resistance to incorporation especially in new democratic times, as such regional planning remains rare and difficult. But it is needed, so regions find other ways such as coordinating cycling paths, public transport systems with uniform tarrif system, planning recreational and protected areas, etc

\bisection{Urban social policy} In 19th century, there were limited social policies, only insurances for the elite. In 20th century, in the inter-war period, pioneering systems start providing social health, education, housing. Social policies popularise post-1945 with a new type of welfare system under capitalism, impinging on planning

\bisection{Urban economic policy} In the 19th century, urban economic policies are non-existant and investors are free to do what they want. In 1920s, private supply of elementary infrastructure is questionned as inefficient, unreliable, expensive. Post-1945, State focuses on building heavy industry (public gas, metal), until 1970s when States start attracting investors and international political institutions through locationalised policies\alignedmarginpar{UN City Vienna}

\bisection{Welfare and housing regimes} Nation States become ambitious and create many welfare policies, including housing, education, health. Different regimes organise their welfare systems differently\alignedmarginpar{Welfare triangle}. Liberal regime centered on the market; social democratic regime centered around state supply; conservative regime centered aroundfamily supply and informal connections. In housing sectors, unitary vs. dualist regimes where public and private markets influence each other to various degrees

\bisection{Stages of urban development} Dynamics of urban/suburban/desuburban/deurban-isation change population of core and ring zones, and are dynamics to be tamed and managed with planning

\bisection{Socialist and post-socialist urbanism} Political systems such as communism impact urban development, with socialist cities having different characteristics and urban dynamics.

%%%%%%%%%%%%%%%%%%%%%%%%%%%%%%%%%%%%%%%%%%%%
\pagebreak
\section{Urban development plans}
%%%%%%%%%%%%%%%%%%%%%%%%%%%%%%%%%%%%%%%%%%%%

\textit{Starting in 1960s-70s. The urban development plans are ambitious plans for everything `urban', but also embedded in a broader, fully-developed national welfare state (not just the embrionic welfare city of the 1920s). Development plans coordinate a multitude of actors/investors and developments, and don't refrain from steering the market. The state is leading urban renewal, and not private investors. Today, urban development plans take many, many forms, is more or less flexible and precise, and includes private actors and isn't only state dominated}

\bisection{Planning culture} tool is urban development plans (massive planning documents), using project management at all levels (from individual buildings to entire city) for coordination

\bisection{Golden age of welfare}\alignedmarginpar{Sweden's distinct regime} large investments from state into pillars of the welfare state: housing, health, education. Urban renewal (of city centre and new suburbs) are state-led, and not from private investors

\bisection{Housing welfare} myriad of land banking systems; direct housing subsidies where State gives money to developers and individuals to construct dwellings (`aide à la pierre', object subsidies), or money given directly to individuals proportional to income (`aide à la personne'); as well as indirect subsidies with stax rebates

\bisection{General plans}\alignedmarginpar{Sweden as inspiration, fingerplans} started in 1920s, popularised post-WWII. Planning expanded, development axes growing towards new suburban districts/satellite towns along transport stops, opening of subways, pedestrian and cycling paths, redevelopment of run-down city centres; all state led

\bisection{Urban development planning} plans are comprehensive, area-wide, project oriented, influencing markets. The goal is to: maximise choice in welfare supply\alignedmarginpar{both public \textit{and} private transport, subway and highway}, improve efficiency of welfare administration, more mixed zoning, city planning coordinates all sectoral planning, focus on technical infrastructure, deepen academic involvement (sociologists), new planning culture involves more than experts (public meetings, debates)

\bisection{Coordination} urban development plans coordinate public actors actors, such as: government departments, local/regional departments, appointed agencies, private enterprises, community interests. Also coordinate developments like public transport systems

\bisection{City developments}\alignedmarginpar{Donau City} high rise buildings appear in 70s, as offices and council housing. Require strong integration with urban development planning because they require an agreement from the city: requires access to public-transport, no obstruction of protected views

\bisection{Mass social housing} considered a pillar of the welfare state, social housing production reached its peak in 1970s; it is not only a social policy but also regional planning, energy policy, economic policy

\bisection{State-driven urban renewal}\alignedmarginpar{Sweden's distinct housing regime} the state focused on inner city renewal, created historic preservation zones, lowered densities in certain districts, introduced parking management. The rediscovering inner city centre and the gentrification movement were (inadvertedly?) started by the state, and regional and urban agencies try to attract firms and influence the market

%%%%%%%%%%%%%%%%%%%%%%%%%%%%%%%%%%%%%%%%%%%%
\pagebreak
\section{Urban management, collaborative planning}
%%%%%%%%%%%%%%%%%%%%%%%%%%%%%%%%%%%%%%%%%%%%

\textit{From 1990s, planning is impinging on society in a different way than it used to. Planning has to coordinate many actors and is strategic (think PPP). Economic forces are leading urban development much more than before, the welfare state is restructured, the EU is making steps to integration and enlargements by funding large urban projects, cities are concerned with urban competition and branding, sustainability concerns integrated in projects.}

\bisection{Welfare restructuring/retrenchment} the golden age of welfare is over, states become more greedy and distributes benefits only after proof of need, and these requirements are more and more restrictive

\bisection{Collaborations} increasing collaboration between markets, states and individuals. The number of relevant actors are increasing, as well as interactions and complexity

\bisection{Planning culture} based on strategic projects, planning is locally focused on one area, and thus pays less attention to the rest of the city/region (from regional to project planning). Planners grasp chances when they can, anywhere in the world and takes initiatives.

\bisection{Neoliberalisation, commodification, financialisation} tendency towards secrecy, less democratic and more elite-drive priorities to avoid disagreements (non-public agreements bypassing local democratic assemblies). Projects are helped by marketing, selective deregulation\alignedmarginpar{urban development corporations, PPP contracts}, there's a shift from social to economic policy, with growth coalitions and growth machines

\bisection{ASID} Agency, Structure, Institutions, Discourse 

\bisection{Entrepreneurialism}\alignedmarginpar{Harvey 1989} an emerging type of urban governance, in which comprehensive urban development planning is taken over by planned urban `fragments'. Inter-urban competition reduces relative autonomy of local states, fosters growth coalitions/machines and PPPs, and competition regulations make it necessary to enter in EU/global market. 
Urban entrepreneurs are making use of dense networks of cities for tourism and attraction of new classes, and urban spectacles are important

\bisection{Urban development corporations}\alignedmarginpar{London Docklands} some territories are excluded from regulations as special economic zones

\bisection{Growth coalitions and growth machines}\alignedmarginpar{GaWC} include local businesses, politicians, local media, urban services, unis, cultural institutions. Cities try to attract businesses and wealthy inhabitants to feed government funds, through local boosterism that broadcasts attractity of the city, using rankings of investment climate, attractive transport connections, instrumentalisation of cultural events, urban branding, visionary plans

\bisection{Urban branding and imagineering} ideas of cities are transported with the help of media and social media, and communicating optimism. Contributing to this is the festivalisation of urban developments

\bisection{Festivalisation of urban development planning}\alignedmarginpar{Seestadt festival, Donau City, Bilbao effect} is done with: mega-events, a tradition starting with world exhibition, rebranded into eg. cultural capitals, sports events kicking off projects; flagship projects; starchitecture

%%%%%%%%%%%%%%%%%%%%%%%%%%%%%%%%%%%%%%%%%%%%
% 						NOTES
%%%%%%%%%%%%%%%%%%%%%%%%%%%%%%%%%%%%%%%%%%%%

\begin{comment}
\begin{chronology}[10]{1900}{2010}{100ex}[\textwidth]
\event{1930}{Bismarckian (pioneers)}
\event{1960}{Beveridgian (divergence)}
\event{1997}{Convergence}
\end{chronology}
\end{comment}

\printbibliography

\end{document}
