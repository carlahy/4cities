\documentclass{article}

\linespread{1}
\usepackage[utf8]{inputenc}
\usepackage[left=1.5in,right=1.5in,bottom=1in]{geometry}
\setlength\parindent{0pt}
\setlength{\parskip}{1em}
\setcounter{secnumdepth}{0}
\usepackage{outlines}
\usepackage{graphicx}
\graphicspath{ {imgs} }
\usepackage{hyperref}
\usepackage{color,soul}

\usepackage[
backend=biber,
style=apa,
citestyle=authoryear,
sorting=nyt,
]{biblatex}
\addbibresource{refs.bib}

\usepackage{comment}
\specialcomment{topicsen}{\begingroup\bfseries\scriptsize}{\endgroup}
%\excludecomment{topicsen}

\newcommand{\alignedmarginpar}[1]{%
        \marginpar{\raggedright\small #1}
    }

\title{Principles of Urban Planning and Urbanism}
\author{Carla Hyenne}

\begin{document}

\maketitle

\tableofcontents

\pagebreak

yellow = urban planning

\section{Mesopotamia}

...

\section{Planning the Ancient City}

\subsection{Greek Urban Planning}
...

\subsection{Roman Urban Planning}

\textit{March 15th 2022}

Volterra, Tuscany: an Etruscan city in 7th century BC shows a grid system with large avenues.

Rome's Golden Age lasted 200 years, from 50BCE to 150CE\alignedmarginpar{(Before) Common Era}. It was an unplanned city which gradually improved. The site of Rome was on hills in a flood-prone low-land, and flooding levels were engraved on building walls.


The first priority for such a large city as Rome, was to get rid of the sewage which is a health hasard. \textit{Cloaca Maxima} was introduced as the sewer of Rome.

The second priority was water, for which aqueducts were built (first in 312). The system was not water efficient, and the volume of water that flowed was much more than necessary. 14 aqueducts were built, delivering 1 billion litres/day. It flowed continuously because they didn't have proper taps, no water storage, and aqueducts were not pressurised. Since water flowed in the streets, the sidewalks were high and raised pavement stones served as crosswalks.

The third concern was food. Daily food doles were distributed, pork and 1.4kg of bread. 

Infrastructures like the forum, Colosseum and city walls were built and expanded under different emperors. 

Housing was socially mixed, and to accommodate the population growth, buildings were built wit additional stories (from 3 stories in 3rd century BCE to 5 stories in 1st century BCE). These constructions became unstable due to the additional weight, even though they were built from solid brick.
Residential buildings had an enclosed patio. The ground floor had shops and storage rooms, and people lived above.
The city is noisy, dark at night, and polluted.

\subsubsection{Pompeii}

An unplanned old town, and a planned new city. They had the same raised sidewalks and crossing stones to avoid the flowing water, housing with atriums.

\subsubsection{Novaesium, a Legion's Camp}

Novaesium, or Neuss, is a city in North-Rhine-Westphalia near Düsseldorf, with many Roman sites.

The camps had exactly the same layout so that soldiers would always find their way around. There were regulations on where and how bathrooms, for eg., should be built.
They were built on elevated grounds to avoid flooding like in Rome. They were built along the Danube (from the Rhine, down to the Black sea) which created a boundary. 

Outside the camp, there were civilian settlements around, which ca be seen in Vienna's city centre.

\subsubsection{Ancient cities in contemporary knowledge}

What is the point of this knowledge nowadays? Tourism and events, but also urban imaginaries. Knowledge is being recycled to create importance and awareness in the city.

\section{Planning the Medieval City}

\section{Planning the Absolutist City}

\section{Fighting the Evils of the Early Industrial City}

\section{Urban Engineering and Urban Design}

\section{Readings}

\subsection{Rome: \textit{The Establishment of the Urban Order, The Imperial City} \parencite{hall1998cities}}

\begin{outline}
	\1 Ancient Rome was the first mega city, estimated at 1 million people. Although it wasn't extremely technologically advanced, its success (in terms of population, and power I guess) is in part due to its urban infrastructure and developments
	\1 It was an extremely unequal society
	\1 Rome had features we recognise in modern cities: dense living spaces with (abusive) landlords, courts, theatres, basilicas, large gathering spaces (colosseum), forums, sewage systems,...
	\1 Housing conditions
		\2 Apartments were a hasard, because they were made of wood and we prone to fires. Rome had `vigils' who were firemen and looked out for fires. Some kind of red clay bricks were banned because they were not sturdy enough building materials
		\2 Richer households had more luxurious houses, but the comforts were still limited (eg. most lacked running water). Rich people purchased agricultural land outside the (western side?) of the city, which limited sprawl because the fields were in the way
	\1 Shops and baths
		\2 Baths were incredibly noisy place and a ``social necessity'' (p. 631), and emphasised the importance of a water system. Acquaducts were built to serve them. Baths started as privately run places, then came enterprises, and then came public baths
		\2 The streets were super narrow (2.9m wide), and always congested. Some streets were pedestrian only, and the rich travelled in cushioned chairs (sedan chairs) carried by men
		\2 Days were divided into 12 hours: first hour at sunrise, midday when sun was highest, and last hour at sunset. Thus hours were longer during the summer, and shorter during the nigh
	\1 Two main and complex challenges for Rome: to feed and water its large population. It managed to do both, even though it might not have been done the most efficiently
	\1 Water
		\2 Aqueducts were built to supply the city with water, supplying up to 1 billion litres of high quality water daily. Distribution is estimated to have been: 26\% for the emperor's service area, 32\% public use, 44\% private use. They were expensive to build and run, and a team of people were appointed to keep them running and maintained
		\2 Households used 60 times more water then, than today's standards, which was attributed mostly to the baths. Overflow water from public fountains was used to clean the streets, beneficial because there was no sewage system, and fountains also served to cool the city in the hot summers
	\1 Funding
		\2 Aqueducts were expensive and land had to be bought, and mostly funded through war booty (Rome fought many wars, in Europe, North Africa, Asia...), also through selling land
	\1 Sewers
		\2 People emptied their chamber pots outside, and thus residential areas didn't smell nice. The city relied on the heavy inflow of water through the aqueducts to dilute the sewage enough that it wouldn't be a problem when it arrived in the rivers/sea
	\1 Health
		\2 Average life expectancy was 27, infant mortality rates were 20\%, and there was a class difference in life expectancy: the poor lived 20, the rich 30, mostly because the poor didn't have enough to eat, and the living conditions spread diseases (cholera, plague, malaria...)
	\1 Food
		\2 Rome wasn't abundant in food, and didn't drain food products from the surrounding regions as can be seen with urbanisation trends in North Africa, for example (p. 649). It took a long time to travel (either by foot, or with ox or mule carriages) and thus food must have come from a 20-40 mile radius
		\2 Rome had colonies who produced grain, three quarters of the grain consumption must have come from outside of the city vicinity, usually by river and sea from other (today Italian) regions and countries, from public (imperial) and private merchants
		\2 Food was sold predominantly at the markets. The poor didn't have storage space, thus had to depend on small shops in the city, who probably got their stocks on the market
		\2 Daily food doles: free and rationed food (like corn) given to the masses by the emperor, to ensure good order; free pork for 150 days a year; child rations $\rightarrow$ a socialist system, and an unequal society?
	\1 Administration, public order
		\2 The emperor and the rich needed to keep poor people, the plebs, content, if they didn't want them to revolt. This included policing
		\2 Senators administrated Rome, and delegated the administrative tasks that kept public services running, managed the treasury, planned public works...
		\2 When the city became too large to effectively rule by one body, it was divided in 14 boroughs
		\2 The taxation system financed doles, games, public services, with money from the wealthier populations
	\1 Rome ended when the capital was moved to Constantinople. Interestingly, Rome didn't have many technological innovations, but perhaps this is due to its large pool of slave labour
\end{outline}

\subsection{Athens: \textit{The City as Cultural Crucible} \parencite{hall1998cities}}

\begin{outline}
	\1 Greeks were first, ie. introduced, many things: democracy, philosophy, political philosophy, systematic written history, systematised scientific and medical knowledge, comedy and tragedy, lyric poetry, naturalistic art, architecture... These were possible because of the environment, of the whole, which allowed people to trust their own judgement and trust the system (democracy), and think freely (critical debates)
	\1 Philosophy, or love of knowledge : humans moved away from believing in mythological stories, to questioning nature and the origins of the universe, asking themselves `why', and were critical of dogmas, myths, traditions, conventions.
	\1 Art was part of everyday life, and taken for granted. Theatre, comedy, depicted and talked about everyday life and people, and were able to criticise people (dead or contemporary). There was a theory behind the art, a mathematical logic, and understanding the correct proportions in drawing or painting human bodies was technically complicated
	\1 Architecture was like an art, and required a patronage. Athenians searched for natural and ultimate beauty, epitomised by the Pantheon. 

\end{outline}

\subsection{}

\begin{outline}
	\1
\end{outline}


\subsection{}

\begin{outline}
	\1
\end{outline}


\subsection{}

\begin{outline}
	\1
\end{outline}


\subsection{}

\begin{outline}
	\1
\end{outline}


\subsection{}

\begin{outline}
	\1
\end{outline}

\printbibliography

\end{document}
