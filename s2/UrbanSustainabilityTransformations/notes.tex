\documentclass{article}

\linespread{1.15}
\usepackage[utf8]{inputenc}
\usepackage[left=1.5in,right=1.5in,bottom=1in]{geometry}
\setlength\parindent{0pt}
\setlength{\parskip}{1em}
\setcounter{secnumdepth}{0}
\usepackage{outlines}
\usepackage{graphicx}
\graphicspath{ {imgs} }
\usepackage{hyperref}
\usepackage{color,soul}
\usepackage[normalem]{ulem}

\usepackage[
backend=biber,
style=apa,
citestyle=authoryear,
sorting=nyt,
]{biblatex}
\addbibresource{refs.bib}

\usepackage{comment}
\specialcomment{topicsen}{\begingroup\bfseries\scriptsize}{\endgroup}
%\excludecomment{topicsen}

\newcommand{\alignedmarginpar}[1]{%
        \marginpar{\raggedright\small #1}
    }

\title{Urban Sustainability Transformations}
\author{Carla Hyenne}

\begin{document}

\maketitle

\tableofcontents

\pagebreak

\section{What are Urban Sustainability Transformations?}

(ref. \cite{mcphearson2021radical})

\subsection{Urban Sustainability Transformations}

UST is an umbrella framework rather than a clearly defined pathway. It can range from changes in infrastructure, transportation, energy systems, food security, health issues, climate change. 

The transformations towards sustainability are non-linear expressions of complex interactions and consequences of a wide range of processes. Sustainability is a process, not an end-point, and can be a constantly shifting target.
The objectives of UST can also vary from city to city.

\subsubsection{What are transformations?}

Is it the process, the result? What starts or triggers the transformation? What are the roles of small-scale activities, can they lead to greater scale transformations, and if so why and how?

A transformation is a move from one path, to another better path. But what is better, who defines it, in terms of what? Is transformation always normative, does it always determine what direction we should be going?

\subsubsection{Why are transformations necessary?}

There are massive global challenges, such as climate change, that are putting increasing pressures on cities. To address these challenges, we require much more than small tweaks and incremental changes.
It is much more than scaling up current initiatives and innovations. This leads to a more holistic, intertwined social-ecological-technological systems (SETS).

\textbf{Radical change} necessitates investments in: knowledge, technology, institutions, modes of business, personal and socio-cultural behaviours and meanings.

\textbf{Radical transformative thinking} is required: provides systemic leverage, actionable ideas, and supportive governance processes to develop pathways for how local, regional and national innovations can be upscaled to drive global-scale sustainability transformations.

In transformative research, you actively contribute to ongoing or planned transformation. Often trans-disciplinary research, interventions, active involvement.

\subsubsection{How to achieve transformations?}

The goal is to provide conceptual and methodological pathways for radical change. 
The principles are to rethink growth, efficiency, the state, the commons, and justice, and to address these together and not in isolation.]]

There are five key actions to achieve transformations:

\begin{enumerate}
	\item Take a systems approach to all sustainability research
	\item Go beyond interdisciplinary research
	\item Co-produce and co-design sustainability research with communities
	\item Recognise and take actions that can push your research to question
	\item Create positive tipping points in urban and regional systems
\end{enumerate}

\subsection{Multi-Level Perspective}

The multi-level perspective is a framework to explain the complex, causal relations and processes. It is trying to visualise how different levels are interlinked and influence each other, and the social-technical system.

Macro-level, the landscape:

Meso-level, the regime: where we are living

Micro-level, the niches:

\subsection{Transformation vs. Transitions}

How does advertisement provoke worry, fear, or false advertising?


\section{Readings}

\subsubsection{McPhearson et al., \textit{Radical changes are needed for transformations to a good Anthropocene}, 2021}

\begin{outline}
	\1 
\end{outline}


\printbibliography

\end{document}

\subsubsection{ \textit{}}
\begin{outline}
	\1
\end{outline}


